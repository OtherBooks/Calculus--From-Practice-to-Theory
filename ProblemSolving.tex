\chapter*{Problem Solving: Some Advice From the Authors}
\label{cha:problem-solving}
\pagestyle{myheadings}
\markboth{{\sc Advice for Students}}{{\sc Advice for Students}}

Our purpose here is to give you our very best advice on how to succeed
in this course and a realistic impression of the level of difficulty
of the course material. Please read this carefully and take our advice. We,
the authors of this textbook, know what we are talking about.

Because the topic we will be studying is called
\underline{\bf{}Calculus} there is a strong tendency to assume that
that \emph{only} the concepts and techniques of Calculus are needed to
solve the problems that will be posed in this course. Nothing could be
further from the truth. You will need all of the mathematics you
know. And probably more. Really.

As a demonstration consider the following problem which we've chosen
because it is neither particularly easy nor particularly difficult. It
is typical of the kinds of ``medium level'' problems that appear in
most Calculus textbooks.

\noindent{}\underline{\bf{} Problem:}
  The lengths of two sides of a triangle are $a$ and $b.$ If the third
  side is chosen in such a way that the area of the triangle is as
  large as possible what is the length of the third side?

  You may be able to intuit the correct answer to this problem. That's
  ok, but you should try to solve it too. By ``solve'' we mean ``be
  able to explain why your answer is correct'' to someone with the
  same mathematical skills you have at the moment. Before reading
  further do your best to solve this problem by yourself. We'll wait.

  \vfill{} {\Huge\bf{} No, really. Give this problem a serious shot
    \underline{before} you turn the page. It is important that you
    try, whether you succeed or not.}  
  \vfill{} 
  \newpage{}

\noindent{}\underline{\bf{}Partial Solution:}

It can be difficult at first to see where to begin. This will frequently
be the case in the problems you encounter in mathematics or anywhere
else\footnote{That's why they're called problems.}.  \underline{Don't
  let this stop you!} In our experience the most common mistake
mathematics students make is giving up too soon. \underline{Don't do
  that.}  Keep thinking. 

\begin{wrapfigure}[]{r}{2in}
\captionsetup{labelformat=empty}
\centerline{\includegraphics*[height=1in,width=2in]{Figures/Advice1}}
\label{fig:Advice1}
\end{wrapfigure}
Since we know two sides of our triangle let's draw it. The sketch at
the right would be typical. The question is,
what is the length of the third side, $c,$  that
makes the total area of the triangle as large as possible.

Now what?

Well, this looks like a right triangle doesn't it? If so then we can
find the length of $c$ via the Pythagorean Theorem:
$$
c=\sqrt{a^2+b^2},
$$
right?

Can it really be that simple? Think about this for a moment. Can you
see any flaws in our reasoning?
\vfill{}
{\Huge\bf{} Once again, before you turn the page take a moment and really
  try to find any flaws in our reasoning. It is more important that
  you \underline{try} than that you succeed.}
\vfill{}
\newpage{}
Sure you can. There is no reason to believe that the triangle we seek
must be a right triangle. That was just an accident of our diagram. If
this seems like a really dumb mistake of the sort that you would never
make, be careful. It \emph{is} a mistake but it is an easy mistake to
make, especially when the problems get more complicated. This is not
dumb mistake. It is just a mistake, no dumber, and no smarter, than
any other mistake.

By definition, mistakes are wrong. They all look dumb
\underline{\bf{}after} you see them, but not before.  Everyone makes
mistakes in the course of solving a problem. The process of making
mistakes, recognizing them as mistakes, and figuring out why they are
mistakes is called {\sc learning}.  The very best, smartest people
(like Newton, Leibniz, Galileo, Fermat, or Einstein for example) made
lots of mistakes.  Making mistakes -- {\bf and then figuring out what was
  wrong} -- is how they got to be so smart.

Newton was once asked how he had been able to solve problems that no
one before him had. His reply: ``By thinking, and thinking, and
thinking about them.'' Of course, when he described his solutions left
out all of his errors because, who cares about that? As a result
making mistakes doesn't typically get the attention it deserves.

But make no mistake {\sc (Gasp!)} about it. It is very important to
make mistakes, but that is just the first step. You then need to
resolve them.

An expert is someone who has made every possible mistake. This is why
your teacher will seldom err\footnote{And will be embarrassed when
  (s)he does.}. Each mistake reflects the level of your current
understanding of the problem. Each mistake you make takes you a little
closer to expertise.  {\bf Embrace your mistakes and make lots of them.}
They are proof that you are making progress.

Sadly, for some reason mistakes are too often seen as a source of
embarrassment. Too many students berate themselves as stupid every
time they make a mistake. Don't do that. It is counter-productive. All
it will do is destroy your self confidence. Don't do it\footnote{Did
  we mention that you shouldn't do this. You really shouldn't.}. You
would not have made it this far if you couldn't do this, but it
\underline{\bf{}is} hard and there is still much to learn.  Keep
working. \underline{Keep making mistakes}, ask for help when you need
it, and don't give up. You have not failed until you stop trying.

\begin{wrapfigure}[]{l}{2.5in}
\captionsetup{labelformat=empty}
\centerline{\includegraphics*[height=1.5in,width=2.5in]{Figures/Advice3}}
\label{fig:Advice3}
\end{wrapfigure}
The triangle we seek might look like the first one we drew on
page~\ref{fig:Advice1}, or it might look like either of the ones drawn
in the diagram at the left. Or myriad others. We simply don't have
enough information to decide at this point. That is the problem.

But from the diagrams we've drawn so far we can see that side
$a$ is pinned at the left end of side $b$ so for each angle between
$a$ and $b$ we have a different possible triangle. 

Do you see how that worked? Yes, our first attempt was simple-minded.
But by drawing our first sketch and then figuring out why it won't work
we were led to this insight: We can think of $b$ as static, and we can
think of $a$ as swinging freely while pinned to the end of $b.$

\begin{wrapfigure}[]{r}{3in}
\captionsetup{labelformat=empty}
\centerline{\includegraphics*[height=1in,width=3in]{Figures/Advice4}}
\label{fig:Advice4}
\end{wrapfigure}
Clearly we need to find the angle between $a$ and $b$ -- let's call it
$C$ as in the sketch at the right -- that makes the triangle largest.
But which angle does that?

Whenever triangles are involved it is a good idea to recall your
Trigonometry. After all, that's what trigonometry is about, isn't it?
Also, since we're trying to maximize the area we should probably write
down the area formula: $A=\frac12(\text{base})(\text{height}).$ Since
we are thinking of $b$ as fixed we may as well use it as the base,
which makes the height equal to $a\sin(C).$ Thus the area of the
triangle is $A=\frac{1}{2}ab\sin(C).$

% Our diagram now looks like this:\\
% \centerline{\includegraphics*[height=2in,width=5in,angle=-1]{Figures/Advice4}}

Is it clear that our initial guess was correct? The angle, $C,$ that
maximizes the area will be the one whose sine is as large as
possible. That would be $\phi=\frac{\pi}{2}$ so $c=\sqrt{a^2+b^2}$
provides the maximum area.

This is the correct solution, but now we'd like to try to convince you
that it isn't.

\begin{wrapfigure}[]{l}{2in}
\captionsetup{labelformat=empty}
\centerline{\includegraphics*[height=1in,width=2in]{Figures/Advice5}}
\label{fig:Advice5}
\end{wrapfigure}
If we increase angle $\phi$ just a bit in the counter-clockwise direction
we add just a bit more area to the triangle, don't we? To be sure we
subtract some as well as shown in the sketch at the left, but we'll be
adding more than we subtract, won't we?

\begin{wrapfigure}[]{r}{2in}
\captionsetup{labelformat=empty}
\centerline{\includegraphics*[height=1in,width=2in]{Figures/Advice6}}
\label{fig:}
\end{wrapfigure}
Also, we didn't have to use $b$ as the base. We could have used $c$
instead.  In that case our diagram will look like the sketch at the
right.

In this case angle $\phi$ is clearly less than $\frac{\pi}{2},$ but
the area is the same, isn't it? This is puzzling, no? 

Can you account for either of these objections? If not, then the
problem remains unsolved.

As we said above, $C=\frac{\pi}{2}$ is correct so $c=\sqrt{a^2+b^2}$
is the maximum area of the triangle. But we need to \emph{know} this,
not just believe it. Unless we can show an irrefutable argument -- an
argument that addresses all possible objections -- then we just
believe it. And belief is not knowledge.

We will return to this problem and solve it completely on
page~\pageref{}. Until then, although we believe that
$c=\sqrt{a^2+b^2}$ is the maximum area of the triangle, it is only a
conjecture so far.

\noindent{}\underline{\bf{}Our advice, a synopsis:}\\

In order to solve a problem, any problem you must:
\begin{description}
\item[Have an idea.]     Solving a mathematical problem is a bit like getting dropped in
    the woods without a cell phone and being told to find your way
    out\footnote{Except, of course, that you won't die of exposure,
      and you always have the option of declaring the math problem
      ``Stupid'' and doing something more fun. If you are dropped in
      the woods you don't have that option. You \underline{must} solve
      the problem or you will die. It is astonishing how quickly that
      knowledge will focus the mind.}. First you have to find a
    path. Once a path is found you have to follow it. Nothing else
    really has a chance of working. 

    Having an idea is like finding a path in the woods. It's a start,
    but that's all it is. You still have to  follow it, and you
    probably still have a lot of hard work to do before you get out of
    the woods. 

    If an idea occurs to you, follow it. Most likely you will not hit
    on the best approach right away. You may not even hit on a good,
    or even a workable approach the first time. Or the first three
    times. That's ok. Keep thinking about the problem anyway. It is
    frustrating and it doesn't feel like progress, but it is. As long
    as you are having and discarding ideas you are making progress.
  \item[No, really. Have an idea. If you don't have a good idea, then
    use a bad one.] 

    Ok, we hear you say, but what if I don't have \emph{any} good
    ideas, just dumb ones? What do I do then?

    Use the dumb idea. 

    This is what you've been doing all of your life anyway isn't it?
    You just didn't tell anyone because you were sure you were doing
    something wrong, right? 

    You weren't. 

    The most important thing you can do is get started. That's what
    having an idea is for: getting started. All we did to start
    this     problem was draw the lines $a$ and $b.$ Look back up at
    the beginning of our solution and see. The first thing we did was
    make this drawing:\\
      \centerline{\includegraphics*[height=1in,width=2in]{Figures/Advice1}}

    Can you think of anything more simple-minded? And yet it
    worked. It didn't work right away, but it got us moving in the
    right direction.
  \item[Have another idea.] 
    Because this is a textbook we couldn't really waste time and pages
    by running down blind alleys, so we started off with an idea that
    we knew would take us in the right direction. In real life this
    usually won't happen. Most often your first (two or three or four)
    ideas aren't going in the right direction. That's ok. Figure out
    what is wrong with them. Something about the problem led you to
    your idea. Figuring out why it didn't work will clarify things for
    you just a bit. That will help. 

    So have another idea that is based on what you learned from your
    first idea. And another. And another after that. Keep having ideas
    until you find one that works. Another way to say this
    is, \underline{\sc{}Don't give up.}
  \item[Ask for help when you get stuck, not before.] Sometimes ideas
    just won't come to you. That's ok. Sometimes you will need
    help. Ask for it. Ask your teacher, another student, a tutor if
    one is available, your Mom, your Dad, a former teacher.

    Ask. For. Help.

    But ask constructively. If you are asking another student, a peer,
    it is fine to ask, ``How do I do this problem?'' But if you are
    asking your teacher, or a tutor, this sounds a lot like, ``This
    problem looks hard, my friends are meeting up in half an hour and
    besides I don't really want to spend any more time on it. Please,
    do this problem and let me watch so I can turn it in and go have
    fun with my friends\footnote{To be clear, your teacher knows that
      a serious student does not mean this, that you really just want
      to learn how to do the problem. The difficulty is that there are
      always some students who have no qualms about asking their
      instructor to do problem after problem with no intention of
      learning anything. They just want to copy the answers down and
      quit. Teachers quickly learn to identify these students from the
      way they ask questions. If you approach your instructor in the
      same fashion that a non-serious student does the response you
      get won't be as helpful as you would like.

      Teachers \emph{want} to help serious students, but non-serious
      students are a waste of time. Be sure you sound like a serious
      student.}.'' This is not usually effective. Moreover some
    teachers will get a little testy about it. Instead ask something
    like, ``I've tried this, and this, and this, but I keep getting
    stuck here. Can you give me some direction?''  That shows that you
    have already put real effort into solving the problem and are
    willing to continue working on it.

    Or, if you find the problem so mystifying that you can't even
    think of a first idea ask, ``I really don't know where to begin on
    this problem. Can you point me in the right direction to get me
    started?''

\end{description}




%%% Local Variables: 
%%% mode: latex
%%% outline-minor-mode: t
%%% TeX-master: "Calculus"
%%% End: 
