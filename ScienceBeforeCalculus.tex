\chapter{Science Before Calculus }
\label{cha:science-before-calc}
\markboth{{\sc Science Before Calculus}}{{\sc Science Before Calculus}}

\section*{Science Before Calculus}

{\bf{}Warning for teachers and students alike!  }

\begin{wrapfigure}[]{O}{1.5in}
%\vskip-.7cm{}
\captionsetup{labelformat=empty}
\includegraphics*[height=2in,width=1.5in]{Figures/NovaScientia}
%\caption{test}
\label{fig:Figures/NovaScientia}
\end{wrapfigure}
n
There is a real temptation to skip this section and get to ``the
Calculus part.'' Don't do it.

This is an error for two reasons.  First, the story of Calculus, what
sorts of problems it was invented to address and why the tools of
Calculus were developed in the way that they were, is very helpful in
understanding what Calculus is and what it is not. Second, the real
world  problems Calculus was invented to solve are not ``Calculus
problems\footnote{By which we mean that they are not drill problems
  designed to exercise you Calculus skills. There were, and still are,
very hard problems in science and technology that the tools of
Calculus can be used to solve, but only if your skills are already
honed.}.''
That is, to solve these problems the techniques of Calculus are not
the only mathematical tools that will we need. Indeed, frequently they
will not be the first tools will need, but the last.

This section addresses both precalculus\footnote{Define
  precalculus and pre-calculus.} and pre-calculus.  The techniques
predate Calculus and were subsequently supplanted by more general
calculus techniques, but these techniques allow students to practice
(relearn?) algebra and analytic geometry techniques while solving
actual problems instead of just practice.  Calculus would not exist
without these precalculus techniques, and fluency in them will be
crucial in applying Calculus to solve problems.  So, enjoy the entire
story of how Calculus became one of the great human achievements and
ushered in a golden period in mathematics and science.

\noindent{\bf A Problem to Think About}\\
Mathematics is such an integral part of the training of any scientist,
that it is hard to believe that the notion of using mathematics to
study physical phenomena is a relatively new notion.  Consider, for
example, the title page of the \emph{Nova Scientia (New Science)} written by
Niccolo Tartaglia in $1537. $

There is much allegory in this picture of which we will point out some
items.  Inside the large ring is a group of Muses including Geometria
and Arithmetica and others surrounding Tartaglia observing the
trajectory of a cannonball.  This represents the fact that this was
one of the first works which studied the science of projectile motion
(artillery fire) using mathematical principles rather than empirical
data.  At the door of the ring is Euclid, representing the notion that
one can only reach this inner ring through an understanding of
Euclid's Elements (geometry).  Clearly the man trying to scale the
wall does not know geometry as his ladder is woefully short.  The
smaller ring, separate from the larger ring is occupied by Philosophia
on a throne.  Of course, the only entrance to the ring of philosophy
is through the larger ring of mathematics.    At that gate are
Aristotle (on the larger ring side) and Plato (on the philosophy
side).  On the banner is the motto of Plato's Academy, ``Let no one
ignorant of geometry enter.''  Of course, such allegory is open to
interpretation, but the message is clear that mathematics would play
an important role in the ``new science.''   

But in order to apply mathematical principles to study physical
phenomena there must be some sort of rules which seem to govern
nature.  One of the tenets that scientists hold onto nowadays is
{\sc{}NATURE IS LAZY!}  Nature tries to do things in the most
efficient manner.  Whether this is true or not, it gives a starting
point to applying mathematics to study phenomena.  This was a driving
force to mathematicians in the $1600$s as they looked for techniques
for optimization.  We will look at some of this presently, but let's
start with some optimization problems of our own.

\noindent{\bf Example 1:}  Out of all rectangles with a fixed
perimeter, which one encompasses the greatest area?    

One might guess that the answer is a square.  To see that this is
true, consider a square whose side is $s.$  The perimeter of this square
is $4s$ and the area is $s^2.$  Suppose we now increase the length by $x.$
To maintain the same perimeter, we must now decrease the width by the
same $x.$  This $(s-x)$ by $(s+x)$ rectangle still has a perimeter of $4s,$
but its area is $(s-x)(s+x)=s^2-x^2<s^2.$  Thus the square has a larger
area than this rectangle with the same perimeter.  Consider the
following variation on the problem, which can be done in a somewhat
similar manner. 

\noindent{\bf Problem 1:}  Out of all rectangles with a fixed
perimeter, which one has the shortest diagonal?  Justify your answer. 

As mentioned above, Problem $1$ can be done in a manner similar to
Example $1.$  But what about this problem?

\noindent{\bf{}Example 2.}  Consider all square based boxes with a
fixed surface area $S.$ Does the cube enclose the largest volume?
Justify you answer.

This problem may or may not be able to be solved using techniques
similar to the above, but it is this type of optimization problem
which prompted mathematicians from the $1600$'s to develop techniques
that ultimately led to the invention of Calculus.  We will come back
to this problem later in this section.  

\noindent{\bf \underline{Fermat's Method of Aedequality}}\\

One of the mathematicians developing techniques for solving such
problems was Pierre de Fermat (1601-1665).  To address the problem of
optimizing a function, Fermat developed a method often called the
\emph{Method of Adequality.}  The modern word ``adequate'' has its origin from
the Latin verb \emph{adaequ\={a}re} which means to equalize.  When something is
adequate, then it is essentially being made equal to whatever your
needs are.  The equalizing appears in Fermat's method as follows.
Fermat noticed that the values of a function $f(x)$ are very nearly
equal near a maximum or minimum value.  This is noticeable during the
year with the number of hours of daylight.  At the summer and winter
solstices, the daily change in the amount of daylight is relatively
small as compared to that at the equinoxes.  This can be seen if we
look at a graph of this. 

%\centerline{\includegraphics*[height=3in,width=5in]{Figures/SantaBarbaraDaylight}}
\begin{wrapfigure}[]{l}{3in}
\captionsetup{labelformat=empty}
\centerline{\includegraphics*[height=1.75in,width=3in]{Figures/SantaBarbaraDaylight}}
\label{fig:}
\end{wrapfigure}
Notice that for days near the maximum which occurs at day $172$ and the
minimum which occurs at day $356,$ the number of hours of daylight does
not vary too much.  This reasoning is the basis nbehind Fermat's idea
to determine where a maximum or minimum occurs.  In general, for a
function $f(x),$ we set $f(x+h)=f(x).$  After performing some algebra, we
set $h=0$ to ``make it correct.''
  
In Example $1,$ we showed that out of all rectangles with a fixed
\begin{wrapfigure}[]{O}{1.2in}
%\vskip-.7cm{}
%\captionsetup{labelformat=empty}
\includegraphics*[height=.75in,width=1in]{Figures/GenericRectangle}
%\caption{test}
\label{fig:GenericRectangle}
\end{wrapfigure}
perimeter, the one with the largest area is a square.  We will use
Fermat's Method of Adequality to examine a related question:

\noindent{\bf{}Example 3.} Out of all rectangles with a fixed area,
does a square have the smallest perimeter? 

To look at this question, we need to transform it into a mathematical
problem where we can apply Fermat's method.  To do this, consider a
rectangle whose length is given by x and width is given by $y.$  The
area of the rectangle is given by $A=xy$ and the perimeter is given by
$P=2x+2y.$

The mathematical problem is to minimize $P$ given that $A$ is fixed.  To
apply Fermat's method, we need to use our constraint $A=xy$ to eliminate
one of the variables in $P.$  Solving for $y,$ we get $y=A/x.$  Substituting
into $P,$ we get 
$$
P=P(x) = 2x+\frac{2A}{x}.
$$
We used $P(x)$ to emphasize that $P$ is a function of $x$ alone.
(Remember that $A$ is fixed.)  

Fermat's method says to first set $P(x+h)=P(x):$
$$
2(x+h)+\frac{2A}{x+h}=2x+\frac{2A}{x}.
$$

Performing some algebra, we get 
\begin{align*}
  2x+2h+\frac{2A}{x+h}&=2x+\frac{2A}{x}\\
  2h&=\frac{2A}{x}-\frac{2A}{x+h}\\
  h&=\frac{A(x+h)-Ax}{x(x+h)}\\
   &=\frac{Ah}{x(x+h)}.\\
\intertext{Dividing by $h,$ we get}   
1&=\frac{A}{x(x+h)}.
\end{align*}
Setting $h=0,$ ``to make it correct,'' we get $1=A/x^2,$ so $x^2=A.$
Substituting into the expression for $y,$ we get $y=A/x=x^2/x=x.$  


Putting this back into context, we have that $P$ is optimized when
$y=x;$ that is, when the rectangle is a square.  The fact that $P$ is
minimized comes from the context of the problem; a moment's thought
tells us that by making the length as long as we wish and the width
appropriately short, there is no maximum value for $P.$

Reconsider the problem from earlier in the chapter.

\begin{embeddedproblem}{}
\begin{wrapfigure}[]{r}{.75in}
% %\vskip-.7cm{}
\captionsetup{labelformat=empty}
\includegraphics*[height=1.5in,width=.75in]{Figures/RectSolid}
% %\caption{test}
\label{fig:RectSolid}
\end{wrapfigure}
  Consider the following rectangular box with a square base.\\
  \begin{description}
  \item[(1)] Find a formula for the volume of the box, $V.$
  \item[(2)] Find a formula for the surface area of the box, $S.$
  \item[(3)] 	Assuming that $V$ is held fixed, use Fermat's Method
    of Adequality to find the dimensions of the box which will
    minimize $S.$  Is this box a cube?    
  \end{description}
\end{embeddedproblem}

If you think about Fermat's method, you quickly realize that Fermat is
exploiting the fact that at a maximum or minimum, the tangent line to
the curve is horizontal.  In fact, Fermat modified his method to
determine (the slope of) the tangent line to a curve.  We will
modernize Fermat's ideas as he talked of things such as subtangents
and we have replaced that notion with slope. 

As you know, we need two points to determine the slope of a line, so
with Fermat in mind, we choose two points on the curve $y=f(x)$ very
close together and use them to compute the slope of the line joining
them.

\begin{wrapfigure}[]{r}{3in}
\captionsetup{labelformat=empty}
\centerline{\includegraphics*[height=2in,width=2in]{Figures/Aedequality1}}
\caption{slope of secant line: $\frac{f(x+h)-f(x)}{x+h-x} = \frac{f(x+h)-f(x)}{h}$}
\label{fig:}
\end{wrapfigure}

After we perform some algebra, we set $h=0$ to ``make it correct.''
Setting $h=0$ essentially merges the two points together and the
secant line becomes the desired tangent line.  Let's try this on the
curve $y=x^2.$ We have that the slope of secant line joining $(x,x^2)$
and $(x+h,(x+h)^2)$ is given by
\begin{align*}
  \frac{(x+h)^2-x^2}{h}&=\frac{x^2+2xh+h^2-x^2)}{h}\\
  &=2x+h.
\end{align*}
Setting $h=0$ to ``make it correct,'' we have that the slope of the
tangent line at $(x,x^2 )$ is given by $2x.$ If we substitute the
values $x=-2, -1, 0, 1,$ and $ 2$ we obtain slopes of $-4, -2, 0, 2,$
and $4,$ respectively.  This does seem to be consistent with what we
can see on  the graph.
% \begin{wrapfigure}[]{r}{2in}
% \captionsetup{labelformat=empty}
% \centerline{\includegraphics*[height=1in,width=2in]{figures/aedequality2}}
% \label{fig:}
% \end{wrapfigure}

\begin{embeddedproblem}{}
  \begin{enumerate}
  \item Use Fermat's method for tangents to compute the slope of the
    tangent line to $y=x^3$ at the point $(x,x^3).$
  \item Substitute the values $x=-2, -1, 0, 1,$ and $2$ into the formula you
    obtained for the slope and graph this to see if it makes sense.
    You should notice something odd about the tangent lines.
  \end{enumerate}
\end{embeddedproblem}

As mentioned in the problem above, there is something a bit odd about
the tangent lines when you graph them, though the slopes appear to be
correct.  There is also a logical problem with just letting $h=0$ ``to
make it correct.''  If you've spotted it, good for you (maybe your
teacher will want to discuss it).  If not, that is ok as well, for the
method does provide the correct answer (even though the logic behind
it is suspect).  Examining these issues is for a later chapter as we
look at the foundations of the calculus.  For now, we will develop the
rules and learn how to use them.

While Fermat's Method of Tangents has a fundamental flaw (though it
provides the correct answer), a contemporary of his, Ren\'e{} Descartes
$(1596-1650),$ provided an alternative method which is free of logical
issues.

\subsection{Tangent Lines: Decartes's Method of
    Normals}
% It is difficult to explain to students the
%       necessity of the language invented for mathematics. Obviously we
%       could just say ``perpendicular'' and mean the same thing. Why
%       use ``normal''? The answer to this is much deeper than you might
%     think. The fact is that ``perpendicular'' has  a very specific,
%     geometric meaning. 

\label{sec:descartes-normals}

  \noindent{\bf Apollonius and Conic Sections}
  The problem of finding the tangent line to a curve is a geometry
  problem, so it is not surprising that some of the first attempts
  were geometric. The Greek mathematician Apollonius of Perga (circa
  262 BC to 190 BC) used geometric methods to construct tangent lines to
  the classical conic sections: the ellipse, the hyperbola, and the
  parabola.


  Since algebra had not yet been invented, Apollonius's methods were
  entirely geometric, and could not be easily generalized for curves
  that were not conic sections. 


Ren\`e{} Descartes (1596-1650)  was one of the pioneers in applying the
techniques of algebra to solve geometry problems.  In his \emph{La
  Geometrie} he remarked that the problem of finding the tangent to a
curve was ``. . .  not only the most useful and most general problem
in geometry that I know, but even that I have ever desired to know.''

Descartes' technique is often called the Method of Normals\footnote{In this context, ``normal'' means
      perpendicular.} because
what he actually finds is the line normal to (perpendicular to) a
curve.  Once the normal is obtained, then the tangent line is
perpendicular to that.

His method for finding the normal to a curve can be described as
follows.

Given a curve and a point $P$ on that curve, we find the circle
passing through $P$ whose center is on the $x$ axis. The line from the
center of that circle throught the point $P$ will be the normal (and
its perpendicular through the point (P) will be the tangent.

Let's take a look at how this might be done in a particularly simple
situation. Consider the line $y=3x+2$ and the point $(1,5)$ on that
line.

\InsertGraphic{}

We know that the slope of the line normal to this curve\footnote{Yes,
  we know that lines aren't curved. It is convenient
  to think of a line as a special case of a curve.} is $-1/3.$ Thus
the normal line passes through the point $(1,5)$ with slope $-1/3$ so
the equation of this line is 
$
y-5=-\frac13(x-1)
$
or 
$$
y=-\frac13 x+\frac{16}{3}
$$
which crosses the $x$-axis at $(16,0).$ Since the distance from
$(1,5)$ to $(16,0)$ is $\sqrt{122}$ we see that the equation of the
circle with center $(16, 0)$ and radius $\sqrt{122}$ is $(x-16)^2 +y^2
= 122,$ as seen below.

\InsertGraphic{}

We now have a point on our tangent line, $(1,5),$  and it's slope,
$3,$  so we can write down the equation of the tangent line:
\begin{align*}
  y-5&=3(x-1), \text{ or, in slope-intercept form}\\
  y&=3x+2.
\end{align*}

Wait a minute! That can't be! This is the equation of our original
line!

But think about it for a moment. When we say ``tangent line'' what do
we really mean? The first definition we usually encounter is ``a line
that touches a curve at exactly one point.'' But this can't possibly
work as a definition for a tangent line. To see why not consider the
tangent to the curve $y=x^3-3x$ at the point $(-1,2).$ It passes
through two points on the curve: $(-1,2)$ and $(0,2).$

When we use the word ``tangent'' in ordinary speech we usually mean
something more like, ``getting off of the main point, but in the same
direction that things are currently moving.'' When we think
of ``tangent'' in this way is is clear that a line and it's tangent
must be the same line. After all, there is only one line \emph{through a
given point} that \emph{always} points in a given direction.

But we've gotten off on a tangent.

Returning to the problem at hand it is clear that we didn't really
need to find the equation of our circle. Once we have the slope of the normal line,
the slope of the tangent is known. But this is an artifact of the
simplicity of our problem. Straight lines are easy. When the curve is
more complex we \emph{will} need the circle.

This is because there is one special curve, the circle, whose tangent line is
obvious. Pick a point on a circle and draw the radius to that
point. It is well known that the tangent line at that point will be
the line perpendicular to the radius.



% To find the tangent line to a general curve Descartes found the
% tangent circle to a curve and used that to determine the tangent line.
We'll illustrate Descartes' idea with the following example.

\begin{myexample}
Find the slope of the normal (and tangent) line to the curve $y=\sqrt{2x}$ at the
point $(2,2).$

\centerline{\includegraphics*[height=2.7in,width=4in]{Figures/DescartesCircle}}


Following Descartes' approach we look at the family of circles whose
centers lie on the $x$ axis and which pass through the point $(2,2).$
The dashed circle represents a generic member of that family of
circles.  Notice that typically these circles intersect the parabola twice.  We
are searching for the circle that hits the curve only once.  This is
the solid circle in our picture.  If we can find the center of that
circle then its radius through the point $(2,2)$ will be normal to the curve and we can find its
slope (and the tangent's slope).

% \begin{wrapfigure}[]{O}{3in}
% %\vskip-.7cm{}
% \includegraphics*[height=2in,width=3in]{Figures/CassegrainTelescope}
% \caption{}
% \label{fig:CassegrainTelescope}
% \end{wrapfigure}

If we let $(a,0)$ denote the coordinates of the center of a generic circle in that
family, then the equation of the circle with center $(a,0)$ is
$(x-a)^2+y^2=r^2$ where $r$ is the radius of the circle. Since we
require our circle to pass through the point $(2,2)$ this radius will
be the distance from $(2,2)$ to $(a,0).$ That is
$r=\sqrt{(2-a)^2+2^2}.$
Thus we have
$$
(x-a)^2+y^2=(2-a)^2+2^2.
$$

Substituting $y=\sqrt{2x}$ into the equation of the circle, we get
\begin{align*}
  (x-a)^2+2x&=(2-a)^2+4\\
  x^2-2ax+a^2+2x&=4-4a+a^2+4\\
  x^2+(2-2a)x+(4a-8)&=0
\end{align*}

It is tempting at this point, to use the quadratic formula to solve
for $x$ and get (typically two) values for $x$ in terms of $a.$
However, before do a bunch of unnecessary work, let's keep in mind
what we are trying to do.  We want to find the value of $a$ where the
circle and the curve intersect exactly once.  We really don't care
about $x$ at this point.  But think about this for a moment. If we use
the quadratic formula to solve this equation then we will get only one
solution precisely when the discriminant (the part under the square
root) is zero.  For our problem, the discriminant is
$(2-2a)^2-4(4a-8).$ Setting this equal to zero and solving, we get
\begin{align*}
  4-8a+4a^2-16a+32&=0\\
  4a^2-24a+36&=0\\
  4(a-3)(a-3)&=0\\
  a&=3.
\end{align*}

 Thus the center of the circle which intersects the curve exactly once
 is $(3,0)$ and the slope of the normal line is  $(2-0)/(2-3)=-2.$
 So the tangent line to the curve $y=\sqrt{2x}$ at $(2,2)$ has slope
 $1/2.$  

 Since we have both a point on the tangent line: $(2,2),$ and the
 slope of the tangent line we can use the point-slope form of the
 equation of a line to write down the equation of the tangent
 line: $$y=\frac{1}{2}x+1.$$
\end{myexample}

 \begin{embeddedproblem}{}
   Use Descartes' Method of Normals to find the slope of the tangent
   line to the curve $y=\sqrt{x}$ at the point $(4,2).$
 \end{embeddedproblem}

 \begin{embeddedproblem}{}
   In a variation of Descartes' method, find the tangent line to the
   curve $y=x^2$ at the point $(3,9)$ by considering the family of all
   the lines passing through the point $(3,9).$ Out of all those lines, only
   two will intersect the curve exactly once, the vertical line and
   the tangent line.  Use this method to find the tangent line to
   $y=x^2$ at $(3,9).$ What would happen if you tried this idea for
   the curve $y=x^3?$ (Maybe this explains why Descartes used
   circles.)
 \end{embeddedproblem}

 Descartes' method is not only clever, it is completely algebraic.
 But it is also unwieldy. And for curves that are more complex than
 parabolas it is often simply impossible to use.

 Descartes' Method of Normals was not the only technique developed for
 finding tangent lines before Calculus was invented, but it is
 typical in both it's  applicability and in the computational effort
 required.

 \begin{embeddedproblem}{}
   Use Descartes' Method of Normals to verify Fermat's result that the
   slope of the tangent line to $y=x^2$ at the point $(1,1)$ is given by
   $2.$  [You probably want to look at circles that are centered on the
   $y$-axis instead of the $x$-axis.
 \end{embeddedproblem}

 Whereas Descartes' Method is free of the logical pitfalls that
 Fermat's Method has, it is algebraically cumbersome to find where a
 polynomial has a double root.  A tool to examine this was found by
 Johann van Waveren Hudde $(1628-1704).$  We will state a special case
 of this rule.

\noindent{\bf{}Hudde's Rule:} Consider any polynomial $p(x)=a_0+a_1 x+ \cdots +a_n x^n.$ Form the following ``Hudde polynomial'' 
$$
p^* (x)=a_1 x+2a_2 x^2+\ldots+na_n x^n.
$$

If $r$ is a double root of $p(x),$ then $r$ is a root of the
Hudde\footnote{\textcolor{red}{\bf{}Put in Hudde problem later in the section on
    the product rule and the chain rule.  It could look like this:}

	Recall that we introduced Hudde's Rule on page XX to search for the double root of a polynomial (if it has one).  In fact, Hudde stated a more general rule than what we gave before.

\noindent{\bf{}Hudde's Rule:}  Consider the polynomial $p(x)=a_0+a_1x+a_2x^2+\ldots{}+a_n x^n.$ Let $a,b$ be any real numbers and form the Hudde polynomial 
$$
p^*(x)=a_0 a+a_1 (a+b)x+a_2 (a+2b) x^2+a_3 (a+3b) x^3+\ldots+a_n
(a+nb) x^n.
$$
If $r$ is a double root of $p(x)$ then $r$ is a root of the Hudde
polynomial $p^*(x)$ 

	Hudde proved this is a completely algebraic manner before the invention of Calculus, but with calculus, we can polish this off pretty readily.

        \begin{embeddedproblem}{}
          \begin{enumerate}
          \item Show that if $r$ is a double root of the polynomial
            $p(x)$ 
            then it is a root of $\dfdx{p}{x}.$  [Hint: If $r$ is a double root
            of $p(x),$ then $p(x)=(x-r)^2 q(x)$ for some polynomial
            $q(x).$]
          \item Show that the Hudde polynomial
            $p^*(x)=ap(x)+bx\dfdx{p}{x}$ 
            and use this to prove Hudde's Rule.
          \end{enumerate}
      \end{embeddedproblem}
} polynomial $p^*(x).$

\begin{embeddedproblem}{}
Suppose $p(x)$ is a quadratic polynomial with a double root $r.$  Thus $p(x)=a(x-r)^2$
\begin{enumerate}
\item Expand $p(x)$ and use this to form the Hudde polynomial $p^*(x).$ 
  Show that $p^*(r)=0$ (so that $r$ is a root of $p^*(x)).$
\end{enumerate}
\end{embeddedproblem}

Another contemporary of Fermat and Descartes, Gilles Personne de
Roberval $(1602-1675)$ developed a method for constructing tangents
which is a bit more dynamic in nature.  We will explore this dynamical
viewpoint later when we talk about Newton.  For now, let's look at
Roberval's idea.  Roberval thought of a curve in the plane as being
generated by a point whose motion is compounded from two known
motions.  By combining the ``velocity vectors'' of these motions, the
tangent vector (and thus line) will result.  We'll apply this concept
to a parabola.  You may think of a parabola as being determined by a
quadratic formula (which it is), but there are other ways to describe
it.  For example, one definition is that a parabola is the set of
points in a plane which are equidistant from a fixed point called the
\begin{wrapfigure}[]{r}{2in}
\captionsetup{labelformat=empty}
\centerline{\includegraphics*[height=1in,width=2in]{Figures/GeometricParabola1}}
\label{fig:}
\end{wrapfigure}
focus and a fixed line called the directrix.  A diagram is in Figure
1.  In Figure 1, the point $P$ on the parabola is the same distance from
the focus $F$ as it is from the directrix $d.$  This distance changes as $P$
moves along the parabola, but the distance from $P$ to $F$ will always
equal the distance from $P$ to $d.$

If we think of the motion of $P$ as being a combination of the motion
away from $F$ and the motion away from $d,$ then we can modify our
drawing to reflect this [Figure 2].  The resultant of these will
provide the tangent to the curve at $P.$ 
\begin{wrapfigure}[]{r}{2in}
\captionsetup{labelformat=empty}
\centerline{\includegraphics*[height=1in,width=2in]{Figures/GeometricParabola2}}
\label{fig:}
\end{wrapfigure}


While this geometric approach does not produce the equation of the
tangent line, it does have an interesting consequence in optics.  To
see this, let's examine how light reflects off a mirror.  As we had
noted earlier, when applying mathematics to natural phenomena, people
often start with the assumption that nature is lazy; that is, nature
tries to optimize.  Take, for example, a beam of light.  If one were
to shine a laser pointer in a room, then it would be reasonable to
assume that light travels in a straight line which minimizes distance
traveled.  This premise is reflected\footnote{Pun intended.}  in light
reflecting off a mirror.  Assuming that light travels the shortest
path, we can examine what path light would travel in this case.
Mathematically, we are given two points $A,$ $B$ and a line $m$ (for
mirror, not slope).  We wish to find the shortest path from $A$ to $m$
to $B.$

\centerline{\includegraphics*[height=1.5in,width=5in]{Figures/ReflectingMirror1}}

This can be done without Calculus by reflecting $B$ across $m.$  We will
denote this reflection by $B_R.$  Clearly the shortest path from $A$ to $m$
to $B_R$ is a straight line.

\centerline{\includegraphics*[height=2in,width=5in]{Figures/ReflectingMirror2}}

Specifically, this says that the solid straight line path in the above
picture is shorter than the dotted path.  If we reflect the parts of
the paths that are below line $m,$ then the solid path will still be
shorter than the dotted path.

\centerline{\includegraphics*[height=2in,width=5in]{Figures/ReflectingMirror3}}

This solid path represents the shortest path from $A$ to $m$ to $B.$  Notice
that on this shortest path, we have the following situation.

\centerline{\includegraphics*[height=2in,width=5in]{Figures/ReflectingMirror4}}

To sum up, this says that by traveling the shortest path, light
reflects off of a mirror so that the angle of incidence, $\angle 1,$ is
congruent to the angle of reflection, $\angle 3.$

\begin{embeddedproblem}{}
\begin{wrapfigure}[]{l}{2in}
\captionsetup{labelformat=empty}
\centerline{\includegraphics*[height=1in,width=2in]{Figures/GeometricParabola3}}
\label{fig:}
\end{wrapfigure}
  \begin{enumerate}
  \item Explain why the parallelogram in Figure 2 is a rhombus.
  \item	Given that the parallelogram in Figure 3 is a rhombus, show
    that the two angles, $\alpha,$ $\beta$ are congruent.
  \end{enumerate}
\end{embeddedproblem}
Given what we noted about how light reflects, this result says that
any light ray, radio wave, sound wave, etc. which comes in parallel to\\
% \begin{wrapfigure}[]{r}{2in}
% \captionsetup{labelformat=empty}
\centerline{\includegraphics*[height=1in,width=2in]{Figures/Telescopes}}
% \label{fig:}
% \end{wrapfigure}
the axis of the parabola (perpendicular to the directrix) will reflect
off the parabola to the focus.  Among other things, this is why
satellite dishes and primary mirrors in reflective telescopes are
parabolic.

% \begin{wrapfigure}[]{r}{2in}
% \captionsetup{labelformat=empty}
% \centerline{\includegraphics*[height=1in,width=2in]{Figures/Satellite2}}
% \label{fig:}
% \end{wrapfigure}

\begin{embeddedproblem}{}
Consider that an ellipse is the set of points in a plane, the sum of
  whose distances from two fixed points (foci) is a constant.  Using
  Roberval's idea, we can think of the motion of a point $P$ on the\\
% \begin{wrapfigure}[]{r}{2in}
% \captionsetup{labelformat=empty}
\centerline{\includegraphics*[height=1in,width=2in]{Figures/Ellipse}}
% \label{fig:}
% \end{wrapfigure}
  ellipse as being the result of a motion away from one focus and
  toward the other focus.

  Explain why the parallelogram in the figure must be a rhombus and
  use this to show that angle $\alpha$ is congruent to angle $\beta.$
\end{embeddedproblem}

\begin{wrapfigure}[]{r}{4in}
\captionsetup{labelformat=empty}
\centerline{\includegraphics*[height=1in,width=2in]{Figures/Kidney1} \includegraphics*[height=1in,width=2in]{Figures/Kidney2}}
\label{fig:}
\end{wrapfigure}
Again, applying this to optics, this says that any light ray, sound
wave, etc. emanating from one focus will reflect off of the ellipse to
the other focus.  Among other things, this reflective property is used
to treat kidney stones.  The treatment is called Extracorporeal Shock
Wave Lithotripsy (ESWL) and is really what the name says.  Lithotripsy
literally means ``rock grinding,'' and this technique uses shock waves
which are generated outside the body [See figures below]
% \begin{wrapfigure}[]{l}{2in}
% \captionsetup{labelformat=empty}
% \centerline{\includegraphics*[height=1in,width=2in]{Figures/Kidney2}}
% \label{fig:}
% \end{wrapfigure}

\begin{wrapfigure}[]{l}{2in}
\captionsetup{labelformat=empty}
\centerline{\includegraphics*[height=1in,width=2in]{Figures/CassegrainTelescope}}
\label{fig:}
\end{wrapfigure}
As you can see in the second figure, the table contains a cup which is
in the shape of part of an ellipse.  At one focus of the ellipse is an
electrode which generates shock waves.  These shock waves reflect off
the elliptical cup and converge on the other focus.  When the kidney
stone is positioned at that focus, the shock waves will pummel it into
small pieces which can be passed relatively painlessly through the
ureter. 


A hyperbola is the set of points in the plane, the difference of whose
distances from two fixed points (foci) is constant. This idea is
utilized in the secondary mirror of Cassegrain telescopes.

\begin{embeddedproblem}{}
% \begin{wrapfigure}[]{l}{2in}
% \captionsetup{labelformat=empty}
\centerline{\includegraphics*[height=2in,width=4in]{Figures/Hyperbola1}}
% \label{fig:}
% \end{wrapfigure}

Using the ideas of Roberval, show that any light ray, sound wave, etc. directed at one focus will reflect off of the hyperbola toward the other focus.
\end{embeddedproblem}

We've examined the reflection of light (without Calculus); what about
refraction of light as it passes from one medium to another, say from
water to air.

\centerline{\includegraphics*[height=2in,width=5in]{Figures/Refraction1}}

Notice that light does not travel the shortest path anymore, as this
would be a straight line from the fish to the kingfisher.  This seems
incongruous with our tenet before that nature will be lazy and light
will travel the shortest path.  We will get to that in a moment, but
for now we will try to determine what the path of refracted light is.

The physical law is known as Snell's Law, named after the Dutch
Astronomer Willebrord Snellius $(1580-1626)$ though it was accurately
described before that time.  In modern terms, Snell's Law can be
stated as follows.

\noindent{\bf Snell's Law of Refraction:} Suppose that light travels
with a velocity of $v_1$ in medium $1$ and velocity $v_2$ in medium
$2.$\\
\centerline{\includegraphics*[height=2in,width=4.5in]{Figures/Refraction2}}
Then the path that light follows satisfies 
$$
\frac{\sin(\theta_1)}{v_1} = \frac{\sin(\theta_2)}{v_2 }.
$$

A number of mathematicians (including Snell) provided derivations of
this, but we will focus (a bit) on Fermat.  To attack this problem,
Fermat assumed that light would travel the fastest path, not
necessarily the shortest path.  Notice that this is not incongruous
with our observations concerning reflecting light.  In that case, the
speed of light was constant so the shortest path was, in fact, the
fastest path.

Let's see what happens when we try to use Fermat's Method of
Adequality to find this fastest path.

\begin{embeddedproblem}{}
  \label{FermatSnell}
  Consider the following labeling in the above diagram:\\
  \centerline{\includegraphics*[height=2in,width=4.5in]{Figures/Refraction3}}
  Find an expression for the time $T$ traveled along the path from $A$ to
  $B$ in terms of the variable $x$ and the constants $a, b, c, v_1, v_2.$
\end{embeddedproblem}

If you try to apply Fermat's Method of Adequality to minimize T, you
will run across some pretty daunting algebra.  You should try to start
this, but don't get discouraged if you can't complete the algebra. (We
couldn't do it either.)  The square roots involved make the algebra
difficult at best.

Somehow Fermat was able to minimize this and derive Snell's Law, but
there aren't many Fermats around.  What about the rest of us?  This is
where Newton and Leibniz come in.

Deriving Snell's Law was the second application in Leibniz's $1684$ paper
on Calculus, \emph{A New Method for Maxima and Minima, as Well as Tangents,
Which is Impeded Neither by Fractional nor Irrational Quantities, and
a Remarkable Type of Calculus for This.}

The title may be long, but it is very revealing.  Clearly, the problem
of optimizing (finding maxima and minima) is of central importance (as
we mentioned before) and somehow finding tangent lines to curves is
involved.  Also, it is not impeded by fractional (exponents) nor
irrational quantities (square roots).  In this paper, Leibniz remarks,
``Other very learned men [such as Fermat] have sought in many devious
ways what someone versed in this Calculus can accomplish in these
lines as by magic.''

We will come back to the problem of deriving Snell's Law, utilizing
these systematic rules and techniques, but first we need to develop
these rules and become fluent with them before we can perform our
``magic.''

%%% Local Variables: 
%%% mode: latex
%%% TeX-master: "Calculus"
%%% End: 
