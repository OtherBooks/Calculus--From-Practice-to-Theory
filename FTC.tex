\chapter{The Fundamental Theorem of Calculus }
  \label{chapt:FTC}
\markboth{{\sc The Fundamental Theorem of Calculus}}{{\sc The Fundamental Theorem of Calculus}}
\aptta{}
\endaptta{}
Consider the region bounded by $y=x^2,$ $y=0,$ $x=1,$ $x=4$ as below\\
\centerline{\includegraphics*[height=2in,width=5in]{Figures/FTC-1}}
The problem at hand is to compute the area of this region.  With this
in mind, suppose we place an infinitely skinny box in that region as
below.\\ 
\centerline{\includegraphics*[height=2in,width=5in]{Figures/FTC-2}}
The height of this generic box is y.  As we've done in the past with
differentiation, we will denote the width of the box by $\d x.$  Thus the
area of this generic box is thus $y\d x.$  If we envision this region as
being composed of such boxes, then the area of this region would be
the (infinite) sum of the areas of these boxes, as below\\
\centerline{\includegraphics*[height=2in,width=5in]{Figures/FTC-3}}

  You might wonder, looking at the picture, how we can say that the
  area of the (curved) region can be the sum of rectangular boxes.
  Bear in mind that the drawing above is not totally accurate as the
  boxes are infinitely skinny and there are infinitely many of them.
  Later we will examine this more carefully and use this observation
  to approximate areas under more complicated curves, but for now
  let's imagine that the boxes are infinitely skinny.  We will see, as
  mathematicians saw in the 1700's, that this will be very fruitful.
  We need some notation for this sum.  Originally, Leibniz used the
  notation omn which was short for the Latin word omnis meaning all.
  Shortly after, he replaced omn with an elongated s which stood for
  the latin word summa (sum).  We use this notation today and write
  the sum of these infinite areas as 
$$
\text{Total area } =\int_{x=1}^4 y \d x =\int_{x=1}^4x^2 \d x.
$$

As we will see, having a good notation will help when we apply this
technique to more applications that go beyond computing areas, but it
doesn't tell us how to add infinitely many, infinitely small areas.
Surprisingly, we actually have the tools to accomplish this task, and
this realization was one of the things that made calculus such a great
tool from the 1700's until now.  To do this with our above example,
let's choose a random value $x=b$ and let $A(b)$ denote the area under the
curve $y=x^2$ from $x=1$ to $x=b.$  In other words,  
$$
A(b)=\int_{x=1}^bx^2 \d x.
$$
Suppose we increase $b$ by the infinitely small differential $\d b.$
If $\d A$
denotes the corresponding change in area, then we have $\d A=y(b)\d x=b^2
\d b.$ [See below.] \\
\centerline{\includegraphics*[height=2in,width=4in]{Figures/FTC-4}}
To recap, we have this function $A(b),$ we don't have a formula for it,
but we know some things about it. 
$$
\dfdx{A}{b}=b^2,\ \      A(1)=0
$$
We know how to solve this antidifferentiation problem.
$$
A(b)=\frac{b^3}{3}+C
$$
where $C$ is a constant. Using the fact that $A(1)=0,$ we get
\begin{align*}
  0&=A(1)\\
   &=\frac{1^3}{3}+C \text{ so that}\\
  C&=-\frac{1^3}{3}.\\
\intertext{Thus we have}
\int_{x-1}^bx^2\d x &= A(b)\\
&= \frac{b^3}{3}=\frac{1^3}{3}.\\
\intertext{Letting $b=4,$ we get} 
\int_{x-1}^bx^2\d x &= \frac{4^3}{3}-\frac{1^3}{3}\\
&= \frac{63}{3}= 21.
\end{align*}

We could apply the same technique to compute
$$
\int_{x=0}^\pi\sin x\d x.
$$
Letting $A(b) = \int_{x=0}^b\sin x\d x$ and letting $\d b$ denote an
infinitely small change in $b,$  we have $\d A=\sin b\d b$ which leads
to the antidifferentiation problem 
$$
\dfdx{A}{b} = \sin b,\ A(0)=0.
$$
Solving this, we get 
$$
A(b) =-\cos b+C.
$$
Using the initial condition $0=A(0)=-\cos 0+C,$ gives $C=-(-\cos0)$ so
that
$$
\int_{x=0}^b\sin x\d x = A(b) = -\cos b-(-\cos 0)
$$
and so
$$
\int_{x=0}^b\sin x\d x = -\cos\pi-(-\cos 0)=1+1=2.
$$
Notice that we deliberately did not simplify the above equation right
away as it demonstrates how our technique can be generalized.
Specifically, suppose we have a function, $y=y(x)$ with an antiderivative $Y,$ that
is $\dfdx{Y}{x}=y.$  To compute $A(b)=\int_{x=a}^by\d x,$ notice that
we have 
$$
\dfdx{A}{b}=y(b), \ A(a)=0.
$$
Solving this antidifferentiation problem, we have
$$
A(b)=Y(b)+C.
$$
Using the initial condition, we get 
\begin{align*}
  0&=A(a)\\
   &=Y(a)+C \text{ so that }\\
  C&=-Y(a).
\end{align*}
Thus
$$
\int_{x-a}^by\d x = A(b)=Y(b)-Y(a).
$$
The amazing fact that an area problem can be reduced to an
antidifferentiation problem is called the Fundamental Theorem of
Calculus, and can be summarized as follows. 
\begin{mytheorem}{\bf{}The Fundamental Theorem of Calculus}
  If $Y$ is an antiderivative of $y,$ then 
$$
\int_{x=a}^b y\d x = Y(b)-Y(a).
$$
\end{mytheorem}
Notice that the Fundamental Theorem states that $Y$ is
\underline{an} antiderivative of $y$ not \underline{the}
antiderivative.  This is because if a function has an antiderivative,
then it has many\footnote{Infinitely many, in fact.}. The theorem does
not require that we find a particular antiderivative. Any of them will
do.

\section{The Natural Logarithm}
\label{sec:natural-logarithm}

Suppose we have a single microbe living in a petri dish and that it
divides once every hour. Then after one hour we have two microbes and
our population has grown by one microbe per hour. After the second
hour our population is now $4$ microbes, and the rate of growth during
that second hour is $3$ microbes per hour. If this continues for $10$
hours our microbe population will have grown according to the
following chart\marginpar{This is supposed to be $P(t)=(5/2)^t.$ Check the numbers.}:
$$
\begin{array}{|ccc||ccc|}
\hline
  \text{hours}&\text{population}&\text{rate of
                                  growth}&\text{hours}&\text{population}&\text{rate
                                                                          of
                                                                          growth}\\\hline{}
  1&2.5&1.5\frac{\text{\tiny microbes}}{\text{\tiny hour}}&6&244.1&97.2\frac{\text{\tiny microbes}}{\text{\tiny hour}}\\[1mm]
  2&6.3&2.5\frac{\text{\tiny microbes}}{\text{\tiny hour}}&7&610.4&242.9\frac{\text{\tiny microbes}}{\text{\tiny hour}}\\[1mm]
  3&15.6&6.2\frac{\text{\tiny microbes}}{\text{\tiny hour}}&8&1525.9&607.2\frac{\text{\tiny microbes}}{\text{\tiny hour}}\\[1mm]
  4&39.1&15.5\frac{\text{\tiny microbes}}{\text{\tiny hour}}&9&3814.7&1518.0\frac{\text{\tiny microbes}}{\text{\tiny hour}}\\[1mm]
  5&97.7&38.9\frac{\text{\tiny microbes}}{\text{\tiny hour}}&10&9536.7&3795.1\frac{\text{\tiny microbes}}{\text{\tiny hour}}\\[1mm]\hline
\end{array}
$$
Can we find a function $P(t)$ which  tells us our microbe
population at any time?

\subsection{Logistic Growth}
Before, we looked at exponential growth models, that is models
governed by the equation
$$
\dfdx{p}{t}=rp.
$$


Here, $p=p(t)$ is the population level at time $t$ and $r$ is the
(continuous) growth rate of the population.  Recall that the solution
to this is the exponential function $p=p_0 e^rt$ where $p_0$ is the
initial population.  As we pointed out, this model is really only
applicable in the short run, as this function grows exponentially.
For example, if $r=10=.1$) and $p_0=1$ gram of a bacteria population,
and $t$ is measured in hours, then $p(100)=e^.1(100)\approx{}22026$ grams and
$p(200)=e^.1(200)\approx{}485165195$ grams.  Clearly, there are limitations to
this model.  This is the case with all mathematical models which, of
necessity, must be simplified to make tractable.  People who model
with mathematics understand this and the mantra is to start simple,
and add on extra complications.  While no model will be exactly real
life, the idea is to put in enough complexity to approximately real
life, while keeping the model simple enough to use.  We will do this
in this case.

For example, consider the differential equation
$$
\dfdx{p}{t}=.1p\left(1-\frac{p}{500}\right).
$$
Notice that this is similar to the differential equation for
exponential growth, but has an extra factor, namely $(1-p/500).$  The
$500$ is called the carrying capacity of the environment (notice that it
must be measured in grams to match the unit of $p$).  This number
represents the limitation placed on the population by the environment,
and does not allow the population to grow exponentially without bound.

\begin{embeddedproblem}{}
  \begin{description}
  \item[(a)] Show that for $0<p<500$ the population is growing, and
    for $p>500$ the population is shrinking.  Does the name carrying
    capacity make sense now?  Explain.  Plot a graph of growth rate
    vs. time.
  \item[(b)]  For what level of p is the population growing at the fastest
    rate?  What is this rate?
  \item[(c)]  For what values of p is the growth rate increasing (that is,
    the population's level is accelerating)?  For what values is the
    population's growth rate decreasing?
  \item[(d)] According to the model, what happens to the population if it
    should ever hit $500?$  What happens to the growth rate as p
    approaches $500?$
  \item[(e)] On the same set of axes, draw reasonable graphs of the
    population level $p(t)$ with the given initial values:
    $p(0)=100,$ $p(0)=300,$ $p(0)=500,$ $p(0)=600.$  (It is a fact from the
    theory of differential equations that these trajectories will
    never cross.  You can assume this when you draw your graphs.)
  \end{description}


This model is called a logistic growth model, and is a refinement of
the original exponential model.  Notice that for small values of $p,$ 
the factor $(1-p/500)\approx 1$ and the model is very much like exponential
growth.  However, this model is more refined than the original as it
better illustrates the fact that the population cannot grow
exponentially without bound.  Notice also, that this differential
equation is a bit harder to solve (we will learn integration
techniques to solve it later).  Nonetheless, we were able to use it to
provide reasonable qualitative graphs of $p(t)$ without actually having
an equation for $p(t).$ 
\end{embeddedproblem}

\begin{embeddedproblem}{}
  Let $f(t)$ represent the amount of a certain
  species of fish in a lake (in tons) at time $t.$  If fishing is
  allowed at a (uniform) rate of $20$ tons per year, then the growth
  rate of the population can be modeled with the differential equation
$$
\dfdx{f}{t}=0.25f\left(1-\frac{f}{500}\right)-20.
$$
\begin{description}
\item[(a)] What is the carrying capacity of the population?
\item[(b)] Plot the graph of $\dfdx{g}{t}$ vs. $t.$
\item[(c)] What will happen to the population over time if initially
  $f(0)=150?$
\item[(d)] Suppose initially $f(0)=150,$ but fishing is allowed at a
  uniform rate of $30$ tons per year.  How would the differential
  equation be altered and what would the model predict about the
  population over time?
\end{description}

  Alas, with every model, the logistic model also has limitations.
  Consider the following refinement of that model.
$$
\dfdx{p}{t} = 0.1p\left(1-\frac{p}{500}\right)\left(\frac{
p}{20}-1\right).
$$

  The number $20$ is called the minimum viability level of the
  population.
\end{embeddedproblem}

\begin{embeddedproblem}
  Plot the graph of $\dfdx{p}{t}$ vs. $t.$  In
  particular, where is the growth rate positive, negative, and zero.
  Explain why $20$ is called the minimum viability level of the
  population?
\end{embeddedproblem}

\begin{embeddedproblem}
  Use the graph of $\dfdx{p}{t}$ to plot
  reasonable graphs of $p(t)$ for various initial values.  Specifically
  use $p(0)=15, 20, 150, 400, 500, 600.$ 


  Notice that even without having a formula for $p(t),$ we were able
  to understand it from information given by the function
  $\dfdx{p}{t}.$ This function has a special name, the derivative. The
  history of this name is interesting and represents a shift in the
  study of calculus.  Throughout the 1700's, infinitesimal
  differentials were utilized to help explain physical and
  mathematical phenomena (as you have experienced in this book so
  far).  There were issues as to what an infinitely small quantity is,
  but it was overshadowed by its power and utility.  Too make a long
  story short, foundational issues as to what an infinitely small
  difference is grew in importance.  To help address these issues,
  Joseph Louis Lagrange developed the concept of a {\it fonction d\'eriv\'e{}e},
  that is a function derived from the original function.  Thus instead
  of focusing on curves, this marked the beginning of focusing on
  functions and derivatives instead of infinitely small differentials
  (useful as they are).  He even developed an new notation, $f^\prime(x),$
  instead of $\dfdx{f}{x}.$  Today these symbols are synonymous, but
  in Lagrange's day, it represented an attempt to suppress the
  infinitely small.  In practice, it is still advantageous to utilize
  the older differential notation, but there are times where the
  ``prime'' notation has its advantages, especially when one wants to
  focus on the derivative as a function rather than a process
  (differentiation).  One of the first places is what we've been
  doing, namely gathering information from a function by examining its
  derivative.

\end{embeddedproblem}

\subsection{Compound Interest and the Exponential Functions}
Suppose I invest money in a bond which nominally pays 5\% annually,
compounded quarterly.  The effective yield would be 5.09\%.  What does
any of this mean?

To start, the effective yield represents the actual amount of money I
will have earned at the end of one year.  For example, if the
investment was not compounded at all, then at the end on one year,
every dollar invested would yield $\$1.05$ in return.  If the investment
is compounded semiannually, then this means that half of interested
earned in paid out halfway through the year.  This money is then
re-invested for the next half year.  This is summarized in the
following table.
\begin{table}[h]
  \centering
  \begin{tabular}{|c|c|}\hline{}
    Time $t$ in years&Amount investment is worth in dollars\\\hline
    $1/2$ & $\$1+\$1\$\left(\frac{.05}{2}\right)=\$1\left(1+\frac{.05}{2}\right)$\\\hline
    1 & $\$1\left(1+\frac{.05}{2}\right)\left(1+\frac{.05}{2}\right)=\$1\left(1+\frac{.05}{2}\right)^2$\\\hline
  \end{tabular}
  \caption{Semiannual compounding}
  \label{tab:1}
\end{table}

If the investment was compounded three times in a year, we would have
the following modified table.
\begin{table}[h]
  \centering
  \begin{tabular}{|c|c|}\hline{}
    Time $t$ in years&Amount investment is worth in dollars\\\hline
    $1/3$ & $\$1+\$1\$\left(\frac{.05}{3}\right)$\\\hline
    2/3 & $\$1\left(1+\frac{.05}{3}\right)^2$\\\hline
    1 & $\$1\left(1+\frac{.05}{3}\right)^3$\\\hline
  \end{tabular}
  \caption{Compounding three times annually}
  \label{tab:2}
\end{table}
\newpage{}
Compounded quarterly means that the interest is re-invested three
times.  Our table looks like this:
\begin{table}[h]
  \centering
  \begin{tabular}{|c|c|}\hline{}
    Time $t$ in years&Amount investment is worth in dollars\\\hline
    $1/4$ & $\$1+\$1\$\left(\frac{.05}{4}\right)$\\\hline
    1/2 & $\$1\left(1+\frac{.05}{4}\right)^2$\\\hline
    3/4 & $\$1\left(1+\frac{.05}{4}\right)^3$\\\hline
    1 & $\$1\left(1+\frac{.05}{4}\right)^4\approx \$1.0509$\\\hline
  \end{tabular}
  \caption{Compounding quarterly}
  \label{tab:3}
\end{table}

The effective yield comes from the last entry in the table.  That is,
it is the difference between the amount the investment is worth at the
end of the year and what it is worth at the beginning of the year:
$1.0509-1=.0509=5.09\%.$

Following the same reasoning, if the investment was compounded daily,
then the effective yield would be $(1+.05/365)^{365}-1\approx.0513=5.13\%.$   

In general, if we compound n times in a year, then the return the
investment on one dollar at the end of one year would be
$$
\left(1+\frac{.05}{n}\right)^n.
$$

\noindent{\bf Alternative 1.}  Doing exponential function first, then
logarithm

Going one step further, if we let $p(t)$ denote the value of the investment of one dollar earning $5\%$ annually, compounded $n$ times per year, after $t$ years, then we have
$$
p(t)=\left(1+\frac{.05}{n}\right)^{nt}, \ \ t=0,1,2,\ldots
$$

What if the investment was compounded continuously?  What does this
mean, and what would the value $p(t)$ be?  To examine this, it will be
convenient to write $p(t)$ as
$$
p(t)=\left(1+\left(\frac{.05}{n}\right)\right)^{\frac{n(.05t)}{.05}}
=\left[\left(1+\frac{.05}{n}\right)^{\frac{n}{.05}}\right]^{.05t}
=\left[\left(1+\frac{1}{\frac{n}{.05}}\right)^{\frac{n}{.05}}\right]^{.05t}
$$

Let's rewrite $m=\frac{n}{.05},$ and consider what would happen if we
compounded a very large number of times (let $n$ be very large).
Table~\ref{tab:4} looks at some values of
$\left(1+\frac{1}{m}\right)^m,$ where $m$ is large.
\begin{table}[h]
  \centering
  \begin{tabular}{|c|c|}\hline{}
     $m$        & $\left(1+\frac{1}{m}\right)^m$\\\hline
    $100$       & $2.70481382942$ \\\hline
    $1000$      & $2.71692393224$ \\\hline
    $10000$     & $2.71814592683$ \\\hline
    $100000$    & $2.71826823717$ \\\hline
    $1000000$   & $2.71828046932$ \\\hline
    $1000000$   & $2.71828169255$ \\\hline
  \end{tabular}
  \caption{Compounding quarterly}
  \label{tab:4}
\end{table}

It would appear from Table~\ref{tab:4}, that $(1+\frac{1}{m})^m$ approaches a fixed number
as $m$ becomes indefinitely large.  This will be given more precision
later, but for now let's assume that it does and call this number $e.$ 
By the table, it appears that $e\approx{}2.71828169255.$  All of this suggests
that if the investment is being continually compounded, then
$$
p(t)=e^.05t.
$$
Let's examine the situation from a different perspective. If we abuse
notation by letting $p(t)$ be the amount that the investment is worth
at time $t$ in Tables~\ref{tab:1}, \ref{tab:2}, \ref{tab:3} as well,
then Table~\ref{tab:1} says
$$
p(0)=1,\ p(1/2)=(1+.05/2),p(0), \ p(2/2)=(1+.05/2)p(1/2).
$$
Likewise, Table\ref{tab:2} says 
$$
p(0)=1,\ p(1/3)=(1+.05/3)p(0),\ p(2/3)=(1+.05/3)p(1/3),\ p(3/3)=(1+.05/3)p(2/3). 
$$
and Table~\ref{tab:3} says 
$$
p(0)=1,\ p(1/4)=(1+.05/4)p(0),\ p(2/4)=(1+.05/4)p(1/4),\ p(3/4)=(1+.05/4)p(2/4),\ p(4/4)=(1+.05/4)p(3/4).  
$$
If we let $\Delta t=1/n$ denote the change in time, then this can be
written more succinctly as 
$$
\Delta p=p(t+ \Delta t)-p(t)=(1+.05\cdot\Delta
t)p(t)-p(t)=.05\cdotp(t)\cdot\Delta t.
$$

Extrapolating this to the case of continuous compounding where we use
the infinitesimal quantity ``$\d t$'' instead of the increment $\Delta t,$ we have
that the amount the investment is worth at time $t$ must satisfy the
differential equation $``\d" p=.05p\cdot ``\d" t.$

All of this suggests that $p=e^{.05t}$ satisfies the differential
equation $``d" p/``d" t=.05p$ with initial condition $p(0)=1.$  In
general, we would have $p(t)=e^{rt}$ satisfying the equations $``d" p/``d"
t=rp,\ \ p(0)=1.$

\begin{embeddedproblem}{}
  What would the effective yield be for a bond nominally rated at
  $5\%$ annually compounded continuously?  How does this compare to
  the effective yield of an investment compounded daily?
\end{embeddedproblem}

\begin{embeddedproblem}{}
  Suppose we had two investments growing continually with nominal
  rates of $5\%$ and $10\%$ annually?  Would the effective yield of
  the second investment be twice that of the first?  Justify your
  answer.
\end{embeddedproblem}

For $r=1,$ we have the function $p(t)=e^t,$ which is called the
natural exponential function and it satisfies $``d" e^t /``d" t=e^t,\
\ e^0=1.$

Such a function which grows this way constitutes exponential growth.
You might have heard the phrase ``growing exponentially'' referring to
something which grows very fast.  The following graph of $p(t)=e^t$
illustrates this\\
\centerline{ \includegraphics*[height=2.5in,width=3in]{Figures/ExpGraph}}
Indeed, Table compares values of $f(x)=x^2$ and $p(x) = e^x$ for
various values of $x.$
\begin{table}[h]
  \centering
  \begin{tabular}{|c|c|c|}\hline{}
     $x$ & $f(x)=x^2$&$p(x)=e^x$\\\hline
    $1$  & $1$ &2.71828\\\hline
    $2$  & $4$ &7.38906\\\hline
    $3$  & $9$ &20.08554\\\hline
    $4$  & $16$ &54.59815\\\hline
    $5$  & $25$ &148.41316\\\hline
    $6$  & $36$ &403.42879\\\hline
    $$   & $\vdots{}$ &\\\hline
    $100$& $10000$ &$2.68817\times 10^{43}$\\\hline
    $1000$   & $1000000$ &$1.97007 \times 10^{434}$\\\hline
  \end{tabular}
  \caption{Compounding quarterly}
  \label{tab:5}
\end{table}

Not only do continuously compounded investments grow in this manner,
but exponential growth can be used to model other phenomena (at least
in the short run).  As an example, consider the following situation. 

\begin{myexample}
  Consider a strain of bacteria growing in a Petri dish.  Initially
  there are $10$ mg of the bacteria growing continually at a (nominal)
  hourly rate of 25\%.  How much bacteria would there be at time $t$
  hours.

\noindent{\bf{}\underline{Solution:}}  Let $b=b(t)$ denote the amount of bacteria (in mg) at time $t$ hours.  We have that $b$ satisfies the following differential equation and initial condition
$$
\dfdx{b}{t}=.25b,\ \ b(0)=10.
$$

This says that  $b(t)=10e^.25t.$
\end{myexample}

The term nominal growth rate, though appropriate for investing money,
is not typically used in such a biological example.  Typically, such
problems are stated more in this fashion.



\begin{myexample}
  Consider a strain of bacteria growing in a Petri dish.  Initially
  there is $10$ mg of the bacteria and it is noticed that in $3$ hours,
  the amount has doubled.  How much will there be at any time $t.$ 

  \noindent{\bf{}\underline{Solution:}} As before, we let $b=b(t)$
  represent the amount of bacteria (in mg) at time $t$ hours.  Assuming
  that the bacteria grows at a rate proportional to the amount
  present, we have the following
$$
\dfdx{b}{t}=rb\ \ b(0)=10.
$$
where $r$ is some as yet to be determined constant.  At this point, we
know that $b=10e^rt,$ and we need to determine $r.$  To do this, we have
a different piece of information, namely, $b(3)=20.$  Substituting this
in, we get
$$
20=b(3)=10b^{3r}
$$
What we need to do is find the inverse of this exponential function.
With this in mind, we define the \emph{natural logarithm} function, denoted
$\ln x,$ as follows.
\centerline{$y=\ln x$ exactly when  $x=e^y.$}

Another way of saying this is that $\ln x$ is the exponent so that
$e^{\ln x}=x.$
For example, we have some fairly straightforward observations:
%   \centerline{$\ln 1=0$ since $e^0=1$ }
%   \centerline{$\ln e=1$ since $e^1=e$ }
%   \centerline{$\ln e^a=a$ since $e^a=e^a$}

At this point it would be useful to compare the graphs of $y=e^x$ and
$y=\ln x.$  Since they are inverse functions then one graph is the
reflection of the other about the line $y=x.$ 
\centerline{ \includegraphics*[height=2.5in,width=3in]{Figures/LogAndExp}}
\end{myexample}

Before we delve further into the natural logarithm function, let's see
how it applies to our problem above. 

We needed to solve the equation $20=10e^3r.$  To do this, we get
\begin{align*}
2&=e^3r\\
\ln 2&=\ln e^3r =3r\\
r&=ln2/3 \approx 2.463
\end{align*}

Thus the amount of bacteria at time t is given by 
$$
b\approx 10e^{2.463t}.
$$
Of course, as with any mathematical model, this is only an
approximation of real life.  As we noticed, this exponential function
would grow exponentially, and there is definitely a limit to the
amount of bacteria that will grow in a Petri dish.  At best, this
model can only be used for relatively small values of $t.$  Later we
will look at more complicated models for this growth, where we put
limitations in.  But for now, notice that we can play with our
function mathematically, we can get an equation that might be a bit
more appealing.
$$
b=10e^{\ln2/3 t}=10\left(e^{\ln 2}\right)^(t/3)=10\cdot2^{t/3}
$$

This general exponential function is a bit more appealing as it
encapsulates better the idea that the amount doubles every $3$ hours.
The natural logarithm function is actually the key to a number of
related properties of general exponential functions.  But before we do
that, let's consider the case of exponential decay.  You may have
heard of utilizing radioactive carbon dating to determine the age of
various fossils.  What does this entail and how is it related to what
we have been talking about?  Radioactive carbon dating was invented by
Willard Libby in the late $1940$'s at it became a standard tool for
archeologists to date their discoveries.  Libby received the Nobel
Prize in Chemistry in $1960$ for his work on this.  The idea is based on
the fact that radiocarbon ($C_{14}$) is constantly being created in the
atmosphere by the interaction of cosmic rays with atmospheric
nitrogen. The resulting radiocarbon combines with atmospheric oxygen
to form radioactive carbon dioxide, which is incorporated into plants
by photosynthesis; animals then acquire $C_{14}$ by eating the plants. When
the animal or plant dies, it stops exchanging carbon with its
environment, and from that point onwards the amount of $C_{14}$ it contains
begins to reduce as the $C_{14}$ undergoes radioactive decay. Measuring the
amount of $C_{14}$ in a sample from a dead plant or animal such as piece of
wood or a fragment of bone provides information that can be used to
calculate when the animal or plant died 
(http://en.wikipedia.org/wiki/Radiocarbon\_dating).  Exponential decay
comes from the fact that radioactive materials decay at a rate which
is proportional to the amount of radioactive material left.  This rate
is measured by the materials half-life.  For example, $C_{14}$ has a
half-life of about $5730$ years.  This says that if we have an amount of
$C_{14},$ then in $5730$ years we would have half of that amount, in another
$5730$ years we would have half of that (one quarter the original), and
so on.  The following example illustrates how the technique works.

\begin{myexample}
Let $c=c(t)$ represent the amount of Carbon-14 in a sample at any time
$t$ years.  If we let $c(0)=c_0$ denote the original amount in the
sample, then how much is in the sample at any time $t.$ 

\noindent{\bf{}\underline{Solution:}} We have that $c$ satisfies the
differential equation $\dfdx{c}{t}=rc.$  This says that $c=c_0e^{rt}.$
To determine $r,$ notice that $c(5730)=1/2 c_0.$  This gives us 
$1/2c_0=c_0 e^{5730r},$ $r=\ln(1/2)/5730.$

Thus we have
$$
c=c_0 e^{\ln(1/2)/5730 t}=c_0 \left(e^{\ln(1/2)}\right)^{t/5730}=c_0
(1/2)^{t/5730}.
$$

This is consistent with the idea that the amount halves every $5730$
years.  Of course, we can apply this to other situations as well. 
  
\end{myexample}

\begin{myexample}
  Suppose we have a wooden handle from a tool that contains 80\% of
  the amount of $C_{14}$ that it should contain.  How old is the tool?

\noindent{\bf{}\underline{Solution:}}  Applying the previous formula, we have
$$
c_0 e^{\ln(1/2)/5730 t}=.8c_0
$$
and we need to find $t.$  Solving we get:
\begin{align*}
  e^{\ln(1/2)/5730 t}&=.8\\
  \ln(1/2)/5730 t&=\ln(.8)\\
  t&=5730\ln(.8)/\ln(.5) \approx 1845 \text{years}
\end{align*}

\begin{embeddedproblem}{}
  How old would the tool be if the amount of $C_{14}$ was 70\% or 90\%?
\end{embeddedproblem}


\end{myexample}

%%% Local Variables: 
%%% mode: latex
%%% outline-minor-mode: t
%%% TeX-master: "Calculus"
%%% End: 

