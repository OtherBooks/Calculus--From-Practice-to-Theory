\chapter*{Introduction: Calculus is a Rock}
  \label{chapt:metaphors}
\pagestyle{myheadings}
\markboth{{\sc Calculus is a Rock}}{{\sc Calculus is a Rock}}

\begin{wrapfigure}[]{r}{2in}
%\vskip-.7cm{}
\captionsetup{labelformat=empty}
\includegraphics*[height=1.5in,width=2in]{Figures/Rechentisch}
\caption{By moving the pebbles around
  on the board you can add, subtract, multiply and divide numbers.}
\end{wrapfigure}

Commercial transactions were hard to do using Roman numerals. So in
Europe during the Middle Ages such computations were usually done
using pebbles on a small table called a counting board. The board was
marked off so that placing a pebble here we meant ``one'', there meant
``ten'' and so on. Think of an abacus. Obviously the only skills
needed to perform a given computation, was how to read the board and
how to move the pebbles around on the board to perform the
calculation.  The Latin word for ``rock'' or ``pebble'' is
``calculus.'' Thus the English word ``calculus'' has come to refer to
any computational scheme where it was not really necessary to
understand the computations in order to apply them, at least for basic
applications.

\begin{wrapfigure}[]{l}{2in}
%\vskip-.7cm{}
\captionsetup{labelformat=empty}
\includegraphics*[height=1.5in,width=2in]{Figures/abacus}
\caption{A traditional Chinese abacus (suanpan).}
\label{fig:abacci}
\end{wrapfigure}
  There are many calculuses (calculi?) in mathematics but the one we will be
  studying is of such profound and fundamental importance that it has
  come to be called {\bf Calculus} or {\bf The Calculus}.

  Nevertheless, it is a calculus in the original sense. That is, it is
  possible to simply memorize the rules for manipulating symbols
  (``move the pebbles'') and thereby solve a great many otherwise very
  difficult problems without any very deep understanding. There is
  nothing wrong with \underline{using} Calculus this way. Indeed, this
  is precisely what it was designed for. The computational rules you
  will learn shortly, once mastered, can be used to solve quite
  substantial problems fairly easily. The difficulty is that those who
  learn Calculus in that manner are in a position similar to the
  merchants of the Middle Ages. They can do the calculations but they
  don't understand anything except how to manipulate the pebbles (use
  the notation). So, if a problem comes up that is outside the reach
  of their pebbles they are completely helpless.



  This was not a problem for say, Marco Polo. He rarely had to do
  much mathematics besides add, subtract, multiply and divide. A deep
  understanding of these operations is not needed for the simple
  financial transactions of the Middle Ages. However, in the modern
  era such routine, rudimentary computations are done by machines. The
  role of humans is to figure out how to apply the fundamental
  concepts in novel arenas and in novel ways. Simply learning to
  compute, without a deeper understanding of the principles involved,
  is not a foundation upon which a modern education can be built.

  So, you will need a profound understanding of the underlying
  principles of Calculus.  An excellent way to learn these is to start
  with the computational techniques themselves. This is how we will
  begin.

  It is easy to mistake the rules for using the notation of Calculus
  (using the pebbles\footnote{OK. We're pretty sure you understand the
    metaphor by now so we'll stop giving the parenthetical cues.})
  with the Calculus itself.

  They are not.

  These are simply the rules for using Calculus which can be used
  blindly, with no understanding at all.
  % In fact, these days, the
  % computational rules can be programmed into a computer and used far
  % more efficiently than we can.
  But this does not make them unimportant. If anything it is more
  important than ever that you learn these computational rules because
  it is through mastering the fundamentals that the underlying
  concepts emerge and become clear.

  This process is slow. It takes time and practice and can be
  incredibly frustrating. Prepare yourself.

  On the other hand, at the other end, when it all comes together,
  there is a transcendent beauty to The Calculus which is impossible
  to convey to the uninitiated. A poet once wrote, ``Euclid alone has
  looked on beauty bare,'' but this is not true. Everyone who has
  studied and understood mathematics beyond the level of moving the
  pebbles has seen the same beauty that Euclid did. If you have
  never seen it, this is your chance. But you must focus on
  \underline{understanding} not mere \underline{computation}.

%   We will begin with problems. We will do this quickly with a
%   minimum of explanation because the primary goal here is to acquire
%   skill in using the notation. While we do this we will look at some
%   of the kinds of problems that we can solve using Calculus that are
%   \emph{ very} difficult without it.


    
% After that we will look more deeply into the underlying logical
% foundations supporting Calculus. In taking this approach we follow the
% history of our topic. 
  \label{MurkyFoundation}
  It may surprise you to learn that when it was first invented the
  validity of Calculus was very suspect. Indeed, the foundations of
  the topic were so murky that the only reason Calculus remained
  viable was the simple fact that it \underline{worked}. Or, at least,
  it seemed to. As you will soon see, many otherwise very difficult
  problems become straightforward if we know how to ``move the pebbles''
  correctly.

  % For example, for values of \(x\) between \(-\frac12\) and
  % \(\frac12\) the
  % \label{page:witch-of-agnesi-approx}
  % function $f(x) = \frac{1}{1+x^2}$ closely approximated by the
  % polynomial \(1-x^2+x^4,\)
  % as can be seen in figure~\ref{fig:Taylor1}. So for values of \(x\)
  % in that range we can replace \(\frac{1}{1+x^2}\)
  % with \(1-x^2+x^4\) with very little loss of accuracy.

  % This probably does not appear to be an especially useful fact
  % because the function $f(x)=\frac{1}{1+x^2}$ is particularly hard to
  % work with. However as the course wears on we will encounter some
  % functions which will be quite hard to compute exactly. It will be
  % very helpful if we can replace these with a simple polynomial which
  % approximates it to a high degree of accuracy.



%\end{chapter*}
%%% Local Variables: 
%%% mode: latex
%%% outline-minor-mode: t
%%% TeX-master: "Calculus"
%%% End: 
