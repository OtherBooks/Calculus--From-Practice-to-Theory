\chapter{Science Before Calculus}
\label{chapt:pre-calculus}
\markboth{{\sc Pre-Calculus}}{{\sc Pre-Calculus}}
\aptta{}
Mathematics is such an integral part of the training of any scientist,
that it is hard to believe that the notion of using mathematics to
study physical phenomena is a relatively ``new'' notion.  Consider, for
example, the title page of the \emph{Nova Scientia (New Science)} written by
Niccolo Tartaglia in 1537. 

\centerline{\includegraphics*[height=3in,width=2in]{Figures/NovaScientia}}

There is much allegory in this picture of which we will point out some
items.  Inside the large ring is a group of muses including Geometria
and Arithmetica and others surrounding Tartaglia observing the
trajectory of a cannonball.  This represents the fact that this was
one of the first works which studied the science of projectile motion
(artillery fire) using mathematical principles rather than empirical
data.  At the door of the ring is Euclid, representing the notion that
one can only reach this inner ring through an understanding of
Euclid's Elements (geometry).  Clearly the man trying to scale the
wall does not know geometry as his ladder is woefully short.  The
smaller ring, separate from the larger ring is occupied by Philosophia
on a throne.  Of course, the only entrance to the ring of philosophy
is through the larger ring of mathematics.    At that gate are
Aristotle (on the larger ring side) and Plato (on the philosophy
side).  On the banner is the motto of Plato's Academy, ``Let no one
ignorant of geometry enter.''  Of course, such allegory is open to
interpretation, but the message is clear that mathematics would play
an important role in the ``new science.''

Of course, how does one apply
mathematical principles to study physical phenomena?  To do so, there
must be some sort of rules which seem to govern nature.  One of the
tenants that scientists hold onto nowadays is NATURE IS LAZY!  Nature
tries to do things in the most efficient manner.  Whether this is true
or not, it gives a starting point to applying mathematics to study
phenomena.  This was a driving force to mathematicians in the 1600's
as they looked for techniques for optimization.  We will be looking at
some of these presently, but let's start with some optimization
problems of our own.
 
Consider this: Out of all rectangles with a fixed perimeter, which one
encompasses the greatest area? Give this some thought before reading
further.


One might guess that the answer is a square.  To see that this is
true, consider a square whose side is $s$.  The perimeter of this
square is $4s$ and the area is $s^2.$  Suppose we now increase the
length by $x.$  To maintain the same perimeter, we must now decrease
the width by the same $x.$  This $s-x$ by $s+x$ rectangle still has a
perimeter of $4s,$ but its area is $(s-x)(s+x)=s^2-x^2<s^2.$  Thus the
square has a larger area than this rectangle with the same perimeter. 


Consider the following variation on the problem.
 \begin{embeddedproblem}{}
   Out of all rectangles with a fixed perimeter, which one has the
   shortest diagonal? (Hint: This can be done in essentially the same
   manner as the previous problem.)
 \end{embeddedproblem}

Now how about this one:
\begin{embeddedproblem}{}
  Consider all square based boxes with a fixed surface area $S.$ Does
  the cube enclose the largest volume?  Justify you answer.
\end{embeddedproblem}

This problem may or may not be able to be solved using techniques
similar to the above, but it is this type of problem which prompted
mathematicians from the 1600's to develop techniques that ultimately
led to the invention of calculus.  We will come back to this problem
toward the end of the chapter.  
\endaptta{}

The interest in optimization problems -- finding the biggest or
smallest values of some variable quantity -- set the stage for the
invention of Calculus which solves exactly those kinds of problems. It
does much more of course, but it solves optimization problems
spectacularly well.

It is widely accepted that the Calculus was the invention (discovery?)
of two men, Isaac Newton (1643-1727) and Gottfried Wilhelm von Leibniz
(1646-1716). Newton developed his ideas at his home of Woolsthorpe
Manor in Lincolnshire while Cambridge University was closed by a
plague in the summer of $1665.$ In a period of less than two years he
made the extraordinary advances while still under the age of $25.$
Independently Leibniz developed his ideas during the period 1673-1675
while serving as a diplomat in Paris. He wrote the first paper about
calculus in $1684.$

  % As with most scientific discoveries, it would be a mistake to assume
  % that these two men worked in a vacuum.  Indeed, Newton is quoted as
  % saying, ``If I have seen further than others, it is by standing upon
  % the shoulders of giants.''  Who are these giants?

  The rather cumbersome title of Leibniz $1684$ paper: \emph{A New Method for Maxima
  and Minima, as Well as Tangents, Which is Impeded Neither by
  Fractional nor Irrational Quantities, and a Remarkable Type of
  Calculus for This,}  is very revealing.  Clearly,
  the problem of optimizing (finding maxima and minima) is of central
  importance and somehow finding tangent lines to curves is involved.
  That this is a \underline{\emph{new}} method means that other
  methods were available first. Indeed, Leibniz remarks in his paper that
  ``Other very learned men have sought in many devious ways what
  someone versed in this calculus can accomplish in these lines as by
  magic.''  Who are these learned men and what are these devious ways?

  The purpose of this section is two-fold.  Before we can exploit the
  ideas of calculus, we will examine some of the precalculus
  techniques from this pre-calculus\footnote{We will use the
    unhyphenated ``precalculus'' to refer to the mathematical methods
    which came before Calculus, and the hyphenated ``pre-calculus'' to
    refer to the time period before Calculus was invented.} period.
  This will not only give us some perspective when we move on to the
  computational rules of calculus, but will also will provide a
  framework for attacking mathematical problems with or without
  Calculus.


  The problem of finding the tangent line to a curve is a geometry
  problem, so it is not surprising that some of the first attempts
  were geometric. The Greek mathematician Apollonius of Perga (circa
  262 BC to 190 BC) used geometric methods to construct tangent lines to
  the classical conic sections: the ellipse, the hyperbola, and the
  parabola.


  Since algebra had not yet been invented, Apollonius's methods were
  entirely geometric, and could not be easily generalized for curves
  that were not conic sections. However we since we do have access to
  algebraic techniques we can derive his result for the parabola
  rather easily.

  \begin{myexample}
    Consider the curve $y=x^2:$\\
\centerline{\includegraphics*[height=2in,width=2in]{Figures/Quadratic}}
    We wish to find an equation of the line tangent to this curve at
    the point $(3,9).$
    It is clear from the graph that the line tangent to this curve at
    the point $(3,9)$ will also pass through the $y-$axis, say at
    the point $(0,b).$ If we can find $b$ we have the tangent line. 
    
    The slope of the tangent line is $m=\frac{9-b}{3}$ so the equation
    of the tangent line will be 
    \begin{equation}
      y=\left(\frac{9-b}{3}\right)x+b\label{eq:TanSq}
    \end{equation}
    for one, and only one, value of $b.$ If $b$ is anything other than
    that one value then equation~\ref{eq:TanSq} will intersect the parabola
    twice: once at the point $(3,9),$ and once at another point, say
    $(x,x^2).$ At this second point we have 
    $$
    x^2=\left(\frac{9-b}{3}\right)x+b.
    $$
    
    Rearranging this slightly we have 
    $$
    x^2-\left(\frac{9-b}{3}\right)x-b=0
    $$
    which is a quadratic and we can solve quadratics.

    Before charging blindly forward though, let's stop and think about
    what we're trying to accomplish. We want to find the value of $b$
    which guarantees that there is only \emph{one} solution. We're
    not really interested in the actual solution itself. We just want
    to ensure that there is only one.

    From the Quadratic Formula we see that:
    $$
    x=
    \frac{\left(\frac{9-b}{3}\right)\pm\sqrt{\left(-\left(\frac{9-b}{3}\right)\right)^2+4b}}{2},
    $$
    which has one solution precisely when the discriminant\footnote{The part under
    the square root symbol.} is zero. Thus
  \begin{align*}
   \left(\frac{9-b}{3}\right)^2+4b&=0\\
     \frac{9^2-18b+b^2}{9}+4b&=0\\
    9^2-18b+b^2+36b&=0\\
    b^2+18b+9^2&=0\\
    (b+9)^2&=0,
  \end{align*}
so $b=-9.$ Thus an equation of the line tangent to $y=x^2$ at the point
$(3,9)$ is
$$
y=6x-9.
$$

  

\begin{embeddedproblem}{}
  Use this method to find an equation of the line tangent to
  $y=x^2$ at each of the following points:\\
  \begin{description}
  \item[(a)] $(2,4)$
  \item[(b)] $(-3,9)$
  \item[(c)] $(1,1)$
  \item[(d)] $\left(\frac12,\frac14\right)$
  \item[(e)] $(a,a^2)$
  \end{description}
\end{embeddedproblem}  

\begin{embeddedproblem}{}
  Use this method to find an equation of the line tangent to the given
  curve  at the given point:\\
  \begin{description}
  \item[(a)] $y=x^2+x$ at $(2,6).$
  \item[(b)] $y=x^2+x$ at $(\tau,\tau^2+\tau).$
  \item[(c)] $y=x^2-3x+2$ at $(\tau,\tau^2-3\tau+2).$
  \item[(d)] $y=ax^2+bx+c$ at $(\tau,a\tau^2+b\tau+c).$
  \end{description}
\end{embeddedproblem}
  %   For example, he showed that the line tangent to the parabola at the
  % point $(0,b)$ in the following diagram will always pass through the
  % point $(0,-b).$
\end{myexample}

As we mentioned, although his methods were much different than this,
Apollonius was able to construct the tangent line to each of the conic
sections (the parabola, ellipse and the hyperbola). Construction the
tangent line to the ellipse and the hyperbola is still rather
difficult to do, even with algebraic methods we used above, so we will leave
them for another time.

\noindent{\bf Decartes's Method of Normals}

There is one special case where the tangent line of a conic section is
obvious. This is the special case of the circle. Pick a point on a
circle and draw the radius to that point. It is well known that
the tangent line at that point will be the line perpendicular to the
radius.

Ren\`e{} Descartes (1596-1650) used this fact to good effect in his
Method of Normals\footnote{``Normal'' means perpendicular in this
  context.}  Descartes was one of the pioneers in applying the
techniques of algebra to solve geometry problems.  In his \emph{La
  Geometrie} he remarked that the problem of finding the tangent to a
curve was ``. . .  not only the most useful and most general problem
in geometry that I know, but even that I have ever desired to know.''
% Descartes' technique is often called the Method of Normals because it
% uses algebra to find the normal (perpendicular) to a curve.  Once the
% normal is obtained, then the tangent line is perpendicular to that.

% His method for finding the normal to a
% curve can be described as follows.

% Given the equation of a curve $f(x,y)=0$ and a point $P$ on that
% curve, we can consider the family of circles whose centers are on the
% $x$ axis and passing through $P.$ We can algebraically find the center
% of the circle which intersects the curve only once.  The radius of
% that circle will be the normal (and its perpendicular through the
% point will be the tangent.)


To find the tangent line to a curve Descartes found the tangent circle
to a curve and used that to determine the tangent line.  We'll
illustrate Descartes' idea with the following example.

\begin{myexample}
Find the slope of the normal (and tangent) line to the curve $y=\sqrt{2x}$ at the
point $(2,2).$

\centerline{\includegraphics*[height=2.7in,width=4in]{Figures/DescartesCircle}}


Following Descartes' approach we look at the family of circles whose
centers lie on the $x$ axis and which pass through the point $(2,2).$
The dashed circle represents a generic member of that family of
circles.  Notice that typically these circles hit the curve twice.  We
are searching for the circle that hits the curve only once.  This is
the solid circle in our picture.  If we can find the center of that
circle then its radius will be normal to the curve and we can find its
slope (and the tangent's slope).

% \begin{wrapfigure}[]{O}{3in}
% %\vskip-.7cm{}
% \includegraphics*[height=2in,width=3in]{Figures/CassegrainTelescope}
% \caption{}
% \label{fig:CassegrainTelescope}
% \end{wrapfigure}

If we let $(a,0)$ denote the coordinates of the center of a generic circle in that
family, then the equation of the circle with center $(a,0)$ is
$(x-a)^2+y^2=r^2$ where $r$ is the radius of the circle. Since we
require our circle to pass through the point $(2,2)$ this radius will
be the distance from $(2,2)$ to $(a,0).$ That is
$r=\sqrt{(2-a)^2+2^2}.$
Thus we have
$$
(x-a)^2+y^2=(2-a)^2+2^2.
$$

Substituting $y=\sqrt{2x}$ into the equation of the circle, we get
\begin{align*}
  (x-a)^2+2x&=(2-a)^2+4\\
  x^2-2ax+a^2+2x&=4-4a+a^2+4\\
  x^2+(2-2a)x+(4a-8)&=0
\end{align*}

It is tempting at this point, to use the quadratic formula to solve
for $x$ and get (typically two) values for $x$ in terms of $a.$
However, before we go doing a bunch of unnecessary work, let's recall
what we are trying to do.  We want to find the value of $a$ where the
circle and the curve intersect exactly once.  We really don't care
about $x$ at this point.  But think about this for a moment. If we use
the quadratic formula to solve this equation then we will get only one
solution precisely when the discriminant (the part under the square
root) is zero.  For our problem, the discriminant is
$(2-2a)^2-4(4a-8).$ Setting this equal to zero and solving, we get
\begin{align*}
  4-8a+4a^2-16a+32&=0\\
  4a^2-24a+36&=0\\
  4(a-3)(a-3)&=0\\
  a&=3.
\end{align*}

 Thus the center of the circle which intersects the curve exactly once
 is $(3,0)$ and the slope of the normal line is  $(2-0)/(2-3)=-2.$
 Thus the tangent line to the curve $y=\sqrt{2x}$ at $(2,2)$ has slope
 $1/2.$  

 Since we have both a point on the tangent line: $(2,2),$ and the
 slope of the tangent line we can use the point-slope form of the
 equation of a line to write down the equation of the tangent
 line: $$y=\frac{1}{2}x+1.$$
\end{myexample}

 \begin{embeddedproblem}{}
   Use Descartes' Method of Normals to find the slope of the tangent
   line to the curve $y=\sqrt{x}$ at the point $(4,2).$
 \end{embeddedproblem}

 \begin{embeddedproblem}{}
   In a variation of Descartes' method, find the tangent line to the
   curve $y=x^2$ at the point $(3,9)$ by considering the family of all
   the lines passing through the point $(3,9).$ Out of all those lines, only
   two will intersect the curve exactly once, the vertical line and
   the tangent line.  Use this method to find the tangent line to
   $y=x^2$ at $(3,9).$ What would happen if you tried this idea for
   the curve $y=x^3?$ (Maybe this explains why Descartes used
   circles.)
 \end{embeddedproblem}

 Descartes' method is not only clever, it is completely
 algebraic.  It is also  unwieldy for curves that are of
 higher degree than that of a parabola (in our case $y^2=2x$). The
 search for finding a double root of a general polynomial led the
 Dutch mathematician Johann Hudde (1628-1704) to algebraically show
 that a double root of the polynomial $p(x)=a_0+a_1 x+a_2 x^2++a_n
 x^n$ must also be a root of the polynomial
 \begin{align*}
   q(x)&=a_1 x+2a_2 x^2+3a_3 x^3+\ldots+na_n x^n\\
       &=x\left(a_1+2a_2 x+3a_3x^2+\ldots+na_n x^{n-1}\right)
 \end{align*}

We won't derive this result, but we will return to it later when we
have built up some Calculus.

Recall that in the title of Leibniz' first paper on Calculus, there was
mention of a new method for maxima and minima, as well as tangents.
Well, we've certainly looked at older methods for tangents.  What
about older methods for maxima and minima; that is, to find values
which optimize a function (quantity).  As we will see, these problems
are intimately related.

A contemporary of 
%Roberval and 
Descartes was Pierre de Fermat
(1601-1665).  To address the problem of optimizing a function, Fermat
developed a method often called the Method of Adequality.  The modern
word ``adequate'' has its origin from the Latin verb adaequare which
means to equalize.  When something is adequate,  it is
 equal to whatever your needs are.   Equalizing appears
in Fermat's method as follows.  Fermat noticed that the values of a
function $f(x)$ are very nearly equal near a maximum or minimum value.
This is noticeable during the year with the number of hours of
daylight.  At the summer and winter solstices, the daily change in the
amount of daylight is relatively small as compared to that at the
equinoxes.  This can be seen if we look at a graph of
the hours of sunlight vs. day of the year~\footnote{http://www.geog.ucsb.edu/ideas/OtherImages/DaylightLengthInSantaBarbaraGraph500w.png}.

\centerline{\includegraphics*[height=2in,width=3in]{Figures/SantaBarbaraDaylight}}


Notice that for days near the maximum which occurs at day $172$ and
the minimum which occurs at day $356,$ the number of hours of daylight
does not vary too much.  This reasoning is the basis behind Fermat's
idea to determine where a maximum or minimum occurs.  In general, if we
are near a maximum, then for $h$ relatively small essentially $f(x+h)=f(x).$ After
performing some algebra, we set $h=0$ to ``make it correct.''  Let's
see how this works with an example.

Suppose we want to divide $10$ into two lengths so that the product of
the lengths is a maximum.  If we let $x$ denote one of the lengths, then
the other length would be $10-x.$  Thus the product of the lengths is
given by $P(x)=x(10-x)=10x-x^2,$ $0<x<10.$

It is important to notice, explicitly, what  values $x$ can have.
The domain of the function $P(x)$ itself is $(-\infty,\infty),$ but
since we are dealing with lengths, this restricts our domain.
According to Fermat, we should set $P(x)=P(x+h).$ Doing this we have
\begin{align*}
  10x-x^2&=10(x+h)-(x+h)^2\\
  10x-x^2&=10x+10h-x^2-2xh-h^2\\
  0&=10h-2xh-h^2\\
  0&=10-2x-h.
\end{align*}
Setting $h=0$ ``to make it correct,'' we get $0=10-2x,$ so $x=5$ and
$f(5)=5(10-5)=25$ 
provides the maximum product.  (We know it is a maximum and not a
minimum by the context of the problem.)

\begin{embeddedproblem}{}
  Consider all square based boxes with a fixed surface area $S.$
  Does the cube enclose the largest volume?  Justify you answer.
\end{embeddedproblem}

As a side note, notice that in the graph of the number of hours of
daylight that at the maximum and minimum points on the graph, the
tangent line is horizontal.  As you go further, you will see that this
concept is what drives Fermat's method.  We will be exploiting this
idea later after we learn some calculus tools.

These are a few of the ``devious methods''
sought by ``learned men'' to solve such problems.  This begs the
following question, ``If these techniques were already known, then why
do Newton and Leibniz get credit for inventing the Calculus?''


The answer is that pre-calculus people were using ad-hoc precalculus
methods to solve specific problems.  Newton and Leibniz certainly knew
of these methods.  What they did was to develop a systematic way of
solving these sorts of problems in general.  This is what Leibniz
refers to as ``A Remarkable Type of Calculus for This'' in his $1684$
paper. As we've observed the word ``calculus'' means ``pebble.'' and has
come to mean a set of rules for calculating.  Our purpose is to first
develop and then
exploit these rules/tools to solve these and related problems. By
using Calculus won't have to be as clever as these pre-calculus
``giants.''  They were all brilliant people. But brilliance is a lot
to ask of merely ordinary people like us. We'll have to settle for
being educated.



To apply Fermat's Method of Adequality to this problem, we must create
an equation for the volume so we can maximize it.  With this in mind,
let the dimensions of the box be $x$ by $x$ by $y$ and let the fixed surface
area be given by $S.$

\includegraphics*[height=2in,width=2in]{Figures/Parallelepiped}

Thus we want to maximize the volume $V=x^2y$ subject to the constraint
that  $2x^2+4xy=S.$

\begin{problem}{}
  Use the constraint to write $V$ in terms of $x$ alone and use Fermat's
  method to find the dimensions of the box that will maximize the
  volume.  Is this box a cube? 
\end{problem}

\marginpar{\textcolor{red}{Bob says everything from ``We had'' to the
    Problem section should go after we do the differentiation
    rules. Not clear to me where he means. We must discuss.}} We had noted earlier that to apply calculus to natural phenomena,
people often start with the assumption that nature is lazy, that is
nature tries to optimize.  Take, for example, a beam of light.  If one
were to shine a laser pointer in a room, then it would be reasonable
to assume that light travels in a straight line which minimizes
distance traveled.  This premise is reflected (pun intended) in light
reflecting off a mirror.  Assuming that light travels the shortest
path, we can examine what path light would travel in this case.
Mathematically, we are given two points $A, B,$ and a line $m$ (for mirror,
not slope).  We wish to find the shortest path from $A$ to $m$ to $B.$   

\includegraphics*[height=2in,width=2in]{Figures/Fermat1}
% These rules/tools were developed to be used on expressions (functions
% and equations).  With that in mind, we will have you practice the
% ``precalculus'' task of creating equations of one independent variable
% which model non-equation scenarios.

This can be done without calculus by reflecting $B$ across $m.$  We will
denote this reflection by $B_R.$  Clearly the shortest path from $A$ to $m$
to $B_R$ is a straight line.

\includegraphics*[height=2in,width=2in]{Figures/Fermat2}

Specifically, this says that the solid straight line path in the above
picture is shorter than the dotted path.  If we reflect the parts of
the paths that are below line $m,$ then the solid path will still be
shorter than the dotted path. 

\includegraphics*[height=2in,width=2in]{Figures/Fermat3}

This solid path represents the shortest path from $A$ to $m$ to $B.$
Notice that on this shortest path, we have the following situation.

\includegraphics*[height=2in,width=2in]{Figures/Fermat4}

To sum up, this says that by traveling the shortest path, light
reflects off of a mirror so that the angle of incidence, $\angle 1,$ is
congruent to the angle of reflection, $\angle 3.$  We've examined the
reflection of light (without calculus); what about refraction of light
as it passes from one medium to another, say from water to air.

\includegraphics*[height=2in,width=2in]{Figures/Fermat5}

Notice that light does not travel the shortest path anymore, as this
would be a straight line from the fish to the kingfisher.  This seems
incongruous with our tenant before that nature will be lazy and light
will travel the shortest path.  We will get to that in a moment, but
for now we will try to determine what the path of refracted light is.

\begin{ProblemSection}
  \begin{problem}{}
    The area of a rectangle is $400 m^2.$  Find a formula for the
    perimeter of the rectangle in terms of one of the sides of the
    rectangle.  Remember to define what your variables are (along with
    their units).  Also remember to include the possible values for
    your independent variable (the domain of your function).
  \end{problem}
  \begin{problem}{}
    \begin{description}
    \item[a)] You throw a ball straight up with an initial velocity of
      $50 m/s.$ Assuming that there is no air resistance and that the
      ball accelerates toward the earth at a constant rate of
      $9.8 ((m/s))/s,$ what would the ball's velocity be at $t$
      seconds.
   \item[b)] How would your formula change if you threw the ball
     downward?
    \end{description}
  \end{problem}
  \begin{problem}{}
    If a promoter charges $\$50$ each for concert tickets, then he/she
    will sell $15000$ tickets.  For each $\$1$ increase in price per
    ticket, $100$ tickets less will be sold.  Determine a formula for
    the revenue based on the price per ticket, assuming that the price
    will be at least $\$50$ per ticket.
  \end{problem}
  \begin{problem}{}
    The perimeter of an isosceles triangle is $100$ units.  Find the
    area of the triangle as a function of the base.  Don't forget to
    define your variables and give the domain of the function.
  \end{problem}
  \begin{problem}{}
    A cylindrical tank must hold $500$ cubic feet of water.  The ends of
    the tank are cut from two individual squares of metal, and the
    side is constructed from one rectangular sheet of metal.  The
    metal for the ends costs $\$5$ per square foot and the metal for the
    side costs $\$4$ per square foot.  The weld for the seams costs $\$2$
    per linear foot.  What is the cost of the tank (a) As a function of
    its height?  (b)  As a function of the radius?  Don't forget
    to identify variables and to provide the domain.
  \end{problem}
  \begin{problem}{}
    Consider a rectangular box with a square base whose volume is
    $500~ m^3.$ Find the surface area of the box as a function of its
    height.  Now find it as a function of the length of the side of
    the base.  Don't forget to identify your variables and to provide
    the domain.
  \end{problem}
  \begin{problem}{}
    \begin{description}
    \item[(a)] The fixed cost for publishing a specific calculus book
      is $\$100,000.$ These are costs that the publisher would incur
      no matter how many books were printed.  Beyond that, it costs
      $\$20$ per book to actually print the book.  If we let n denote
      the number of books printed and $C=C(n)$ denote the cost in
      dollars from printing this many books, then give a formula for
      the total cost to print $n$ books.  What would the cost per book
      be?
    \item[(b)] Suppose the company plans to charge $\$150$ per book.
      Assuming the company can sell as many books as it produces (as
      with a print-on-demand company), write a formula for the profit
      $P$ in dollars that the company would have from printing and
      selling $n$ books.  How many books would need to be sold to
      break even?
    \end{description}
  \end{problem}
\end{ProblemSection}
%%% Local Variables: 
%%% mode: latex
%%% outline-minor-mode: t
%%% TeX-master: "Calculus"
%%% End: 
