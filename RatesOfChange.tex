\chapter{Motion, Rates of Change, Tangent Lines}
  \label{chapt:RatesOfChange}
\markright{{\sc Motion, Tangent Lines, and Shapes}}


It is a truth universally acknowledged that the Calculus was the
invention (discovery?) of two men, Isaac Newton $(1643-1727)$ and
Gottfried Wilhelm von Leibniz $(1646-1716).$ Newton developed his
ideas at his home of Woolsthorpe Manor in Lincolnshire while Cambridge
University was closed by a plague in the summer of $1665.$ In a period
of less than two years he made the extraordinary advances while still
under the age of $25.$ Independently Leibniz developed his ideas
during the period 1673-1675 while serving as a diplomat in Paris. He
wrote the first paper\footnote{The rather cumbersome title of Leibniz
  $1684$ paper: \emph{A New Method for Maxima and Minima, as Well as
    Tangents, Which is Impeded Neither by Fractional nor Irrational
    Quantities, and a Remarkable Type of Calculus for This,} is very
  revealing.  Clearly, the problem of optimizing (finding maxima and
  minima) is of central importance and somehow finding tangent lines
  to curves is involved.  That this is a \underline{\emph{new}} method
  means that other methods were available first.} about calculus in
$1684.$


As with most scientific discoveries, although they are given credit,
it is a mistake to believe that these two men worked in isolation.
Indeed, Newton is frequently quoted as saying, ``If I have seen
further than others, it is by standing upon the shoulders of giants,''
and, Leibniz remarks in his paper that ``Other very learned men have
sought in many devious ways what someone versed in this calculus can
accomplish in these lines as by magic.''

\section{Rates of Change}
\label{sec:rates-change}

One of the ``learned giants'' that Newton and Leibniz referred to was
Galileo Galilei (1564-1642).

Galileo was interested in questions like these: Suppose you throw a
ball straight up into the air with an initial velocity of \(15\)
meters per second.  How high will the ball go?  How long will it take
for it to hit the ground?  What will its velocity be at impact?  If
you throw the ball up with twice as much initial velocity will it go
twice as high?

% These are interesting on their own. However Newton observed that many
% physical problems that are not about velocity or acceleration can be
% analyzed as if they are. That is, the mathematics is the same. We need
% only change the way we think about the meaning of our symbols.

We will spend some time examining how Calculus is used to describe the
motion of objects. If physics isn't your particular interest be
assured that this is only a starting point. Calculus is applicable to
the real world far beyond physics. However it was first invented as a
way to describe motion, and the language of Calculus reflects
this. Moreover, motion will serve as the foundational metaphor for
many of the other applications so it is well worth out time to
understand this fundamental problem very well before proceeding.

Of course, an object falling freely under the influence of Earth's
gravity gets moving pretty quickly so it's speed can be hard to
measure and the tools for measuring time in Galileo's day were limited
to say the least. He slowed things down by letting small balls roll
down a ramp assuming, correctly, that friction with the ramp would not
significantly influence his measurements or computations. He noticed
that the balls always accelerated at a constant rate which depended on
the steepness the ramp. That is, the ball's velocity increased at a
constant rate. More precisely, the acceleration of the rolling ball varied with the
angle $\theta,$ as in this diagram. \\
\centerline{\includegraphics*[height=2in,width=1in,angle=90]{Figures/GalileoRamp}}


%%% working wraptable example
%  \begin{wraptable}[]{r}{1.75in}
%  \begin {tabular}{llc}
%  \hline %
%  \multicolumn{2}{c}{Sample} & Roughness $ R_ a $\\
%  & & ( nm )\\\hline %
%  A & ring & 385\\\cline{2 -3}
%  & plate & 397\\\hline
%  B & ring & 376\\\cline{2 -3}
%  & plate & 390\\\hline
%  \end {tabular}
%  \end{wraptable}
 \begin{wraptable}[]{r}{1.75in}
 \begin{tabular}{c|c}
 \hline
   $\theta$ &Acceleration\\
      (in radians)      & in $(m/s^2)$\\[2mm]
 \hline{}
   $\pi/3$&4.9\\
   $\pi/4$&6.93\\
   $\pi/6$&8.49\\
   $\pi/12$&9.47\\
   $\pi/24$&9.72\\
   $\pi/48$&9.78\\
   $\pi/100$&9.795\\
   $\pi/200$&9.799\\
   $\pi/300$&9.799\\\hline
 \end{tabular}
 \end{wraptable}
When $\theta=\pi/2$ clearly the ball doesn't move so its acceleration
is zero. When $\theta=0$ the ball accelerates freely under the
influence of gravity. Galileo measured the acceleration associated
with steeper and steeper ramps (that is for $\theta$ closer and closer
to zero) obtaining a table of values for similar to the one at the right. 
From this table it is clear that as $\theta$ gets closer and closer to
zero the acceleration is getting closer and closer to $9.8.$ Thus
Galileo  deduced that when there is no ramp (when
$\theta =0$) the \emph{velocity} will increase each second by $9.8\,
m/s.$ 

That is, the \emph{acceleration} of an object falling under the
influence of the earth's gravity is $9.8$ meters per second each
second. This is usually abbreviated as $9.8\,m/s^2.$ 
\begin{embeddedproblem}{}
  Actually, Galileo wouldn't have made that many measurements. Instead
  he would have made one measurement very carefully and deduced the
  vertical acceleration, $g=9.8 m/s^2,$  from that. 

  Use the first line of the table above and the following diagram, and
  the knowledge that $\cos\left(\frac{\pi}{6}\right)=\frac12$\\ 
\InsertGraphic{}
  to deduce that $g=9.8.$
\end{embeddedproblem}
This means that as an object falls its \emph{velocity} increases at the
constant rate of \(9.8\, m/s\) every second so we have $v=9.8t.$ If we
drop the ball from some height, then initially its initial velocity is
zero. It is not moving. After one second it will be falling at a rate
of \(9.8\,m/s,\) after two seconds, its speed will be
\(9.8\times2=19.6\,m/s,\) etc.

In general, if the ball's \underline{velocity} in meters per second
($m/s$) is changing at a constant rate, say $r$ meters per second, per
second ($m/s^2$) and $t$ is the number of elapsed seconds then
$$
v=rt.
$$
We can verify that this is correct\footnote{Or nearly so.} by looking
at the  units of measurement. Since the acceleration, $r,$ is measured
in $m/s^2$ and $t$ is measured in seconds when they are multiplied the
obvious symbolic cancellation gives
$$
m/s^2\cdot s = m/s
$$
or meters per second, which are the units of measurement of velocity
as they should be.

Of course the specific value of $9.8\,m/s$  is an artifact of the units
we've chosen to measure distance (meters) and the fact that we are on
the surface of a particular planet. If we measure distance in feet
instead then a falling object will accelerate at $32 ft/s^2.$ If we
are on the surface of the moon it will accelerate at $1.6\, m/s.$

\begin{embeddedproblem}{}
  Show that if we measure distance in feet the acceleration constant
  on the moon is $5.2 m/s^2.$
\end{embeddedproblem}


\subsection{The Simplest Case: A Dropped Ball}
\label{subsec:simpl-case:-dropp}

To begin we will only consider the case of an object falling
vertically under the influence of gravity.  Suppose we drop a ball
from a height of $9.8$ meters. Will it hit the ground at the end of
that first second?

Clearly not. To reach the ground after $1$ second the ball would have
to average $9.8\,m/s$  during the entirety of that first second.  But at
first it is not moving at all (velocity $=0$). After one second its
velocity has increased to $9.8 m/s.$ So for the entire duration of
that first second the ball's \emph{average} velocity is \emph{less
  than} $9.8 m/s.$ So it has not yet hit the ground. But exactly how
far it has fallen is not yet clear.

Reasoning similarly, at the beginning of the next second the ball is
already falling at $9.8\,m/s$  and thereafter it increases, so it falls
\emph{at least} $9.8$ meters during the next second. So, while we
cannot yet tell \emph{exactly} how far it falls during either second,
we can say that it will hit the ground some time during the second
second\footnote{How the word ``second'' came to have two such
  different meanings is interesting. Google it.}

But how can we find out exactly when the ball strikes the ground?

Clearly, the formula $v=rt$ which tells us the ball's velocity at any
time $t$ is not sufficient.  But if we had a similar expression for
$p,$ the ball's position at any time $t$ we would be in much better
shape. 

\begin{mynotation}{Function}
  Since $p$ and $t$ are the quantities of interest, and $p$ depends on
  $t$ we will use the functional notation, $p(t)$ to represent all of
  this succinctly.  This is just a shorter way to indicat that
  position $p,$ depends on the elapsed time $t.$
\end{mynotation}
Now suppose we know that the graph of $p(t)$ looks like this:

\centerline{\includegraphics*[height=2in,width=2in]{Figures/GenericVelocity}}


We are not suggesting that the graph above actually is the graph of
$p(t).$ After all, if we knew that our problem would be solved. 
But suppose it is. Can we tell from this
what the velocity, $v(t),$ will be at a given time, $t?$

Sure we can. 

All we need are the differentials that we introduced in
Chapter~\ref{chapt:differentials}. That is, if $p(t)$ is the ball's
position at time $t,$ and we allow time to tick forward by an
infinitesimal amount, $\d{t}$ then we have the following picture:

\centerline{\includegraphics*[height=2in,width=2in]{Figures/GenericVelocityWithDiffTri}}


We know from our work in Chapter~\ref{chapt:differentials} that
$\dfdx{p}{t}$ is the derivative of $p(t).$

Moreover $\d{p}$ is measured in meters while $\d{t}$ is measured in
seconds so $\dfdx{p}{t}$ is measured in meters per second. The obvious
conclusion is that if $p(t)$ gives the position of the ball then
$\dfdx{p}{t}$ is it's velocity.

\begin{embeddedproblem}{}
Use an argument  analogous to this to show that the derivative of
velocity is acceleration.
\end{embeddedproblem}

Our problem of course is exactly the reverse of this. We have the
velocity
$$
v(t) = 9.8 t = \dfdx{p}{t}
$$
and we need to construct the position function, $p(t).$ This might seem
a little daunting at first but it really isn't too bad. We need only
ask ourselves what $v(t) = 9.8t$ is the derivative of? Clearly it is 
$$
p(t) = \frac{9.8}{2}t^2=4.9t^2
$$ 
since 
$$
\dfdx{4.9t^2}{t}=2(4.9)t=9.8t.
$$

So we can now solve our problem, right? All we have to do is find the
time, $t,$ when our position $p(t),$ is zero. That is, we solve
\begin{align*}
  p(t)&=0\\
  4.9t^2&=0\\
  t^2&=0\\
  t&=0.
\end{align*}
So the ball reaches the ground after zero seconds.

{\sc Wait!} That can't be right.

At zero seconds we've just released the ball $9.8$ meters above the
ground so our position at $t=0,$ $p(0)$ should be $9.8$ not zero. And
$p(t)$ should be zero for some \emph{positive} value of $t,$  since
the ball strikes the ground some time \emph{after} we drop it.

Something seems to have gone terribly wrong with our analysis of the
problem\footnote{Most likely you already see what's gone wrong. If so,
good for you! If not, no big deal. Keep reading.}. What could it be?

There really isn't anything wrong with our \emph{reasoning.} The
difficulty is with some \emph{assumptions} we made that we weren't
really aware of as we worked through the problem.

Specifically, we concluded that the ball's position is given by
$p(t)=4.9t^2.$ This is actually true but we need to remember that
position is always given as a distance from somewhere else, some other
location. In this case since at $t=0$ we have $p(0)=0,$ our $p(t)$
gives us the distance from our starting position. Thus when we solved
$p(t)=0$ what we found was how long it takes for the ball to reach its
starting position. Since it is already there, naturally this is $t=0$
seconds.

Apparently what we need to solve is $p(t)=9.8.$ This will tell us how
long it takes the ball to travel the $9.8$ meters to the
ground. Solving this gives
\begin{align*}
  p(t) &= 9.8\\
  4.9t^2&=9.8\\
  t^2&=2\\
  t&=\sqrt{2}\approx 1.414
\end{align*}
or slightly less than one and a half seconds.

\noindent\underline{\bf Coordinate Systems}\\
It is tempting to leave things there, but this problem of implicit
assumptions -- assumption we are not aware we are making -- is a
difficulty that will come back over and over again, sometimes in much
more complex situations so we will take a few pages to understand it
now in this much simpler context.


When we observed that the velocity $v(t),$ is the derivative of
position, $p(t),$ we asked what $9.8t$ is the derivative of. The
answer we found was $p(t) =4.9t^2.$ But this is only one
possibility. The derivative of $4.9t^2+5$ is also $9.8t;$ and of
$4.9t^2+12;$ and of $4.9t^2-700;$ and in particular the deriviative
of $4.9t^2+9.8$ is also $9.8t.$ In fact if $k$ is any constant
whatsoever then the derivative of $4.9t^2+k$ is still $9.8t.$

When we chose $p(t)=4.9t^2$ as our position function we implicitly
assumed that our constant was zero. However we could have chosen $k$
to be any constant whatsoever. All that changes is what distance is being measured. 


\begin{wrapfigure}[]{r}{1in}
\captionsetup{labelformat=empty}
\includegraphics*[height=2in,width=1in]{Figures-HandDrawn/FallingBall1}
\label{fig:FallingBall1}
\end{wrapfigure}
For example either $p_1(t) =4.9t^2$ or $p_2(t)=4.9t^2-9.8$ will work
as a position function. What is the difference between them?

If we use $p_1(t) = 4.9t^2$ we are assuming that at time $t=0$ our
position is also zero. Physically we have the following situation:\\
where at $p(0) =0$ and $p(\sqrt{2}=9.8.$ We have
\begin{wrapfigure}[]{l}{1in}
\captionsetup{labelformat=empty}
\includegraphics*[height=2in,width=1in]{Figures-HandDrawn/FallingBall2}
\label{fig:FallingBall2}
\end{wrapfigure}
implicitly assumed that at time zero our starting position is also
zero. After releasing the ball time moves forward as usual so $t$
increases through positive numbers, and since $p(t)$ measures the
distance from our starting point, it also increases through positive
numbers.

On the other hand, if we use $p_2(t)=4.9t^2-9.8$ to measure distance
then 
we have implicitly assumed that our starting position ($t=0$) is
at $-9.8$ meters, that is $9.8$ meters above the ground, so that
$p_2(t)$ measures how far the ball is away from the ground. 

\begin{embeddedproblem}{}
  If other values of $k$ are used then the starting position (position
  when $t=0$) and ending position (position when $t=\sqrt{2}$) of the
  ball are implicitly changed, but the falling time will not. Find the
  position function for each of the following values of $k.$ Also give
  a sketch like the two above for each value of $k,$ and show that in
  every case the falling time is $\sqrt{2}.$
  \begin{multicols}{2}
    \begin{description}
    \item[(a)] $k=5$
    \item[(b)] $k=12$
    \item[(c)] $k=-700$
    \item[(d)] $k=9.8$
    \item[(e)] $k=-9.8$
    \end{description}
  \end{multicols}
  Which of these seems to you like the best choice\footnote{There is
    no correct answer to this question. That is the point.}.
\end{embeddedproblem}

In either case, we are asking how long it takes the ball to fall $9.8$
meters. In the first case we are asking how long it takes the ball
to move $9.8$ meters \underline{from} the starting position. In the
second we are asking how long it takes to move \underline{to} the
ending position.

The former probably feels more natural. This is because we habitually
measure our position in the future by assigning it a value (in meters)
which tells us how far we have moved from our starting position. We do
this every when we tell a friend something like, ``You're about
five miles away from me.''

But just because this is a common habit of mind does not mean that it
is necessarily \emph{always} the best way to measure position.

For example, one consequence of this for our current problem is that
the ball falls in the \emph{positive} direction. This is true whether
we use $p_1(t)$ or $p_2(t)$ In the former case is that as $t$
increases the position, $p_1(t),$ changes from zero to $9.8$ meters -- it becomes
``more positive.'' In the latter it also becomes more positive but in
this case $p_2(t)$ changes from $-9.8$ to zero meters\footnote{Perhaps
we should call this ``less negative?''}. In either case the position
now is represented by a number that is greater than the position
previously, so the positive direction is \emph{downward} when we use
either $p_1(t)$ or $p_2(t).$  

\begin{embeddedproblem}{}
We don't normally think of down as positive. Find a new position
function $p_3(t)$ which models the same falling ball but has down as
the negative direction. There is more than one way to do this so
verify that you have a valid function by solving for the time needed
to reach the ground. You should still get $\sqrt{2}.$
\end{embeddedproblem}

Ultimately what this all comes down to is choosing a coordinate system
to work in. You are used to thinking of coordinates as an ordered pair
of numbers $(x,y),$ that specify a location in the plane. But you were
using a coordinate system long before you started graphing equations
in the plane. 

Choosing a coordinate system is nothing more or less than choosing a
distinguished point as a reference from which the location of all
other points is measured. So as soon as you were taught to think of
the real numbers ``as points strung out on a line'' you were using a
coordinate system. The point at zero is specified (and called the
origin) and all other points are named ($1, 2, 3, \ldots$) according
to how far they are from the origin. 

This is what we did when we chose the {\bf k}onstant above. When we
chose $k=0$ so that $p(t)=4.9t^2$ we placed the origin at $9.8$ meters
above the surface of the earth. When we chose $k=-9.8$ so that
$p(t)=4.9t^2-9.8$ we placed the origin at the surface of the
earth. But of course neither gravity nor the ball care what coordinate
system we use so it falls to the ground in $\sqrt{2}$ seconds no
matter where we decide to place the origin.

In the first case since $p(t)=4.9t^2$ we see that $p(0) = 0$ and
$p(\sqrt{2}) = 9.8$ so the ball moved from $0$ to $9.8$ meters. It
follows that ``down'' must be the positive direction and ``up'' the
negative direction.

Similarly if we use $p(t)=4.9t^2-9.8$ we see that $p(0) = -9.8$ and
$p(\sqrt{2}) = 0.$ Although the position of the ball is always
negative, it moves from a more negative position, $-9.8,$ to a
non-negative position, $0.$ So again ``down'' must be the positive and
``up'' the negative direction.

This seems natural when we think about a falling ball, but of course
when we choose a coordinate system we get to choose both the origin
and  the directionality. In the next section it will be more natural
to take ``up'' as the positive direction.

\subsection{A Tossed Ball: How High Will the Ball Go?}
\label{subsec:tossed-ball}

Instead of simply dropping the ball suppose we toss it vertically
upwards with some initial velocity of say \(15 m/s.\) Then after one
second it will have \emph{lost} \(9.8 m/s\) of its initial
velocity. That is, its velocity after \(1\) second will be
\(15-9.8=5.02\,m/s.\) After \(2\) seconds, its velocity will be
\(15-9.8\times{}2=-4.58 m/s.\)

In general, after \(t\) seconds its velocity will be \(15-9.8t\, m/s.\) 
Notice that the velocity is positive at the one second mark $(t=1)$
and negative at the two second mark $(t=2.)$

As we observed at the end of the last section, it thus seems natural
to interpret \underline{positive} velocity as climbing (moving in the
positive direction). We'll need to keep this in mind as we proceed.

So the height of the ball increases until the velocity drops to
zero. At that time the ball's height has reached its zenith. After
that the ball will continue to drop. But now its velocity will become
larger in the negative direction (because the ball is falling, or
moving in the negative direction), Although we don't know exactly what
the height is at any given time we can see that there is a point where
the ball reaches its zenith. The ball's height increases (velocity is
positive) until it reaches its zenith and then decreases (velocity is
negative) thereafter. The zenith occurs when the velocity has
decreased to zero.

Since we know that the velocity is positive at one second and negative
at two seconds we can say that the ball must reach its zenith at
somewhere between \(1\) second and \(2\) seconds after it is released.
To find out the exact time, notice that this is precisely when the
velocity equal to zero because for an instant at the top of its climb
the ball's velocity is zero. Solving for $t$ gives:
\begin{align*}
  15-9.8 t&=0 \\
t=15/9.8&\approx 1.53 
\label{eq:accel}
\end{align*}
Thus the zenith is reached after approximately $1.53$ seconds.

\begin{embeddedproblem}{}
  Suppose the initial velocity is double what it was originally.
  Would it take twice as long for the ball to reach its maximum
  height?  What if the initial velocity was half?
\end{embeddedproblem}

So far, so good.  But now we ask: How high will the ball go? 

As before what we need is not the velocity function, $v(t) =15-9.8t,$
but the position function, $p(t).$ Since we know how long it takes to
reach the maximum height ($\approx 1.53$ seconds) all we need to do is
compute $p(1.53).$

Since we know that position is the derivative of velocity all we need
to do is ask ourselves; What is $v(t) = 15-9.8t$ the derivative of?
Give this a few moment's thought before reading on. See if you can
figure it out for yourself. We'll wait.

It should be clear that the position function is: $p(t) = 15t-4.9t^2.$
Is it?

But wait a moment. Remember the trouble we had before with the
konstant, $k?$ Let's think about this a bit more. The derivative of 
$$
p(t) = 15t-4.9t^2 +k
$$
is
$$
v(t) =15-9.8t
$$
no matter what the numerical value of $k$ is, so let's see if we can
find the value of $k$ that makes things easiest for
us. If\footnote{Notice that we've rearranged the order of the
  terms. This isn't really necessary, but it's a good idea to always
  write polynomial in the standard form, unless there is a compelling
  reason not to. This makes things easier to read when you come back
  looking for the errors that will inevitably happen.}  $p(t) =
-4.9t^2+15t +k$ then at time $t=0$ the ball's position is $p(0) =
-4.9(0)^2+15(0) +k = k.$ Since we're standing on the ground when we
toss our ball we'll probably release the ball about $5$ feet above the
ground\footnote{Both authors are over six feet tall. If you are much
  taller or much shorter you should adjust this number as
  appropriate.}, or about $1\frac{2}{3} = 5/3$ meters. So it seems to
make sense to set $k=5/3.$ Thus our position function is:
$$
p(t) = -4.9t^2+15t +  5/3
$$

Therefore the maximum height of the ball is approximately $p(1.53) =
-4.9(1.53)^2+15(1.53) +  5/4 \approx 13.15$ meters.

\begin{embeddedproblem}{}
  Suppose you toss the ball twice as hard so that its initial velocity
  (its velocity when it leaves your hand) is $30 m/s.$ 
  \begin{description}
  \item[(a)] How long will it take to reach its maximum height?
  \item[(b)] What is its maximum height?
  \item[(c)] How long does it take the ball to hit the ground?
  \item[9c)] What is its velocity when it hits the ground?
  \end{description}
\end{embeddedproblem}

We now have everything we need to describe the motion of a ball thrown
vertically. We know the following facts:
\begin{enumerate}
\item The derivative of the position $p(t),$ is the velocity $v(t).$
\item The derivative of the velocity $v(t),$ is the acceleration.
\item At the surface of the earth the acceleration due to gravity is
  $9.8\,m/s^2.$ To keep our analysis as general as possible we will
  just say that the acceleration is a constant and us $\alpha$ to
  represent it.
\end{enumerate}

First, to find the velocity function $v(t),$ we ask, what is the
constant $\alpha$ is the derivative of? Clearly this is
$$
v(t) =\alpha t+ v_0
$$
where $v_0$ is an unknown constant. To determine the value of $v_0$
we evaluate the velocity when $t=0.$ Since 
$$
v(0) = \alpha\cdot0+v_0 = v_0
$$
we see that $v_0$ is the \underline{initial velocity} of the ball;
its velocity just as it is released.

Next we ask what is 
$$
v(t) = \alpha t+v_0
$$ 
the derivative of? Clearly this is 
$$
p(t) = \frac{\alpha}{2}t^2+v_0t+p_0
$$
where $p_0$ is the \underline{initial position} by a similar argument.

\begin{myexample}
  Suppose you toss the ball vertically with an  initial velocity of
  $30\,m/s$ while standing at the edge of a $120$ meter building.
  (its velocity when it leaves your hand) is $30 m/s.$ 
  \begin{description}
  \item[(a)] How long will it take to reach its maximum height?
  \item[(b)] What is its maximum height?
  \item[(c)] How long does it take the ball to hit the ground?
  \item[d)] What is its velocity when it hits the ground?
  \end{description}
\end{myexample}

\subsection{Motion in Parametric Form}
\label{subsec:moti-param-form}

\begin{wrapfigure}[]{r}{2in}
\captionsetup{labelformat=empty}
\includegraphics*[height=2in,width=2in]{Figures/FallingBallGraph1}
\label{fig:FallingBallGraph1}
\end{wrapfigure}
Suppose we toss a ball upward from the ground with an initial velocity
of $15\,m/s.$ Then if we place the origin at ground level, and the
position  function is given by:
$$
p(t) = 5/3 + 15t-4.9t^2,
$$
and its graph is shown at the right.

Before we go on take a moment to notice that this is \emph{\bf{}not}
the graph of the ball's trajectory. Remember the ball is moving
straight up and down. Rather this graph  represents a fairly abstract
representation of the purely vertical motion of the ball. Notice that
the horizontal axis of this graph represents $t,$ the elapsed time.

That is if we start our clock at the moment the ball leves our hands
then at time $t=0$ we have $p(0)=5/3.$ so the point $(0,5/3)$
represents the fact that when we start our clock the ball is $5/3$
meters above ground.

Moreover, as we observed previously sin in this graph the vertical is
measured in meters and the horizontal in seconds, it should be clear
that the slope of $p(t)$ at $t=0$ is the velocity at $t=0.$ That is,
$$
\left.\dfdx{p}{t}\right|_{t=0} = \left.v\right|_{t=0} = 15-9.8(0)^2= 15
\text{ meters.}
$$

Our point is that although this graph contains a great deal of
information about the ball's motion it is \underline{not} a picture of
the ball's \emph{actual path} through the air if it is thrown
vertically. Recall that the ball moves vertically upward from our
hands and then vertically downward to the ground.

However we all know that it is nearly impossible to toss a ball
perfectly vertically. Inevitably the ball will receive some non-zero
lateral (horizontal) velocity. When that happens the path the ball
follows will indeed look very much like the previousl graph. This can
lead to  confusion so it is worth taking a few moments to clarify this
a bit. 

If we were to releast the ball in such a way that its initial
\underline{vertical} velocity is $15 m/s$ as before, but also with an
initial \underline{horizontal} velocity of $1 m/s$ then the path the
ball follows looks exactly like the previous graph:

\begin{wrapfigure}[]{l}{2in}
\captionsetup{labelformat=empty}
\centerline{\includegraphics*[height=1in,width=2in]{Figures/FallingBallPath1}}
\label{fig:FallingBallPath1}
\end{wrapfigure}
\noindent{}\underline{except} for the labelling on the axes. Notice that the
horizontal axis shown is labelled ``$x$'' and the vertical is
``$y.$'' This is because the horizontal axis now represents the
physical, horizontal distance the ball has traveled, rather than the
number of seconds elapsed since it was released.

The vertical axis still represents the vertical position of the ball
but since the position now requires two coordinates, $x$ and $y,$ we
now simply label the vertical axis ``$y$'' as per tradition. 

The reason the two graphs are identical (except for the axis labels,
of course) in this case is that we chose an initial horizontal
velocity of $v(0)=1.$ If we also choose our horizontal origin to be
where we are standing when we release the ball, then the horizontal
position is given by:
$$
x=t.
$$
Since the initial vertical velocity is unchanged (as is the
acceleraton due to gravity: $9.8 m/s^2$) the vertical position is
still 
$$
p(t) = 5/3 + 15t-4.9t^2.
$$

We use $y$ (instead of $p$) to represent the vertical position of the
ball because the ball's position in this situation requires
two\footnote{Actually we need $3.$ Do you see why we can ignore the
  third this time? If not, pause here and think about it for a moment.
We'll wait.}
coordinates, $x$ \underline{and} $y,$ whereas previously we only
needed one, the $y$ coordinate.

If we want to use $p$ to represent the ball's position at time $t$ (we
do) then $p(t)$  will have to return two coordinates for each value of
$t:$
$$
p(t) = \left(x(t), y(t)\right).
$$

To see why the two graphs above coincide we need only observe that
since 
$$
x=t
$$
it doesn't matter whether we label the horizontal axis $t,$ in which
caxse the graph gives the vertical position at any time $t,$ or $x,$
in which case the curve gives the actuall position of the eball. They
\underline{look}  the same even if they mean different things.

We got $x=t$ because we assumed an initial horizontal velocity,
$v_x(0)$ of $1 m/s$ and an initial horizontal position, $x_0$ of
zero. 

\begin{wrapfigure}[]{l}{2in}
\captionsetup{labelformat=empty}
\centerline{\includegraphics*[height=1in,width=2in]{Figures/FallingBallPath2}}
\label{fig:FallingBallPath2}
\end{wrapfigure}
Suppose we set our horizontal origin $2$ meters to the left of where
we released the ball. Then we get
\begin{align*}
  x&=t+2\\
  y&=5/3 + 15t-4.9t^2,
\end{align*}
so the ball follows the given trajectory.
\begin{embeddedproblem}{}
For each set of conditions given below, answer the following three questions:
    \begin{description}
    \item[(1)] What are the $x$ and $y$ coordinates of the ball's
      position at time, $t.$
    \item[(2)] Plot the ball's position on the $x-y$ plane.
    \item[(3)] Plot the \underline{vertical position only} of the ball
      on the $p-t$ plane. (This should not necessarily look the same
      as the graph you drew for part (b). (You should be curious about
      the difference. In fact, you should be so curious that you will
      spend a few minutes trying to puzzle out why and how they are
      different. If you are not curious, pretend that you are and do
      it anyway.)
    \item[(4)] At what time does the ball reach its maximum height?
    \item[(5)] What is the maximum height?
    \item[(6)] How long before the ball hits the ground?
    \item[(7)] Where does it strike the ground?
    \item[(8)] What is it's horizontal and vertical velocity at the
      moment it strikes the ground?
    \end{description}

  \begin{description}
   \item[(a)] Suppose our initial horizontal velocity is $2 m/s,$ and that
    the initial vertical velocity is $15 m/s.$
  \item[(b)] Suppose our initial horizontal velocity is $1/2 m/s,$ and that
    the initial vertical velocity is $15 m/s.$
  \item[(c)] Suppose our initial horizontal velocity is $0 m/s,$  that
    the initial vertical velocity is $15 m/s,$ but
    that the ball is accelerating to the right at a constant rate of
    $2 m/s^2.$
 \end{description}
\end{embeddedproblem}

The position of a moving object must always be given as a function of
time because we always need to know when it arrives at some position
along with the position itself.

When, like our falling ball, the object is moving both horizontally
and vertically we clearly need two coordinates in order to specify its
position. That is, the $x$ and $y$ coordinates depend on $t$ as a
common parameter.

Thus the equations 
\begin{align*}
  x&=2t-3\\
  y&=5/3 + 15t-4.9t^2,
\end{align*}
are called  \underline{Parametric Equations.}

We have barely scratched the surface of this important topic but this
is enough for now. We will return to this in more depth later.

% \begin{wrapfigure}[]{l}{2in}
% \captionsetup{labelformat=empty}
% \centerline{\includegraphics*[height=1in,width=2in]{Figures/velocity}}
% \label{fig:velocity}
% \end{wrapfigure}


% % \begin{wrapfigure}[]{R}{2in}
% \includegraphics*[height=1.75in,width=2in]{Figures/parab-sec}
% % \caption{}
% % \label{fig:parab-sec}% \end{wrapfigure}

% \centerline{\includegraphics*[height=2in,width=2in]{Figures/parab-tan}}

% % % \begin{wrapfigure}[]{L}{2in}
% \centerline{\includegraphics*[height=2in,width=2in]{Figures/ProposedGraph}}
% % % \caption{Proposed graph of $h=h(t)$}
% % % \label{fig:proposed}
% % % \end{wrapfigure}

% Because the change in velocity, $v,$ is so simple, $v=15-9.8t$ it is easy to
% discover \emph{when} the ball reaches it maximum height,
% but what \emph{is} the maximum height?  This is a bit trickier, as the
% change in position is not as simple.  Notice that
% initially the velocity was
% \(15 m/s.\) If the ball traveled that speed throughout the first
% second (which it doesn't), then the ball would travel \(15\)
% meters in the first second.  At one second, the velocity was \(5.02\,m/s,\)
%  so if it stayed that velocity (which it doesn't), then in the
% next second the ball would again travel $5.02$) meters. But if the
% ball is accelerating it will actually go further during the second
% $1-$second interval than the first. We need to figure out some way to
% measure the distance traveled for an object whose velocity is
% continually changing.


% We will develop the tools for this soon.  But first let's look at this
% problem graphically.  If we plot the velocity
% $v=15-9.8t$ as a function of time, we have 

% Notice that the slope of this line is $-9.8,$ the (constant) acceleration due
% to gravity.  This makes sense since the slope of this line is the
% change in velocity divided by the change in time
% ($\frac{\text{m}}{\text{s}}/s = m/s^2$), which as we have seen is the
% acceleration, and Galileo showed that the acceleration is
% constant\footnote{Actually it isn't. The acceleration of falling
%   objects depends on their distance from the center of the
%   Earth. However, \emph{near} the surface of the Earth the change is
%   so small that Galileo couldn't measure it with the tools he had. In
%   any case, the change is so slight that it can be taken to be
%   constant for our purposes.}.


% But what about the graph of the height, \(h(t)?\)
% A graph of what \(h=h(t)\)
% might look like is given in the following figure:

% \centerline{\includegraphics*[height=2in,width=2in]{Figures/ProposedGraph}}

% This is not the trajectory of the ball (remember it is moving straight
% up and down), but is a plot of height versus time.  Since the velocity
% is continually changing, the graph of the height versus time will not have a constant slope the
% way that the graph of velocity did.

% There is a relationship between
% the graphs in these figures which is easiest to see by considering
% what the horizontal and vertical axes measure. In the graph
% \centerline{\includegraphics*[height=1in,width=2in]{Figures/velocity}}
% the vertical axis is the velocity which is measured in meters per
% second, $m/s$ and the horizontal axis is measured in seconds.
% In the graph
% \centerline{\includegraphics*[height=2in,width=2in]{Figures/ProposedGraph}}
% the vertical axis is the height, $h,$ which is measured in meters, and again
% the horizontal axis is the time, $t,$ which is measured in seconds.
% The first graph measures velocity directly. However Since the velocity is the change in
% distance divided by the change in time, we can use the information in
% the second graph to deduce the velocity at some time $t_0$ as
% follows.  We follow Leibniz's example and think of the graph as made
% up of infinitely small straight line segments, infinitesimals. In that
% case the slope of the second graph at $t_0$ is given by $\dfdx{h}{t}.$
% Since this fraction is measured in meters per second ($m/s$) the slope
% of the graph of height versus time is the velocity of the ball at the
% time $t_0.$ Or, to put it another way, the vertical coordinate of the
% first graph (velocity versus time) at any time $t_0$ gives the slope
% of the the second graph at the same time, $t_0.$

% More abstractly, we can say that the velocity (vertical coordinate of
% velocity versus time) at time $t_0$ is the slope of the line tangent
% to the graph of position (height versus time) at time $t_0.$
% This abstraction is what we have really been aiming for all along
% however before we go too far down the rabbit hole of abstraction,
% let's review what we've learned from this very concrete physical
% problem.


% then it seems that the slope
% of the curve in the second at any point \((t,h(t))\) should be the
% value \(v(t).\) But what do we mean by the slope of a curve?  This
% where our physical depiction of throwing a ball into the air might
% help.


% As we noted, on the way up the ball is continually slowing down.  On
% the way down, it is continually speeding up.  We can find the average
% velocity over any time interval \([t_0,t_1]\)
% by computing the ratio \(\frac{h(t_1 )-h(t_0 )}{t_1-t_0}.\)
% On our proposed graph of \(h(t),\) 
% this would be the slope of the line connecting the points
% \((t_0,h(t_0))\)
% and \((t_1,h(t_1))\)

% \centerline{\includegraphics*[height=1.75in,width=2in]{Figures/parab-sec}}


% \begin{embeddedproblem}{}
%   What would happen to the
%   slope of the line segment joining \((t_0,h(t_0 ))\) to \((t_1,h(t_1 ))\) in
%   relation to the slope of the tangent line?  How does this translate
%   into the velocities of the ball in the original problem?
% \end{embeddedproblem}

% However, as we noted before, this is not the velocity of the ball at
% time \(t_0\) since that ball is slowing down.

% If we could somehow force the ball to stop slowing down, if we could
% somehow turn off gravity, at the instant \(t_0\) then from that time
% on the ball would move in a straight line with the velocity it has at
% the instant \(t_0.\) A moment's reflection makes it clear that this
% line must be the tangent line to the curve \(h(t)\) at the point
% \((t_0,h(t_0 )).\)


% Figure 5

% Bud – later when we talk about limits, we can come back to this and use a series of drawings where the time interval gets smaller.  Or do you think we should get into this now?  One possibility is the following problem

% While it is true that we can't physically make the ball stop
% slowing down, this 'thought experiment' gives us insight into
% the relationship between the graph of the ball's velocity and height
% versus time. Clearly the
% values of \(v(t)\) represent the slopes of the tangent lines to the curve
% \(h=h(t).\)

% \begin{wrapfigure}[]{L}{2in}
%   \includegraphics*[height=2in,width=2in]{Figures/parab-tan}
% \caption{}
% \label{fig:parab-tan}
% \end{wrapfigure}



% Using the tools of  Calculus we can exploit this relationship between
% the graph of a curve and the slope of its tangent line.  This
% fact led to one of ``golden periods'' of mathematics in the
% 1700s where the power of Calculus was used to examine all sorts of
% geometric and physical problems.  In order to exploit this
% relationship, we will need to establish rules for computing.  Indeed,
% as we observed in the previous section, this is what a 'calculus' is a
% set of rules for computation.  In the case of differential Calculus,
% it is literally a set of rules for computing infinitely small
% (infinitesimal) differences.  The key turns out to be relating
% everything back to the graphs of straight lines since straight lines
% are the graphs of quantities whose rates of change are constant and
% thus are easily measured.

% Developing these rules will allow us to examine the relationship
% between the graphs of \(v=v(t)\)
% and \(h=h(t)\)
% algebraically.  In particular, we will see how to find the equation of
% one curve, for example the height of a falling object, if we are given
% the other, for example its velocity.  Once we develop these tools, we
% will be ready to see how Calculus led to this golden period of
% mathematics. 

% \subsection{Tangent Lines}
% \label{sec:tangent-lines}

% In many ways, the invention of calculus was the culmination of the
% merging of geometry and kinematics with algebra, while exploiting
% infinitesimals.  While it is true that algebraic methods are the
% predominant tool of a calculus class, the blending of it with geometry
% and kinematics helps students adapt these techniques to physical and
% geometric applications.  In a curriculum that has downplayed the role
% of geometry, having precalculus problems which bring it back can pay
% future dividends.

% Consider the method of Gilles Personne de Roberval (1602-1675) for
% constructing the tangent to a curve.  In a blend of kinematics and
% geometry, Roberval considered a curve as generated by a point whose
% motion is compounded from 2 known motions.  The ``velocity vectors'' of
% the motions will combine to give the resultant tangent vector to the
% curve [Boyer, p. 390].  For example, consider that a parabola is the
% set of points in the plane equidistant from a fixed point (focus) and
% fixed line (directrix).  We can think of the parabola as generated by
% point moving so that it maintains the same distance from the focus as
% it does from the directrix.\\
% \centerline{\includegraphics*[height=3in,width=2.5in]{Figures/Roberval1}}

% Since the point maintains the same distance from the focus as it does
% from the directrix, the velocity vectors of these two motions (away
% from $F$ and $d$) will be the same length and the tangent vector will be
% the diagonal of the rhombus.  While this method is limited from an
% algebraic point of view (in terms of computing a slope), it does have
% a nice geometric consequence.  Since the diagonal of a rhombus bisects
% the angle, then we have $\angle 1\cong\angle2\cong\angle3.$  This provides the reflective
% property of a parabola which says that any light ray, sound wave,
% radio wave, etc. parallel to the axis of the parabola will be
% reflected to the focus.  This is important to the design of satellite
% dishes and reflective telescopes. \\
% \centerline{\includegraphics*[height=3in,width=5in]{Figures/Telescopes}}
% \begin{verbatim}
% First image from http://media-2.web.britannica.com/eb-media/82/72982-004-9191583E.jpg. 
% Second image from http://astro-canada.ca/_en/_illustrations/a4304_newton_en_p.jpg.
% \end{verbatim}

% \begin{myproblem}{}
%   Given that an ellipse is the set of points in the plane, the sum of
%   whose distances from two fixed points (foci) is constant, use
%   Roberval's idea to construct the tangent to the ellipse.  Notice
%   that one of the motions is away from a focus and the other is
%   towards the other focus.  Why?\\
% \centerline{\includegraphics*[height=2in,width=2in]{Figures/Roberval2}}

% Conclude that the parallelogram must be a rhombus and use this to
% demonstrate the reflective property of the ellipse, namely that any
% ray emanating from one focus of the ellipse is reflected by the
% ellipse to the other focus.   
% \end{myproblem}

% The aforementioned reflective property is noticed in a whispering
% gallery, where a person at one focus can speak quietly and the person
% at the other focus can hear it.  It is also used in the treatment of
% kidney stones through a procedure called extracorporeal shock wave
% lithotripsy (ESWL).  As imposing as the title sounds, the name makes
% sense.  Lithotripsy literally means “rock grinding” and the procedure
% utilizes shockwaves generated outside the body to crush kidney stones
% into granules that can pass relatively painlessly.  Ellipses are
% utilized to direct the shock waves from the generator to the stone as
% can be seen below.\\
% \centerline{\includegraphics*[height=2in,width=2in]{Figures/Lithotripsy}}

% Image 1 from http://www.hussainkidney.com/images/LithotripsyVSsurgery.jpg.
% Image 2 from http://blogs.nejm.org/now/wp-content/uploads/2012/07/Electrohydraulic-Lithotripter.jpg

% As can be seen from the images, shock waves are generated outside the
% body at one focus of the elliptical reflector and are reflected to the
% other focus where the stone is located.  The procedure is done on an
% outpatient basis and is a very effective, non-invasive procedure for
% treating kidney stones.

% As another (future) application, consider the following image.
% \centerline{\includegraphics*[height=2in,width=2in]{Figures/MagRes}}

% Image from http://physics.bu.edu/~condmat/pictures/mirage.jpg .

% This image represents the magnetic resonance of 36 cobalt atoms placed
% in an elliptical ring called a quantum corral by scientists at IBM
% Almaden labs.  When researchers bombarded a cobalt atom placed at one
% focus with electrons, the electron partial waves reflected off of the
% elliptical ring to produce a weaker “mirage” at the other focus even
% though there was no atom present
% [http://cerncourier.com/cws/article/cern/28197].  IBM is hoping to
% utilize this “Quantum Mirage” effect to transmit information on a
% nanoscale, where normal circuitry does not work.  This would lead
% toward nano-sized processors.  Below is an artist’s rendition of a
% nanobot attaching itself to a damaged nerve and becoming a prosthetic
% nerve
% [http://www.cg4tv.com/animation/stock-images-3d/neurons-nanobot-high-resolution.jpg].
% \\
% \centerline{\includegraphics*[height=2in,width=2in]{Figures/Nanobot}}
% This technology doesn’t yet exist, but it demonstrates how even old
% ideas can lead to future developments.

% \begin{myproblem}{}
%   A hyperbola is the set of points in the plane, the difference of
%   whose distances from two fixed points (foci) is constant.  Use
%   Roberval's idea to construct the tangent to the hyperbola and derive
%   the reflective property of a hyperbola, namely that any ray directed
%   at one focus will reflect off of the hyperbola to the other focus.\\
% \centerline{\includegraphics*[height=2in,width=2in]{Figures/Roberval3}} 
% \end{myproblem}

% The aforementioned reflective property is utilized in the secondary
% reflector of a cassegrain antenna.\\
% \centerline{\includegraphics*[height=2in,width=2in]{Figures/Telescope2}}
% Image 1 from http://www.antedo.com/images/13-Meter-(TCS).jpg .
% Image 2 from http://www.rfwireless-world.com/images/cassegrain-feed-antenna.jpg .


% \subsection{}

% \begin{wrapfigure}[]{O}{3in}
% %\vskip-.7cm{}
% \includegraphics*[height=1.75in,width=3in]{Figures/CosApprox2}
% \caption{Approximating $\cos x$ with a quadratic polynomial}
% \label{fig:CosApprox2}
% \end{wrapfigure}



% \begin{wrapfigure}[]{O}{3in}
% %\vskip-.7cm{}
% \includegraphics*[height=1.75in,width=3in]{Figures/CosApprox4}
% \caption{Approximating $\cos x$ with a quartic polynomial}
% \label{fig:CosApprox4}
% \end{wrapfigure}

% \marginpar{These last paragraphs should be somewhere else. But where?}
% For example, for values of \(x\)
% between \(-\frac\pi2\) and \(\frac\pi2\) the
%   \label{page:cosine-approx}
%   cosine of \(x\) is closely approximated by the polynomial
%   \(1-\frac{x^2}{2}+\frac{x^4}{24}.\) So for values of \(x\) in that
%   range we can replace \(\cos(x)\) with
%   \(1-\frac{x^2}{2}+\frac{x^4}{24}\) with very little loss of
%   accuracy. This is useful because, for example, \(\cos(.7)\) can be
%   very hard to compute whereas \(1-\frac{(.7)^2}{2}+\frac{(.7)^4}{24}
%   \approx 0.77\) is relatively easy.

%   % Once known this approximation is easy to confirm and to use.  But
%   % coming up with it in the first place, finding the right polynomial
%   % to use is quite difficult, until you've learned a little Calculus as
%   % we will see in section~\ref{subsec:tangent-lines}.

%   % This approximation is easy to confirm and to use \emph{once it is
%   %   known.}  But coming up with it in the first place is quite
%   % difficult until you've learned a little Calculus as we will see in
%   % section~\ref{subsec:tangent-lines}.\marginpar{A word about the use
%   %   of technology should be added here.}

% \subsection{}
% Of particular note is the following technique of
%   Gilles Personne de Roberval (1602-1675).  Roberval thought of a
%   curve in the plane as being generated by a point whose motion is
%   compounded from two known motions.  By combining the ``velocity
%   vectors'' of these motions, the tangent vector will
%   result. With the tangent vector in hand the tangent line is easily
%   computed.


% We'll apply this idea to a parabola.  You probably think of a parabola
% as being determined by a quadratic function (which it is), but that is
% the modern approach. The older way to define a parabola is geometric: A
% parabola is the set of points in a plane which are equidistant from a
% fixed point called the focus and a fixed line called the directrix. In
% the figure, the point $P$ on the parabola is the same
% distance from the focus $F$ as it is from the directrix $d.$ These
% lengths  change as $P$ moves along the parabola, but the distance
% from $P$ to $F$ will always equal the distance from $P$ to $d.$
% \begin{embeddedproblem}{}
%   \begin{description}
%   \item[a.] Explain why the parallelogram in the figure is
%     a rhombus.
%   \item[b.] Given that the parallelogram in the  figure is
%     a rhombus, show that the two angles $\alpha,$ and $\beta$ are
%     congruent.
%   \end{description}
% \end{embeddedproblem}

% If we think of the motion of $P$ as a combination of motion away from
% $F$ and motion away from $d,$ then we can modify our drawing to
% reflect this as in the figure. While this geometric approach does
% not produce the equation of the tangent line, it does have the
% interesting consequence
% \begin{wrapfigure}[]{O}{2in}
% %\vskip-.7cm{}
% \includegraphics*[height=1.5in,width=2in]{Figures/Ellipse}
% \caption{}
% \label{fig:ellipse}
% \end{wrapfigure}
% that any light ray, radio wave, sound
% wave, etc. which comes in parallel to the axis of the parabola
% (perpendicular to the directrix) will reflect off of the parabola to
% the focus.  Among other things, this is why satellite dishes and
% primary mirrors in reflective telescopes are parabolic.  
% \begin{embeddedproblem}{}
% Consider that an ellipse is the set of points in a plane, the sum of
%   whose distances from two fixed points (foci) is
%   constant. Using Roberval's idea, we can think of the
%   motion of a point P on the ellipse as being the result of a motion
%   away from one focus and toward the other focus.
%   Explain why the parallelogram in the figure must be a rhombus and
%   use this to show that angle $\alpha$ is congruent to angle $\beta.$
% \end{embeddedproblem}

% \begin{wrapfigure}[]{O}{2in}
% %\vskip-.7cm{}
% \includegraphics*[height=1.5in,width=2in]{Figures/RockGrinding}
% \caption{}
% \label{fig:RockGrind}
% \end{wrapfigure}

% This says that any light ray, sound wave, etc. emanating from
% one focus will reflect off of the ellipse to the other focus.  Among
% other things, this reflective property is used to treat kidney stones.
% The treatment is called Extracorporeal Shock Wave Lithotripsy (ESWL)
% and is really what the name says.  Lithotripsy literally means ``rock
% grinding,'' and this technique uses shock waves which are generated
% outside the body (See
% Figures~\ref{fig:RockGrind})\footnote{http://us.medispec.com/patient/kidney-stones/treatment-using-eswl/
%   and\\
%   http://bme240.eng.uci.edu/students/09s/ysantoro/CurrTechn.html} As
% you can see in the figure, the table contains a cup which is in
% the shape of part of an ellipse.  At one focus of the ellipse is an
% electrode which generates shock waves.  These shock waves reflect off
% of the elliptical cup and converge on the other focus.  When the
% kidney stone is positioned at that focus, the shock waves will pummel
% it into small pieces which can be passed relatively painlessly through
% the ureter.
% \begin{embeddedproblem}{}
%   An hyperbola is the set of points in the plane, the difference of
%   whose distances from two fixed points (foci) is constant.  Using the
%   ideas of Roberval, show that any light ray, sound wave,
%   etc. directed at one focus will reflect off of the hyperbola toward
%   the other focus.
% \end{embeddedproblem}
% This idea is utilized in the secondary mirror of Cassegrain
% telescopes.

% \begin{wrapfigure}[]{O}{2in}
% %\vskip-.7cm{}
% \includegraphics*[height=2in,width=2in]{Figures/ReflectingHyperbola}
% \caption{}
% \label{fig:ReflectingHyperbola}
% \end{wrapfigure}

% Roberval's idea to use distances from fixed objects to describe the
% location of a point on a plane is the essence of analytic geometry.
% You are accustomed to plotting points by determining the distances
% from two fixed axes to give the coordinates of a point.  This allows
% us to identify geometric curves by their corresponding equations.
% Specifically, a point will lie on a curve provided its coordinates
% satisfy a certain equation.  For example, the point $P$ with coordinates
% $(2,4)$ lies on the parabola with equation $y=x^2$ since $4=2^2,$ whereas
% the point $Q$ with coordinates $(2,5)$ does not since $5\neq2^2.$
% Notice that, remarkably, we were able to answer a geometry question
% without drawing a picture!  We're not promoting not drawing pictures,
% but it does illustrate the power of analytic geometry; we can bring
% the tools and techniques of algebra to bear on solving the geometry
% problems.  

% \begin{myproblem}{}
%   Before the invention of calculus, Galileo tried to determine the
%   motion of a projectile as it fell to the ground.  Efforts of people
%   such as Galileo in handling specific problems such as this helped
%   Newton and Leibniz to generalize the ideas to form differential
%   calculus.  We will recreate some of these ``pre-calculus'' ideas.
%   To start, by experimenting with a ball rolling down a plane, Galileo
%   surmised that an objected falling under the influence of gravity
%   accelerated at a constant rate (ignoring air resistance, friction,
%   etc.).  Today we know that rate to be $9.8$ (meters/second)/second.
%   Suppose we stand at some height and drop a ball.  This says that the
%   velocity of the object at time t seconds is given by $v(t)=9.8t$
%   meters/second.  Figuring out how far the ball has fallen is a bit
%   trickier as the velocity is not constant.
%   \begin{description}
%   \item[(a)] Suppose we divide the time interval $[0,t]$ into four
%     equal intervals.  Determine that average value of $v(0), v(t/4),
%     v(2t/4), v(3t/4), v(4t/4)$ is $4.9t.$ What if we divided the time
%     interval into $8, 16,$ or $32$ equal intervals?
%   \item[(b)] Assuming that the average velocity during the time
%     interval $[0,t]$ is given by $4.9t\,m/s,$  show that the distance
%     the ball has fallen at time $t$ is given by $y(t)=4.9t^2.$
%   \item[(c)] To check if Galileo is correct that $y(t)=4.9t^2$ meters,
%     compute the average velocity of the ball from time $t$ to time
%     $t+\Delta t$ for $\Delta t=.1, .01, .001,$ and $.0001.$
%   \item[(d)] Based on what you obtained in part (c), what would the
%     instantaneous velocity be at time $t,$ (i.e., when $\Delta t=0$).
%     Does this agree with what we had as $v(t)$ before?
%   \end{description}
% \end{myproblem}


%   \begin{myexample}
%     Consider the curve $y=x^2:$\\
% \centerline{\includegraphics*[height=2in,width=2in]{Figures/Quadratic}}
%     We wish to find an equation of the line tangent to this curve at
%     the point $(3,9).$
%     It is clear from the graph that the line tangent to this curve at
%     the point $(3,9)$ will also pass through the $y-$axis, say at
%     the point $(0,b).$ If we can find $b$ we have the tangent line. 
    
%     The slope of the tangent line is $m=\frac{9-b}{3}$ so the equation
%     of the tangent line will be 
%     \begin{equation}
%       y=\left(\frac{9-b}{3}\right)x+b\label{eq:TanSq}
%     \end{equation}
%     for one, and only one, value of $b.$ If $b$ is anything other than
%     that one value then equation~\ref{eq:TanSq} will intersect the parabola
%     twice: once at the point $(3,9),$ and once at another point, say
%     $(x,x^2).$ At this second point we have 
%     $$
%     x^2=\left(\frac{9-b}{3}\right)x+b.
%     $$
    
%     Rearranging this slightly we have 
%     $$
%     x^2-\left(\frac{9-b}{3}\right)x-b=0
%     $$
%     which is a quadratic and we can solve quadratics.

%     Before charging blindly forward though, let's stop and think about
%     what we're trying to accomplish. We want to find the value of $b$
%     which guarantees that there is only \emph{one} solution. We're
%     not really interested in the actual solution itself. We just want
%     to ensure that there is only one.

%     From the Quadratic Formula we see that:
%     $$
%     x=
%     \frac{\left(\frac{9-b}{3}\right)\pm\sqrt{\left(-\left(\frac{9-b}{3}\right)\right)^2+4b}}{2},
%     $$
%     which has one solution precisely when the discriminant\footnote{The part under
%     the square root symbol.} is zero. Thus
%   \begin{align*}
%    \left(\frac{9-b}{3}\right)^2+4b&=0\\
%      \frac{9^2-18b+b^2}{9}+4b&=0\\
%     9^2-18b+b^2+36b&=0\\
%     b^2+18b+9^2&=0\\
%     (b+9)^2&=0,
%   \end{align*}
% so $b=-9.$ Thus an equation of the line tangent to $y=x^2$ at the point
% $(3,9)$ is
% $$
% y=6x-9.
% $$

  

% \begin{embeddedproblem}{}
%   Use this method to find an equation of the line tangent to
%   $y=x^2$ at each of the following points:\\
%   \begin{description}
%   \item[(a)] $(2,4)$
%   \item[(b)] $(-3,9)$
%   \item[(c)] $(1,1)$
%   \item[(d)] $\left(\frac12,\frac14\right)$
%   \item[(e)] $(a,a^2)$
%   \end{description}
% \end{embeddedproblem}  

% \begin{embeddedproblem}{}
%   Use this method to find an equation of the line tangent to the given
%   curve  at the given point:\\
%   \begin{description}
%   \item[(a)] $y=x^2+x$ at $(2,6).$
%   \item[(b)] $y=x^2+x$ at $(\tau,\tau^2+\tau).$
%   \item[(c)] $y=x^2-3x+2$ at $(\tau,\tau^2-3\tau+2).$
%   \item[(d)] $y=ax^2+bx+c$ at $(\tau,a\tau^2+b\tau+c).$
%   \end{description}
% \end{embeddedproblem}
%   %   For example, he showed that the line tangent to the parabola at the
%   % point $(0,b)$ in the following diagram will always pass through the
%   % point $(0,-b).$
% \end{myexample}

\chapter{What the Derivative Tells Us}
\label{cha:what-deriv-tells}
\section{Newton's View: Motion as a Metaphor}
\label{sec:newtons-view:-motion}


Newton invented Calculus for a specific purpose. He was trying to
describe the physical motion. Naturally this affected the way that he
thought about Calculus and the language he used to talk about
it. Although he quickly realized that his Calculus could be used for
other purposes his approach is deeply tied to the idea of motion.

Newton's kinematic approach can be extended to other types of
problems. Recall that in order to find out how high the tossed ball
will go we first needed to discover how long it takes for the ball's
velocity to drop to zero. This is, obviously, because the ball
continues to rise until its velocity is zero. Thereafter it will
fall. Once we know that the velocity is zero at time, say
$t_{\text{max}},$ then we have $v(t_{\text{max}})=0,$ and the
position, $p(t_{\text{max}}),$ at the same time is the maximimum
height of the ball's flight.

This is all fairly intuitive. 

\begin{wrapfigure}[]{r}{2in}
\captionsetup{labelformat=empty}
\centerline{\includegraphics*[height=2in,width=2in]{Figures/metaphor1}}
\label{fig:metaphor1}
\end{wrapfigure}
Can we use this same kind of reasoning to find the coordinates of the
highest point on the graph at the right?

Notice that the horizontal and vertical axes are no longer
``position'' and ``time.'' Since there is no motion taking place there
is clearly no velocity, so we can't simply set our velocity function
equal to zero an solve for the time, $t,$ when velocity is zero like
we did for the falling ball. Not only is there no velocity function,
there is no time $(t)$ either!

Or can we? Yes, it is true that there is no motion, and no time
involved. But so what? Once we plot velocity against time we just have
a curve. The only place where either velocity or time appear is in the
labelling of the axes. 

So let's just change the labels.

That is, let's pretend -- for now -- that this \emph{is} a motion
problem, that the vertical axis \emph{is} position, and that the
horizontal axis \emph{is} time. In that case we can proceed as before:
Find the velocity function (derivative), set it equal to zero to find
the time (number of chocolate bars) when the velocity (derivative) is
zero, put that back into the position (anxiety) function, $A(C),$ to
find the maximum level of anxiety.

This was Newton's idea.  Any such problem can be thought of as a
problem of motion. In this case we think of our anxiety level as
dependent on how many chocolate bars we have. If the number of
chocolate bars changes then the anxiety level increases (or decreases)
accordingly.

But it is silly to rename our functions just to do the analysis and
then change them back. Rather than explicitly pretending that motion
is involved we keep that conceit in the background and adapt our
notation and language to our motion metaphor.

In particular we have seen that $\dfdx{p}{t}=v,$ (velocity is the
derivative of position). Since velocity is by definition the rate of
change of position ``with respect to time'' we adopt the convention,
in this case, of interpreting the symbol $\dfdx{A}{C}$ as the rate of
change of anxiety ``with respect to the number of chocolate bars we
own.''  

In this way the metaphor of motion is simply built into the language
and notation we use, and we say things like ``As the number of
chocolate bars increases our anxiety decreases, then increases, and
then decreases again,'' just as if $C,$  (the number of chocolate
bars) is constantly increasing in the same way that $t,$ (time) does.


\begin{wrapfigure}[]{l}{2in}
\captionsetup{labelformat=empty}
\centerline{\includegraphics*[height=2in,width=2in]{Figures/metaphor2}}
\label{fig:metaphor2}
\end{wrapfigure}
It can be illuminating to look at the graph of both our function and
its derivative at the same time. The sketch at the left shows the graph
of $A(C)$ (in green) and $\dfdx{A}{C}$ (in orange) together on the
same set of axes.

Since $A(0) = 5$ when $C=0$ we seem to be fairly anxious. This is probably
because we have no chocolate. As our number of chocolate bars
increases\footnote{Notice how the language of motion is naturally
  co-opted for our purposes. We either own chocolate or we
  don't. Nothing is changing, but we pretend that as time goes on we
  are getting more chocolate.}
our anxiety level is decreasing so the value of $\dfdx{A}{C}$ is
negative (below the horizontal axis). When $\dfdx{A}{C} =0,$ at
slightly less than one chocolate bar, our anxiety level ``bottoms
out.'' Presumably this is because we're looking forward to eating our
chocolate. As we gain chocolate our anxiety level rises again,
probably because in these uncertain times you never know who is
coveting your chocolate, or when it might be taken from you just because you
have a lot of it. As we accumulate more our anxiety level drops again
because we have plenty to share so we don't need to worry about some
theif taking it from us. 

This is a silly example of course, but the point of this silliness is
quite serious: The derivative of a curve can tell us a lot about the
curve. If  we simply pretend that the curve is the position of some
object and that the derivative is its velocity then it is clear what
the derivative can tell us:
\begin{enumerate}\label{list:FirstDerivTest-Newton}
\item If the derivative is positive then the object is moving in the
  positive direction. Usually this will be \emph{up} or \emph{to the
    right}, but in any case the object moves in the positive direction
  whatever that is in our coordinate system.
\item If the derivative is negative then the object is moving in the
  negative direction. Usually this will be \emph{down} or \emph{to the
    left}, but in any case the object moves in the negative direction
  whatever that is in our coordinate system.
\item If the derivative is neither positive nor negative -- that is,
  if it is zero -- then the object has stopped moving\footnote{Our
    metaphor actually breaks down a little bit here. No one would say
    that a vertically tossed ball is not moving at the top of its
    flight since it immediately begins to fall. Nevertheless, at the
    top of its flight its velocity is zero and it does ``stop'' just
    for an instant.
  }.
\end{enumerate}

\subsection{Population Growth}
\label{sec:population-growth}


Suppose we have a single microbe living in a petri dish and that it
divides once every hour. Then after one hour we have two microbes and
our population has grown by one microbe per hour. After the second
hour our population is now $4$ microbes, and the rate of growth during
that second hour is $3$ microbes per hour. If this continues for $10$
hours our microbe population will have grown according to the
following chart\marginpar{This is supposed to be $P(t)=(5/2)^t.$ Check the numbers.}:
$$
\begin{array}{|ccc||ccc|}
\hline
  \text{hours}&\text{population}&\text{rate of
                                  growth}&\text{hours}&\text{population}&\text{rate
                                                                          of
                                                                          growth}\\\hline{}
  1&2.5&1.5\frac{\text{\tiny microbes}}{\text{\tiny hour}}&6&244.1&97.2\frac{\text{\tiny microbes}}{\text{\tiny hour}}\\[1mm]
  2&6.3&2.5\frac{\text{\tiny microbes}}{\text{\tiny hour}}&7&610.4&242.9\frac{\text{\tiny microbes}}{\text{\tiny hour}}\\[1mm]
  3&15.6&6.2\frac{\text{\tiny microbes}}{\text{\tiny hour}}&8&1525.9&607.2\frac{\text{\tiny microbes}}{\text{\tiny hour}}\\[1mm]
  4&39.1&15.5\frac{\text{\tiny microbes}}{\text{\tiny hour}}&9&3814.7&1518.0\frac{\text{\tiny microbes}}{\text{\tiny hour}}\\[1mm]
  5&97.7&38.9\frac{\text{\tiny microbes}}{\text{\tiny hour}}&10&9536.7&3795.1\frac{\text{\tiny microbes}}{\text{\tiny hour}}\\[1mm]\hline
\end{array}
$$
Can we find a function $P(t)$ which  tells us our microbe
population at any time?

\section{Leibniz's View: Geometry and Differentials}
\label{sec:leibn-view:-geom}
\begin{wrapfigure}[]{r}{2in}
\captionsetup{labelformat=empty}
\centerline{\includegraphics*[height=1in,width=2in]{Figures/LeibnizGraph1}}
\label{fig:LeibnizGraph1}
\end{wrapfigure}
Newton's dynamical view of the Calculus is very useful. However it is
only one way to approach the topic. Leibniz had an entirely different
approach. In Leibniz's view every point on the graph  at the right is
already in place so the graph can be thought of as a completed
entity. It is not ``brought into being'' by the motion of a point, a
pixel, or anything else. If we want to know the shape of this graph
. . . well, that's easy isn't it? Just look at it.

Of course that only works if we have a picture of the graph in hand,
or if we can generate one easily. But suppose we hadn't shown you the
above graph. Suppose instead we'd asked you to describe the graph of:
$$
y=-\frac15(4x^3-16x^2+11x-9).
$$

Ok. Sure. Since you're living after the invention of computing
technology you'd have just pulled some electronic device out of your
pocket, opened your favorite graphing app, punched in the formula
above, held it out to us and said, ``Here it is.''

Touch\^{e} You got us.

Obviously though, neither Newton nor Leibniz could have done that, but
so what? We can, so we're done, right?

Well maybe.

Suppose now that we'd said there is  curve whose derivative
is
$$
\dfdx{y}{x}=-\frac{12}{5}x^2 +\frac{32}{5}x-\frac{11}{5},
$$
and asked, ``What is its shape?''

This is harder, but Leibiz's differential notation is extremely
well-suited to express his approach. The expression $\dfdx{y}{x}$
clearly expresses the idea that the deriviative is the slope of the
tangent to\footnote{Leibniz would have said ``tangent of'' rather
than ``tangent to.'' In his view he was computing the slope of the
curve itself.} the curve $y$ at the point $.$

We know a lot about lines. For example we know that:
 \begin{enumerate}\label{list:FirstDerivTest-Lines}
\item If the slope of a line is positive then the $y$ coordinate is
  increasing.
\item If the slope is negative then the $y$ coordinate is decreasing.
\item If the slope is zero then the $y$ coordinate is not changing.
\end{enumerate}

Of course, lines are easy because the slope of a line never
changes. But we're interested in curves. The slope of a curve changes
from point to point so how does this list help us?

This is where Leibniz's conception of the derivative as a ratio of
differentials is helpful. Since $\dfdxat{y}{x}{x=a}$ is the slope of
the function at the point $x=a$ it is clear that the above list of
statements generalizes as follows:
\begin{enumerate}\label{list:FirstDerivTest-Leibniz}
\item At every point $x=a$ where $\dfdxat{y}{x}{x=a}>0$  the curve is
  increasing. (Notice that we can only claim that the curve is
  increasing at the single point $x=a.$ 
\item At every point $x=a$ where $\dfdxat{y}{x}{x=a}<0$  the curve is
  decreasing. (Notice that we can only claim that the curve is
  decreasing at the single point $x=a.$ 
\item At every point $x=a$ where $\dfdxat{y}{x}{x=a}=0$  the curve is
  horizontal.
\end{enumerate}

\begin{wrapfigure}[]{r}{2in}
\captionsetup{labelformat=empty}
\centerline{\includegraphics*[height=2in,width=2in]{Figures/MaxMin1}}
\label{fig:MaxMin1}
\end{wrapfigure}
Regarding item 3) above, if
$\dfdxat{y}{x}{x=a}=0$ then clearly near $x=a$ the graph of $y$ has
one of the two shapes at the right, which confirms that we can find
the maximum height of a tossed by by setting 
$$
v(t) = \dfdx{p}{t} =0
$$
and solving for $t.$ This is only common sense.

Unfortunately our common sense fails us here. It is, in fact,
\underline{\bf not} true that if
$$
\dfdxat{y}{x}{x=a}=0
$$ 
then $y$ is either a maximum or a minimum at $x=a$ as the following
problem demonstrates.
\begin{embeddedproblem}{}
  Consider the polynomial
$$
y = (x+1)^3(x-1)(x+3).
$$
\begin{description}
\item[(a)] Show that $\dfdxat{y}{x}{x=-1}=0.$ (Hint: You can save
  yourself a lot of work on this problem by keeping you eye on the
  goal. You are not asked to compute $\dfdx{y}{x}.$ The problem is to
  show that $\left.\dfdx{y}{x}\right|_{x=-1}=0.$ You don't need to
  fully expand $\dfdx{y}{x}$ to do that.)
\item[(b)]  Graph $y$ near $x=1$ to convince yourself that $y$ is
  neither a maximum nor a minimum at $x=-1$.
\item[(c)] We don't really need a polynomial as complicated as the one
  given in part (a) to make the point we're getting at. Repeat parts
  (a) and (b) with the polynomial $y=x^3$ at $x=0.$
\end{description}

\end{embeddedproblem}

It is important to notice, as we did above, that when we are dealing
with curves rather than lines we can only state that the curve is
increasing at the single point where the derivative is positive.

% \subsection{Tangent Lines: Decartes's Method of
%     Normals}
% % It is difficult to explain to students the
% %       necessity of the language invented for mathematics. Obviously we
% %       could just say ``perpendicular'' and mean the same thing. Why
% %       use ``normal''? The answer to this is much deeper than you might
% %     think. The fact is that ``perpendicular'' has  a very specific,
% %     geometric meaning. 

% \label{sec:descartes-normals}

%   \noindent{\bf Apollonius and Conic Sections}
%   The problem of finding the tangent line to a curve is a geometry
%   problem, so it is not surprising that some of the first attempts
%   were geometric. The Greek mathematician Apollonius of Perga (circa
%   262 BC to 190 BC) used geometric methods to construct tangent lines to
%   the classical conic sections: the ellipse, the hyperbola, and the
%   parabola.


%   Since algebra had not yet been invented, Apollonius's methods were
%   entirely geometric, and could not be easily generalized for curves
%   that were not conic sections. 


% Ren\`e{} Descartes (1596-1650)  was one of the pioneers in applying the
% techniques of algebra to solve geometry problems.  In his \emph{La
%   Geometrie} he remarked that the problem of finding the tangent to a
% curve was ``. . .  not only the most useful and most general problem
% in geometry that I know, but even that I have ever desired to know.''

% Descartes' technique is often called the Method of Normals\footnote{In this context, ``normal'' means
%       perpendicular.} because
% what he actually finds is the line normal to (perpendicular to) a
% curve.  Once the normal is obtained, then the tangent line is
% perpendicular to that.

% His method for finding the normal to a curve can be described as
% follows.

% Given a curve and a point $P$ on that curve, we find the circle
% passing through $P$ whose center is on the $x$ axis. The line from the
% center of that circle throught the point $P$ will be the normal (and
% its perpendicular through the point (P) will be the tangent.

% Let's take a look at how this might be done in a particularly simple
% situation. Consider the line $y=3x+2$ and the point $(1,5)$ on that
% line.

% \InsertGraphic{}

% We know that the slope of the line normal to this curve\footnote{Yes,
%   we know that lines aren't curved. It is convenient
%   to think of a line as a special case of a curve.} is $-1/3.$ Thus
% the normal line passes through the point $(1,5)$ with slope $-1/3$ so
% the equation of this line is 
% $
% y-5=-\frac13(x-1)
% $
% or 
% $$
% y=-\frac13 x+\frac{16}{3}
% $$
% which crosses the $x$-axis at $(16,0).$ Since the distance from
% $(1,5)$ to $(16,0)$ is $\sqrt{122}$ we see that the equation of the
% circle with center $(16, 0)$ and radius $\sqrt{122}$ is $(x-16)^2 +y^2
% = 122,$ as seen below.

% \InsertGraphic{}

% We now have a point on our tangent line, $(1,5),$  and it's slope,
% $3,$  so we can write down the equation of the tangent line:
% \begin{align*}
%   y-5&=3(x-1), \text{ or, in slope-intercept form}\\
%   y&=3x+2.
% \end{align*}

% Wait a minute! That can't be! This is the equation of our original
% line!

% But think about it for a moment. When we say ``tangent line'' what do
% we really mean? The first definition we usually encounter is ``a line
% that touches a curve at exactly one point.'' But this can't possibly
% work as a definition for a tangent line. To see why not consider the
% tangent to the curve $y=x^3-3x$ at the point $(-1,2).$ It passes
% through two points on the curve: $(-1,2)$ and $(0,2).$

% When we use the word ``tangent'' in ordinary speech we usually mean
% something more like, ``getting off of the main point, but in the same
% direction that things are currently moving.'' When we think
% of ``tangent'' in this way is is clear that a line and it's tangent
% must be the same line. After all, there is only one line \emph{through a
% given point} that \emph{always} points in a given direction.

% But we've gotten off on a tangent.

% Returning to the problem at hand it is clear that we didn't really
% need to find the equation of our circle. Once we have the slope of the normal line,
% the slope of the tangent is known. But this is an artifact of the
% simplicity of our problem. Straight lines are easy. When the curve is
% more complex we \emph{will} need the circle.

% This is because there is one special curve, the circle, whose tangent line is
% obvious. Pick a point on a circle and draw the radius to that
% point. It is well known that the tangent line at that point will be
% the line perpendicular to the radius.



% % To find the tangent line to a general curve Descartes found the
% % tangent circle to a curve and used that to determine the tangent line.
% We'll illustrate Descartes' idea with the following example.

% \begin{myexample}
% Find the slope of the normal (and tangent) line to the curve $y=\sqrt{2x}$ at the
% point $(2,2).$

% \centerline{\includegraphics*[height=2.7in,width=4in]{Figures/DescartesCircle}}


% Following Descartes' approach we look at the family of circles whose
% centers lie on the $x$ axis and which pass through the point $(2,2).$
% The dashed circle represents a generic member of that family of
% circles.  Notice that typically these circles intersect the parabola twice.  We
% are searching for the circle that hits the curve only once.  This is
% the solid circle in our picture.  If we can find the center of that
% circle then its radius through the point $(2,2)$ will be normal to the curve and we can find its
% slope (and the tangent's slope).

% % \begin{wrapfigure}[]{O}{3in}
% % %\vskip-.7cm{}
% % \includegraphics*[height=2in,width=3in]{Figures/CassegrainTelescope}
% % \caption{}
% % \label{fig:CassegrainTelescope}
% % \end{wrapfigure}

% If we let $(a,0)$ denote the coordinates of the center of a generic circle in that
% family, then the equation of the circle with center $(a,0)$ is
% $(x-a)^2+y^2=r^2$ where $r$ is the radius of the circle. Since we
% require our circle to pass through the point $(2,2)$ this radius will
% be the distance from $(2,2)$ to $(a,0).$ That is
% $r=\sqrt{(2-a)^2+2^2}.$
% Thus we have
% $$
% (x-a)^2+y^2=(2-a)^2+2^2.
% $$

% Substituting $y=\sqrt{2x}$ into the equation of the circle, we get
% \begin{align*}
%   (x-a)^2+2x&=(2-a)^2+4\\
%   x^2-2ax+a^2+2x&=4-4a+a^2+4\\
%   x^2+(2-2a)x+(4a-8)&=0
% \end{align*}

% It is tempting at this point, to use the quadratic formula to solve
% for $x$ and get (typically two) values for $x$ in terms of $a.$
% However, before do a bunch of unnecessary work, let's keep in mind
% what we are trying to do.  We want to find the value of $a$ where the
% circle and the curve intersect exactly once.  We really don't care
% about $x$ at this point.  But think about this for a moment. If we use
% the quadratic formula to solve this equation then we will get only one
% solution precisely when the discriminant (the part under the square
% root) is zero.  For our problem, the discriminant is
% $(2-2a)^2-4(4a-8).$ Setting this equal to zero and solving, we get
% \begin{align*}
%   4-8a+4a^2-16a+32&=0\\
%   4a^2-24a+36&=0\\
%   4(a-3)(a-3)&=0\\
%   a&=3.
% \end{align*}

%  Thus the center of the circle which intersects the curve exactly once
%  is $(3,0)$ and the slope of the normal line is  $(2-0)/(2-3)=-2.$
%  So the tangent line to the curve $y=\sqrt{2x}$ at $(2,2)$ has slope
%  $1/2.$  

%  Since we have both a point on the tangent line: $(2,2),$ and the
%  slope of the tangent line we can use the point-slope form of the
%  equation of a line to write down the equation of the tangent
%  line: $$y=\frac{1}{2}x+1.$$
% \end{myexample}

%  \begin{embeddedproblem}{}
%    Use Descartes' Method of Normals to find the slope of the tangent
%    line to the curve $y=\sqrt{x}$ at the point $(4,2).$
%  \end{embeddedproblem}

%  \begin{embeddedproblem}{}
%    In a variation of Descartes' method, find the tangent line to the
%    curve $y=x^2$ at the point $(3,9)$ by considering the family of all
%    the lines passing through the point $(3,9).$ Out of all those lines, only
%    two will intersect the curve exactly once, the vertical line and
%    the tangent line.  Use this method to find the tangent line to
%    $y=x^2$ at $(3,9).$ What would happen if you tried this idea for
%    the curve $y=x^3?$ (Maybe this explains why Descartes used
%    circles.)
%  \end{embeddedproblem}

%  Descartes' method is not only clever, it is completely algebraic.
%  But it is also unwieldy. And for curves that are more complex than
%  parabolas it is often simply impossible to use.

%  Descartes' Method of Normals was not the only technique developed for
%  finding tangent lines before Calculus was invented, but it is
%  typical in both it's  applicability and in the computational effort
%  required.

 Compare this with the ease of finding the equation of the tangent
 line to $y=x^3$ in the problem above using Calculus: We have $y=x^3$
 from which we have $\dfdx{y}{x} = 3x^2.$ So the slope of the tangent
 line at $(3,9)$ is $\left.\dfdx{y}{x}\right|_{x=3} = 3(3)^2=27.$ Thus
 the equation of the line tangent to $y=x^3$ at the point $(3,9)$ is
$$
y-9=27(x-3). \text{ (Tada!)}
$$

It is easy to see how the methods of Calculus replaced previous
efforts.



% The
%  search for finding a double root of a general polynomial led the
%  Dutch mathematician Johann Hudde (1628-1704) to algebraically show
%  that a double root of the polynomial $p(x)=a_0+a_1 x+a_2 x^2++a_n
%  x^n$ must also be a root of the polynomial
%  \begin{align*}
%    q(x)&=a_1 x+2a_2 x^2+3a_3 x^3+\ldots+na_n x^n\\
%        &=x\left(a_1+2a_2 x+3a_3x^2+\ldots+na_n x^{n-1}\right)
%  \end{align*}

% We won't derive this result, but we will return to it later when we
% have built up some Calculus.





%%% Local Variables: 
%%% mode: latex
%%% outline-minor-mode: t
%%% TeX-master: "Calculus"
%%% End: 
