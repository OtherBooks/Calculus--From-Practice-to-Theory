\chapter{Power Series }
\label{cha:power-series}

\section{Representation of Numbers}
\label{sec:repr-numb}

The decimal place-value representation of a number like $43527$ is so
familiar that it feels easy and natural, even simple. It is anything
but. Our familiar  base ten number representation notation is actually
extremely sophisticated. It only seems simple because we learn it in
childhood and use it every day for  all of our lives. 

For example the notation $"43527"$ is actually a condensed form of
$$
4\cdot10^4+3\cdot10^3+5\cdot10^2+2\cdot10^1+7\cdot10^0,
$$
but numbers written in this form are very difficult to work with. We
get our usual representation $"43527"$ by observing that the powers of
ten needn't be explicitly written down since they are clearly
indicated by the position of each digit. Hence the name ``place
value.'' 

However, once the place value system is adopted we do lose
some flexibility, since the digits  must appear in the proper
order. For example $"72534"$ is a completely different number than
$"43527,"$ but 
$$
4\cdot10^4+3\cdot10^3+5\cdot10^2+2\cdot10^1+7\cdot10^0
$$
and 
$$
7\cdot10^0+2\cdot10^1+5\cdot10^2+3\cdot10^3+4\cdot10^4
$$
are the same. It will be convenient for us to use this last ordering
soon.

As long as our base  is $10$ the place value notation prevents us
from confusing the number $"43527"$ with, say, $"18263."$

But suppose our base is $8.$ The then number 
$$
43527 = 4\cdot8^4+3\cdot8^3+5\cdot8^2+2\cdot8^1+7\cdot8^0 = 18263.
$$

Clearly we can't allow this kind of ambiguity. We can't allow the same
set of digits, written in the same order, to mean both ``forty-three
thousand five hundred twenty-nine,'' \underline{and} ``eighteen
thousand two hundred sixty three,''  \underline{and} ``sixty-three
thousand one hundred ninety-one,'' which is what we would get if we
interpreted the digits $"43527"$ using ``11'' as the base. To prevent
this sort of confusion we will use subscripts. That is, 
$$
43527_{10} = 4\cdot10^4+3\cdot10^3+5\cdot10^2+2\cdot10^1+7\cdot10^0
$$
whereas
$$
43527_{8} = 4\cdot8^4+3\cdot8^3+5\cdot8^2+2\cdot8^1+7\cdot8^0 = 18263_{10}, 
$$
and 
$$
43527_{11} = 4\cdot11^4+3\cdot11^3+5\cdot11^2+2\cdot11^1+7\cdot11^0 = 63191_{10}.
$$
Naturally as long as we only use base 10 notation there is no
ambiguity. Since this is the usual situation we usually suppress the
subscripts.

But sometimes it is necessary to use a base other than $10$ and it is
necessary to convert from one base to another. This conversion can be
difficult at first, mainly because it is unfamiliar. That is, in base
$10$ the number one hundred twenty-one is written $121_{10}$ but in
base $5$ it is $441_5.$

\begin{embeddedproblem}{}
  Verify that $121_{10} = 441_5.$
\end{embeddedproblem}

If we did not know the base $5$ representation how could we find it?

This sounds harder than it is. If we write  
$$
121_{10} = a_0 +a_1\cdot5 + a_2\cdot5^2
$$ 
we can find the unknown coefficients in the
order given as follows. Divide each side of the above equation by $5.$
On the right we get 
$$
a_1+a_2\cdot5, \text{ with a remainder of $a_0$.}
$$
On the left we get $\frac{121}{5} = 24$ with a remainder of $1.$ So
$a_0 = 1$ and 
$$
24 = a_1+a_2\cdot5.
$$ 

Dividing both sides of this last formula by $5$ again gives a
remainder of $a_1$ on the right and of $4$ on the left, and $a_2=4$ as
well. Therefore
$$
121_{10} = 441_5.
$$
\begin{embeddedproblem}{}
  Convert $121_{10}$ to each of the following bases.
  \begin{multicols}{3}
    \begin{description}
    \item[(a)] $3$
    \item[(b)] $4$
    \item[(c)] $8$
    \item[(d)] $9$
    \item[(e)] $11$
    \item[(f)] $2$
    \end{description}
  \end{multicols}
\end{embeddedproblem}

\section{Representations of Polynomials}
\label{sec:repr-polyn}

Notice that when we write $121_{10}$ as 
$$
121_{10} = 1+2\cdot10+1\cdot10^2
$$
the expression on the right has the form of a polynomial. That is, if
we replace each instance of the base $10$ with $x$ we get the polynomial
$1+2\cdot x+1\cdot x^2.$ Polynomials can be thought of as numbers
where the base is $x$ (or that it is unspecified).

We will very soon find it very convenient to to be able to convert
polynomials to different bases just like we converted numbers in the
last section. Fortunately, the method we've just developed carries
over unchanged.

\begin{myexample}
  The polynomial $p(x) = 1+2x+x^2$ is represented  with  $x$ as the
  base. Convert it to the base $(x-1).$

  As before, we want to find coefficients $a_0, a_1,$ and $a_2$ so
  that 
$$
 1+2x+x^2 = a_0 + a_1(x-1) + a_2(x-1)^2.
$$
Dividing both sides by $x-1$ we get
\begin{description}
\item[\sc{}On the right:] $a_1+a_2(x-1)$ with the remainder $a_0.$
\item[\sc{}On the left:] $x+3$ with the remainder $4.$
\end{description}
So $a_0=4$ and $x+3=a_1+a_2(x-1).$ Dividing again by $x-1$ gives
\begin{description}
\item[\sc{}On the right:] $a_2$ with the remainder $a_1.$
\item[\sc{}On the left:] $1$ with the remainder $4.$
\end{description}
Therefore
$$
1+2x+x^2 = 4+4(x-1)+(x-1)^2.
$$
\end{myexample}
\begin{embeddedproblem}{}
  Confirm the result in the previous example.
\end{embeddedproblem}

For reasons that we will make clear later, we don't normally refer to
this re-representation of polynomials as a ``change of base\footnote{even
though that is really what it is.}.'' 
Instead, when we convert from base $x$ to base $x-a$ for some $a$ we
say we are ``expanding the polynomial about the number $a.$'' When the
base is $x$ we say that the polynomial is expanded about  the number
$0.$

A more substantial example is in order.
\begin{myexample}
  Expand the polynomial $p(x) = x^5+3x^4-15x^2+1$ about the number
  $2.$

  As before we have 
$$
x^5+3x^4-15x^2+1 =
a_0+a_1(x-2)+a_2(x-2)^2+a_3(x-2)^3+a_4(x-2)^4+a_5(x-2)^5.
$$
Dividing both sides by $x-2$ gives
\begin{description}
\item[\sc{}On the right:]
  $a_1+a_2(x-2)+a_3(x-2)^2+a_4(x-2)^3+a_5(x-2)^4$ with remainder $a_0.$
\item[\sc{}On the left:] $x^4+5x^3+10x^2+5x+10$ with remainder $21.$
\end{description}
Thus $a_0 = 21$ and 
$$
x^4+5x^3+10x^2+5x+10 = a_1+a_2(x-2)+a_3(x-2)^2+a_4(x-2)^3+a_5(x-2)^4.
$$
Dividing again we have
\begin{description}
\item[\sc{}On the right:] $a_2+a_3(x-2)+a_4(x-2)^2+a_5(x-2)^3$ with
  remainder $a_1.$
\item[\sc{}On the left:] $x^3+7x^2+24x+53$ with remainder $116.$
\end{description}
Thus $a_1=116$ and 
$$
x^3+7x^2+24x+53 = a_2+a_3(x-2)+a_4(x-2)^2+a_5(x-2)^3.
$$
Continuing in this fashion gives $a_2= 137$ and
$$
x^2+9x+42 = a_3+a_4(x-2)+a_5(x-2)^2,
$$
and then $a_3=64$ and 
$$
x+11 = a_4+a_5(x-2),
$$
from which it is clear that $a_5= 1$ and $a_4=13.$

Therefore
$$
x^5+3x^4-15x^2+1 =
21+116(x-2)+137(x-2)^2+64(x-2)^3+13(x-2)^4+(x-2)^5.
$$
\end{myexample}
\begin{embeddedproblem}{}
  Verify the result in the previous example.
\end{embeddedproblem}

If we expand the polynomial $p(x) = x^3$ about the number $1$ we get
$$ 
x^3 = a_0+a_1(x-1) +a_2(x-1)^2+a_3(x-1)^3.
$$
Proceeding as before we find that $a_0=1$ and $a_2= 3.$ If, rather
than completing the conversion we stop here it is reasonable to expect
that the graphs of $l(x) = a_0+a_1(x-1) = 1+3(x-1)$ and $p(x) = x^3$
should be related, and indeed, when we graph both polynomials on the
same set of axes we see the following.\\
\centerline{\includegraphics*[height=2in,width=2in]{Figures/LinearTaylor}}

Hey! Wait a second! We've seen things like this before!

This appears to be the graph of $p(x)=x^3$ and its tangent line at
$x=1.$ Is this just an artifact of this particular problem or is it
generally true?

Clearly this is general. If we have an unspecified polynomial expanded
about the number $a,$ $p(x) = a_0 + a_1(x-a) + a_2(x-a)^2 + \cdots +
a_m(x-a)^n$ then clearly $p(a) = a_1$ and $\dfdx{p}{x}(a) = a_1.$ So
it seems that in computing the coefficients $a_0$ and $a_1$ we have
found the line tangent to $p(x)$ at $x=a.$ What do you suppose we will
find when we compute $a_2?$
\begin{embeddedproblem}{}
 Find $a_2$ graph $x^3$ and $a_0 + a_1(x-a) + a_2(x-a)^2$
    on the same set of axes. What do you observe?
\end{embeddedproblem}
%%% Local Variables: 
%%% mode: latex
%%% outline-minor-mode: t
%%% TeX-master: "Calculus"
%%% End: 

