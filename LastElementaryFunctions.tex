\chapter{The Last of the Elementary Functions}
\label{cha:last-elem-funct}
\markright{{\sc Last Elementary Functions}}

\aptta{}
Previously (p. XX)  we showed that the catenary (hanging chain)
$y=y(x)$ satisfied the equation 
$$
\dfdx{y}{x}{2}=\frac{w}{H} \sqrt{1+  \left(\dfdx{y}{x}\right)^2 }
$$
where $w$ is the weight density of the chain, and $H$ is the constant
(magnitude of the) horizontal tension.  We also showed that a parabola
does not satisfy this equation.  In this chapter we will look at
functions which help us to determine the solution.
\endaptta{}

\section{The Natural Exponential}
\label{sec:exponential}

Early efforts to find the line tangent to a curve focussed on what
was  called the subtangent problem. That is, given a curve, $y,$ as in
this figure\\
\centerline{\includegraphics*[height=1in,width=2in]{Figures/subtan1}}
where $B$ is on the curve, $y,$ $AB$ is perpendicular to $CA$ and $BC$
is tangent to $y$ at $B,$ we'd like to find the coordinates of the
point $C.$ The line segment $AC$ is called the subtangent, and finding
the coordinates of $C$ is equivalent to finding the derivative of $y$
at $B.$  This is because if $B$ and $C$ are known then the slope of
the line segment $BC$ is the derivative of $y$ at $B.$

As a technique the subtangent problem quickly fell out of favor with
the invention of Calculus. However the following subtangent problem is
instructive: We ask is there a curve with the property that it
subtangent always has length one?\\
\centerline{\includegraphics*[height=1in,width=2in]{Figures/subtan2}}
It is instructive because, $\dfdx{y}{x},$ the derivative of $y$
appears very naturally. If we take $\d y$ to be an infinitesimal
vertical displacement of the curve $y$ at $B,$ and $\d x$ to be a
horizontal displacement\\
\centerline{\includegraphics*[height=1in,width=2in]{Figures/subtan3}}
then by proportional triangles we have $\dfdx{y}{x} = \frac{y}{1},$ or
\begin{equation}
  \label{eq:ExpDiffeq}
  \dfdx{y}{x} = y.
\end{equation}
For obvious reasons this is called a \emph{differential equation.} In
fact, this is the same differential equation we examined in
section~\ref{sec:eulers-method} so we already have an approximate
graph of this curve. Recall that in section~\ref{sec:eulers-method} we
found that near the point $(0,1)$ the curve looks like this:
\centerline{\includegraphics*[height=2in,width=2in]{Figures/exp-piecewise-linear}}
A graphical approximation is nice but we'd really like an explicit formula. 

% Ok. At this point you have memorized all of the general differentiation
% rules. Really. All of them. Using them in tandem you can take the
% derivative of any function that can be written down as an algebraic combination
% of polynomials, trigonometric function and their inverses.

% So what?  What is this good for?

% Just like Geometry, Algebra or Trigonomety, Calculus was invented to
% solve a particular type of problem. When you drop a ball it is clear
% that how fast the ball is moving depends on how far it has fallen. To
% state this more carefully: The ball's \emph{rate of change} is related
% to its position. This is the sort of problem Calculus
% was invented to address. 

% An example is in order.

% \begin{myexample}
%   Suppose we have some quantity, $y,$ which depends on another
%   quantity, $x.$ We don't really care what $x$ or $y$ represent for
%   this first example. We have seen that the symbol $\dfdx{y}{x}$ is
%   the \emph{rate of change of y} with respect to $x$ so let's make the
%   relationship between $y$ and its rate of change, $\dfdx{y}{x},$ as
%   simple as we possibly can. Let's say they are equal. 

% That is, suppose


We would like to find a function, $f(x),$ that satisfies
equation~\ref{eq:ExpDiffeq}. Nothing really obvious comes to mind as a
solution but before we give up, let's see if we can 
guess a solution.

No, really. Let's try guessing. It can be surprisingly effective. You
know this of course. You've been doing it all of your life, but you've
probably been discouraged from it in the past so you tend to deny
doing it. It's ok. Guess anyway. We won't tell anyone.

The fact is that guessing is a tried and true solution technique and
we encourage you to use it regularly. If you guess wrong, which is
most likely, then you have this related problem to think about:
All of your intuition said this was a good guess.  Why didn't it work?
The answer to that question almost always gives some insight into the
original problem.  Even a bad guess can be useful. 

Bad guesses are
easy. If you don't know what else to do, make a bad guess. But all by
itself a bad guess is a waste of time. You must also figure out why
your bad guess doesn't work.

\underline{\Large\bf But} the real danger in guessing -- the reason
it is usually discouraged -- is that when you do guess correctly
it is {\bf very} tempting to just move on from there. \underline{Don't
  do that.} When you guess correctly it is crucial that you take a few
moments to ask yourself what was the intuition that lead you to that
guess. Guessing correctly means that your intuition was very good. But
intuition is unconcious. If you don't stop and ask yourself why this
seemed like a good idea your thought process will remain unconscious
and thus nearly useless. It  is just as important to understand where a good
guess came from  as it is to understand why a bad guess didn't work.

Stop. Take a breath.  Think. Then move on.

So let's take a guess. There is no need to get really, really crazy
though. let's try this guess: $y = 1.$ Obviously this won't work
because $\dfdx{y}{x}$ is equal to zero, not one. But wait! The
derivative of $x$ is $1$ so if we we guess $y=1+x$ we get $\dfdx{y}{x}
= 1.$ This almost works.  When we differentiate we recover the first, constant
term ($1$) but lose the second, linear term ($x$).

Can we adjust this last guess so as to recover the second term?
Sure. Let's try $y=1+x+\frac{x^2}{2}.$ In that case $\dfdx{y}{x} =
1+x$ and again, upon differentiating we get everything back but the
last term. 

Clearly we can pick up that term by adding $\frac{x^3}{3\cdot2}$ to
our last guess. That is, if we take
$y=1+x+\frac{x^2}{2}+\frac{x^3}{3\cdot2}$ then $\dfdx{y}{x} =
1+x+\frac{x^2}{2}.$ But again, the last missing term is a
problem. Continuing to add terms won't help as the following problem
shows. 

\begin{embeddedproblem}{}
Compute $\dfdx{y}{x}$ and $y-\dfdx{y}{x}$ for each of the following.
  \begin{multicols}{2}
    \begin{description}
    \item[(a)] $y= 1+x+\frac{x^2}{2!}+\frac{x^3}{3!}$
    \item[(b)] $y= 1+x+\frac{x^2}{2!}+\frac{x^3}{3!}+\frac{x^4}{4!}$
    \item[(c)]
      $y= 1+x+\frac{x^2}{2!}+\frac{x^3}{3!}+\frac{x^4}{4!}+\frac{x^5}{5!}$
    \item[(d)]
      $y= 1+x+\frac{x^2}{2!}+\frac{x^3}{3!} + \ldots +
      \frac{x^{n-1}}{(n-1)!} + \frac{x^n}{n!}$
    \end{description}
  \end{multicols}
\end{embeddedproblem}

  Clearly, we've almost got something here. If we take
  \begin{align*}
   y&= 1+x+\frac{x^2}{2!}+\frac{x^3}{3!} + \ldots +
  \frac{x^{999999}}{999999!} +   \frac{x^{1000000}}{1000000!} \\
\intertext{  then}
  \dfdx{y}{x} &= 1+x+\frac{x^2}{2!}+\frac{x^3}{3!} + \ldots +
  \frac{x^{999999}}{999999!}, \\
\intertext{and}
    y-\dfdx{y}{x} &= \frac{x^{1000000}}{1000000!}.
  \end{align*}
  So the  difference between $y$ and $\dfdx{y}{x}$ is that stupid
  one-millionth term! Surely we can handle that somehow!

  Sadly no. The fact is that there is no polynomial that solves the
  differential equation $\dfdx{y}{x}=y.$

  So a fair question is, ``Why have we led you down this wild goose
  chase?''

As we mentioned earlier differential equations can be extremely
difficult to solve. But often in real applications an approximate
solution is almost as good. Clearly we have an approximate solution
here. One million is a very large number, so one million factorial
($1000000!$) is inconceivably large. Thus for any reasonable value of
$x$ that one millionth term, $\frac{x^{1000000}}{1000000!},$ is
  vanishingly small and can probably be ignored in real applications. 

In fact, we don't really have to take anywhere near that many
terms. Depending on the application involved
$y=1+x+\frac{x^2}{2!}+\frac{x^3}{3!}$ might be entirely adequate as an
approximate solution. Indeed, for some applications the approximation
$y=1+x$ is all that is really needed. Although we haven't
\underline{solved} the problem we have found a very good
approximation.


If there is no \emph{polynomial} that solves
equation~\ref{eq:ExpDiffeq} does that mean there is no solution at
all? Certainly not. In fact, since  that last term of the polynomial
was always the stumbling block it should be clear that all we need to
do is \underline{not have a last term}. That is, the solution of
equation~\ref{eq:ExpDiffeq} is:
$$
   y= 1+x+\frac{x^2}{2!}+\frac{x^3}{3!} + \cdots
$$ 
where the dots at the end mean that the summation goes on
forever so there is no last term. 
\begin{embeddedproblem}{}
Differentiate
$$
   y= 1+x+\frac{x^2}{2!}+\frac{x^3}{3!} + \ldots
$$ 
term-by-term to show that 
$$
   \dfdx{y}{x}= y.
$$ 
\end{embeddedproblem}

% \digress{Power Series}
% But this ``solution'' doesn't really help us understand anything at
% all. It is (for now) just empty symbol manipulation. 

% As soon as we ask ourselves what it means it is clear that we don't
% really know. For example, and most profoundly, what does it mean to
% add up infinitely many monomial terms? That is, what is an ``infinite
% polynomial.'' Does such an expression actually mean anything?

% These kinds of questions have probably never come up in your
% mathematics education before, but they will become more common from
% this point on and we will have to watch for them. Writing down an expression like 
% $$
%    y= 1+x+\frac{x^2}{2!}+\frac{x^3}{3!} + \ldots
% $$ 
% is rather like asking, ``Which is bigger or a giraffe?'' It looks like
% standard English. All the parts are there: subject, verb, etc.. But
% when you put it all together it is simply meaningless. 


% If $ y= 1+x+\frac{x^2}{2!}+\frac{x^3}{3!} + \ldots $ it is to have any
% meaning we will have to decide, and agree on, what meaning we want it
% to have.
% \enddigress{}

% This approach to solving equation~\ref{eq:ExpDiffeq} will turn out to
% be very fruitful. In fact, there \underline{is} a solution and it
% \underline{can} be express as the ``infinite polynomial''
% $$
%    y= 1+x+\frac{x^2}{2!}+\frac{x^3}{3!} + \cdots
% $$
% but before we can use this ``solution'' we will have to find a way to
% give meaning to the idea of ``adding up'' infinitely many terms. As
% you might imagine this will be rather tricky, so we will let it be for
% now. 

% On the other hand, the solution of equation~\ref{eq:ExpDiffeq} is of
% profound importance in all of mathematics and all of science. We will
% need to understand it as thoroughly as possible so let's try to come
% at this problem in a new way. 

% The formula 
% $$
% \dfdx{y}{x} =y
% $$ 
% says that the slope of the tangent line of our curve
% $\left(\dfdx{y}{x}\right)$ at every point is exactly equal to the $y$
% coordinate at the same point. What could such a curve look like?

% For example suppose our curve passes through the point $(0,0).$ Then
% the tangent line at $(0,0)$ will be horizontal (slope $=0.$) It is
% easiest to think of the curve as being traced out by the motion of a
% point for this example.

% We are now in a
% position to work out the essential properties of the solution of
% equation~\ref{eq:ExpDiffeq} using only the concept of the derivative
% so we we will do that now. When we are done we will have another, much
% simpler, expression for
% $ y= 1+x+\frac{x^2}{2!}+\frac{x^3}{3!} + \ldots. $


% \underline{\sc{}Fair Warning:} The development of the solution of
% equation~\ref{eq:ExpDiffeq} that you are about to read will be
% difficult to follow. You'll need to slow down and be sure you follow
% each step to fully understand what we're doing.

% Now consider the following table which gives the population of
% snaggle-toothed hornpops over a period of $10$ hours.
% $$
% \begin{array}{|ccc||ccc|}
% \hline
%   \text{hours}&\text{population}&\text{rate of
%                                   growth}&\text{hours}&\text{population}&\text{rate
%                                                                           of
%                                                                           growth}\\\hline{}
%   0&1&&6&64&32\frac{\text{\tiny hornpops}}{\text{\tiny hour}}\\[1mm]
%   1&2&1\frac{\text{\tiny hornpops}}{\text{\tiny hour}}&7&128&64\frac{\text{\tiny hornpops}}{\text{\tiny hour}}\\[1mm]
%   2&4&2\frac{\text{\tiny hornpops}}{\text{\tiny hour}}&8&256&128\frac{\text{\tiny hornpops}}{\text{\tiny hour}}\\[1mm]
%   3&8&4\frac{\text{\tiny hornpops}}{\text{\tiny hour}}&9&512&256\frac{\text{\tiny hornpops}}{\text{\tiny hour}}\\[1mm]
%   4&16&8\frac{\text{\tiny hornpops}}{\text{\tiny hour}}&10&1024&512\frac{\text{\tiny hornpops}}{\text{\tiny hour}}\\[1mm]
%   5&32&16\frac{\text{\tiny hornpops}}{\text{\tiny hour}}&11&2048&1024\frac{\text{\tiny hornpops}}{\text{\tiny hour}}\\[1mm]\hline
% \end{array}
% $$

% Clearly there is a pattern to these numbers and the pattern is easy to
% see. If $t$ is the number of hours elapsed then the population,
% $P = 2^t,$ and the rate of change of the population
% $\dfdx{P}{t}=2^{t-1}.$ So it appears that $\dfdx{P}{t}=2P.$ It is
% interesting that $P$ and $\dfdx{P}{t}$ seem to be so closely
% related. Let's examine that relation.

% % $$
% % \begin{array}{|ccc||ccc|}
% % \hline
% %   \text{hours}&\text{population}&\text{rate of
% %                                   growth}&\text{hours}&\text{population}&\text{rate
% %                                                                           of
% %                                                                           growth}\\\hline{}
% %   1&2.5&1.5\frac{\text{\tiny hornpops}}{\text{\tiny hour}}&6&244.140625&97.1533224297\frac{\text{\tiny hornpops}}{\text{\tiny hour}}\\[1mm]
% %   2&6.25&2.4871250542\frac{\text{\tiny hornpops}}{\text{\tiny hour}}&7&610.3515625&242.883306074\frac{\text{\tiny hornpops}}{\text{\tiny hour}}\\[1mm]
% %   3&15.625&6.2178126355\frac{\text{\tiny hornpops}}{\text{\tiny hour}}&8&1525.87890625&607.208265186\frac{\text{\tiny hornpops}}{\text{\tiny hour}}\\[1mm]
% %   4&39.0625&15.5445315888\frac{\text{\tiny hornpops}}{\text{\tiny hour}}&9&3814.697265631&1518.02066296\frac{\text{\tiny hornpops}}{\text{\tiny hour}}\\[1mm]
% %   5&97.65625&38.8613289719\frac{\text{\tiny hornpops}}{\text{\tiny hour}}&10&9536.74316406&3795.05165741\frac{\text{\tiny hornpops}}{\text{\tiny hour}}\\[1mm]\hline
% % \end{array}
% % $$

% % The first thing we notice about this table is the utter futility of 
% % displaying so many decimals. The table is so overrun with digits
% % that it is nearly impossible to see any pattern to the numbers. Here
% % is the same chart with each number rounded to the nearest tenth.


% % That's better, but only slightly. 

% % There seems to be a relationship
% % between $\dfdx{P}{t}$ and $P$ similar to the relationship we noticed
% % in the bacteria problem earlier. However in this case it is clearly
% % not true that  $\dfdx{P}{t}=2P.$ 

% Clearly the coefficient of proportiality, $2,$ is special to this
% particular problem. Let's generalize things a bit by supposing instead
% that $\dfdx{P}{t}=kP$ for some unknown constant, $k.$

% Ok. Now what? It seems that we've made the problem even harder by
% introducing $k.$ At least we know something about the number $2,$ but
% $k$ is an arbitrary constant. It could be anything.

% Remembering that we're ultimately interested in the more general
% problem let's make this as simple as we can, for now, by assuming that
% $k=1.$ If we can solve this simpler problem maybe it will give us some
% insight into the general situation.


% \digress{}
% Moreover, once we have
% decided what is meant by an expression like 
% $ y=
% 1+x+\frac{x^2}{2!}+\frac{x^3}{3!} + \ldots $ we will have 
% $$
% e^x=
% 1+x+\frac{x^2}{2!}+\frac{x^3}{3!} + \ldots
% $$
% as well.
% \enddigress{}

At this point we have two approximate answers to the question, ``What
curve satisfies the differential equation $\dfdx{y}{x}=y?"$ By Euler's
Method we've found several approximate points on the curve and by
trying to guess a solution we've found that a polynomial of the form
$$
y=1+x+\frac{x^2}{2!}+\frac{x^3}{3!}+ \cdots + \frac{x^n}{n!}
$$
is also probably a good approximation as well. If both of those
statements are true then the graphs should be similar. In the following
figure we have both our Euler approximation and the graph of
$y=1+x+\frac{x^2}{2!}+\frac{x^3}{3!}.$\\
\centerline{\includegraphics*[height=2in,width=6in]{Figures/ExpEulerCubic}}
As you can see they are indeed similar, at least near the point
$(0,1),$ so we can be confident that we are on the right track.

A disadvantage of Euler's Method is that a great deal of effort is
needed to find just a few (approximate) points on our curve. But now
that we have convincing evidence that the polynomials
$y=1+x+\frac{x^2}{2!}+\frac{x^3}{3!}+ \cdots + \frac{x^n}{n!}$ also
make good approximations we can use those to see what our function
looks like farther from $(0,1).$ Here is the graph of the polynomial
$y=1+x+\frac{x^2}{2!}+\frac{x^3}{3!}:$ \\
\centerline{\includegraphics*[height=2in,width=2in]{Figures/CubicExpApprox}}
and here is the graph of $y=1+x+\frac{x^2}{2!}+\frac{x^3}{3!}+ \cdots
+ \frac{x^{10}}{10!}:$\\
\centerline{\includegraphics*[height=2in,width=2in]{Figures/TenthExp}}

Do these graphs make sense to you? Stop and think about it for a
moment.

Our differential equation is $\dfdx{y}{x}=y.$ This says that
the slope of the curve at any point is equal to its
$y-$coordinate. Put another way, as the graph rises is will get
steeper.

Here is the graph of $y=1+x+\frac{x^2}{2!}+\frac{x^3}{3!}+ \cdots +
\frac{x^{50}}{50!}.$ You can see for yourself that this is what happens.\\
\centerline{\includegraphics*[height=2in,width=2in]{Figures/FiftiethExp}}

But wait a minute! Have you seen a graph like this before?

Of course you have. The graph of $y=10^x$ looks very similar:\\
\centerline{\includegraphics*[height=2in,width=2in]{Figures/TenToTheX}}
Now this is interesting. If everything we've done so far is valid the
the solution of $\dfdx{y}{x}=y$ is probably something like
$$
y=e^x
$$
where $e$ is some, currently unknown, number.

Let's proceed on that assumption\footnote{Obviously this is actually
  true or we wouldn't have spent this much time on it.} for now,
recognizing that this is only a conjecture for now. That is, we make
the following conjecture.

\begin{myconjecture}
  There is a real number we'll denote by $e,$ with the property that
  $e^x$ satisfies equation~\ref{eq:ExpDiffeq}. That is
$$
\dfdx{(e^x)}{x} = e^x.
$$
\end{myconjecture}

Depending on your mathematical background you may have seen this
number before and you may already know that it is approximately
$2.7128.$ This is true but it isn't especially useful information for
our purposes right now.

What is useful is that we now have strong evidence that the solution
of equation~\ref{eq:ExpDiffeq} is $y=e^x,$  and that this can be
interpreted as an exponential function just like $10^x$ or $2^x.$
Because of this the expression $e^x$ has been named the ``natural
exponential.''

We will admit that nothing seems ``natural'' about $y=e^x$ being the
natural exponential function.  One could say that satisfying $dy/dx=y$
makes it natural, but there is more to it than that.  First, consider
that by the chain rule, if $x=kt$ for some constant $k,$ then
$\dfdx{y}{t}=\dfdx{e^kt}{t}=\dfdx{e^x}{x}\dfdx{x}{t}=e^x\cdot{}k=ky.$
If we interpret $t$ as time, then this says that the quantity
$y=y(t)=e^kt$ grows at a rate proportional to the amount present.  In
here, the constant $k$ is called the ``relative (nominal) growth
rate.''  The name makes sense if you notice that
$k=\frac{\dfdx{y}{t}}{y}=\frac {\frac{\d{y}}{y}}{t},$ and often this
growth rate is stated as a percentage.  Consider the following
example.

\begin{myexample}
  Suppose a strain of bacteria, under controlled conditions, grows
  continuously with a relative growth rate of 5\% per hour.  If we
  start with $2$ grams of this bacteria, how much would we have in $4$ 
  hours?

    If we let $B$ represent the amount of bacteria (in grams) at
    time $t$ hours, then we have that this must satisfy 
    $$
    \dfdx{B}{t}=.05 B, \ \     B(0)=2.
    $$

    You can check that $B=Ce^.05t$ will satisfy this differential
    equation for any arbitrary constant $C.$  To find $C,$ we will utilize
    the initial condition $B(0)=2.$  Substituting $t=0,$ and  $B=2$ into our
    equation $B=Ce^kt$ we get
    $$
    2=Ce^{.05\cdot0}=C\cdot1=C
    $$
    and so the amount of bacteria at any time is given by $B=2e^.05t$ 
    and after $4$ hours we have $B(4)=2e^{.05\cdot4}\approx 2.44$ grams.  Notice that
    after $400$ hours, the model predicts that we would have
    $B(400)=2e^{.05\cdot400}\approx 970330390.8$ grams.  This illustrates the
    notion that something growing exponentially grows really fast, and
    that our model is not a good predictor for large values of $t.$ 
    This limitation makes sense as the bacteria could not sustain a
    growth rate of $5$\% indefinitely.  There are modifications to this
    model which we will explore later to take this into account, but
    for now we will utilize this exponential growth within reason.
  \end{myexample}
  
We will explore some more examples of exponential growth models.

\begin{myexample}
  Suppose that any investment grows continuously at a nominal rate of
  $7$\% per year.  What would be the effective yield of this
  investment?  That is, how much would the investment actually pay in
  one year?

  Solution: Let $A=A(t)$ denote the value of the investment (in dollars)
  at time $t$ years and let $A_0$ denote the initial amount.  Then we have
  that $A$ must satisfy
  $$
  \dfdx{A}{t}=.07A,\ \  A(0)=A_0
  $$
  Thus, as before, we have $A(t)=A_0e^.07t.$  Thus in one year, the
  investment is worth $A(1)=A_0e^.07\approx 1.0725A_0.$  Thus, our annual
  yield is actually $1.0725A_0-A_0=.0725A_0$ which represents and
  effective yield of $7.25$\% as opposed to a nominal rate of $7$\%.

\marginpar{  Bud,  we could put some of these at the end of the section on
  logarithms, including some standard exponential growth and decay,
  Newton's law of cooling.}

  Of course, the relative (nominal) rate is not always given so neatly
  and often we need to glean it from the information given.  To do
  this, we will need to look at the inverse of the natural exponential
  function, namely the natural logarithm function.  But before we do
  this, we will utilize our new-found friend to settle the matter of
  the catenary.
\end{myexample}
\begin{embeddedproblem}{}
  \begin{description}
  \item[(a)] Show that the curve
  $$
  y=\frac{H}{w} \left(\frac{e^{\frac{w}{H}x}+e^{-\frac{w}{H}
        x}}{2}\right)
  $$
  really does satisfies the equation of the catenary
  $$
  \dfdxn{y}{x}{2}=\frac{w}{H} \sqrt{1+(dy/dx)^2}
  $$
  where $w$ is the weight density of the chain, and $H$ is the
  constant (magnitude of the) horizontal tension.
\item[(b)]   Assume $w=1,$ and  $H=1$ and graph this curve.  Does it look like a chain?
  What happens to the graph if we use $w=1, H=2$ or $w=2, H=1.$  Does this
  make sense physically?  Why or why not?
\end{description}

\end{embeddedproblem}

\section{The Natural Logarithm}
\label{sec:natural-logarithm-1}

You may not have thought of it this way before, but exponentiation
changes addition to multiplication and multiplication to
exponentiation.  Specifically, we have
$$
b^{x+y}=b^x b^y,\ \ \ b^xy=(b^x )^y.
$$
The inverse function (a logarithm) will do exactly the opposite; it
will change multiplication to addition and exponentiation to
multiplication.  Here are the details.

As we mentioned in the last section, we need to look at the inverse of
an exponential function.  This inverse is call a logarithm.  The word
logarithm is a combination of two Greek words logos (reckoning) and
arithmos (number), so literally a logarithm is a ``reckoning number.''
What does this mean?

Logarithms were invented by John Napier $(1550-1617)$ as a device which
changes multiplication to addition.  Nowadays, with the availability
of handheld calculators, this may not seem like a big deal.  However,
back in the $16$th and $17$th centuries, where astronomy was growing and
calculations were completed by hand, this was a major innovation.
Think about it, is it easier to multiply astronomical numbers or to
add them?  As put forth by the mathematical historian Howard Eves, the
invention of logarithms literally ``doubled the life of the
astronomer.''

While logarithms' computational value has diminished somewhat, the
idea of changing multiplication to addition (and exponentiation to
multiplication) is a valuable mathematical tool and we will have
opportunities to exploit this.  For now, let us see what the
connection is to exponential functions.


\begin{definition}{}
  Let $b>0,$ and  $x>0.$  Then $y=\log_b x$ exactly when $x=b^y.$ 
\end{definition}

A few things to note on this definition.
\begin{enumerate}
\item The above definition says
$$
  b^{\log_b x} =b^y=x ,\  \log_b b^y =\log_b x=y.
$$
\item   The number $b$ is called the base of the logarithm (and the
  exponential function) and is relegated to being a positive number.
  Furthermore, since $x=b^y,$ then $x>0$ is necessary as well.
\end{enumerate}
To see that the logarithm fulfills Napier's desire to change
multiplication into addition and exponentiation into multiplication,
we have the following result.

\begin{embeddedproblem}{}
  Let $b>0,$ $x>0,$ $y>0$ and  let $c$ be any real number.
  \begin{description}
  \item[(a)] Show that $\log_b  xy=\log_b x+\log_b y.$  [Hint: Consider
    $b^{\log_b x+\log_b y}$  and use properties of exponents to show that
    this equals $xy.$]

    Show that $\log_b x^c =c \log_b x.$  [Hint: Consider $b^{c
      log_b x}$ and use properties of exponents to show that this
    equals $x^c.$]
  \end{description}
\end{embeddedproblem}

Just as $e^x$ is called the natural exponential function, $y=\log_e x$
is called the natural logarithm, and is denoted by $y=\ln x.$  This is
just what we need to deal with exponential growth (and decay) when the
relative (nominal) growth rate is not explicitly given.

\begin{myexample}
Suppose we have a bacteria culture which grows at a rate proportional
to the amount present.  We noticed that initially we have $2$ grams, and
$24$ hours later we have $3$ grams.  How much would we have in $48$ hours? 

Before we solve this, what is your guess?  Will it be $4$ grams? More?
Less?  Let's see.

As before, let $B=B(t)$ be the amount of bacteria (in grams) at time $t$ 
(hours).  Again, we know $B(0)=2.$  Though we don't know what the
relative growth rate is, we can call it something, say $k.$  Thus we
have 
$$
\dfdx{B}{t}=kB,\  B(0)=2,\text{ and} B(24)=3.
$$

The differential equation is satisfied by $B=Ce^kt$ for some constant
$C.$ Utilizing our initial condition, we can determine $C.$ 
$$
2=B(0)=Ce^{k\cdot0}=C\cdot1=C.
$$

So $C=2$ and we have $B=2e^kt.$  To determine $k,$ we will utilize our
other bit of information. 
\begin{align*}
  3&=B(24)=2e^24k \\
  3/2&=e^24k\\
  \ln \frac{3}{2}&=24k\\
  \frac{1}{24} \ln \frac{3}{2}&=k.
\end{align*}

Thus $B=2e^{\left(\frac{1}{24} \ln \frac{3}{2}\right)t}.$  This is kind
of cumbersome, so let's use properties of exponents and logarithms to
clean it up before we look at $B(48).$ 
$$
B=(e^{\left(\frac{1}{24} \ln \frac{3}{2}\right)t} = \left(e^{\ln(3/2)}\right)^\frac{t}{24}  =\left(\frac{3}{2}\right)^{\frac{t}{24}}
$$
$$
B(48)=\left(\frac{3}{2}\right)^(48/24)=\left(\frac{3}{2}\right)^2=\frac{9}{4}=2.25
\text{ grams}.
$$
How does this match your guess?  
\end{myexample}

A natural question is to ask how long
will it take for the culture to reach $4$ grams.  Translating this into
a mathematical question, we want $t$ when $B=4.$ 
\begin{align*}
  4&=\left(\frac{3}{2}\right)^{\frac{t}{24}}\\
 \ln 4&=\ln \left(\frac{3}{2}\right)^{\frac{t}{24}}  \\
 &=\frac{t}{24} \ln \left(\frac{3}{2}\right) \\
t&=\frac{24  \ln 4}{\ln \left(\frac{3}{2}\right)}\\
 &\approx 82.06 hours.
\end{align*}
Of course, this assumes that the exponential model holds for that long
a period of time. 

Exponential functions are not only used to model growth, but also
decay.  This is especially important when speaking of radioactive
decay.  You might have heard terms such as radiocarbon dating and
half-life, but what do these terms mean?  Let's look at both.

Radiocarbon dating is a technique for dating organic material from the
past.  It was developed by Willard Libby in $1949$ and revolutionized
the field of archeology.  He was awarded the Nobel Prize in Chemistry
in $1960$ for this development. Radiocarbon $(C_{14})$ is a
radioactive isotope of carbon which is constantly being created in the
atmosphere by the interaction of cosmic rays with atmospheric
nitrogen. The resulting radiocarbon combines with atmospheric oxygen
to form radioactive carbon dioxide, which is incorporated into plants
by photosynthesis; animals then acquire $C_{14}$ by eating the
plants. When the animal or plant dies, it stops exchanging carbon with
its environment, and from that point onwards the amount of $C_{14}$ it
contains begins to decrease as the $C_{14}$ undergoes radioactive
decay. Measuring the amount of $C_{14}$ in a sample from a dead plant
or animal such as a piece of wood or a fragment of bone provides
information that can be used to calculate when the animal or plant
died
\begin{verbatim}
[https://en.wikipedia.org/wiki/Radiocarbon_dating].
\end{verbatim}


This is where the notion of half-life comes into play.  The half-life
of $C_{14}$ is about $5,730$ years.  This means that given any initial
amount of $C_{14},$ in $5,730$ years, half of the radiocarbon will
remain (the other half having decayed).  Of course, the decaying
process is gradual.  It is reasonable to assume that the rate of decay
of the radiocarbon (or any radioactive material) is proportional to
the amount present.  After all, the more material there is, the more
there is to decay.  All of these ideas are incorporated in the
following example.

\begin{myexample}
  The Shroud of Turin is a religious relic which is reputed to be the
  burial shroud of Jesus and bears the image of a man which is alleged
  to be that of Jesus.  The Vatican agreed to subject pieces of the
  shroud to radiocarbon dating in $1987.$ To examine this, let
  $A=A(t)$ be the amount of $14C$ (in mg) at time $t$ years, where $t=0$ 
  represents when the shroud was used.  Let $A(0)=A_0$ be the initial
  amount of $C_{14}$ present in the sample.  How much $C_{14}$ would be present
  if the shroud was 2000 years old?  Mathematically, we have
$$
\dfdx{A}{t}=kA,\ \ \ A(0)=A_0,\ \ \text{and}   A(5730)=\frac{1}{2} A_0,
$$
and we want to know what $A(2000)$ is.

The solution of the differential equation and initial condition is $A=A_0e^kt.$  To determine $k,$ we utilize the half-life.
\begin{align*}
  1/2 A_0&=A(5730)\\
  &=A_0 e^5730k \\
 \frac{1}{2}&=e^5730k \\
k&=1/5730 ln \frac{1}{2}.
\end{align*}
Notice that this is negative since $\ln (1/2)=-ln 2.$  This makes
sense. This the growth rate should be negative since the amount of
$C_{14}$ is shrinking.   Thus 
$$
A=A_0 e^{\frac{1}{5730}\ln (1/2) }t=A_0 \left(\frac{1}{2}\right))^\frac{t}{5730}
$$
and so
$$
A(2000)=A_0 \left(\frac{1}{2}\right)^{\frac{2000}{5730}} \approx 0.785A_0.
$$
This says that an authentic shroud should contain about $78.5\%$ of  the original amount.

\begin{embeddedproblem}{}
  One of the samples contained $88.9\%$ of the original $C_{14}.$ How
  old would this sample be?
\end{embeddedproblem}
The technique of carbon dating is not in dispute, but there are issues
about the quality of the sample and other issues.  Interested readers
can read
https://www.usatoday.com/story/news/world/2013/03/30/shroud-turin-display/2038295/.
\end{myexample}

\marginpar{Bud, do we want to put in Newton's Law of cooling here, or wait until we talk more about solving differential equations. I can insert it here if you think we shouldn't wait.
}

% Of course, $y=ln x$ has uses in other places, so we should treat it like
% any other function we have come across.  In particular, what do we get
% if we differentiate it?

You should already be familiar with logarithms as the inverses of
exponentials. That is, if $f(x)=10^x$ and $\inverse{f}(x)=\log_{10}(x)$
then
$$
f\left(\inverse f(x)\right) = 10^{\log_{10}(x)}=x
$$
and
$$
\inverse f\left(f(x)\right) = \log_{10}(10^x) = x.
$$

We will define the ``natural logarithm'' in precisely the same way.

\begin{definition}{}
  If $f(x) = e^x$ then $\inverse f(x) = \log_e(x),$ where 
$$
f\left(\inverse f(x)\right) = e^{\log_{e}(x)}=x
$$
and
$$
\inverse f\left(f(x)\right) = \log_{e}(e^x) = x.
$$
\end{definition}
\begin{mynotation}{Natural Logarithm}
  The natural logarithm is often shortened to $\ln(x).$ We will use
  this shortened form hereafter.
\end{mynotation}

A natural question to ask is, ``What is the derivative of $\ln(x)?$ At
first this may seem to be a difficult question to answer but  
it is actually quite straight-forward. In fact, we will use the same
trick that we used to obtain
$\dfdx{\inverse\sin(x)}{x}=\frac{1}{\sqrt{1-x^2}}.$

First let $y=\ln(x)$ and then we simply differentiate both sides of 
$$
e^y=x
$$
to obtain
\begin{align*}
  e^y\dfdx{y}{x} &=1\\
  \intertext{from which }
  \dfdx{y}{x}    &= \frac{1}{e^y}\\
                 &= \frac{1}{e^{\ln(x)}}\\
                  &= \frac{1}{x}.
\end{align*}

Thus we have the following:
\begin{mytheorem}
  \begin{enumerate}
  \item Suppose that $y=e^x.$ Then $\dfdx{y}{x} = e^x.$
  \item Suppose that $y=\ln(x).$ Then $\dfdx{y}{x} = \frac1x.$
  \end{enumerate}
\end{mytheorem}

Aside from being the inverse function of an exponential, a logarithm
is an important function in its own right, as illustrated in the
following problem.  

\begin{embeddedproblem}{}\label{prob:tractrix1}
Consider the top view of a tractor-trailer.  Initially, the center of
the rear axle of the tractor is at the origin and the center of the
rear axle of the trailer is at the point $(1,0).$
  \centerline{\includegraphics*[height=3in,width=4in]{Figures/Tractrix}}
  \begin{description}
  \item[(a)] 	Suppose the tractor pulls the front wheels up the
    $y$-axis and that the rear wheels don't slip.  Show that the path
    $y=y(x)$ that the center of the rear axle of the trailer follows
    must satisfy the equations 
    \begin{align*}
      \dfdx{y}{x}&=-\frac{\sqrt{1-x^2}}{x}\\
      y(1)&=0.
    \end{align*}
    This path is called a tractrix from the Latin verb trahere ``to
    drag or pull,'' the same root that gives us farm tractors and
    ``tractor beams'' from science fiction.
  \item[(b)] 	Show that  
    $$
    y=\ln\left(\frac{1+\sqrt{1-x^2}}{x}\right)-\sqrt{1-x^2}
    $$
    satisfies the above differential equation and the condition $y(1)=0.$
    Graph this.  Does it make sense?
  \item[(c)] How far will the tractor have gone (in trailer
    lengths) before the trailer is within one degree of vertical?
  \end{description}
\end{embeddedproblem}

%%% Local Variables: 
%%% mode: latex
%%% outline-minor-mode: t
%%% TeX-master: "Calculus"
%%% End: 
