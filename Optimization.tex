\section{The Shape of Things}
\label{sec:optim-shape-things}
\markright{{\sc The Shape of Things}}
% \section{The Abstract Viewpoint}
% \label{sec:abstract-viewpoint-1}
Whether we use Newton's dynamic view or Leibniz's more geometric
interpretation of the techniques of Calculus is largely a matter of
choice, and personal preference. For some problems Newton's
interpretation is more natural and easier to use, for others Leibniz's
viewpoint is better. What is important is that we recognize that
neither approach \underline{\bf{}is} The Calculus. They are ways of
interpreting, or ``thinking about'' The Calculus. 

In this section we will try to think about Calculus directly, without
any metaphors between us and the topic. For the most part we will
fail, but never mind that. This is just our first attempt. We will get
better.
 
In the previous two sections\footnote{specifically, on
pages~\pageref{list:FirstDerivTest-Newton},
\pageref{list:FirstDerivTest-Lines}
and~\pageref{list:FirstDerivTest-Leibniz}} we saw that the derivative
gives us information about curves in much the same way that the slope
of a straight line gives us information about the line. The only
difference is that since the numerical value of the derivative changes
at each point it only gives us \emph{local} information about the
curve at that point. Whereas the slope gives us global information
about the whole line.

Moreover, since we reached the same conclusions regardless of whether
we used Newton's or Leibniz's point of view, we can be reasonably
confident that our conclusions are independent of which point of view we
use. That is, we can be sure that what we have found  are fundamental
truths of The Calculus itself. We repeat these conclusions in the
abstract:
\begin{description}
\item[Increasing Functions:] If the derivative of a function is
  positive at some point then the function is increasing at that
  point.
\item[Decreasing Functions:] If the derivative of a function is
  negative at some point then the function is decreasing at that
  point. 
\end{description}

As simple and intuitive as this seems there are two problems with
it. We will need to proceed carefully.

\begin{wrapfigure}[]{r}{2in}
\captionsetup{labelformat=empty}
\centerline{\includegraphics*[height=2in,width=2in]{Figures/Abstract1}}
\label{fig:}
\end{wrapfigure}
The first problem is, what do we mean by increasing and decreasing? 

No, really. What do we mean?

You've been taught all of your life that increasing means ``rising to
the right'' and that decreasing means ``dropping to the right.'' Thus
the graph at the right is usually said to be is increasing from the
far left to a point somewhere between zero and one, decreasing from
that point to a point somewhere between three and four, and then
increasing again thereafter.  That interpretation is deeply tied to
the convention of saying that as we go farther to the right on the
$x$-axis the values of $x$ get bigger. But this is only a convention.

Conventions are handy, but they are not reality. If, for some reason,
we had adopted the convention that moving farther to the \emph{left}
causess the value of $x$ to increase then we would say (reading
right-to-left) that this graph decreases first, then increases for a
while, and then decreases again.

The good news here is that we \emph{will} be adopting the usual
conventions so we can say, loosely, that a graph is increasing if it
rises as we move to the right. \underline{But} keep in mind that what
this really means is that a function, $f(x),$ gets bigger if $x$ gets
bigger. Which direction means ``bigger'' depends entirely on the
context of a given problem.

Resolving this issue was simply a matter of recognizing our own
convention. Such difficulties are usually easy to resolve once we
realize that's what they are.  However, distinguishing between a
convention (especially one as old and comfortable as this one) and
reality can be very difficult because conventions, once we've gotten
comfortable with them, begin to feel as if they are {\bf actual}
reality. They are not.

But we have another problem and this one cannot be resolved by
noticing that we've mistaken a convention for reality. We stated above
that if the derivative of some function $f(x),$ is positive at some
point $x=a,$ then $f(x)$ is increasing \emph{at that point.}

The problem is, what does it mean to be increasing at a single point?

In order to know that $f(x)$ has increased we need to move away from
$x=a$ to some point (to the right of $a$ in our adopted convention),
say $x=a+h,$ and see if $f(a+h)$ is bigger than $f(a).$ But even if
$f(a+h)$ \emph{is} bigger the only information we have is that
$f(a+h)>f(a).$ This does not guarantee that $f(x)$ was increasing the
whole time. That is, there might be some point between $a$ and $a+h$
-- let's call it $\alpha,$ -- where $f(\alpha)<f(a).$ Even worse there
might be some point, $\beta$ between $a$ and $\alpha$ where
$f(\beta)>f(a).$ Yet again there might \emph{another} point $\gamma$
between $a$ and $\beta$ where $f(\gamma) < f(a),$ again. And so on
$\ldots$

If the distance between $a$ and $a+h$ is very small (that is, if $h$
is very small) and if $f(a+h)>f(a),$  it is very easy to
\emph{believe} that $f(x)$ increased through all of the points from
$a$ to $a+h.$ But belief is not knowledge. In fact, we have no
information whatsoever about the behavior of $f(x)$ between $a$ and
$a+h.$ We simply know that when $x=a+h$ the function was greater than
it was at $x=a.$

This is getting very hard to think about isn't it? That's because what
we are getting caught on here is one of Zeno's
Paradoxes\footnote{Specifically, the arrow paradox. Briefly, Zeno said
  that an arrow in flight can't really move. This is because to get
  from where it is now, to any other point in space it has to traverse
  all of the (infinitely many) points in between. But this is also
  true of all of those points. Thus before the arrow can move once it
  has to move infinitely many times. That's crazy.}. If you have never
heard of the Greek philosopher Zeno, you really should \google{}
him. He's an interesting guy. 

But wait until you are done with your Calculus for today.

Here's how we get around this difficulty. We all know what we mean
when we say that a curve is increasing at a point. The mental image
that is generated is very clear. The problem is that any attempt to
put this into words runs headlong into Zeno and his paradox. We cannot
solve this problem, so we will simply go around it. That is, we first
recognize that the usual intuitive meaning of the word increasing does
not apply at a single point, since we need at least two points to say
we've increased from one to the other.  We are thus free to define the
phrase ``increasing at a point'' in any manner we choose. We choose
the following:
\begin{definition}{}
  Consider the graph of the function $y(x),$ at some point $(a, y(a)).$
  \begin{enumerate}
  \item The graph is said to be \emph{increasing} at $x=a$ if $\dfdxat{y}{x}{x=a} > 0$
  \item The graph is said to be \emph{decreasing}\footnote{Notice that
      our definition does \underline{\bf{}not} address the situation
      $\dfdxat{y}{x}{x=a}=0.$ Since the definition does not address
      this we say that the function is neither increasing nor
      decreasing when $\dfdxat{y}{x}{x=a}=0.$ } at $x=a$ if
    $\dfdxat{y}{x}{x=a} < 0$
  \end{enumerate}
  \label{def:first-deriv-test}
\end{definition}


Does this feel like cheating to you? It probably does.  After all,
what have we really accomplished? We took our highly intuitive
observation above, that if the derivative is positive then the
function is increasing, and turned it around to make it a
definition. It probably seems like we're blathering on about nothing
at all, making easy things seem difficult and getting lost in a jungle
of logical pecadillos. This is the usual criticism brought against
mathematicians and mathematics.

But stop and think about this just a little more deeply. Is that
really all that we've done?

Suppose we we would like to know when a function is increasing and
when it is decreasing (we do). In the absence of any other tools,
our only option is to look at the value of the function here, then
look at it over there, and decide which is bigger. Unfortunately, as
Zeno pointed out this tells us exactly nothing about what happened in
between. So we are obliged to look at a point in between, and another,
and another $\ldots$

On the other hand, if we know the derivative of our curve all we have
to do is check, to see for which values of $x$ the derivative is
positive (function is increasing) and for which it is negative
(function is decreasing). Most of the time we don't have to look at
the points one at a time. We can consider entire classes of points all
at once.  


\begin{myexample}
  For example, consider the function whose derivative is $\dfdx{y}{x}=
  3x^2-3.$ Can you tell us where this curve is increasing and where it
  is decreasing?
  
  Sure you can. Since $\dfdx{y}{x}$ is positive when $x<-1$ and when
  $x>1,$ but negative when $-1<x<1.$ Thus our curve increases until
  $x=-1,$ decreases from $x=-1$ to $x=1$ and increases thereafter.
  \begin{embeddedproblem}{}
    Do the computations necessary to show that $\dfdx{y}{x}$ is
    positive when $x<-1$ and when $x>1,$ but negative when $-1<x<1.$

    Then find at least one function whose derivative is $3x^2-3,$
    graph it, and confirm that it increases when $x<-1$ and when
    $x>1,$ and decreases when $-1<x<1.$
  \end{embeddedproblem}
\end{myexample}

Definition~\ref{def:first-deriv-test} is sometimes called the First
Derivative Test, but such an imposing name seems unnecessarily
overblown to us. On the other hand, we mention the name because it
suggests that there might be something called a Second Derivative
Test, and indeed there is. Here's what it says: If the second
derivative of a function is positive at a point then the graph of the
function is concave upward at that point. If the second
derivative of a function is negative at a point then the graph of the
function is concave downward at that point. 

Clearly this will not make sense until we've understood all of the
words being used. For example, what can ``concave upward'' possibly
mean. We will illuminate this with the following examples.

Each of the following graphs:\\
\centerline{\includegraphics*[height=2in,width=4in]{Figures/ConcaveUp}}
exemplifies the idea of ``concave up,'' whereas the following graphs are ``concave down.''\\
\centerline{\includegraphics*[height=1.5in,width=4in]{Figures/ConcaveDown}}

Obviously concave up means ``a bowl that opens up,'' and concave down
means ``a bowl that opens down. However it is easy to get the wrong
impression from simple examples so let's look at some more.

Is this graph\\
\centerline{\includegraphics*[height=1.75in,width=3in]{Figures/ConcaveUpDown}}
concave up or concave down? 

Clearly it is both. On the interval $(0,\pi)$ it is concave down and
on the interval $(\pi,2\pi)$ it is concave up. So this notion is like
increasing/decreasing in that we will always need to specify the
interval on which a given function is either concave up or concave down.

Now what about these two?\\
\centerline{\includegraphics*[height=2in,width=4in]{Figures/ConcaveUpDownUnclear}}

The graph on the left is clearly\footnote{Never mind that we can't
  actually see \emph{all} of the graph.}  concave up when $x<0,$ but
what about when $x>0?$ Is it concave up there? Concave down?  Neither?
Both? It is very hard to tell from the graph.  The one on the right is
similarly vague in the region around $x=0.$

Clearly, if it is important for us to distiguish concave up from
concave down\footnote{It is.} we will need a more precise method
than simply looking at graphs.

Let's proceed slowly. Since we're studying Calculus the derivative
will probably be instrumental in our understanding of concavity, so we
ask what we can determine about the derivative of a function whose
graph is concave up.

Consider our first example $y(x) = x^2+1.$ This is concave up
everywhere so we will consider how its derivative changes as $x$ moves
increases. The following sketches show the graph and
its tangent at $x=1,$ $x=2,$ and $x=3.$\\
\centerline{\includegraphics*[height=2in,width=4in]{Figures/ConcaveUpWithTangent}}
Notice that as $x$ increases the tangent line gets steeper and
steeper. That is to say its slope gets larger. Or, even more formally,
the derivative increases.
\begin{embeddedproblem}{}
  Look back up to the graph of the other ``concave up'' examples and
  convince yourself that this is true of all of them: As $x$ increases
  the derivative of each function also increases.
\end{embeddedproblem}

Does that mean that the derivative of a concave down function will
decrease as $x$ increases? Yes, of course it does.
\begin{embeddedproblem}{}
  Look back up to the graph of the  ``concave down'' examples and
  convince yourself that this is true of all of them: As $x$ increases
  the derivative of each function decreases.
\end{embeddedproblem}

So, apparently in order to distinguish where a function is concave up
from where it is concave down we need to locate where the derivative
is increasing and where it is decreasing.  If only we had a tool to
tell us when a function is increasing and when it is decreasing the
problem would be solved.

But we do! That's exactly what the First Derivative Test tells us! If
we want to know where the graph of $y(x)$ is increasing or decreasing
we look at its derivative, $\dfdx{y}{x}$ and find where this is
positive or negative.

To find out where the graph of $y(x)$ is concave up or concave down,
we ask where its derivative, $\dfdx{y}{x},$ is increasing and where it
is decreasing. Where the derivative is increasing the graph of $y(x)$
will be concave up and where it is decreasing the graph of $y(x)$ will
be concave down.

To determine where the derivative is increasing or decreasing we need
to look at the \emph{derivative of the derivative,} $\dfdxn{y}{x}{2}$
and ask where it is positive and where it is negative.


So apparently the derivative of $y(x)$ can tell us where a function is
increasing, decreasing, concave up, and concave down. This is actually quite a lot to tell 

% \section{The Shape of Things}
% \label{sec:shape-things}
% Can we graph the function in the previous example? Well, almost. One
% possibility is the graph at the right.


\begin{ProblemSection}
  \begin{myproblem}{}
    Find all values of $x$ for which each of the following curves is
    increasing and for which it is decreasing. \\
         Hint:
         Recall that the definition of the absolute value is: 
         $$
         \abs{x} =
         \begin{cases}
           x& \text{if } x\ge0 \\
           -x& \text{if } x<0 
         \end{cases}.
         $$
    \begin{multicols}{3}
      \begin{description}
       \item[(a)] $y=\frac{1}{x^2}$
       \item[(b)] $y=\frac{x^2}{x+1}$
       \item[(c)] $y=\abs{x}$
       \item[(d)] $y=\cos^2x -\cos x, 0\le x\le 1\pi$
       \item[(e)] $y=$
       \item[(f)] $y=$
       \item[(g)] $y=$
       \item[(h)] $y=$
       \item[(i)] $y=$
       \item[(j)] $y=$
       \item[(k)] $y=$
       \item[(l)] $y=$
      \end{description}
    \end{multicols}
\end{myproblem}

  \begin{myproblem}{} 
    Consider the top view of an airplane $P$ starting at the point
    $(1,0)$ and traveling up the line $x=1$ at a constant speed $v.$
    When the plane is at $(1,0),$ a homing missile $M$ is fired from
    the origin directly at the plane. Assuming that the missile
    travels at a speed which is $k$ times the speed of the plane
    $(k>1)$ and is always aimed directly at the plane, find the path
    the missile takes.  [Such a path is called a pursuit curve.]  The
    diagram below shows the situation at time $t.$\\
    \centerline{\includegraphics*[height=3in,width=4in]{Figures/PursuitCurve}}
    \begin{description}
    \item [(a)]  If we let $s$ denote the distance the missile has traveled
      at time $t,$ show that the pursuit curve must satisfy the
      differential equation
      \[\dfdx{y}{x}=\frac{(\frac{s}{k}-y)}{1-x}\]
      and use this to show that the pursuit curve must satisfy the
      differential equation
      \[(1-x) \dfdxn{y}{x}{2}=1/k \dfdx{s}{x}=1/k \sqrt{1+(\d y/\d x)^2}\]
      with the initial conditions $y(0)=0,$ $\dfdx{y}{x}(0)=0.$
    \item[(b)]  Solve the differential equation in part (a).  [Hint: Let
      $z=\dfdx{y}{x}$ and solve the resulting equations for $z.$]
    \item[(c)]  How long does it take for the missile to catch the plane?
    \end{description}
  \end{myproblem}

\begin{myproblem}{}
  \begin{description}
  \item[(a)] Show that $x^2=1+2(x-1) +(x-1)^2.$
  \item[(b)] Find the equation of the straight line tangent to the
    graph of $y=x^2$ at the point $(1,1).$
  \item[(c)] How is your formula in part (b) related to the formula in
    part (a)?
  \end{description}
\end{myproblem}

\begin{myproblem}{}
  Consider the functions:\marginpar{This belongs with the foreshadowing of power series}
\[    y(x) = \frac{x^2}{x^2+1} \text{ and } p(x)=Ax^2+Bx+C.\] 
We want to find values for the parameters, $A, B,$ and $C$ that make
the graphs of these two functions as similar as possible near the
point $(2, 4/5).$
  \begin{description}
  \item[(a)] First compute $\dfdx{y}{x},$ and $\dfdxn{y}{x}{2}$ at the
    point $x=2.$
  \item[(b)] Naturally we'll want both functions to pass through the
    point $(2, 4/5)$ so we set 
    \begin{align}
      p(2) &= f(2).\label{TS:1}
      \intertext{To guarantee that the graphs are ``pointing in the
        same direction'' we set their derivatives at the point $(2,
        4/5)$ equal as well.}
      y^\prime(2) &= p^\prime(2)\label{TS:2}
      \intertext{To guarantee that the graphs are ``similarly curved''
        we set their second derivatives at the point $(2,
        4/5)$ equal as well.}
      y^{\prime\prime}(2) &= p^{\prime\prime}(2)\label{TS:3}
    \end{align}
    Solve equations~\ref{TS:1},~\ref{TS:2}, and~\ref{TS:3} to find $A,
    B,$ and $C.$
  \item[(c)] On the same set of axes graph $f(x)$ and $p(x).$ How are
    these graphs similar and how are they different?
  \end{description}

\end{myproblem}

    \begin{myproblem}{}
      \marginpar{This should be somewhere else.}  L'Hopital's Rule
      problem: $\limit{x}{0}{(1-ax)^{\frac1x}}.$ (Form: $0^0$ but is
      \underline{not} equal to $1.$
  \end{myproblem}

\end{ProblemSection}

% \subsection{Precalc vs. Pre-calc}
% An argument could be made that a student who is not comfortable with
% precalculus topics such algebra and analytic geometry would find the
% going in calculus difficult and anecdotal evidence suggests that
% students that students do not possess the necessary fluency with these
% topics.  Efforts such ``just in time'' algebra have tried to address
% this issue.  Perhaps looking at Pre-calculus would provide another
% opportunity while setting the stage for the course.  After all,
% calculus in its present form would not exist without the achievement
% of utilizing algebra to solve geometry problems (and vice versa).
% Furthermore, both Newton and Leibniz acknowledged the contributions of
% their predecessors.   

% Consider, for example, Leibniz's 1684 paper on calculus A New Method
% for Maxima and Minima as Well as Tangents, Which is Impeded Neither by
% Fractional No by Irrational Quantities, and a Remarkable Type of
% Calculus for This.  The title alone is telling and should be shared
% with students to pique their interest and to set the stage for the
% course, while providing opportunities to review pre-calculus topics.
% For example, notice the word New.  What were the old methods and why
% did this supplant them?  What actually is a Calculus?  Even more
% curious are the following excerpts from the article [Calinger,
% p. 387-392].

% ``Knowing thus the Algorithm (as I may say) of this calculus, which I
% call differential calculus, all other differential equations can be
% solved by a common method. We can find maxima and minima, as well as
% tangents without the necessity of removing fractions, irrationals, and
% other restrictions, as had to be done according to the methods that
% have been published hitherto.''

% ``Other very learned men have sought in many devious ways what someone
% versed in this calculus can accomplish in these lines as by magic.''

% If we make the claim to students that calculus is one of the pinnacles
% of human thought, and that it is important to learn, then it behooves
% us to look at the methods that this supplanted to set the stage for
% its development and to put the rules of calculus into context.  It
% also affords the opportunity to practice precalculus skills as the
% following problems provide. 

% \begin{myproblem}{}
%   Descartes' Method of Normals: [Boyer, p. 166-167] In his work
%   G\'e{}om\'e{}trie [1637], Ren\'e{} Descartes (1595-1650) presented a
%   completely algebraic method for finding the normal (perpendicular)
%   to a curve.  From there it is straightforward to find the tangent to
%   the curve, a problem which ``I dare say that this problem
%   (constructing tangents) is not only the most useful and general
%   problem in geometry that I know, but even that I have ever desired
%   to know.''  Given a curve f(x,y)=0 and a point $P$ on that curve, we
%   can find the normal line by considering the family of circles whose
%   centers lie on the $x$ axis and passing through $P.$ We can
%   algebraically find the center of the circle with intersects the
%   curve only once.  The radius of that circle will be the normal and
%   its perpendicular through $P$ will be the tangent to both the circle
%   and the curve.  Consider the point $(4,2)$ lying on the parabola
%   $y=\sqrt{x}.$
%   \begin{description}
%   \item[(a)] Find the equation of the circle passing through
%     $(4,2)$ centered at $(a,0).$\\[1mm]
% \centerline{\includegraphics*[height=2.5in,width=5in]{Figures/DescartNorm}    }
%   \item[(b)] Find the value of $a,$ so that the parabola and the
%     circle only intersect once.  Compute the value of the slope of the
%     line joining $(4,2)$ and $(a,0),$ and use this to determine the
%     slope of the tangent line to the parabola at $(4,2).$
%   \end{description}
% \end{myproblem}

% It should be noted that Descartes' Method requires one to find the
% double root of an equation, which is tedious at best.  Fortunately,
% algorithms were developed to address this problem.  One such rule was
% developed by the Dutch mathematician Johann Hudde (1628-1704):

% Given the polynomial $f(x)=a_0+a_1 x+\ldots+a_n x^n$ and any arithmetic
% progression $a,a+b,a+2b,\ldots,a+nb,$ form the polynomial
% $$ 
% f^* (x)=a_0 a+a_1 (a+b)x+a_2 (a+2b) x^2+\ldots+a_n (a+nb) x^n
% $$

% Huddes' rule states that any double root of f must be a root of $f^*$
% [Katz, p. 474].  Huddes' rule can be verified algebraically, but this
% might not be appropriate for the beginning of a calculus class.
% However, It can be verified later by noticing that $f^*
% (x)=af(x)+bf^\prime(x).$ This leads to the following problem which can be
% assigned after learning the product rule.

% \begin{myproblem}{}
%   \begin{description}
%   \item[(a)] Consider the polynomial $f(x)=a_0+a_1 x+\ldots+a_n x^n.$
%     Show that if $x=r$ is a double root of $f(x)$ then it is a root of
%     $f^\prime(x).$
%   \item[(b)] Use part (a) to verify Hudde's rule, namely that if $x=r$ is a double root of $f(x),$ then it is a root of
% $$  
% f^*(x)=a_0 a+a_1 (a+b)x+a_2 (a+2b) x^2+\ldots+a_n (a+nb) x^n.
% $$

%   \end{description}
% \end{myproblem}

% It should be noted that Hudde's rule was a precursor to calculus and
% that Newton and Leibniz would certainly have been aware of it when
% they were developing their calculus rules.




\subsection{Rainbows}
Did you ever notice a double rainbow?  
\InsertGraphic{} 
Look carefully at the order of the colors in the primary rainbow and
the secondary rainbow.  Do you notice that it is reversed?  Why does
this happen?  There are some other questions which readily come to
mind.  Why is the primary rainbow below the secondary one?  Why is the
primary rainbow brighter and why isn't there always a secondary
rainbow?  We will use what we know about the refraction and reflection
of light to examine some of these questions.

When a ray of light enters a raindrop (which is spherical), it gets
both refracted and reflected.
\includegraphics*[height=2in,width=2in]{Figures/Rainbow1}

As you can see in the diagram, the ray of light entering the raindrop
at an angle of radian measure $\alpha$ is refracted to an angle of radian
measure $\beta.$  This is governed by Snell's Law $\sin\alpha/v_a =\sin\beta/v_b,$
where $v_a$ and $v_b$ are the velocities of light in air and water,
respectively.  Actually only a portion is refracted as the other
portion is reflected off the exterior of the raindrop.  Once inside
the raindrop, it reflects off of the back of the raindrop (again only
a portion) so that the angle of incidence equals the angle of
reflection.  A portion of the light then exits the raindrop governed
by Snell's Law again.  The total amount of (clockwise) rotation in
this process is called the deviation angle $D(\alpha).$

\begin{myproblem}{}
  Show that $D(\alpha)=\pi-2(2\beta-\alpha)=\pi-4\beta+2\alpha.$

Notice that $\alpha$ and $\beta$ are not independent, they are governed by Snell's Law which we will rewrite as 
$$
\sin\alpha=v_a/v_b\sin\beta=k \sin\beta
$$
where $k=v_a/v_b$  is the index of refraction (for water).  In the case of white light, this index of refraction is about $4/3.$  Graphing $D(\alpha)$ for this value of $k$ and $0\le\alpha\le \pi/2,$ we have
\includegraphics*[height=2in,width=2in]{Figures/Rainbow2}

Notice from the graph that the minimum of this curve appears to be
around the point $(1.03,2.41).$  This says that the minimum value of
$D(\alpha)$ is about $2.41 radians\approx 138^\circ $ and it occurs when $\alpha\approx 1.03
radians\approx 59.4^\circ.$  Notice from an enlargement the graph near that minimum
that for values of $\alpha$ near $1.03$ the amount of deviation is roughly the
same.
\includegraphics*[height=2in,width=2in]{Figures/Rainbow3}

This says that light rays entering the raindrop at approximately
$59\circ$ have the highest concentration when exiting the raindrop.
At other values of $\alpha$ the dispersion of light is greater.  This
provides the angle that the rainbow makes with an observer.  This
supplementary angle to $D(\alpha)$ is called the rainbow angle,
$$
R(\alpha)=\pi-D(\alpha)  radians=180^\circ-(180^\circ)/\pi D(\alpha).
$$

\includegraphics*[height=2in,width=2in]{Figures/Rainbow4}
\end{myproblem}

\begin{myproblem}{}
  Show that for an index of refraction $k$ that $D(\alpha)$ is minimized when
  $\sin\alpha=\sqrt{(4-k^2)/3}.$  Use this to fill in the following table:
$$  
  \begin{array}[c]{|c|c|c|c|}
    \hline{}
    Color&k&\alpha in degrees&R(\alpha) in degrees\\\hline
     \text{Red}&1.331&&\\\hline
    Orange&1.332&&\\\hline
    Yellow&1.333&&\\\hline
    Green&1.335&&\\\hline
    Blue&1.337&&\\\hline
    Indigo&1.340&&\\\hline
    Violet&1.344&&\\\hline
  \end{array}
$$
\end{myproblem}

The values of $R(\alpha)$ you obtained in the table accounts for the
various color bands that appear in the rainbow as sunlight hits
droplets at various angles.

\includegraphics*[height=2in,width=2in]{Figures/Rainbow5}

The primary rainbow is created by light entering the upper portion of
the raindrop.  If light is bright enough then a secondary rainbow is
created from light entering the lower portion of the raindrop as in
the diagram below.

\includegraphics*[height=2in,width=2in]{Figures/Rainbow6}

This time the deviation angle $D(\alpha)$ is the total amount of clockwise
rotation and the rainbow angle would be $R(\alpha)=D(\alpha)-\pi.$  Note that the
extra reflection accounts for the rainbow being not as bright.

\begin{myproblem}{}
  Show that in the case of the secondary rainbow
$$
D(\alpha)=2\alpha-6\beta+2\pi.
$$
and show that this is minimized when $\cos\alpha\sqrt{(k^2-1)/8.}$ Use
this to complete the following table.
$$  
  \begin{array}[c]{|c|c|c|c|}
    \hline{}
    Color&k&\alpha in degrees&R(\alpha) in degrees\\\hline
     \text{Red}&1.331&&\\\hline
    Orange&1.332&&\\\hline
    Yellow&1.333&&\\\hline
    Green&1.335&&\\\hline
    Blue&1.337&&\\\hline
    Indigo&1.340&&\\\hline
    Violet&1.344&&\\\hline
  \end{array}
$$
Use this table and the table in the previous problem to explain why the secondary
rainbow is above the primary rainbow and its colors are reversed.
\end{myproblem}


\begin{ProblemSection}
  \begin{myproblem}{}
    The area of a rectangle is $400 m^2.$  Find a formula for the
    perimeter of the rectangle in terms of one of the sides of the
    rectangle.  Remember to define what your variables are (along with
    their units).  Also remember to include the possible values for
    your independent variable (the domain of your function).
  \end{myproblem}
  \begin{myproblem}{}
    \begin{description}
    \item[a)] You throw a ball straight up with an initial velocity of
      $50 m/s.$ Assuming that there is no air resistance and that the
      ball accelerates toward the earth at a constant rate of
      $9.8 ((m/s))/s,$ what would the ball's velocity be at $t$
      seconds.
   \item[b)] How would your formula change if you threw the ball
     downward?
    \end{description}
  \end{myproblem}
  \begin{myproblem}{}
    If a promoter charges $\$50$ each for concert tickets, then he/she
    will sell $15000$ tickets.  For each $\$1$ increase in price per
    ticket, $100$ tickets less will be sold.  Determine a formula for
    the revenue based on the price per ticket, assuming that the price
    will be at least $\$50$ per ticket.
  \end{myproblem}
  \begin{myproblem}{}
    The perimeter of an isosceles triangle is $100$ units.  Find the
    area of the triangle as a function of the base.  Don't forget to
    define your variables and give the domain of the function.
  \end{myproblem}
  \begin{myproblem}{}
    A cylindrical tank must hold $500$ cubic feet of water.  The ends of
    the tank are cut from two individual squares of metal, and the
    side is constructed from one rectangular sheet of metal.  The
    metal for the ends costs $\$5$ per square foot and the metal for the
    side costs $\$4$ per square foot.  The weld for the seams costs $\$2$
    per linear foot.  What is the cost of the tank (a) As a function of
    its height?  (b)  As a function of the radius?  Don't forget
    to identify variables and to provide the domain.
  \end{myproblem}
  \begin{myproblem}{}
    Consider a rectangular box with a square base whose volume is
    $500~ m^3.$ Find the surface area of the box as a function of its
    height.  Now find it as a function of the length of the side of
    the base.  Don't forget to identify your variables and to provide
    the domain.
  \end{myproblem}
  \begin{myproblem}{}
    \begin{description}
    \item[(a)] The fixed cost for publishing a specific calculus book
      is $\$100,000.$ These are costs that the publisher would incur
      no matter how many books were printed.  Beyond that, it costs
      $\$20$ per book to actually print the book.  If we let n denote
      the number of books printed and $C=C(n)$ denote the cost in
      dollars from printing this many books, then give a formula for
      the total cost to print $n$ books.  What would the cost per book
      be?
    \item[(b)] Suppose the company plans to charge $\$150$ per book.
      Assuming the company can sell as many books as it produces (as
      with a print-on-demand company), write a formula for the profit
      $P$ in dollars that the company would have from printing and
      selling $n$ books.  How many books would need to be sold to
      break even?
    \end{description}
  \end{myproblem}
\end{ProblemSection}

%%% Local Variables: 
%%% mode: latex
%%% outline-minor-mode: t
%%% TeX-master: "Calculus"
%%% End: 

