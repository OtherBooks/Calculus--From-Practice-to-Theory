\chapter{Differentials, and Differentiation Rules }
  \label{chapt:differentials}
\markboth{{\sc Differentials}}{{\sc Differentiation Rules}}

\begin{wraptable}{r}{3.6in}
  \begin{tabular}{ |c|c|c| }
    \hline
%    \multicolumn{3}{|c|}{} \\
    \multicolumn{3}{|c|}{\bf{}General} \\
    \multicolumn{3}{|c|}{\bf{}Differentiation Rules} \\
%    \multicolumn{3}{|c|}{} \\
    \hline
%    && \\
\small    Rule~\ref{thm:ConstantDifferential}&\small    The differential of &\small If $a$ is a constant then \\
    &\small   a constant is zero.  &\small $\d a = 0.$\\\hline
%    && \\
\small    Rule~\ref{thm:LinearityConstantMultiple}&\small    The constant        &\small If $a$ is a constant and\\
    &\small   multiple rule.       &\small $x$ is a variable then \\
    &\small                        &\small$\d(ax) = a\d{x}.$\\\hline
%    && \\
\small    Rule~\ref{thm:LinearitySum}&\small    The sum             &\small If $x$ and $y$ are variables then \\
    &\small       rule.            &\small $\d(x+y) = \d x + \d y.$\\\hline{}
%    && \\
\small    Rule~\ref{thm:product-rule}&\small    The Product         &\small If $x$ and $y$ are variables then\\
    &\small     Rule.              &\small $\d(xy) = x\d y + y\d x.$\\\hline
%    && \\
\small    Rule~\ref{thm:power-rule}&\small    The Power           &\small If $x$ is variables and\\
    &\small      Rule.             &\small  $n$ is a rational number then\\
    &\small                        &\small  $\d(x^n) = nx^{n-1}\d x$\\
%    && \\
    \hline{}
%    && \\
\small    Rule~\ref{thm:quotient-rule}&\small    The Quotient        &\small If $x$ and $y$ are variables then\\
    &\small           Rule.        &\small  $\d\left(\frac{x}{y}\right) = \frac{y\d x
      - x\d y}{x^2}$\\
%    && \\
    \hline
%    && \\
\small    Rule~\ref{thm:chain-rule}&\small    The Chain        &\small If $y$ depends on
c    $x,$ \\
    &\small    Rule.            &\small and $x$ depends on $t$ then\\
    &\small                     &\small  $\d y = \dfdx{y}{x}\dfdx{x}{t}\d t$\\[2mm]
%    && \\
    \hline
  \end{tabular}
\end{wraptable}
% \begin{wrapfigure}[]{r}{3in}
% \captionsetup{labelformat=empty}
% \centerline{\includegraphics*[height=4.5in,width=3in]{Figures/DifferentiationRules}}
% \label{fig:}
% \end{wrapfigure}
In the last chapter, we looked at some of the pre-calculus techniques
utilized before the invention of the Calculus by Newton and Leibniz.
These techniques were extremely clever and much was accomplished in
applying these techniques to tangents, optimization, and related
problems.  Clever as these were, they were ad-hoc and somewhat limited
in their applicability.  Newton and Leibniz were well-aware of these
techniques, but went further.  Independently, they produced general
rules and techniques which could be systematically applied to the
problems solved by their predecessors, and could be applied to
situations that the ad-hoc methods could not.  Newton was the first to
develop his calculus during a two-year period at his home in
Whoolsthorpe, Lincolnshire, UK when Cambridge was closed due to an
outbreak of plague in $1665.$  He was under $25$ at the time.  Leibniz
independently developed his rules for calculus while employed as a
diplomate in Paris during the period $1672-1674.$  He was $26$ at the
time.  Their results on Calculus are essentially the same, though
their approaches are fundamentally different.  Newton preferred to
think of quantities as changing with time (fluents) and developed
rules for computing their instantaneous rates of change (fluxions),
whereas Leibniz focused on infinitely small differences
(differentials).  They, as well as their contemporaries, recognized
that these approaches were equivalent and we will eventually see this.
Each approach has its advantages, and we will exploit both the
geometric view (Leibniz) and the dynamic view (Newton).  We will start
with Leibniz' approach.  

As was noted in the previous chapter, Leibniz published the first
paper on Calculus in $1684.$  Newton developed his ideas first, but
delayed in publishing his results.  In his paper, {\it{}A New Method for
Maxima and Minima, as Well as Tangents, Which is Impeded Neither by
Fractional nor Irrational Quantities, and a Remarkable Type of
Calculus for This,} Leibniz provides rules for computing and
applications of his differential calculus.  Literally speaking,
differential calculus means ``rules for differences.''  By differential,
Leibniz means an infinitely small (infinitesimal) difference.  So if $x
is$ some quantity, then its differential $\d x$ represents an infinitely
small increase or decrease in $x.$  This may sound a little strange to
you, to talk of infinitesimal changes, and in truth, Leibniz did not
try to describe what an infinitesimal was.  Mathematicians were aware
of the notion before Leibniz and he never could rigorously define what
one was.  But he didn't let this foundational issue paralyze him.  We
want to emphasize that the foundational aspects are not something to
be swept under the rug and we will address them later, but for now, we
don't want to let the foundations paralyze us and keep us from
applying this surprisingly powerful tool to solving problems.  So be
patient on the foundational questions, they will be addressed later.

We do want to point out that you have come across the idea of taking
the foundational aspects for granted before.  How many of you have
questioned the notion of a point which has no length, width, or
height?  This useful notion permeates nearly all of mathematics, even
though it is often referred to as an undefined term.  For now, we will
treat differentials in much the same manner, focusing on its utility
rather than its existence.

The utility of a differential comes not from trying to measure its
(infinitesimal) size, but from looking at their relationship to other
differentials.  Specifically, if we have two (or more) quantities, $x$
and $y,$ related by some equation, then we want to apply the rules of
differential calculus to create a differential equation relating their
differentials.  The process of finding the differential of a quantity
is called differentiation.  In some applications, we will use the
problem to define a differential equation and use this to see how the
quantities are related. This reverse process is called
antidifferentiation.  Before we go any further, let's state the rules
for differentiation:

% Original table begins here
% \begin{center}
%   \begin{tabular}{ |c|c|c| }
%     \hline
%     \multicolumn{3}{|c|}{} \\
%     \multicolumn{3}{|c|}{\Large\bf{}General} \\
%     \multicolumn{3}{|c|}{\Large\bf{}Differentiation Rules} \\
%     \multicolumn{3}{|c|}{} \\
%     \hline
%     && \\
% Rule~\ref{thm:ConstantDifferential}&    The differential of & If $a$ is a constant then \\
% &   a constant is zero.  & $\d a = 0.$\\[2mm]\hline
%     && \\
% Rule~\ref{thm:LinearityConstantMultiple}&    The constant        & If $a$ is a constant and\\
% &   multiple rule.       & $x$ is a variable then \\
% &                        &$\d(ax) = a\d{x}.$\\[2mm]\hline
%     && \\
% Rule~\ref{thm:LinearitySum}&    The sum             & If $x$ and $y$ are variables then \\
% &       rule.            & $\d(x+y) = \d x + \d y.$\\[2mm]\hline{}
%     && \\
% Rule~\ref{thm:product-rule}&    The Product         & If $x$ and $y$ are variables then\\
% &     Rule.              & $\d(xy) = x\d y + y\d x.$\\[2mm]\hline
%     && \\
% Rule~\ref{thm:power-rule}&    The Power           & If $x$ is variables and\\
% &      Rule.             &  $n$ is a rational number then\\
% &                        &  $\d(x^n) = nx^{n-1}\d x$\\[2mm]
%     && \\
%     \hline{}
%     && \\
% Rule~\ref{thm:quotient-rule}&    The Quotient        & If $x$ and $y$ are variables then\\
% &           Rule.        &  $\d\left(\frac{x}{y}\right) = \frac{y\d x
%                            - x\d y}{x^2}$\\[2mm]
%     && \\
%     \hline
%     && \\
% Rule~\ref{thm:chain-rule}&    The Chain        & If $y$ depends on
%                                                     $x,$ \\
%                             &    Rule.            & and $x$ depends on $t$ then\\
%                             &                     &  $\d y = \dfdx{y}{x}\dfdx{x}{t}\d t$\\
%     && \\
%     \hline
%   \end{tabular}
%   \label{tab:DiffRules}
% \end{center}



These rules for differentials are simultaneously straightforward and
subtle.  The subtleties will come later in the foundational aspects,
but we will provide some heuristic arguments for the first seven rules
here.  In fact, the 3 rules are identical with what you know from
regular algebra and regular (finite) differences.  For example, if $y$ 
was constantly $7,$ then the traditional change in $y$  $(\Delta y)$ would be zero
since $\Delta y=7-7=0.$  If we extend this to infinitely small changes, we
would still get $\d y=7-7=0.$  Of course, there is nothing special about
7, so we have the following rule

\noindent{\bf{}Differentiation Rule \#1:}  If $c$ is a constant, then $\d c=0.$
	
Differentiation Rule \#2 is also analogous to a familiar algebraic
rule, the distributive property.  Specifically, if $a$ is a constant,
then $\Delta(ax)=ax_1-ax_0=a(x_1-x_0)=a\Delta x.$  The same distributive popeerty
holds on an infinitesimal level.

\noindent{\bf{}Differentiation Rule \#2:}  If $a$ is a constant, then $\d(ax)=a\d x.$ 

The sum rule is really the associative and commutative properties.  On
the large scale, $\Delta(x+y)=(x_1+y_1)-(x_0+y_0)=(x_1-x_0)+(y_1-y_0)=\Delta x+\Delta y.$  The same idea works with infinitely small differences, so we
have

\noindent{\bf{}Differentiation Rule \#3:}  $\d(x+y)=\d x+\d y.$
Before we go on with the other differentiation rules, let's apply what
we already have to illustrate how differentials relate to slopes and
tangents.  Consider the line $y=4x+7.$ Applying our differential rules
to this we create the differential equation 
$$
\d{y}=\d(4x+7)=\d(4x)+\d(7)=4\d{x}+0=4\d{x}.
$$

Notice that this differential equation can be rewritten as
$\dfdx{y}{x}=4$ which is
the slope of the line.  This compares favorably with the notion that
the slope of a line is the change in $y$ divided by the change in $x.$
Whereas before, you recognized this slope as $\frac{\Delta{}y}{\Delta{}x}=4;$ we are now looking at
infinitesimal (instantaneous) changes producing the same slope.  This
makes sense since the slope of a line is constant.  The distinction
between finite changes $\Delta y,$ $\Delta x,$ and infinitesimal
changes $\d y,$ $\d x,$ comes through when
we try to compute the slope of (the tangent line to) a non-linear
function.  Specifically, notice that we did not distinguish between
finite and infinitesimal differences in the first three rules.  The
distinction between finite and infinitely small differences comes
across in the Product Rule (Rule \#4), $\d(xy)=x\d y+y\d x.$

\begin{wrapfigure}[]{r}{2in}
\captionsetup{labelformat=empty}
\centerline{\includegraphics*[height=1in,width=2in]{Figures/ProductRule1}}
\label{fig:}
\end{wrapfigure}
To see this rule, we will use an area model.  With this in mind,
consider the area $A$ of a rectangle with side lengths $x$ and $y.$ 


If we increase (or decrease) $x$ and $y$ by infinitesimal amounts $\d x,$ and $\d y$
respectively, then the area will change by an infinitely small amount
as well.
\begin{wrapfigure}[]{l}{2in}
\captionsetup{labelformat=empty}
\centerline{\includegraphics*[height=1in,width=2in]{Figures/ProductRule2}}
\label{fig:}
\end{wrapfigure}

The change in area $\d A$ is represented by the shaded $L$ shaped
region (officially called a gnomon).  Algebraically it is given by
\begin{align*}
\d A&=(x+\d x)-(y+\d y)-xy\\
    &=xy+x\d y+y\d y+\d x \d y-xy\\
    &=x\d y+y\d x+\d x \d y.
\end{align*}

The three terms which make up $\d A$ can be seen in the figure at the
left.

\begin{wrapfigure}[]{r}{2in}
\captionsetup{labelformat=empty}
\centerline{\includegraphics*[height=1in,width=2in]{Figures/gnomon1}}
\label{fig:}
\end{wrapfigure}

This is where we deviate from the world of finite differences and
utilize the fact that we are working with infinitesimals.  Since $\d x$
and $\d y$ are both infinitely small, then their product $\d x\d y$ is
infinitely small compared to either of them, so we will conveniently
ignore it.  This leaves us with

\noindent{\bf{}Differentiation Rule \#4 (The Product Rule):}   
$$
\d (xy)=x \d y+y \d x.
$$

If you are suspicious of the sleight of hand in ignoring the term $\d x$ 
$\d y,$ then know that you are in good company.  A number of
contemporaries of Newton and Leibniz were suspicious of their
reasoning and raised objections to it.  Leibniz himself struggled with
the ``correct'' product rule, and did not provide a justification in his
original calculus paper.  He wrote ``The demonstration of this will be
easy to one who is experienced in these matters $\ldots$'' Again, if these
foundational matters bother you, be patient! We will come back to
them.  (Perhaps, you should also declare a math major.)  For now, we
will justify these rules as Leibniz did by noticing that they provide
correct results.  For example, consider the following circle.

\begin{wrapfigure}[]{r}{1.5in}
\captionsetup{labelformat=empty}
\centerline{\includegraphics*[height=1.5in,width=1.5in]{Figures/CircleDifferential}}
\label{fig:CircleDifferential}
\end{wrapfigure}
Next consider the following slightly more complex geometric
problem. let $A$ be the area of a circle with radius $r,$ and suppose
that $r$ is changing in time, as in the diagram at the right.  We wish
to compute $\d A,$ the differential of the area of a circle with
respect to its radius, $r.$

Clearly this is just the area of the ring
between the inner and outer circles in the figure.  Since the width of
this ring is the infinitesimal $\d{r}$ it seems reasonable to
conjecture that the differential of the area will be the circumference
of the circle with radius $r$ times the differential $\d{r}.$ 

That is, it should be clear that
\begin{equation*}
\d A = 2\pi r\d r.%\label{eq:CircDiffAnal}
\end{equation*}
\begin{wrapfigure}[]{l}{1.5in}
\captionsetup{labelformat=empty}
\centerline{\includegraphics*[height=.4in,width=1.5in]{Figures/CutCircleDiff}}
\label{fig:}
\end{wrapfigure}

To see this we simply cut the ring and flatten it out to get the
rectangle at the left.
%\centerline{\includegraphics*[height=.75in,width=2in]{Figures/CutCircleDiff}}
In that case
$$
\d A = 2\pi r\d r
$$
as we've indicated.
Since our reasoning here is  shaky\footnote{ If you are paying attention you see that
  the claim we just made is incorrect. When we flatten out the ring
  the difference in the length of the inner and outer circumferences
  guarantees that we do \underline{not} get a rectangle. We get a
  trapezoid. However for very small values of $\Delta r$ the inner and
  outer circumferences get closer and closer together so that when
  $\Delta r$ becomes the infinitesimal $\d{r}$ the inner and outer
  circumferences become equal and our claim is at least approximately
  correct. That will do for now.

  If this kind of intuitive reasoning makes sense to you,
  good. Continue to use it. It will help.

  If it does not, if you don't worry about it. We will return to this in a
  much more rigorous fashion later.} at best we'll call this a
conjecture:
\begin{myconjecture}
  If $A=\pi r^2$ then $\d A = 2\pi r.$
\end{myconjecture}


Let's try to compute $\d A$ analytically to see if our conjecture is
correct.
\begin{align}
  \d A &= \left\{\text{area of outer circle}\right\} -\left\{\text{area of inner circle}\right\}\nonumber\\
     &= \left\{\pi(r+\d r)^2\right\} - \left\{\pi r^2\right\}\nonumber\\
     &= \pi(r^2+2r\d r + (\d r)^2 -r^2)\nonumber\\
\d A &=2\pi r\d r+ \pi(\d r)^2.\label{eq:CircDiffAnal}
\end{align}

Uh, oh. 

This \emph{almost} agrees with our conjecture but we seem to have the
extra term: $\pi(\d r)^2.$ 

% Oh no! This is a problem!

Clearly we need for $(\d r)^2$ to be zero if our geometric argument is
to be consistent with our analytical argument in
equation~\ref{eq:CircDiffAnal}.

But, just as clearly, $(\d r)^2$ is \underline{not} zero, since it is
the square of the (non-zero) increment of the radius. 

% This is bad! Very bad.

This is indeed a puzzle, but it is not a puzzle we need to solve right
now so we will deal with it pragmatically. That is, we will accept the geometric
argument as correct and leave for a later time the resolution of the
inconsistency between the geometric and the analytical arguments.

As a practical matter this means that we will simply ignore the
term $(\d r)^2$  and assert that 
$$
\d A = 2\pi r\d r
$$
or, in its more useful form
\begin{equation}
\dfdx{A}{r} = 2\pi r.
\label{eq:DiffCirc}
\end{equation}

Actually, once we have the product rule, Rule \#5 follows
pretty readily.

Notice that if we apply the Product Rule (\#4) to the product $x\cdot
x = x^2$ we get
\begin{align*}
  \d (x^2)&=\d (x\cdot x)\\
          &=x\d x+x\d x\\
          &=2x\d x.
\end{align*}
If we extend this idea, we get
\begin{align*}
  \d (x^3)&=\d (x^2\cdot x)\\
          &=x^2\d x+x\d (x^2)\\
          &=x^2\d x+x(2x\d x)\\
          &=3x^2\d x.
\end{align*}

Rule \#4, (The Power Rule) is proved as in the following problem.
\begin{embeddedproblem}{}\ \\
  \begin{enumerate}
  \item Use the above idea to show that $\d (x^4)=4x^3\d x,$   $\d
    (x^5)=5x^4\d x,$ and  $\d (x^6)=6x^5\d x.$
  \item Show that if $\d (x^{99})=99x^{98}\d x$ then $\d
    (x^{100})=100x^{99}\d x.$
  \item If you familiar with induction, provide an induction proof
    of The Power Rule: $\d (x^n)=nx^{n-1}\d x$ \\
    for any positive integer $n.$  [Actually,
    it works for $n=0$ as well, since $\d (x^0)=\d (1)=0=0 x^{0-1}\d x.$ ]
  \end{enumerate}
\end{embeddedproblem}

You might ask, ``What do these rules have to do with tangents and
optimization, as Leibniz said?''  Good question!  Our gateway for this
comes in the form of an old friend, the slope of a line. 



% \begin{enumerate}
% \item $\dfdx{ (\sin x)}{x} = \cos x $
% \item $\dfdx{ (\cos x)}{x} = -\sin x $
% \item $\dfdx{ (\tan x)}{x} = \sec^2 x $
% \item $\dfdx{ (\cot x)}{x} = -\csc^2 x $
% \item $\dfdx{ (\sec x)}{x} = \sec x\tan x $
% \item $\dfdx{ (\csc x)}{x} = -\csc x\cot x $
% \end{enumerate}

% \begin{enumerate}
% \item $\displaystyle\dfdx{ (\inverse\sin x)}{x} =  \frac{1}{\sqrt{1-x^2}} $
% \item $\displaystyle\dfdx{ (\inverse\cos x)}{x} = \frac{-1}{\sqrt{1-x^2}}  $
% \item $\displaystyle\dfdx{ (\inverse\tan x)}{x} = \frac{1}{1+ x^2} $
% \item $\displaystyle\dfdx{ (\inverse\cot x)}{x} = \frac{-1}{1+ x^2}$
% \item $\displaystyle\dfdx{ (\inverse\sec x)}{x} = \frac{1}{\abs{x}\sqrt{x^2-1}} $
% \item $\displaystyle\dfdx{ (\inverse\csc x)}{x} = \frac{-1}{\abs{x}\sqrt{x^2-1}} $
% \end{enumerate}


\section{Differentials and the Slope of a Line}
\label{sec:diff-slope-line}

Now that we have the idea of infinitesimal differentials, the familiar
idea that the slope of a line is the change in y divided by the change
in $x$ can be extended to more general curves.  To find the slope of a
curve, notice that the slope will typically change from point to point
on the curve.  For example, consider the parabola $y=x^2:$
\begin{wrapfigure}[]{r}{2in}
\captionsetup{labelformat=empty}
\centerline{\includegraphics*[height=1in,width=2in]{Figures/parabola1}}
\label{fig:}
\end{wrapfigure}

Notice that on the left side of the $y$ axis, the slope appears to be
negative while on the right side, it appears to be positive.
Furthermore, as you move away from the origin, the curve appears to be
steeper (so the slope would be larger in magnitude).  How does one
compute this slope? 

First, since we are good at dealing with lines, let's look at the
tangent line to the parabola at the point $(x,x^2).$  If you recall
from the previous section, Fermat determined that the slope of this
tangent line is given by $2x.$  This should ring a bell.  We just
utilized Leibniz's rules to show that $\d (x^2)=2x\d x.$  This would
suggest that if $y=x^2,$ then $\d y/\d x=\d (x^2)/\d x=2x\d x/\d x=2x.$  Does this
really represent the slope as Fermat had said?  Basically, $\d x$ is an
infinitely small change in $x,$ while $\d y$ is the infinitely small change
in $y$ along the curve.  Would the quotient of these changes give the
slope of the tangent line?  Since we are talking about infinitely
small changes, let's draw a tangent line to the curve at the generic
point $(x,x^2)$ and zoom in to see what is happening.

\centerline{\includegraphics*[height=4in,width=4in]{Figures/parabola-zoom}}

These images seem to suggest that as we zoom in, it is getting harder
to distinguish between the curve and the tangent line.  If we could
zoom in infinitely much, then this would suggest that the curve and
the tangent line would be indistinguishable, so it would make sense
that the slope of the curve $\dfdx{y}{x}$ is also the slope of the tangent
line.  In the previous chapter, you were asked to use Fermat's method
to determine the slope of the tangent line to the curve $y=x^3$ at the
point $(x,x^3).$  If we utilize Leibniz' rules, then we would have
$\dfdx{y}{x}=\frac{\d(x^3 )}{\d x}=\frac{3x^2 \d x}{\d x}=3x^2.$  Does this match your answer from
before?  Maybe this really is a ``New Method for Tangents.''

\begin{embeddedproblem}{}
  We have all heard that the tangent line to a circle is perpendicular
  to the radius.  Of course, this can be proven geometrically (You may
  have seen such a proof).  We even used this fact in Descartes'
  Method of Normals.  Find the slope $\dfdx{y}{x}$ of the tangent line to the
  circle $x^2+y^2=r^2$ at the point $(x,y)$ and verify that the radius and
  the tangent line are in fact perpendicular.  
\end{embeddedproblem}

  
Solving problems such as the previous ones show that these Calculus
rules seem to work, even if the foundations are questionable.  But
wait!  What about the curve $y=\sqrt{2x}?$  We used Descartes' Method of
Normals to find the slope of the tangent line to this curve at the
point $(2,2).$  Can Leibniz do this problem as well with his ``New
Method?''  First notice that $y=\sqrt{2}\cdot x^(1/2).$  Since $\sqrt{2}$ is a constant,
which we know how to handle, it becomes a matter of differentiating
$x^(1/2).$  Would Leibniz' Power rule work for this as well?  Let's
see.  If we wanted to differentiate $y=x^(1/2),$ we could rewrite this
as $y^2=x.$  Differentiating, we obtain

\begin{align*}
  \d(y^2)&=\d x\\
  2y \d y&=\d x\\
  \d y&=1/2y \d x\\
      &=1/2x^{1/2} \d x\\
      &=1/2 x^{-1/2} \d x\\
      &=1/2  x^(1/2-1) \d x\\
 \intertext{  This that the slope of the tangent line to y=\sqrt{2}x at (2,2) is given
   by}
   \left. \dfdx{y}{x}\right|_{x=2}=\left.\sqrt{2} \left(\frac12
       x^{-1/2}\right)\right|_{x=2}=1/2.
\end{align*}
This should match what you obtained before.

\begin{embeddedproblem}{}
  Suppose $y=x^{p/q},$ where $p,$ and $q$ are positive integers.  Mimic the
  derivation above to show that
$$
\d x^{p/q}=p/q x^{p/q-1} \d x.
$$
Thus, as Leibniz said, ``$\ldots$ Which is Impeded Neither by Fractional nor Irrational Quantities, $\ldots$''
\end{embeddedproblem}

What about negative exponents?  For example, $y=1/x$ represents an
equilateral hyperbola whose axis is the line $y=x.$  Can we use the
rules of calculus to find the slope of the tangent line to this curve
at the point $(x,y)?$  If $y=1/x$ then $xy=1.$  Applying our current rules,
we have
\begin{align*}
  \d(xy)=\d(1)\\
  x\d{y}+y\d{x}&=0\\
  x\d{y}&=-y\d{x}\\
  \dfdx{y}{x}&=-y/x\\
             &=-1/x^2\\
             & =-1x^{-1-1},
\end{align*}
% \begin{wrapfigure}[]{r}{2in}
% \captionsetup{labelformat=empty}
% \centerline{\includegraphics*[height=2in,width=2in]{Figures/Hyperbola2}}
% \label{fig:}
% \end{wrapfigure}
so the power rule works for $n=-1$ as well.

Before we go any further with this, let's see if the answer makes
sense with the things we know about the curve.  Here is a graph of the
curve $y=1/x.$


Notice on the graph that the tangent line at $(x,y)$ is parallel to the
tangent line at $(-x,-y)$ and always has negative slope.  Also, given
the symmetry of the graph around the line $y=x$ that the slope of the
tangent line at $(x,1/x)$ should be the reciprocal of the slope of the
tangent line at $(1/x,x).$

\begin{embeddedproblem}{}
  Verify the observations made in the previous paragraph by examining $\dfdx{x^{-1}}{x}.$
\end{embeddedproblem}

 \begin{embeddedproblem}{}
   Let $r$ be positive rational number.  Mimic the idea in differentiating $y=x^{-1}$  to show that if $y=x^{-r}=1/x^r,$ then
 $$
 \d y=-rx^{-r-1} \d x
 $$
 so that the power rule holds for any rational number, positive or negative.
 \end{embeddedproblem}

 Of course, all the differentiation rules we've presented work in
 concert, not separately.  They must be applied whenever the situation
 arises.  For example
 \begin{align*}
   \d\left(\frac{2-3x^2}{\sqrt{x}}\right)&=\d(\left(\frac{2}{x^{1/2}}
     -\frac{3x^2}{x^{1/2}}\right)\\
   &=\d\left(2x^{-1/2}-3x^{3/2}\right)\\
   &=2\d\left(x^{-1/2}\right)-3\d(\left(x^{-3/2}\right)\\
   &=2\left(-\frac12 x^{-3/2}\right)\d{x}-3\left(-3/2 x^{-5/2}\right)\d{x}\\
   &=-\left(x^{-3/2}+\frac92 x^{-5/2}\right)\d{x}.
 \end{align*}

 What about $\d\left(\frac{\sqrt{x}}{2-3x^2}\right)?$ It is not so
 easy to separate as above and prompts us to derive the quotient rule.

\noindent{\bf{}Differentiation Rule \#5 (The Quotient Rule):}
$\displaystyle \d\left(\frac{x}{y}\right) = \frac{y\d{x}-x\d{y}}{y^2}$

\begin{embeddedproblem}{}
  Derive the Quotient Rule. (Hint: $\frac{x}{y}=x\inverse{y}$
\end{embeddedproblem}

Applying the quotient rule to the above, we have
\begin{align*}
  \d\left(\frac{\sqrt{x}}{2-3x^2}\right) &=\frac{(2-3x^2)\d\left(x^{1/2}\right)-x^{1/2} \d(2-3x^2)}{(2-3x^2)^2} \\
    &=\frac{(2-3x^2)(\frac12 x^{-1/2} )-x^{1/2} (-6x))}{(2-3x^2 )^2}.
  \end{align*}

  By the way, don't simplify things unless you have some reason to do
  so.  In this problem, there is no reason to do so.  If this was in
  the context of some larger problem, then it might force you to
  simplify, but it is not, so don't.  But this assumes that you could
  if you had to.

  \centerline{\bf\Large Put in Practice problems finding equations of}
  \centerline{\bf\Large  tangent lines to curves.  Can include tangents to all the conics}
  \centerline{\bf\Large  and graph these to see if the answers are correct.}

\subsection{Some Applications}
As you might have noticed, the differentiation rules were written in
differential form while in many of the problems you were asked to find
slopes, which involved the quotients of differentials.  You may or may
not think that this is a big deal, but it comes down to the fact that
in verifying the differentiation rules, it is often easier to consider
``differential'' equations of the form $\d y=2x \d x$ whereas for rates of
change (as in slope) it makes more sense to consider quotients like
$\dfdx{y}{x}=2x.$  We don't want to make a mountain out of a molehill at this
point (wait until the foundations), but there is a slight difference
in the two notions.  In the first equation, we are equating two
infinitely small changes.  Their quotient in the second equation is a
finite quantity.  In fact, there is a name for the finite quantity
$\dfdx{y}{x}.$  It is called the derivative of $y$ with respect to $x$ and it
represents the instantaneous rate of change of $y$ with respect to $x.$
Whether you want to write $\d y=2x\d x$ or $\dfdx{y}{x}=2x$ will depend on the
circumstances and it is really up to you.  You will gain more feel for
what is better in a given application with more experience.  What you
{\bf{}cannot} do is to write $\d y=2x.$  This doesn't make sense as $\d y$ is
infinitesimal and $2x$ is finite.  

\begin{wrapfigure}[]{r}{4in}
\captionsetup{labelformat=empty}
\centerline{\includegraphics*[height=2in,width=4in]{Figures/MackinacBridge1}}
\caption{Mackinac Bridge, Michigan}
\label{fig:}
\end{wrapfigure}
To start exploiting the calculus rules we've developed, consider the
shape of the center cable on a stable suspension bridge.

The shape appears to be parabolic, but how can we be sure?  To examine
this, we will assume that the weight of the cable is negligible with
respect to the weight of the deck (which we will assume is horizontal,
for simplicity).  Suppose the deck is made of a material that weighs $W$
newtons per meter.  Since the curve is symmetric about its lowest
point, let's look at half of the cable.  Consider a point $P$ on the
cable and all of the forces involved. 
\begin{wrapfigure}[]{l}{3in}
\captionsetup{labelformat=empty}
\centerline{\includegraphics*[height=2in,width=3in]{Figures/MackinacBridge2}}
\label{fig:}
\end{wrapfigure}

Here $H$ represents the (magnitude of) the horizontal tension on the
cable and is constant throughout the cable (otherwise the cable would
move sideways), $Wx$ represents the weight of the deck from the lowest
point at $x=0$ to the point $P,$ (recall that we are ignoring the weight
of the cable).  $T$ represents (the magnitude of) the tension on the
cable at point $P$ which is tangent to the cable at that point.  For the
cable to be stable, the vertical component of $T$ must equal the weight
of the deck and the horizontal component must equal $H.$

\begin{embeddedproblem}{}
% \begin{wrapfigure}[]{r}{2in}
% \captionsetup{labelformat=empty}
\centerline{\includegraphics*[height=1in,width=2in]{Figures/MackinacBridge3}}
% \label{fig:}
% \end{wrapfigure}
    \begin{enumerate}[label={\bf{}(\alph*)}]
    \item Use the above analysis and  schematic of the
      forces involved
to show that the curve which
      represents the cable must satisfy the differential equation
       $$\dfdx{y}{x}=\frac{Wx}{H}.$$
    \item Show that the parabola  $y=\frac{WH}{2} x^2+b$  satisfies this differential equation
    \end{enumerate}
\end{embeddedproblem}

\begin{wrapfigure}[]{r}{2in}
\captionsetup{labelformat=empty}
\centerline{\includegraphics*[height=1in,width=2in]{Figures/HangingChain1}}
\label{fig:}
\end{wrapfigure}
As long as we are talking about hanging cables, consider the following
related problem.

Galileo Galilei (1564-1642) mentioned in his {\it{}Discoursi} of $1638$ that
the shape of a chain hanging under its own weight was a parabola.
This reflected the common belief at the time.  Let's explore this
belief.


We will apply a similar analysis as we did above concerning the cable
on a suspension bridge.  The difference is that there is no deck, and
the vertical force will be the weight of the chain (cable) from the
lowest point to $P.$

Before we send you off to attack this problem, we need to introduce a
topic that will take on more significance later.  For now, it is more
notational.  If we have $y=x^3+5x-7,$ then differentiating, we have
$$
\dfdx{y}{x}=3x^2+5.
$$
Differentiating again, we have
$$
\dfdx{}{x}\left(\dfdx{y}{x}\right)=\dfdx{(3x^2+5)}{x}=6x.
$$

This ``second'' derivative is denoted by 
$$
\dfdx{}{x}\left(\dfdx{y}{x}\right) =
\dfdx{}{x}\left(\dfdx{}{x}(y)\right) = \dfdxn{y}{x}{2}.
$$

Of course, we could keep going and compute the ``third derivative''
$\dfdxn{}{x}{3} (x^3+5x-7)
=\dfdx{}{x}\left(\dfdxn{}{x}(x^3+5x-7)\right)=\dfdx{}{x} (6x)=6. $ 

\begin{embeddedproblem}{}
    \begin{enumerate}[label={\bf{}(\alph*)}]
    \item If we let $w$ represent the weight density of the chain in
      newtons per meter and $s$ represent the length of the chain from
      the lowest point to $P$ then show that the curve represented by
      the chain must satisfy the differential equation
$$
\dfdx{y}{x}=ws/H
$$
and use this to show that the curve must satisfy the differential equation
$$
\dfdxn{y}{x}{2}=\frac{w}{H}  \dfdx{s}{x}=w/H
\sqrt{1+\left(\dfdx{y}{x}\right)^2 }.
$$
\item Show that the general parabola $y=ax^2+bx+c$ does \underline{not} satisfy this
  differential equation (and so Galileo was mistaken!)  The curve
  which does satisfy it is called a catenary, which is not too
  illuminating as catenary is derived from the Latin word {\it{}catena} which
  means chain.
    \end{enumerate}
\end{embeddedproblem}


\begin{wrapfigure}[]{l}{2in}
\captionsetup{labelformat=empty}
\centerline{\includegraphics*[height=3in,width=2in]{Figures/RefractiveAndReflectiveTelescopes}}
\label{fig:}
\end{wrapfigure}
So, the previous problem says that the catenary is not a parabola.
Clearly, we need to extend our differentiation rules to more
functions.  Before we do that, let's look at another application which
we can handle at this point.  

To set the stage on this application, you may have learned that there
are two types of telescopes, refractive and reflective.

Refractive telescopes (top image) utilizes two lenses to refract
light.  Reflective telescopes (bottom image) utilizes a parabolic
objective (primary) mirror to reflect light toward the focus of the
parabola (remember Roberval) where the flat (secondary) mirror
reflects it to the eyepiece.  Large telescopes tend to be reflective
telescopes as they really don't need the tube (which actually only
holds the two lenses in the refractive telescope in place).  As an
example, consider the Large Binocular Telescope at Mt. Graham, AZ. 

Each parabolic reflector mirror is $8.4$ meters in diameter.  The
following image gives you an idea of the size of one of these mirrors.
% \begin{wrapfigure}[]{r}{2in}
% \captionsetup{labelformat=empty}
% \centerline{\includegraphics*[height=1in,width=2in]{Figures/SpinCasting1}}
% \label{fig:}
% \end{wrapfigure}

How does one make such a large parabolic mirror?  The Steward
Observatory Mirror Laboratory ``spin casts'' these large mirrors.  That
is, they load borosilicate glass into a revolving oven.  [See images
below.]   
% \begin{wrapfigure}[]{r}{4in}
% \captionsetup{labelformat=empty}
% \centerline{\includegraphics*[height=2in,width=2in]{Figures/MountGraham}
% \includegraphics*[height=1in,width=2in]{Figures/SpinCasting1}}
% \label{fig:}
% \end{wrapfigure}

\begin{wrapfigure}[]{r}{2in}
\captionsetup{labelformat=empty}
\centerline{\includegraphics*[height=1in,width=2in]{Figures/SpinCasting2}}
\label{fig:}
\end{wrapfigure}
As the glass melts and spins, the middle goes down and the sides go up
as any liquid would.  What is interesting is that the surface of the
revolving liquid will form a parabola.  We actually have the tools to
show this, as the following problem explores.    

\begin{embeddedproblem}{}
  Consider a cylinder partially filled with a liquid which is rotating
  with an angular velocity of $\omega$ radians per second.  Of course, in
  such a setting the middle will go down and the sides will go up as
  in the diagram below.
\begin{wrapfigure}[]{r}{2in}
\captionsetup{labelformat=empty}
\centerline{\includegraphics*[height=1in,width=2in]{Figures/SpinCasting3}}
\label{fig:}
\end{wrapfigure}
\end{embeddedproblem}
Our task is to find the shape of the surface of the liquid (that is
the shape of the above curve which will be rotated around the $y$ axis.
If we consider a point mass $m$ at point $(x,y)$ on the surface, then the
force from the liquid which keeps that point elevated is perpendicular
to the surface.  (Think about a hose with a hole in it.  The water
sprays out in a stream perpendicular to the hose.)   


If we separate that force into its vertical and horizontal components,
then we get the following diagram.  
% \begin{wrapfigure}[]{r}{2in}
% \captionsetup{labelformat=empty}
% \centerline{\includegraphics*[height=1in,width=2in]{Figures/SpinCasting3}}
% \label{fig:}
% \end{wrapfigure}
The (magnitude of) the vertical component is $mg,$ where $g$ is the
acceleration due to gravity.  This is the force needed to counter the
weight of the particle.  The horizontal force has a magnitude of
$mx\omega^2.$ This is the centripetal force which is keeping the
particle moving in a circle around the axis.  [We will see why this is
the centripetal force later, but for now, take it on faith.]
    \begin{enumerate}[label={\bf{}(\alph*)}]
    \item Use the above information to show that the curve must
      satisfy the differential equation
      $$
      \dfdx{y}{x}=\frac{\omega^2}{g} x
      $$
      and that the curve satisfying this must be a parabola.
    \item What will happen to the parabola as $\omega$ increases?  Does this
      make sense with the physical problem?
    \end{enumerate}
    
Notice that in the problem, we furnished a formula for the centripetal
force.  Where did this come from?  Again, we need to extend our
differentiation rules to beyond polynomials.

Recall that Fermat was able to deduce Snell's Law of Refraction by
minimizing the amount of time it took for light to travel from point $A$
in air to point $B$ in water\pageref{FermatSnell}.  Fermat was able to do this before
the advent of calculus, but in his paper, Leibniz applied his
differential calculus to solve the problem.  Recall
Problem\ref{FermatSnell-mbed} on page\pageref{FermatSnell} from
Chapter\ref{cha:science-before-calc}.
\begin{embeddedproblem}{}
\centerline{\includegraphics*[height=2in,width=4in]{Figures/Refraction3}}
Consider the following labeling in the above diagram.
%   \begin{wrapfigure}[]{r}{2in}
% \captionsetup{labelformat=empty}
% \centerline{\includegraphics*[height=1in,width=2in]{Figures/Refraction3}}
% \label{fig:}
% \end{wrapfigure}
 Find an expression for the time $T$ traveled along the path from $A$
 to $B$ in terms of the variable $x$ and the constants $a,$ $b,$ $c,$ $v_1,$
 and $v_2.$

 The answer to this problem is
 $$
 T=T(x)=\frac{\sqrt{a^2+x^2}}{v_1} +\frac{\sqrt{b^2+(c-x)^2}}{v_2}. 
 $$

As with Fermat, Leibniz knew that if we graphed this curve, then at
the minimum there would be a horizontal tangent line.  This says that
we should set $\dfdx{T}{x}=0$ to find this minimum.  Differentiating, we get
$$
\dfdx{T}{x}=\frac{1}{v_1}   \dfdx{\sqrt{a^2+x^2}}{dx}+\frac{1}{v_2}
\dfdx{\sqrt{b^2+(c-x)^2}}{x}.
$$

We've reached an impasse.  We don't yet know how to differentiate the
quantities involved.  This is where Rule \#7 - The Chain Rule comes in.




\end{embeddedproblem}
%   \centerline{-----------------------------------------------------------}
%   We're willing to bet that when you saw the word ``slope'' in the
%   title of this section your eyes rolled back in your head and you
%   thought, ``Oh my god! Not slopes \emph{again!?}'' In our experience
%   when students reach their first course in Calculus they have been
%   ``sloped'' to death. How hard can it be, after all? Slope is ``how
%   far up'' divided by ``how far over.'' Nothing to it. Why is such a
%   lot of fuss made over it?

% Why indeed?

% The fuss is made because the concept of the slope of a straight line
% is fundamental and definitive for much of modern mathematics, science
% and technology. Really. It is \emph{important.}

% Nevertheless, your initial reaction is a reasonable one. Once the
% slope of a line is understood it isn't very difficult to deal with and
% it does feel like entirely too much emphasis is put on it in algebra
% courses. But we will ask your indulgence while we introduce the slope
% concept one last time. 

% In order not to bore you by repeating formulations that you've
% probably seen too many times already we'll take a different approach
% to the idea of slope\footnote{When we are done, of course, we will
%   still just have the slope. We're just going to come at it from a
%   different direction.}.


% Suppose the equation of a line is given by:
% \[3x+2y+5=0\] and we need to find the slope of this line. One way to
% do this is to find two points on the line -- \((-1, -1)\) and \((3,
% -7)\) will work -- compute $\Delta y,$ the change in $y,$  the change
% in $x,$ $\Delta x,$ and form the ratio: \(\frac{\Delta y}{\Delta x}.\)

% But this is how you've always computed the slope of a line and we said
% we'd come at this differently. So let's suppose the line passes through
% the points \((x_0, y_0)\) and \((x_1, y_1)\) \emph{without} specifying any particular 
% points. After all, the slope is the same no matter where we are on the
% line, so it shouldn't be necessary to use some \emph{particular}
% points to find it.

% Since \((x_0, y_0)\)  is on the line \(3x+2y+5=0\) we see that
% \begin{equation}
%   \label{eq:slope1}
%   3x_0+2y_0+5=0.
% \end{equation}
% Also since \((x_1, y_1)\)  is on the line \(3x+2y+5=0\) we see that
% \begin{equation}
%   \label{eq:slope2}
%   3x_1+2y_1+5=0.
% \end{equation}
% Subtracting equation~\ref{eq:slope1} from equation~\ref{eq:slope2}
% gives
% \begin{align}
% 3x_1+2y_1+5 -   (3x_0+2y_0+5) &=0\nonumber\\
% \intertext{or}
% 3(\underbrace{x_1-x_0}_{=\Delta x})+2(\underbrace{y_1-y_0}_{=\Delta y}) &=0\\
% \intertext{Of course you recognize the expressions \(x_1-x_0\) and \(y_1-y_0.\)
% Respectively, they are the change in $x,$ $\Delta x,$ and the change
%   in $y,$ $\Delta y.$   Changing the
%   notation accordingly, our formula becomes:} 
% \label{eq:SumOfDiffs}3\Delta x + 2\Delta y &=0\\
% \intertext{or}
%   2\Delta y &= -3\Delta x\nonumber\\
% \intertext{so that the slope is given by }
% \frac{\Delta y}{\Delta x} &= -\frac32.\nonumber
% \end{align}

% Some general observations are in order: 
% \begin{enumerate}
% \item First, this discussion actually proves that the slope of a line
%   does not depend on where we are on the line since we didn't need to
%   know $x_0, x_1, y_0,$ or $y_1$ to deduce the slope of our line.
% \item Second if we re-write the last formula slightly as
% \begin{equation}
%   \label{eq:differential1}
%   \Delta y =-\frac32\Delta x
% \end{equation}
% it is clear that changing $x$ a little bit forces a small change in
% $y.$ But (and this is the crucial point) the change in $y$
% \emph{depends on but is not the same as} the change in $x.$ In this
% problem $\Delta y$ is one and a half times larger than $\Delta x$ and
% in the negative direction. That is, if $x$ is moved some fixed amount
% in the positive direction $y$ will move one and a half times as far in
% the negative direction. 
% \item And third, since we have a \emph{straight} line we can think of
% $\Delta x$ and $\Delta y$ as very large, very small, or anything in
% between. 
% \end{enumerate}


% When we move beyond straight lines we will be particularly interested
% in the situation when they are {{\Huge very,} {\huge very,} {\Large
%     very,} {\large very} {\small very,} {\tiny small}.}  In fact, we
% will be considering the case when \(\Delta x\)
% is \emph{infinitely} small, or \emph{infinitesimal.} This is a rather
% difficult thing to conceive of but we won't let that stop us since, for
% now, we are just trying to learn the computational rules we will need
% (how to ``move the
% pebbles'')\index{pebbles!move~the}. However, in order to
% distinguish the finite case from the infinitesimal case we will
% reserve the notation \(\Delta x\)
% for the finite case and use the following new notation in the
% infinitesimal case.
% \begin{mynotation}{Differential}
%   \label{notation:differential}
%   If $x$ is a variable quantity, then the expression $\d x$ is called   ``the differential of
%   \(x\).'' We will think of it as   ``an infinitely small change in
%   $x.$''  
% \end{mynotation}
% We will explore the deeper meaning of differentials in later chapters. For now,
% when we use $\d x$ we are explicitly limiting our attention to infinitely
% small changes in $x.$ In that case equation~\ref{eq:differential1} can
% be rewritten as:
% \begin{equation}
%   \label{eq:differential2}
%   \d y =-\frac32\d x.
% \end{equation}
% Notice that if we divide by $\d x$ we get $\dfdx{y}{x} = \frac{-3}{2}$
% which simply says that the slope of our line is $\frac{-3}{2}.$
% Evidently with \underline{straight} lines the
% \underline{infinitesimal ratio} $\dfdx{y}{x}$ is the same as the
% \underline{finite ratio} $\frac{\Delta y}{\Delta y}.$ When the line
% is curved this will not necessarily be true.
% \begin{embeddedproblem}
% {
%   \begin{description}
%   \item[(a)] 
%   If we'd started with another equation, say
%   \[
%   8x-4y+50=0
%   \]
%   Then we'd have found that
%   \[
%   \d y = 2\d x.
%   \]
%   Check it and see.
% \item[(b)] If we'd started with \(12x-3y+17=0\) we'd have gotten $\d y
%   = \frac14\d x.$ Do you see a pattern? What is it?
% \end{description}
% }
% \end{embeddedproblem}

% Another way to interpret an equation like $\d y  = 2\d x$ is to observe that if the quantity
% represented by $x$ is changing at some rate, say $60$ miles per hour
% then the quantity represented by $y$ is changing twice as fast, or
% $120$ miles per hour. That is, we say that the rate of change of $y$
% \underline{with respect to}
% \index{``with respect to''}
%  $x$ is $2.$ This is also summed up in
% our differential notation by dividing through by \(\d x\): 
% \begin{equation}
% \label{eq:differential3}
% \dfdx{y}{x} = 2.
% \end{equation}
% Evidently we can think of the expression $\dfdx{y}{x}$ as the rate of
% change of \(y\) ``with respect to'' \(x.\) In this case it is 2. From
% equation~\ref{eq:differential2} we see that in our first problem the
% rate of change of $y$ with respect to $x$ was:
% \[\dfdx{y}{x} = -\frac32.\]

% Now suppose we have three variable quantities, $x, y,$ and $t$ and
% that  we have determined that \(\d y = 7\d x.\) If we divide this by
% $\d t$ we get \[\dfdx{y}{t} =
% 7\dfdx{x}{t}.\] 

% The question is, how should we interpret this expression? What does it
% mean?

% From our interpretation of equation~\ref{eq:differential3} it is clear
% that the expression \(\dfdx{y}{t}\) represents the rate of change of
% $y$ with respect to $t$ and that \(\dfdx{x}{t}\) represents the rate
% of change of $x$ with respect to $t.$ Thus the only possible
% interpretation of
%  \[
% \dfdx{y}{t} = 7\dfdx{x}{t}
% \] 
% seems to be that the rate of change of $y$
% with respect to $t$ is seven times the rate of change of $x$ with
% respect to $t.$

% It helps to imagine these things more concretely. Suppose two cars are
% moving away from the same starting point so that the distance of the
% second, let's call it  $y,$  is always seven times the distance of the
% first from the starting point. Then clearly
% \begin{align}
%   y&=7x\label{eq:differential4}\\
% \intertext{and so}
%   \d y&=7\d x.\label{eq:differential5}
% \end{align}

% Next let $t$ represent the elapsed time for both cars. Since $\d y$ is
% the change in the distance $y$ we can find the velocity of the second car by
% dividing by the change in time, $\d t.$ That is, the velocity of the
% second car is $\dfdx{y}{t},$ and similarly the velocity of the first
% car is $\dfdx{x}{t}.$ If we divide both sides of
% equation~\ref{eq:differential4} by $\d t$ we get
% \begin{equation}
% \dfdx{y}{t} = 7\dfdx{x}{t}\label{eq:DifferentialRatio}
% \end{equation}
% which says simply that the velocity of two cars is related. In fact,
% it says that the second car is moving seven times faster than the
% first so, for example, if car \(x\) is moving at 10 miles per hour
% then car \(y\) is moving at 70 miles per hour. 

% % We will find that the ratio of differentials, as in
% % equation~\ref{eq:DifferentialRatio}, is generally more useful than the
% % differentials themselves, as in equation~\ref{eq:differential5}. Thus
% % we will need the following definition:
% % \begin{definition}
% % \label{def:derivative1}
% %  The  ratio of differentials \(\dfdx{y}{x}\) is called the
% %   ``derivative'' of \(y\) with respect to \(x\). 
% %   The derivative represents the rate of change of \(y\) with respect to \(x\).
% % \end{definition}

% Two other facts have emerged from our investigations in
% this section that deserve to be displayed as independent
% theorems. In our derivation of equation~\ref{eq:differential5} there is
% clearly nothing special about the number 7. Thus we have the following
% theorem.

% \begin{mytheorem}
%   \label{thm:LinearityConstantMultiple}
%   The differential of a constant times a variable is that constant
%   times the differential of the the variable. 

%   Symbolically, if $c$ is constant and $x$ is a variable then
% \[
% \d{(cx)} = c\d x.
% \]
% \end{mytheorem}

% It is also clear from equation~\ref{eq:SumOfDiffs} that
% \[
% \d{(3x+2y)} = 3\d x+2\d y,
% \]
% and as before there is nothing special about the numbers 3 and 2 so the
% following theorem is true.
% \begin{mytheorem}
%   \label{thm:LinearitySum}
%   If $c_0$ and $c_1$ are constants and $x$ and $y$ are variables then
% \[
% \d{(c_0x+c_1y)} = c_0\d x+c_1\d y.
% \]
% \end{mytheorem}

% \subsection{Computing Differentials: The Easy Results}
% We\footnote{Don't be fooled by the word ``easy.'' These are only easy
%   in the sense that, once you understand how to apply them, they are
%   easy to use. Gaining understanding may or may not be easy, depending
%   on your background and abilities.}  now ask, ``What is the
% differential of a constant?'' This question seems simple enough. The
% answer is also quite simple. You may already see what it is. However
% the answer to this seemingly innocuous question is important. So we
% will take a few moments to think about this carefully.

% \begin{mytheorem}
%   \label{thm:ConstantDifferential}
%   \index{differentials!of~a~constant}
%   The differential of a non-varying quantity (a constant) is zero.
% \end{mytheorem}
% \begin{proof}
%   Let \(y=c\) so that \(\d y=\d c,\) where $c$ is constant. Observe
%   that the graph of this curve is a horizontal line. As before suppose
%   $(x,y)$ and $(x_0,y_0)$ be two distinct points on this
%   line. Then \[y=c\] and \[y_0=c.\] Subtracting gives \[(y-y_0)=0\]
%   or \[\Delta y =0\] and so in the infinitesimal case \[\d y =0\]
%   also. And since $y=c$ we know that $\d y = \d c$ so \[\d c=0.\]

% %   \noindent{\bf{} Third Proof:} We will find the \underbar{derivative}
% %   of $c$ with respect to another variable, say $x$ first. Suppose
% %   \(c\)
% %   is constant. That it, \(c\)
% %   is not changing. We wish to find \(\dfdx{c}{x}.\)
% %   According  to definition~\ref{def:derivative1} the symbol
% %   \(\dfdx{c}{x}\) represents the rate of change of $c$ with respect to
% %   $x.$ Since $c$ is {\bf{} not changing} it doesn't really matter what
% %   $x$ represents. We always have
% % \[
% % \dfdx{c}{x}=0.
% % \] 
% % Multiplying through by $\d x$ gives
% % \[
% % \d c=0.
% % \] 

% % Computing $\d y$ in the same way we did in
% % equation~\ref{eq:slope1} we have
% % \begin{align*}
% %   y&=c\\
% % \intertext{and}
% %   y+\d y&=c.\\
% % \intertext{Subtracting yields}
% %   \d y&=0\\
% % \end{align*}
% % immediately. And since \(\d y=\d c\)  we have
% % \[
% % \d c=0.
% % \] 
% \end{proof}
% Another way to understand that $\d c=0$ is to observe that by definition $\d c$ represents an
% infinitesimally small change in $c.$ But since $c$ is
% \underline{\bf{}not} changing clearly $\d c=0.$



\begin{ProblemSection}
  \begin{myproblem}{}
    Suppose that \(3x-4y+7z=2.\)
        \begin{description}
        \item[(a)] Assuming that $y$ and $z$ depend on $x,$ compute
          $\dfdx{y}{x}$ and $\dfdx{z}{x}.$
        \item[(b)] Assuming that $x$ and $y$ depend on $z,$ compute
          $\dfdx{x}{z}$ and $\dfdx{y}{z}.$ 
        \item[(c)] Assuming that $x$ and $z$ depend on $y,$ compute
          $\dfdx{x}{y}$ and $\dfdx{z}{y}.$ 
        \item[(d)] Assume that both $x,$ $y,$ and $z$ depend on a third
          variable, $t.$ Find $\dfdx{x}{t}$ in terms of $\dfdx{y}{t},$
          and $\dfdx{z}{t}.$
        \item[(e)] Assume that both $x,$ $y,$ and $z$ depend on a third
          variable, $t.$ Find $\dfdx{y}{t}$ in terms of $\dfdx{x}{t},$
          and $\dfdx{z}{t}.$
        \item[(f)] Assume that both $x,$ $y,$ and $z$ depend on a third
          variable, $t.$ Find $\dfdx{z}{t}$ in terms of $\dfdx{x}{t},$
          and $\dfdx{y}{t}.$
        \end{description}
%         \begin{description}
%         \item[(a)] $\displaystyle\dfdx{x}{y}$
%         \item[(b)] $\displaystyle\dfdx{y}{x}$
%         \item[(c)] $\displaystyle\dfdx{x}{z}$
%         \item[(d)] $\displaystyle\dfdx{y}{x}$
%         \item[(e)] $\displaystyle\dfdx{y}{z}$
%         \item[(f)] $\displaystyle\dfdx{z}{x}$
%         \item[(g)] $\displaystyle\dfdx{x}{z}$
%         \item[(h)] $\displaystyle\dfdx{z}{y}$
%         \item[(i)] $\displaystyle\dfdx{x}{x}$
%         \item[(j)] $\displaystyle\dfdx{y}{y}$
%         \item[(k)] $\displaystyle\dfdx{z}{z}$
%         \end{description}
  \end{myproblem}

  \begin{myproblem}{}
    Find $\dfdx{y}{x}$ for each of the following.
    \begin{multicols}{3}
      \begin{description}
       \item[(a)] $y=3x+2$
       \item[(b)] $y=4x-2$
       \item[(c)] $y=\frac{x}{3}+12$
       \item[(d)] $y=\frac{3x}{2}-7$
       \item[(e)] $y+3x=2$
       \item[(f)] $3x-y=0$
       \item[(g)] $ax-by=0$ ($a$ and $b$ are constants.)
       \item[(h)] $7-\frac65y+\frac85x=0$
       \item[(i)] $y=3x+2$
       \item[(j)] $1/x=5/y$
       \item[(k)] $\frac{y}{x}=\frac23$
       \item[(l)] $\frac{x}{y} -1 = \frac{2x}{3y}+2$
      \end{description}
    \end{multicols}
\end{myproblem}
\begin{myproblem}{}
  Two cars pass the same point at the same time. Car A is traveling at
  $30 m/h$ and Car B is traveling at $50 m/h.$ 
  \begin{description}
  \item[(a)] Use the formula
    $\text{Distance}=\text{Rate}\times\text{Time}$ to find a formula
    for Car A's position, $D_A$ after they pass.
  \item[(b)] Use the formula
    $\text{Distance}=\text{Rate}\times\text{Time}$ to find a formula
    for Car B's position, $D_B$ after they pass.
  \item[(c)] Give a formula for the distance between the cars after
    they pass.
  \item[(d)] Use the formula you found in part (c) to show that the
    cars are separating at a rate\footnote{Yes this is obvious, and
      yes, this is a lot of work just to get something so
      obvious. But, we are just beginnning and it is important that
      you learn how to use the notation of Calculus in a simple
      context before you try to use it on more complex problems. The
      problems will get harder. We promise.} of $20 m/h.$
  \item[(e)] Find the rate of change of Car A \underline{with respect
      to Car B}.
  \item[(f)] Find the rate of change of Car B \underline{with respect
      to Car A}.
  \end{description}
\end{myproblem}

\end{ProblemSection}

\index{Differentiation Rules!Trig Functions}
\begin{center}
  \begin{tabular}{ |c|c||c|c| }
    \hline
    \multicolumn{4}{|c|}{} \\
    \multicolumn{4}{|c|}{\Large\bf{}Differentiation of } \\
    \multicolumn{4}{|c|}{} \\
    \multicolumn{2}{|c||}{\Large\bf{}Trig Functions}  & \multicolumn{2}{|c|}{\Large\bf{} Inverse Trig Functions} \\
    \multicolumn{2}{|c||}{} &    \multicolumn{2}{|c|}{} \\
    \hline
    $y$   &$\dfdx{(y)}{x}$      &$y$&$\dfdx{(y)}{x}$\\
    \hline\hline
    $\sin x$ &$ \cos x $        &$\inverse\sin x$ &
                                                    $\frac{1}{\sqrt{1-x^2}} $\\\hline
$\cos x$&$ -\sin x $& $\inverse\cos x$ & $\frac{-1}{\sqrt{1-x^2}}$ \\\hline
$\tan x$  & $\sec^2 x$      &$\inverse\tan x$&$o\frac{1}{1+ x^2}$\\\hline
$\cot x$  &$-\csc^2 x$      &$\inverse\cot x$&$\frac{-1}{1+
                                               x^2}$\\\hline
$\sec x$ &$ \sec x\tan x $&$\inverse\sec x$&$\frac{1}{\abs{x}\sqrt{x^2-1}}$\\\hline
$\csc x$  &$-\csc x\cot x $ &$\inverse\csc x$&$\frac{-1}{\abs{x}\sqrt{x^2-1}}$\\\hline
  \end{tabular}
\end{center}

\section{The Differential: Geometry and Dynamics}
\label{sec:visualizing-differentials}

% \begin{wrapfigure}[]{R}{2in}
% \vskip.7cm{}
%   \includegraphics*[height=1.5in,width=2in]{Figures/RectangleDifferential}
% \caption{The Differential Area of a Rectangle}
% \label{fig:RectDiff}
% \end{wrapfigure}
% Differentials are perplexing at first\marginpar{Perhaps this should
%   just go away? Can't figure out where it belongs.}.  An infinitely
% small number is a difficult thing to think about, and thinking about
% it more just seems to add to the confusion. There is good reason for
% this. There are some profound difficulties in the very idea of an
% infinitely small number. But since we do not need a precise notion of
% differentials for now we will settle for an intuitive
% understanding. This is sufficient for now.

Historically, Calculus has been viewed in two fundamentally different
ways. Leibniz's approach was built on the classical methods of
geometry, and Newton used a dynamical, physical approach. As a
mindset\footnote{A mindset is just a ``way of thinking'' about
  something. For example, many people have the mindset that ``Math is
  hard because it is just a horrible gobbledy-gook of stupid symbols
  that don't mean anything.  I will never understand it.'' Such a
  mindset prevents many otherwise very capable people from
  understanding even the simplest mathematics.''  Aren't you glad
  you're not making that particular mistake?} neither approach is
either correct or incorrect. They are simply two different ways to
come at the same idea. However some problems are easier to think about
geometrically and some are easier to think about dynamically so it
will be useful to be familiar with both mindsets.

Since most people have a good intuitive feel for how moving objects
behave we will start by adopting Newton's kinematic
viewpoint. However, this will immediately cause a problem since the
notation used by Leibniz has become standard.

Leibniz thought long and hard about inventing a notation that
reflected his ideas and it is his notation that was eventually
adopted. Thus we have this fundamental tension when we learn
Calculus. We begin with Newton's kinematic approach fused with
Leibniz's geometric notation.

% The basic idea is this: We think of a line as the trace of a point in
% motion. That is, if $x$ use a number, say $x=2,$ to represent a
% multitude of geometric objects. It can measure the length of a line
% segment, the area of a region in the plane, or the volume of a region
% in space\footnote{We do not yet have the vocabulary to talk about the
%   'volume' of a region with more than three dimensions so we stop
%   there for now.}. 

\begin{wrapfigure}[]{r}{2in}
\captionsetup{labelformat=empty}
\centerline{\includegraphics*[height=.5in,width=2in]{Figures/DifferentialLine}}
\label{fig:DifferentialLine}
\end{wrapfigure}
Consider a point starting at the origin and moving in a straight line
along the $x$ axis so that its position at time $t$ is given by
$x(t).$ We take the differential $\d{t}$ to mean the ``next instant''
of time\footnote{Please do not think too much about the meaning of the
  phrase ``in the next instant of time.'' You will go mad. We will
  return to this later.}  and the differential $\d{x}$ to be the
position of our point also in the ``next instant'' of time.

\begin{mynotation}{Leibniz's Differential vs. Newton's Dot}
  Leibniz would say that the differentials $\d{x}$ and $\d{t}$ are
  fixed geometric quantities and that their ratio, $\dfdx{x}{t},$ is
  an (infinitesimal) distance divided by an (infinitesimal) time, so
  it is represents the velocity with which our point is moving.

  Newton would say that at any given moment the moving point has some
  velocity which he denoted $\dot{x}.$ Clearly then
  $\dfdx{x}{t}=\dot{x}.$ Just as clearly the mindsets behind these two
  notational schemes are at odds with each other. The Leibniz, or
  differential, notation is static. That is, $\d{x}$ and $\d{t}$ are
  fixed quantities whose ratio is clearly meant to reflect the idea
  that the velocity of the point can be computed via the fraction
  $\frac{\text{distance}}{\text{time}}.$ On the other hand Newton's
  dot notation takes velocity itself as a basic quantity not to be
  computed from simpler concepts, and so it is about motion.

  Again, there is nothing either inherently right, or inherently
  wrong, with either notation. The two notations simply reflect
  different mindsets -- point of view -- that can be adopted as
  needed.

  \begin{definition}{}
    The quantity denoted by either $\dfdx{x}{t}$ or $\dot{x}$ is
    called the derivative\footnote{Newton's dot notation is still
    used by physicists in the special case where motion is
    involved. That is, $\dot{x}$ is used specifically to represent the
    change of position with respect to time, ie., velocity.

    Sadly, it is the only notation invented by Newton which is still
    in common use. Four hundred years later, Leibniz's much more
    carefully designed differential notation has become universal. }
  of $x$ with respect to $t.$ When $x$ depends on only one other
  variable this is often shortened to ``the derivative of $x$''
  since there is no ambiguity. But eventually we will be dealing with
  situations where $x$ might depend on two or more other variables so
  keep in mind that the variable $t$ is an essential part of the
  derivative.
  \end{definition}

  One of the difficulties with learning Calculus is the abundance of
  different notations available for basic concepts like the
  derivative\footnote{These are only two of them. There are
    others.}. This can be a little overwhelming at first so to keep
  things simple we will be using the differential notation,
  $\dfdx{x}{t},$ exclusively at first. This will cause a certain
  awkwardness in some situations. For example, using Newton's dot
  notation if we want to refer to the velocity of our point $5$
  seconds after starting our clock we would simply write $\dot{x}(5)$
  as usual. Using the differential notation we will adopt the
  convention that the velocity at $5$ seconds is written thus:
$$
\left.\dfdx{x}{t}\right|_{t=5}.
$$

This probably seems unnecessarily cumbersome to you but be assured we
have not made this choice lightly, and this notation, while it is a
bit cumbersome, has certain advantages that will become apparent
later.
\end{mynotation}

% That's the one-dimensional situation. In this case $\d{x}$ represents
% an infinitely small displacement of our point and $\d{t}$ represents
% an infinitely small displacement of time. That is $\d{t}$ is an
% instant of time.

\begin{wrapfigure}[]{r}{2in}
\captionsetup{labelformat=empty}
\centerline{\includegraphics*[height=1in,width=2in]{Figures/DifferentialRect}}
\label{fig:DifferentialRect}
\end{wrapfigure}
Now suppose we have a line of fixed length, say $L.$ If we hold it
vertically with its bottom endpoint on the $x$-axis and slide it to
the right we are clearly sweeping out the rectangular area $R$ in the
 diagram at the right. Moreover, it should be clear that 
\begin{equation}
  \label{eq:DifferentialOfRectangle}
  \d R = L\d x.
\end{equation}
since $L$ is the height and $\d{x}$ is the width of the rectangle at
the next instant of time. Notice that since $\d{R}$ is an infinitesimal
$L\d{x}$ must also be an infinitesimal\footnote{But $L\d{x}$ is the
  product of the real, finite number $L,$ and the infinitesimal
  $\d{x}.$ This is very strange and will eventually require some
  explanation, but we needn't concern ourselves with it just now. }.


% When the relationship between $x$ and $y$ is this simple the
% relationship between the differentials is equally simple and can be
% handled by the ideas and techniques we have already learned. 

% But now suppose this relationship is a bit more complex. That is,
% suppose $x(t)$ and $y(t)$ represent positions on the $x$ and $y$ axes,
% respectively, and that $A=xy.$ In the ``next instant'' $x$ will become
% $x+\d x,$ and $y$ will become $y+\d y,$ but what will $A$
% become? Clearly it will become $A+\d A,$ but what, exactly, is $\d A?$
% We will answer this question more fully in the next
% section. For now we only want to find a way to visualize $\d A$ so
% that we can think  about it more clearly. \marginpar{We don't need the full product rule here. Just let one side change.}

% Consider the following visual representation of this problem: \\
% \centerline{\includegraphics*[height=1.5in,width=2in]{Figures/ProductRuleFirstPass}}
% Since $\d A$ is the increment induced in the area $A=xy$ when $x$ is
% incremented by $\d x$ and $y$ is incremented by $\d y$ it should be
% clear that $d A$ is the L-shaped region sketched on the top and right
% of the rectangle whose area is $A=xy.$

The point here is that, while we introduced differentials as
displacements of line segments, they needn't be visualized as
\underline{only} line segments. A differential is simply an
``infinitely small'' number. The same number, $5,$ can refer
to $5$ miles, $5$ gallons of water, or $\$5,$ and differentials are
similarly flexible. In this example $\d{R}$ is clearly an
area. This is plainly visible in our sketch. What is not visible in
our sketch is that since $\d{x}$ is an infinitesimal $d{R}$  is an
``infinitely thin'' rectangle. Whatever that means.

As intriguing as it is to think about infinitely small line segments
and infinitely thin rectangles we will not venture too far down this
path. Differentials, as such, are mostly just a means to an end. For
most of what follows we will be  reformulating formulas like
equation~\ref{eq:DifferentialOfRectangle} as:
$$
\dfdx{R}{x} = L
$$ 
because the ratio of differentials (the derivative) on the the left side of this formula will
actually be far more useful to us than the differentials $\d R,$ or
$\d x$ themselves. 
% The expression $\dfdx{R}{x}$ is called ``the
% derivative of $R$ with respect to $x.$'' 
The entire formula expresses the fact that the rate of change of the
area of a rectangle with a fixed side $(L)$ and a changing side $(x)$
is the length of the fixed side. 
% In most cases it will be this
% respective rate of change that we will be seeking not the
% differentials $\d R,$ or $\d x$ themselves


% \begin{wrapfigure}[]{r}{1.5in}
% \captionsetup{labelformat=empty}
% \centerline{\includegraphics*[height=1.5in,width=1.5in]{Figures/CircleDifferential}}
% \label{fig:CircleDifferential}
% \end{wrapfigure}
% Next consider the following slightly more complex geometric
% problem. let $A$ be the area of a circle with radius $r,$ and suppose
% that $r$ is changing in time, as in the diagram at the right.  We wish
% to compute $\d A,$ the differential of the area of a circle with
% respect to its radius, $r.$

% Clearly this is just the area of the ring
% between the inner and outer circles in the figure.  Since the width of
% this ring is the infinitesimal $\d{r}$ it seems reasonable to
% conjecture that the differential of the area will be the circumference
% of the circle with radius $r$ times the differential $\d{r}.$ 

% That is, it should be clear that
% \begin{equation*}
% \d A = 2\pi r\d r.%\label{eq:CircDiffAnal}
% \end{equation*}
% \begin{wrapfigure}[]{l}{1.5in}
% \captionsetup{labelformat=empty}
% \centerline{\includegraphics*[height=.4in,width=1.5in]{Figures/CutCircleDiff}}
% \label{fig:}
% \end{wrapfigure}

% To see this we simply cut the ring and flatten it out to get the
% rectangle at the left.
% %\centerline{\includegraphics*[height=.75in,width=2in]{Figures/CutCircleDiff}}
% In that case
% $$
% \d A = 2\pi r\d r
% $$
% as we've indicated.
% Since our reasoning here is  shaky\footnote{ If you are paying attention you see that
%   the claim we just made is incorrect. When we flatten out the ring
%   the difference in the length of the inner and outer circumferences
%   guarantees that we do \underline{not} get a rectangle. We get a
%   trapezoid. However for very small values of $\Delta r$ the inner and
%   outer circumferences get closer and closer together so that when
%   $\Delta r$ becomes the infinitesimal $\d{r}$ the inner and outer
%   circumferences become equal and our claim is at least approximately
%   correct. That will do for now.

%   If this kind of intuitive reasoning makes sense to you,
%   good. Continue to use it. It will help.

%   If it does not, if you don't worry about it. We will return to this in a
%   much more rigorous fashion later.} at best we'll call this a
% conjecture:
% \begin{myconjecture}
%   If $A=\pi r^2$ then $\d A = 2\pi r.$
% \end{myconjecture}


% Let's try to compute $\d A$ analytically to see if our conjecture is
% correct.
% \begin{align}
%   \d A &= \left\{\text{area of outer circle}\right\} -\left\{\text{area of inner circle}\right\}\nonumber\\
%      &= \left\{\pi(r+\d r)^2\right\} - \left\{\pi r^2\right\}\nonumber\\
%      &= \pi(r^2+2r\d r + (\d r)^2 -r^2)\nonumber\\
% \d A &=2\pi r\d r+ \pi(\d r)^2.\label{eq:CircDiffAnal}
% \end{align}

% Uh, oh. 

% This \emph{almost} agrees with our conjecture but we seem to have the
% extra term: $\pi(\d r)^2.$ 

% % Oh no! This is a problem!

% Clearly we need for $(\d r)^2$ to be zero if our geometric argument is
% to be consistent with our analytical argument in
% equation~\ref{eq:CircDiffAnal}.

% But, just as clearly, $(\d r)^2$ is \underline{not} zero, since it is
% the square of the (non-zero) increment of the radius. 

% % This is bad! Very bad.

% This is indeed a puzzle, but it is not a puzzle we need to solve right
% now so we will deal with it pragmatically. That is, we will accept the geometric
% argument as correct and leave for a later time the resolution of the
% inconsistency between the geometric and the analytical arguments.

% As a practical matter this means that we will simply ignore the
% term $(\d r)^2$  and assert that 
% $$
% \d A = 2\pi r\d r
% $$
% or, in its more useful form
% \begin{equation}
% \dfdx{A}{r} = 2\pi r.
% \label{eq:DiffCirc}
% \end{equation}

% Essentially we will declare by fiat -- simply because we know that
% this will give us correct answers -- that $(d r)^2=0$ even though it
% can't possibly be true. If you are uncomfortable with this, \emph{good
%   for you!} You are correct to be uncomfortable. This is a substantial
% problem that we will eventually have to address. However it is not a
% simple problem. It took about 200 years for the mathematical community
% to adequately address the problem of the disappearing differentials
% and the solution requires some very deep and abstract ideas. Since our
% goal at the moment is to learn how  to ``move the pebbles''
% \index{pebbles!move~the} so we can harness the full power of Calculus
% we will simply acknowledge that this is a difficulty and move on,
% holding its resolution for another time.

% For now we will only offer the following, admittedly weak,
% justification for setting $(d r)^2=0.$ Observe that when a number
% which is less than one, say $1/2,$ is squared the result is
% \underline{less than} the original number. Thus since $\d r$ is
% infinitely less than one its square is \emph{infinitely less than $\d
%   r$ itself.}  So $(\d r)^2$ is infinitesimally small \emph{even
%   compared to $2\pi r\d r$} and can thus be safely ignored. So we will
% ignore it.

% This is a problem.  We will resolve this
% apparent inconsistency between our intuition and our computation in
% the next section.

% % \begin{wrapfigure}[]{O}{2in}
% % %\vskip-.7cm{}
% %   \includegraphics*[height=2in,width=2in]{Figures/CircleDifferential}
% % \caption{The Differential Area of a Circle}
% % \label{fig:CircDiff}
% % \end{wrapfigure}

% % equation~\ref{eq:CircDiffAnal} and \underline{assert} that $\d A =2\pi r\d r.$
% % To see this consider what happens when we increment the radius of or
% % circle by $\d r.$ The change in area, $\d A,$ will be the thickness,
% % $\d r,$ times the distance around the circle, $2\pi r.$ That is

% You may find this a little unsatisfying. You should. The justification
% for simply throwing away the quantity $\pi(\d r)^2$ is considerably
% more complex than we are making it out to be. We will get to them
% eventually, when we have a little more experience. For now simply keep
% in the back of your mind that not all of the questions that might be
% asked about this have yet been answered. \index{disappearing
%   differentials}



% Be fore we move on take notice of the following:
% Formula~\ref{eq:DiffCirc} says that $\dfdx{A}{r}$ depends not only on
% $\d{r},$ as we would expect, but also on the radius, $r.$ That is, the
% change in area when we increment the radius of a circle by $\d r$
% depends on how big the circle is.

% It is easy to see why this must be so.  If the radius of the circle is
% small when we increment $r$ by $\d r,$ then the area of the outer
% ring, $\d A,$ will be the product of $\d r$ and the distance around
% the circle which will be small when the radius is small and large when
% the radius is large.

% In fact, this will generally be true. Elucidating the nature of the
% relation between $\d{x}$ and $\d{y}$ when the relation between $x$ and
% $y$ is complicated that will be our first goal.


% \begin{wrapfigure}[]{O}{3in}
% \vskip-.7cm{}
%   \includegraphics*[height=2in,width=2in]{Figures/SquareFunction}
% \caption{ $y=x^2$}
% \label{fig:SquareFunction}
% %\centerline{The Product Rule}
% %\vskip1mm{}
% \end{wrapfigure}

\section{The Derivative of $y=x^2$}
\label{sec:DerivativeSquare}
In the previous section we considered only straight lines: $y=mx+b.$ In that
setting we found this simple relationship between $\d x$ and $\d y:$
\[\dfdx{y}{x} = m\]
where $m$ is the slope of the line.

\begin{embeddedproblem}{}
  Actually, we never worked this out in general. Do that now. That is,
  show that if $y=mx+b$ then
  $$
  \d{y} = m\d{x}
  $$ 
so that 
$$
\dfdx{y}{x} =m.
$$
\end{embeddedproblem}
\begin{wrapfigure}[]{r}{2in}
\captionsetup{labelformat=empty}
\centerline{\includegraphics*[height=1in,width=2in]{Figures/SquareFunction}}
\label{fig:SquareFunction}
\end{wrapfigure}
Unfortunately the world consists of more than straight lines so we
must now ask, ``If $y$ depends on $x$ in some {\bf non}-linear way how
are $\d y$ and $\d x$ related?'' As you might imagine the specifics
will depend very much on the nature of the relationship between $y$
and $x.$ Nevertheless, we will still be able to draw some general
conclusions.


We will start with the simplest non-linear relationship we
know. Suppose \(y=x^2.\) 

As  before if we increment $x$
by an infinitesimal amount, $\d x,$ this will cause an infinitesimal
change, $\d y,$ in $y.$ That is 
\[ y+\d y = (x+\d x)^2.\]

% \begin{wrapfigure}[]{O}{1.25in}
% %\vskip-.7cm{}
%   \includegraphics*[height=1.6in,width=1.25in]{Figures/SquareFunctionMag1}
% \caption{ $y=x^2$ magnified.}
% \label{fig:SquareFunctionMag1}
% %\centerline{The Product Rule}
% \vskip1mm{}
% \end{wrapfigure}


To find the relationship between $\d y$ and $\d x$ we compute:
\begin{equation}
y+\d y = x^2+2x\d x +(\d x)^2
\label{eq:SquareDifferential1}
\end{equation}
and since $y=x^2$ we have $\d y =2x\d x +(\d x)^2.$
This is the relationship we were seeking but as before we will prefer
to have it in the form
\begin{equation}
\dfdx{y}{x} = 2x+\d{x}.
\label{eq:SquareDifferential2}
\end{equation}
\begin{wrapfigure}[]{r}{.75in}
\captionsetup{labelformat=empty}
\centerline{\includegraphics*[height=1.3in,width=.75in]{Figures/SquareFunctionMag1}}
\label{fig:SquareFunctionMag1}
\end{wrapfigure}
since, as we observed in the previous section, the derivative form
$\dfdx{y}{x}$ will be more useful to us later.

We'd like to answer two questions about this formula: (1) Clearly
$\dfdx{y}{x}$ is the slope of \emph{something,} but what? (2) What
does the infinitesimal $\d x$ represent?
To answer the first question consider the graph of $y=x^2$ near the
point $(1,1).$

\begin{wrapfigure}[]{l}{.75in}
\captionsetup{labelformat=empty}
\centerline{\includegraphics*[height=1.5in,width=.75in]{Figures/SquareFunctionMag2}}
\label{fig:SquareFunctionMag2}
\end{wrapfigure}
Since the graph is not a straight line it is hard to see, at first,
what $\dfdx{y}{x}$ could possibly be the slope of. But let's look
closer.\\
No, really. Let's actually look closer. If we magnify the indicated portion of the graph near the point
$(1,1)$ we get the picture at the right.


% \begin{wrapfigure}[]{O}{1.25in}
% %\vskip-.7cm{}
%   \includegraphics*[height=1.6in,width=1.25in]{Figures/SquareFunctionMag2}
% \caption{$y=x^2$ magnified again.}
% \label{fig:SquareFunctionMag2}
% %\centerline{The Product Rule}
% \vskip2mm{}
% \end{wrapfigure}
At this scale the curvature of the graph is almost invisible. That is,
the graph appears to be very nearly a straight line. If we magnify the
graph yet again we see that at \emph{this} scale the graph is
essentially indistiguishable from a straight line.


% It is evident from the figure
% that this graph doesn't have a ``slope'' in the usual sense. More
% precisely, the slope of the graph in figure~\ref{fig:SquareFunction}
% changes from point to point. The right side of the graph is clearly
% negative. Since it seem so matter very much which point we look at we
% will begin by picking a single point on the graph to focus on. Let's
% look at the point $(1,1).$ If we magnify the portion of the orginial
% graph near the point $(1,1)$ as in figure~\ref{fig:SquareFunctionMag1}
% we see that at this scale the graph appears to be very nearly a
% straight line. In fact, if we magnify again as in
% figure~\ref{fig:SquareFunctionMag2} we see that the graph is
% essentially indistinguishable from a straight line. 


This is great! We know how to work with straight lines! In fact, we
know that for a straight line 
\[\dfdx{y}{x} = m\]
% or 
%\[ \d y = m\d x.\] 
Compare this to equation~\ref{eq:SquareDifferential2}:
\[m=\dfdx{y}{x} = 2x+\d{x}.\]
If these formulas represent the same things then we must have $m=2x.$
But what is $\d{x}?$ It is tempting to say that $\d{x}=0$ (and this is
essentially what both Newton and Leibniz did) but that is clearly not
so. For if $\d{x}=0$  then $x+\d x$ is
just $x,$  and $y+\d{y}$ is just $y.$ This is no help.


This is getting hard to think about so let's come at it from a
different point of view. Since $\d{y}$ and $\d{x}$ are fixed
infinitesimals we are thinking in a static, geometrical way. Let's try
thinking about motion instead. 

That is, let's think about a point $P,$ constrained to move from left
to right on the parabola $y=x^2.$ Since the coordinates of $P,$ are
$(x, y)$  we know that
\[\dfdx{y}{x} = 2x.\]

 % \begin{wrapfigure}[]{O}{1.5in}
%  %\vskip-.7cm{}
%    \includegraphics*[height=1.5in,width=1.5in]{Figures/SquareFunctionTangent}
%  \caption{$y=x^2$ with tangent line at $(1,1).$}
%  \label{fig:SquareFunctionTangent}
%  %\centerline{The Product Rule}
%  \vskip2mm{}
%  \end{wrapfigure}

% At first this just seems preposterous and it is difficult to
% understand using Leibniz's geometric point of view. So now we will
% switch to Newton's dynamical approach.

% That is, imagine a particle moving from left to right and constrained
% to stay on the graph of $y=x^2.$ 

As $P$ passes through $(0,0)$ it begins to rise. Now imagine that the
graph ends at $(1,1)$ and that as our particle reaches the point
$(1,1)$ it is suddenly freed from its constraint. What happens?

It moves in a straight line! 

In fact it will move in a straight line in the direction it was moving
as it passed through the point $(1,1).$ 

% then it seems that this slope depends on our horizontal position, $x,$
% on the curve. This seems to make some sense. After all, slope is a
% measure of how fast a graph is rising and certainly the graph of
% $y=x^2$ is rising faster at, say $x=4$ than at $x=1.$ But can we talk
% sensibly about the slope of something that is not a straight line?


The formula 
\[\dfdx{y}{x} = 2x\]
\begin{wrapfigure}[]{r}{2in}
\captionsetup{labelformat=empty}
\centerline{\includegraphics*[height=1.5in,width=2in]{Figures/SquareFunctionTangent}}
\label{fig:SquareFunctionTangent}
\end{wrapfigure}
now begins to make sense. It is telling us that if we release the\marginpar{Trim this image.}
particle when $x=1$ it will proceed forward in a straight line with
slope $2\cdot1=2.$ If we release it when $x=2$ we get a slope of
$2\cdot2=4$ and so on. Naturally the slope depends on $x.$ It must!
The released particle will move off in a straight line whose slope
depends entirely on where it is on the graph when it is released.
Moreover, this line is tangent to our curve at the point of release as
we see in the figure.
Thus $\dfdx{y}{x}$ is the slope of that tangent line.

This argument also tells us how to interpret the quantity $\d{x}$
in equation~\ref{eq:SquareDifferential2}. Notice that we found
equation~\ref{eq:SquareDifferential2} by using two points on
\emph{curve} $y=x^2.$ We used two points so close together that we
couldn't \underline{see} the curvature, but that doesn't mean it isn't
there. What this means is that we didn't actually compute the slope
of our tangent line, $\dfdx{y}{x},$ because the point $(x+\d x, (x+\d
x)^2)$ \emph{is not on the tangent line.} It is on the curve. So the
formula
\[\dfdx{y}{x} = 2x+\d{x}\]
is off by a little bit. That little bit, our error, is the quantity
$\d{x}$ so we simply drop it from consideration.

% If we want to
% find the slope of our graph at the point $(1,1)$ apparently all we
% have to do is magnify the portion of the graph near that point until
% it looks like a straight line. Then compute the slope as usual.

% To be sure no matter how much we magnify our graph it will not be a
% 100\%, perfectly straight line. So our computation will necessarily
% entail some error. But the more we magnify the less the error should
% be.
% Now let's find the slope of $y=x^2$ at the point $(1,1).$ That is, we
% magnify our graph around the point $(1,1)$ until it looks like
% figure~\ref{fig:SquareFunctionMag2}, and then compute the slope as
% usual:
\begin{myexample}{}
  Let's see if we can confirm our reasoning above with some actual
  calculations. Since we will be using actual, finite numbers, not 
  infinitesimals we will revert to the $\Delta$ notation. 

  We zoom in on the graph of $y=x^2$ by setting $\Delta x= .001.$ Then
  from equation~\ref{eq:SquareDifferential1} we have
  \begin{align*}
    \Delta y &= 2x\Delta x + (\Delta x)^2.\\[1mm]
    \intertext{To find the slope we divide through by $\Delta x$,
    obtaining:}
    \frac{\Delta y}{\Delta x} &= 2x + \Delta x.\\
    \intertext{We now take $x=1$ and $\Delta x= .001:$}
    \frac{\Delta y}{\Delta x} &= 2\cdot(1) + .001.\\
             &= 2.001.\\
  \end{align*}
  As before this is an approximation since we used two points on our
  \emph{graph} to do the computation, not two points on our
  \emph{tangent line.} The actual slope should be $2$ as we've seen,
  so the error is $.001 = \Delta x,$ just as we expected.
\end{myexample}
\begin{embeddedproblem}{}
  Repeat this computation using other values for $x$ and $\Delta x$
  and verify that the value of the slope, $\frac{\Delta y}{\Delta x},$
  will differ from $2x$ by the quantity, $\Delta x.$
\end{embeddedproblem}



To summarize the previous discussion and example: We have seen that if
we imagine a particle to be tracing out the graph of $y=x^2$ and then
release it at some horizontal coordinate $x=a,$ then the particle will
continue to move along a straight line with slope
$\left.\dfdx{y}{x}\right|_{x=a} = 2a.$ This line will actually be the
line which is tangent to our graph at the point $(a, a^2).$ Moreover
we can approximate the slope of this tangent line by taking $\Delta x$
to be very small and computing $\frac{\Delta y}{\Delta x}.$ The result
of this computation will have an error of exactly $\Delta x.$

But if we use the ratio of differentials (the derivative) then the
error is eliminated for us. In our example the slope of the line
tangent to the graph of $y=x^2$ is thus $\dfdx{y}{x}=2x$ exactly.

A natural question to ask at this point is: Will the same or a similar
argument give us the slope of the tangent line of, say, $y=x^3?$ Or
$y=x^4?$ Or $y=x^n$ for any integer $n?$ The answer to these questions
is yes. But the visual approach we used here is harder to use. We will
need to come at the problem from a slightly different point of view.

% \section{The Product Rule, the Power Rule, and the Quotient Rule}
% \label{sec:prod-pow-rule}

% \subsection{The Product Rule}
% % \begin{wrapfigure}[]{R}{2.5in}
% % %\vskip-.7cm{}
% %   \includegraphics*[height=2in,width=2.5in]{Figures/ProductRule}
% % \caption{The Product Rule}
% % \label{fig:ProdRule}
% % %\centerline{The Product Rule}
% % \vskip1mm{}
% % \end{wrapfigure}

% Suppose
% \[
% A=xy
% \]
% where $x$ and $y$ are variable.

% We wish to find the relationship between $\d A,$ $x, y, \d x,$ and
% $\d y.$ We can interpret the product, $A=xy,$ as the area of a rectangle
% with sides $x$ and $y$ as in the following diagram:
% \centerline{  \includegraphics*[height=2in,width=3in]{Figures/ProductRule}}


% When $x$ and $y$ are incremented by $\d x$ and $\d y,$ respectively
% then $A$ is incremented by a corresponding quantity, $\d A.$ From the
% figure it is clear that $\d A$ is the L-shaped region on the right and
% top sides of the rectangle. 

% Notice that this region decomposes nicely into three rectangles, one
% with area \(x\d y,\)
% a second with area \(y\d x,\)
% and the third with area \(\d x\d y.\) That is,
% \[
% \d A = x\d y + y\d x +\d x\d y.
% \]

% % This formula is similar to equation~\ref{eq:SquareDifferential} in
% % that we have $\d A$ in terms of $x, y, \d x, \d y$ and the quantity
% % $\d y\d x.$
% However, the term $\d y\d x$ plays the same role here that the term
% $(\d x)^2$ did in equation~\ref{eq:SquareDifferential1}. It simply
% measures the error in the computation. You can see this in the figure.
% If $\d x$ and $\d y$ are infinitely small (they are) then the tiny
% rectangle in the upper right hand corner of the figure is
% insignificant. So, as in the last section, we simply drop the quantity
% $\d x\d y.$

% Thus we have\index{Product~Rule}
% \begin{mytheorem}{\bf{}The Product Rule}\\
% \label{thm:product-rule}
%   If $A,$ $x,$ and $y$ are variable quantities and $A=xy$  
% then
% \[ \d A = x\d y + y\d x.\]
% % \begin{center}
% %   \begin{minipage}{.75\linewidth}
% %     \begin{description}
% %     \item[Differential Form:] \(\displaystyle \d A = x\d y + y\d x\)\vskip3mm{}
% %     \item[Derivative Form:] \(\displaystyle \dfdx{A}{t} = x\dfdx{y}{t}
% %       + y\dfdx{x}{t}.\)
% %     \end{description}
% %   \end{minipage}
% % \end{center}
% \end{mytheorem}

% Notice that if both $x$ and $y$ depend on the time elapsed ($t$), then we can
% divide both sides of the Product Rule by $\d t$ to get the formula:
% $$
% \dfdx{A}{t} = x\dfdx{y}{t} + y\dfdx{x}{t},
% $$
% which says that the rate of change of $A$ with respect to $t,$ $\dfdx{A}{x},$ depends not
% only on the values of $x$ and $y,$ but also on their rates of change,
% $\dfdx{x}{t},$ and $\dfdx{y}{t},$ also with respect to $t.$

% A common mnemonic for the Product Rule is,
% ``The derivative of a product is the first factor times the
% derivative of the second plus second factor times the derivative
% of the first.'' {\sc Memorize it any way you can.} You will be using
% it over and over.

% % To relate this to the slope of a tangent line as before, suppose that
% % both $x$ and $y$ depend on some third variable, say $t,$ then the
% % slope of $A$ with respect to $t$ is given by:\[
% %          \dfdx{A}{t} = x\dfdx{y}{t} + y\dfdx{x}{t}.
% % \]


% \subsection{The Power Rule}
% \label{subsec:power-rule}
% We learned earlier that if \(y=x^2\)
% then  \(\d y = 2x\d x.\)
% In other words, reasoning from the graph of \(y=x^2\) we saw that
% \begin{equation}
%  \dfdx{y}{x}=\dfdx{(x^2)}{x} = 2x.\label{eq:sqDeriv}
% \end{equation}
% %We can use the  \ref[thm:product-rule]{Product Rule} to verify
% %this algebraically as follows
% We can use the  Product Rule to verify this algebraically as follows
% \begin{align*}
%   \d(x^2) &= \d(x\cdot x)\\
%   &= x\d x + x\d x \text{ so that}\\
%       \d(x^2) &= 2x\d x \text{ or, finally,}\\
%   \dfdx{(x^2)}{x} &= 2x.
% \end{align*}

% Now suppose that $y=x^3.$ Again, we'd like to find $\dfdx{y}{x}.$ A visual argument
% like the one presented in section~\ref{sec:DerivativeSquare} will work but is rather difficult to
% follow. Instead we will use the Product Rule. The formula $y=x^3$ can
% be written as the \emph{product:}
% \begin{align}
%   \nonumber y&=x^2\cdot x.\\
% \intertext{From the Product Rule we get}
% \nonumber \d y &= x^2\d x + x\d(x^2).\\
% \intertext{Of course we already know that $\d(x^2)=2x\d x$ from
%   equation~\ref{eq:sqDeriv}. Thus}
% \nonumber    \d y  &= x^2\d x + x(2x\d x)\\
% \nonumber    \d y &= 3x^2\d x\\
% \dfdx{y}{x} &= 3x^2.\label{eq:cubeDeriv}
% \end{align}
% Looking closely at equations~\ref{eq:sqDeriv} and~\ref{eq:cubeDeriv}
% there seems to be a pattern emerging. Do you see it? 

% Perhaps one more example will make it clearer. Suppose that
% $y=x^4=x^2\cdot x^2.$ Then from the Product Rule we have
% \begin{align}
% \nonumber \d y&= x^2\d(x^2)+x^2\d(x^2)\\
% \nonumber     &= x^2(2x\d x) + x^2(2x\d x)\\
%           \dfdx{y}{x}&= 4x^3. \label{eq:quartDeriv}
% \end{align}

% \begin{embeddedproblem}{}
%   Use $x^4=x^3\cdot x$ to show that $\dfdx{y}{x}=4x^3.$
% \end{embeddedproblem}

% We have, so far,
% \begin{align*}
%   \dfdx{x^2}{x} &= 2x\\
%   \dfdx{x^3}{x} &= 3x^2\\
%   \dfdx{x^4}{x} &= 4x^3.\\
% \intertext{The pattern should now be clear. If we continue this list
%   we should get}
%   \dfdx{x^5}{x} &= 5x^4.\\
%   \dfdx{x^6}{x} &= 6x^5.\\
% &\ \  \vdots
% \end{align*}
% and so on.

% This gives us our first version of the Power Rule.
% \begin{embeddedproblem}{}
%   \begin{description}
%   \item[(a)] Verify that $\dfdx{x^5}{x} = 5x^4.$
%   \item[(b)] Verify that $\dfdx{x^6}{x} = 6x^5.$
%   \item[(c)] If $n$ is a positive integer and $y=x^n.$
%   Can you show that     \[ \dfdx{y}{x} = nx^{n-1}?\]

%   \end{description}
% \end{embeddedproblem}

% % \begin{center}
% %   \begin{minipage}{.75\linewidth}
% %     \begin{description}
% %     \item[Differential Form:] \(\displaystyle \d y = nx^{n-1}\d x\)\vskip3mm{}
% %     \item[Derivative Form:] \(\displaystyle \dfdx{y}{x} = nx^{n-1}.\)
% %     \end{description}
% %   \end{minipage}
% % \end{center}

% % The obvious question now is, if this is only the first version of the
% % Power Rule, how many more are there and what are they? There is
% % actually only one but we haven't yet seen its full
% % generality. Specifically, 
% The derivation above is clearly only valid if
% $n$ is a positive integer. But the Power Rule also holds if $n=0.$ To
% see this suppose 
% \[
% y=x^0 =1.
% \]

% Since $y$ is constant we see immediately from
% Theorem~\ref{thm:ConstantDifferential} that 
% \[
%   \d y=0= 0x^{0-1}.
% \]
% Since $0x^{0-1}$ is what we would get if we blindly applied the Power
% Rule for \underline{positive} exponents to $x^0$ we see that the Power
% Rule holds if $n=0.$

% What if $n$ is a negative integer?

% The Power Rule holds in this case too but it is a little harder to
% see. Suppose $n$ is a positive integer and
% \[
% y=x^{-n}.
% \]
% It is a little difficult to see how we might find $\d(x^{-n})$ at
% first, but notice that from the properties of exponents we have
% \begin{align*}
%   x^0 &= x^n\cdot x^{-n}.\\
% \intertext{Taking the differential of both sides we have}
%   \d(x^0) &= \d(x^n\cdot x^{-n}).\\
% \intertext{On the left we get zero since $x^0$ is the constant $1.$ On
% the right we use the Product Rule:}
%   0 &= x^n\d(x^{-n}) + x^{-n}\d(x^n).\\
% \intertext{Next observe that since the expression $\d(x^{-n})$ appears in
%   the above formula we can solve for it:}
% \d(x^{-n}) &= \frac{-x^{-n}\d(x^n)}{x^n}.\\
% \intertext{Since $n$ is a positive integer we know that
%   $\d(x^n)=nx^{n-1}\d x$ so}
% \d(x^{-n}) &= \frac{-x^{-n}nx^{n-1}\d x}{x^n}.\\
% &= \frac{-nx^{-1}\d x}{x^n}\\
% \dfdx{(x^{-n})}{x}&= -nx^{-n-1}
% \end{align*}
% which is exactly what we would get if we blindly applied the Power
% Rule for \underline{positive} exponents to $x^{-n}.$ 

% Therefore the Power Rule holds for any integer, $n,$ whether it be
% positive, negative or zero.

% What if $n$ is {\bf{}not} an integer? What if it is a fraction:
% $n=\frac{a}{b}$ where $a$ and $b$ are both integers?

% \begin{embeddedproblem}{}
% Show that the  Power Rule holds when $n=\frac{a}{b}$ where $a$ and $b$
% are integers. (Hint: If  $y=x^{a/b}$ then $y^b=x^a.$)
% % \intertext{Taking the differential of both sides gives, since $a$ and
% %   $b$ are integers:}
% % by^{b-1}\d y &= ax^{a-1}\d x.\\
% % \intertext{As before the expression $\d y$ appears in our formula and
% %   is easy to solve for. So we solve:}
% % \d y &= \frac{ax^{a-1}\d x}{by^{b-1}}\\
% % &=\frac{a}{b}\left(\frac{x^{a-1}}{y^{b-1}}\right)\d x.\\
% % \intertext{Since $y=x^{\frac{a}{b}}$ we see that $
% %   y^{b-1}=x^{\frac{a(b-1)}{b}} $ so that $\frac{x^{a-1}}{y^{b-1}} =
% %   \frac{x^{a-1}}{x^{\frac{a(b-1)}{b}}} = x^{\left[(a-1)-\left(\frac{a(b-1)}{b}\right)\right]}
% %   = x^{\left[\frac{b(a-1)}{b}-\frac{a(b-1)}{b}\right]}=
% %   x^{\frac{-b+a}{b}} = x^{\frac{a}{b}-1}.$ Substituting this into the
% %   equation above gives:}
% % \d y &=\frac{a}{b}x^{\frac{a}{b}-1} \d x\\
% % \end{align*}
% % which is exactly what we would get if we blindly applied the Power
% % Rule for \underline{positive} exponents\index{Power~Rule!for~rational~numbers} to $x^{a/b}.$ 
% \end{embeddedproblem}

% % Therefore the Power Rule works\footnote{Recall that the integers are
% %   just fractions whose denominator is one, so this result subsumes the
% %   two previous results.} as long as the exponent $n$ is any fraction.

% Are you ready for the next question?

% There are numbers which can not be expressed as the ratio of two
% integers. These are the so-called \emph{irrational}
% numbers. $\sqrt{2}$ is one. So is $\pi.$ What if the exponent is one
% of these? Does the Power Rule still hold? As before the answer is
% ``Yes,'' except this time we do not yet have all of the tools we'll
% need to see this. So we will have to defer this derivation to another
% time. Nevertheless the following theorem is true:

% \begin{mytheorem}[The Power Rule]
% \index{Differentiation Rules!Power Rule}
% \label{thm:power-rule}
% Let $\alpha$ be  \underline{any} real number. Then
%     \[ \dfdx{y}{x} = \alpha x^{\alpha-1}\]
% %   \begin{center}
% %   \begin{minipage}{.75\linewidth}
% %     \begin{description}
% %     \item[Differential Form:] \(\displaystyle \d y = \alpha x^{\alpha-1}\d x\)\vskip3mm{}
% %     \item[Derivative Form:] \(\displaystyle \dfdx{y}{x} = \alpha x^{\alpha-1}.\)
% %     \end{description}
% %   \end{minipage}
% % \end{center}
% \end{mytheorem}

% \begin{embeddedproblem}{}
% This problem shows that there are some subtleties, even some
% difficulties, with these rules as we've set them forth in this
% chapter. We will largely ignore these subtleties for now as they tend
% to be anomalous and distracting at first. They are real though and
% will eventually need to be addressed. We will come back to these
% foundational issues after you learn how to apply and use the tools. So
% for now store the apparent anomalies of this problem in the back of
% your mind. We will return to
% them.\index{Foreshadowing!Differentiaton anomalies}
%   \begin{description}
%   \item[(a)] Find the equation of the tangent line to the curve
%     $y=x^{3/2}$ at the point $\left(a, a^{3/2}\right).$ 
%   \item[(b)] It was not mentioned in part (a), but we must have $a\ge0$
%     as     the curve ends at the point $(0,0).$  What would happen if you
%     substituted $a=0$ into your answer to part (a)?  Does this make sense as
%     an answer?  You should use your graphing calculator to help answer
%     this question by graphing the curve.  (Also, there is not a clear
%     cut answer to this question.)
%   \item[(c)] Notice that the curve $y=x^{2/3}$ exists for all values
%     of $x$ since you can take the cube root of a negative number.
%     Find the equation of the tangent line to this curve at the point
%     $\left(a, a^{2/3}\right).$ What happens if you substitute $a=0$
%     into this answer?  Is this consistent with the graph of the curve
%     $y=x^{2/3}?$
%   \end{description}
% \end{embeddedproblem}

% \subsection{The Quotient Rule}
% \label{subsec:quotient-rule}

% Like the Power Rule the Quotient Rule is, at its heart, really just a
% special case of the Product Rule.

% \begin{mytheorem}[The Quotient Rule]
% \index{Differentiation Rules!Quotient Rule}
% \label{thm:quotient-rule}
%    Suppose that $x$ and $y$ depend on $t$ (think of this as time if it
%    helps) and that $Q=\frac{x}{y},$ then
% \[ \dfdx{Q}{t} = \frac{y\dfdx{x}{t}-x\dfdx{y}{t}}{y^2}.\]
% \end{mytheorem}

% \begin{proof}
%   \begin{align*}
%     Q&=\frac{x}{y}= x\left(\frac{1}{y}\right)\\
%     \intertext{we see from the Product Rule that}
%     \d Q &= x\d(y^{-1}) + \frac{1}{y}\d x\\
%     \intertext{and from the Power Rule}
%     \d Q &= x(-y^{-2})\d y  + \frac{1}{y}\d x.\\
%     \intertext{Finally, by rearranging things a bit we get}  
%     \d Q &= \frac{y\d{x}-x\d{y}}{y^2}\\
%     \intertext{and so, finally}
%     \dfdx{Q}{t} &= \frac{y\dfdx{x}{t}-x\dfdx{y}{t}}{y^2}.
%   \end{align*}

% \end{proof}

% The differentiation rules we've seen so far are shown in the following
% table:
% % \begin{center}
% %   \begin{tabular}{ |c|r|c| }
% % \hline{}
% % \hline{}
% % &    The differential of & If $a$ is a constant then \\
% % &   a constant is zero.  & $\d a = 0.$\\[2mm]\hline
% % &    The constant        & If $a$ is a constant and\\
% % &   multiple rule.       & $x$ is a variable then \\
% % &                        &$\d(ax) = a\d{x}.$\\[2mm]\hline
% % &    The sum             & If $x$ and $y$ are variables then \\
% % &       rule.            & $\d(x+y) = \d x + \d y.$\\[2mm]\hline{}
% % &    The Product         & If $x$ and $y$ are variables then\\
% % &     Rule.              & $\d(xy) = x\d y + y\d x.$\\[2mm]\hline
% % &    The Power           & If $x$ is variables and\\
% % &      Rule.             &  $n$ is a constant then\\
% % &                        &  $\d(x^n) = nx^{n-1}\d x$\\[2mm]\hline{}
% % &    The Quotient        & If $x$ and $y$ are variables then\\
% % &           Rule.        &  $\d\left(\frac{x}{y}\right) = \frac{y\d x
% %                            - x\d y}{x^2}$\\[2mm]
% % \hline{}
% %   \end{tabular}
% % \end{center}




% % \begin{center}
% %   \begin{center}
% %     \begin{tabular}{ |c|c| }
% %     \hline
% %     \multicolumn{2}{|c|}{\bf{}Differential Rules} \\
% %     \hline
% %     The differential of & If $a$ is a constant then \\
% %     a constant is zero.  & $\d a = 0.$\\[2mm]\hline
% %     The Constant        & If $a$ is a constant and\\
% %     Multiple Rule.       & $x$ is a variable then \\
% %                          &$\d(ax) = a\d{x}.$\\[2mm]\hline
% %     The Sum             & If $x$ and $y$ are variables then \\
% %        Rule.            & $\d(x+y) = \d x + \d y.$\\[2mm]\hline{}
% %     The Product         & If $x$ and $y$ are variables then\\
% %      Rule.              & $\d(xy) = x\d y + y\d x.$\\[2mm]\hline
% %     The Power           & If $x$ is variables and\\
% %       Rule.             &  $n$ is a constant then\\
% %                         &  $\d(x^n) = nx^{n-1}\d x$\\[2mm]\hline{}
% %     The Quotient        & If $x$ and $y$ are variables then\\
% %            Rule.        &  $\d\left(\frac{x}{y}\right) = \frac{y\d x
% %                            - x\d y}{x^2}$\\[2mm]
% %     \hline
% %   \end{tabular}
% % \end{center}

% \begin{center}
%   \begin{tabular}{ |c|c|c| }
%     \hline
%     \multicolumn{3}{|c|}{} \\
%     \multicolumn{3}{|c|}{\Large\bf{}Differentiation Rules} \\
%     \multicolumn{3}{|c|}{} \\
%     \hline
%  && \\
%     Theorem~\ref{thm:ConstantDifferential}&    The differential of & If $a$ is a constant then \\
%  &   a constant is zero.  & $\d a = 0.$\\[2mm]\hline
%  && \\
%     Theorem~\ref{thm:LinearityConstantMultiple}&    The constant        & If $a$ is a constant and\\
%  &   multiple rule.       & $x$ is a variable then \\
%  &                        &$\d(ax) = a\d{x}.$\\[2mm]\hline
%  && \\
%     Theorem~\ref{thm:LinearitySum}&    The sum             & If $x$ and $y$ are variables then \\
%  &       rule.            & $\d(x+y) = \d x + \d y.$\\[2mm]\hline{}
%  && \\
%     Theorem~\ref{thm:product-rule}&    The Product         & If $x$ and $y$ are variables then\\
%  &     Rule.              & $\d(xy) = x\d y + y\d x.$\\[2mm]\hline
%  && \\
%     Theorem~\ref{thm:power-rule}&    The Power           & If $x$ is variables and\\
%  &      Rule.             &  $n$ is a rational number then\\
%  &                        &  $\d(x^n) = nx^{n-1}\d x$\\[2mm]
%  && \\
%     \hline{}
%  && \\
%     Theorem~\ref{thm:quotient-rule}&    The Quotient        & If $x$ and $y$ are variables then\\
%  &           Rule.        &  $\d\left(\frac{x}{y}\right) = \frac{y\d x
%                             - x\d y}{x^2}$\\[2mm]
%  && \\
%     \hline
%  && \\
%     Theorem~\ref{thm:chain-rule}&    The Chain        & If $y$ depends on
%                                                         $x,$ \\
%  &    Rule.            & and $x$ depends on $t$ then\\
%  &                     &  $\d y = \dfdx{y}{x}\dfdx{x}{t}\d t$\\
%  && \\
%     \hline
%   \end{tabular}
% \end{center}

% Hold on! We slipped one past you there. Did you catch it? 

% The last entry in our table above is called ``The Chain Rule'' and
% refers to Theorem~\ref{thm:chain-rule} which we haven't mentioned
% yet. The reason for this is that we have, in fact, been using the
% Chain Rule all along. We just haven't been calling it that because
% there really hasn't been any need (yet) to have a separate differentiation
% rule called the Chain Rule. Neither Newton nor Leibniz nor any of
% their contemporaries would have recognized a need for what we today call the
% Chain Rule.

\section{The Chain Rule} 
Before the invention of Calculus arithmetic primers gave the following 
simple rule to be used to, among other things, convert money from one
currency to another: Suppose we have $x$ dollars  that we want to
convert to British pounds. If we know that 
\begin{equation}
 z \text{ dollars}  = y \text{ pounds}
 \label{eq:Dollar2PoundConversion}
 \end{equation}
 then if we divide both sides of equation~\ref{eq:Dollar2PoundConversion} by ($z$ dollar) we get 
$$
1=\frac{y \text{ pounds}}{z \text{ dollars} }.
$$
Notice that the left side of this last formula is just the number one. So we start with $x$ dollars, multiply by one in the form $1=\frac{y \text{ pounds}}{ z \text{ dollars} }$ to get
\begin{align*}
  x \text{ dollars} &= x\text{ dollars}\cdot1 = x\text{ dollars}\cdot\frac{y \text{ pounds}}{ z \text{ dollars} }\\
    &= x\cdot y/z\cdot\frac{\text{\cancel{dollars}} \cdot\text{ pounds}}{\text{\cancel{dollars}}}
%   xy \text{ pounds} &= \left(x \text{ dollars}\right)\left(
%                     \frac{y \text{ pounds}}{\text{dollars}}\right)\\
%   &= xy\ (\text{dollars})\left(\frac{\text{pounds}}{\text{dollars}}\right)\\
\intertext{and since the dollars cancel (at least symbolically) we have}
  &= xy/z\text{ pounds.}
\end{align*}

Now  suppose that $1 \text{ dollar}=.85\text{ pounds},$ and that $1 \text{ pound} = 221 \text{ Chinese yuan}.$ Then\footnote{We're making these numbers up. Don't use them to convert
currency.} if we
need to convert $25$ American dollars to Chinese yuan we compute as follows:
\begin{align*}
  25\text{ dollars} &= 25\text{ dollars}\times \frac{.85\text{ 
                      pounds}}{1 \text{ dollar}}\times \frac{221
                      \text{ 
                      yuan}}{1 \text{ pound}}\\
                    &= 25\times.85\times221\left(\text{\cancel{dollars}}\right)
                      \left(\frac{\text{\bcancel{pounds}}}{\text{\cancel{dollars}}}\right)
                      \left(\frac{\text{yuan}}{\text{\bcancel{pounds}}}\right)  \\
                    &= 4696.25 \text{ yuan}  
\end{align*}

The crucial observation here is that in the expression
$\left(\text{dollars}\right)
\left(\frac{\text{pounds}}{\text{dollars}}\right)
\left(\frac{\text{yuan}}{\text{pounds}}\right) $ we have this symbolic
cancellation of both dollars and pounds so that when finished only
yuan are left.  This daisy chain of symbolic cancellation was called
the Chain Rule and it was very helpful in a time when currency
exchange was done by people. Nowadays of course, since these computations
are done by computers a name is  no longer needed. Machines don't
care about such things.

However, with the invention of Calculus, and especially when the
differential notation is used to take the derivative, the same sort of
symbolic cancellation is seen to occur, except that it is done in
reverse; we ``uncancel'' instead of cancelling.  

We are accustomed to ``cancelling'' out $\dfdx{y}{x}\cdot\dfdx{x}{t}$
to get $\dfdx{y}{t}.$ But if if $y$ depends on $x$ which depends on
$t$ and we wish to compute $\dfdx{y}{t}$ we ``uncancel'' as follows:
\begin{align*}
  \dfdx{y}{x}        &=   \dfdx{y}{x}\\
        \d y  &= \dfdx{y}{x}\d x\\
\intertext{and, "uncancelling" $\d t$ on both sides gives,}
  \dfdx{y}{t} &= \dfdx{y}{x}\dfdx{x}{t}.\\
\end{align*}
``Uncancelling'' is a surprisingly powerful tool in mathematics, but
unlike cancelling we don't get to simply cross something off. Instead
we have to figure out what we'd like to have and force it to appear.
The trick is to determine what needs to appear. This comes with
experience.

Notice that \emph{symbollically} we don't seem to have accomplished
much with all of our machinations. On the right side of this last
formula the $\d x$ that appears in the
numerator appears to cancel out the one in the denominator and in the
end we have the (apparently) obvious statement that
$\dfdx{y}{t}= \frac{\d y}{\text{\cancel{$\d x$}}}\frac{\text{\cancel{$\d x$}}}{\d t} = \dfdx{y}{t}.$

However these symbols we're using are much like the pebbles used by
medieval merchants. At one level they are just empty symbols (pebbles)
lying on the page (counting board). Viewed in this manner it is
profoundly obvious that $\dfdx{y}{t} = \dfdx{y}{x}\dfdx{x}{t}.$ Just
do the cancellation. Duh! 

But since a pebble lying here means ``one'' and the same pebble lying
there means ``ten'' and it is necessary to interpret the meaning of a
pebble here or there in order to compute. Similarly, we have invested
a good deal of meaning into our symbols, so at another, more abstract,
level there is much deeper significance to the equation: $\dfdx{y}{t}
= \dfdx{y}{x}\dfdx{x}{t}.$

We understand the symbol $\dfdx{y}{t}$ to mean ``the rate of change of
$y$ with respect to $t$'' so this formula, when properly interpreted
says that the rate of change of $y$ with respect to $t$ is equal to
the rate of change of $y$ \underline{with respect to $x$} multiplied
by the rate of change of $x$ \underline{with respect to $t.$}


For example, suppose that $y=x^2.$ Then as we've seen $\d y= 2x\d x.$
Now suppose that $x=t^3+1.$ Then $\d x = 3t^2\d t.$ Putting this all
together gives:
\begin{align*}
  \d y &= 2x\d x\\
       &= 2x(3t^2\d t).\\
  \intertext{At this point we have $\d y$ in terms of both $x$ and $t.$
             There is nothing wrong with this, \emph{per se} but in
             general we  prefer using one variable to two. So we make
             the substitution $x=t^3+1,$ giving:}
  \d y    &= 2(t^3+1)(3t^2\d t).\\
\end{align*}
And finally, if we want $\dfdx{y}{t}$ we simply divide through by $\d
t$ giving:
$$
\dfdx{y}{t} = 2(t^3+1)(3t^2).
$$

Notice that since $\dfdx{y}{x} =  2(t^3+1)$ (since $x= t^3+1$) and that
$\dfdx{x}{t} = 3t^2$ this is just a particular example of the formula
$$
\dfdx{y}{t} =\dfdx{y}{x}\dfdx{x}{t}.
$$ 
Since this looks so much like the older ``Chain Rule'' for changing
money\footnote{In fact, it really is the same rule if you look at it
  in just the right way. We'll come back to this later.} from one
currency to another it has inherited the name ``The Chain Rule,'' and
is usually stated as a separate theorem as follows.

\begin{mytheorem}[The Chain Rule]
\index{Differentiation Rules!Chain Rule}
  \label{thm:chain-rule}
  If $y$ depends on $x,$ and $x$ depends on $t$ then  
  $$
  \dfdx{y}{t} =  \dfdx{y}{x}\dfdx{x}{t}.
  $$
\end{mytheorem}
\enddigress{Chain Rule}

\subsection{Differentiation Examples}
\label{subsec:DiffExamples}

\begin{myexample}{}
  \label{ex:brute-force}
We wish to find $\dfdx{y}{x}$ if  $y=5(x^2+x)^3.$ 
One way to do this is simple brute force. That is, we compute 
\begin{align*}
  y=(x^2+x)^3 &= x^6+3x^5+3x^4+x^3,
\intertext{so that}
 y &= 5x^6+15x^5+15x^4+5x^3.
\intertext{Applying the Sum Rule  we
  see that}
\d y &= \d \left(5x^6\right)+\d \left(15x^5\right)+\d
\left(15x^4\right)+\d \left(5x^3\right).
\intertext{Next we use the
       Constant Multiple
       Rule, to get}
\d y &= 5\d \left(x^6\right)+15\d \left(x^5\right)+15\d
\left(x^4\right)+5\d \left(x^3\right).
\intertext{Finally, we use the Power Rule
   to get}
\d y &= 30x^5\d x+75x^4\d x+60x^3\d x+15x^2\d x.\\
\d y &= (30x^5+75x^4+60x^3+15x^2)\d x.
\end{align*}
So that, in the end,
\begin{equation}
\dfdx{y}{x} = 30x^5+75x^4+60x^3+15 x^2.\label{ex2:1}
\end{equation}
This works and it gives the correct answer. However there are some
difficulties with this ``brute-force'' procedure that will cause us
some problems later.

Let's find a simpler way.

Observe that if $y=5(x^2+x)^3$ it seems as if we should be able to use
the Power Rule directly on the expression $(x^2+x)^3$ without
expanding it out like we did above. We can, but we need to be
careful. If we set $z=x^2+x$ then we have \[y=5z^3.\] We can now use
the Constant Multiple Rule and then the Power Rule to get
\[\d y = 5(3z^2\d z)= 15z^2\d z.\] Substituting $x^2+x$ back in for $z$
we get
\[\d y = 15(x^2+x)^2\d (x^2+x).\] Computing $\d (x^2+x) = (2x+1)\d x$
and substituting again we have
\[\d y = 15(x^2+x)^2(2x+1)\d x\]  so that
\begin{equation}
\dfdx{y}{x} = 15(x^2+x)^2(2x+1).\label{ex2:2}
\end{equation}


Not only does this involve considerably less algebra (and is thus less
prone to error) it also gives us $\dfdx{y}{x}$ in a partially factored
form. This is important because in applications we will frequently
need to set $\dfdx{y}{x}$ equal to zero and solve for $x.$

Using equation~\ref{ex2:2} this is fairly easy. If \(
15(x^2+x)^2(2x+1) = 0 \) then clearly $x=0, x=-1, $ and $x=-1/2$ are
the solutions.

If we try to use equation~\ref{ex2:1} we get the equation
\[ 30x^5+75x^4+60x^3+15 x^2 =0,\] and the less said about that right
now, the better.
% \begin{embeddedproblem}{}
% For $x, y,$ and $z$ as given in this example:
%   \begin{description}
%   \item[(a)] Compute $\dfdx{y}{z}.$
%   \item[(b)] Compute $\dfdx{z}{x}.$
%   \item[(c)] Show that $\dfdx{y}{x} = \dfdx{y}{z}\dfdx{z}{x}.$
%   \end{description}
% \end{embeddedproblem}
% Since the same sort of symbolic cancellation occurs here as in the
% Chain Rule discussed above it is natural to also call the computation
% in part (c) the Chain Rule as well. 
\end{myexample}

\begin{myexample}{}
 Find $\dfdx{y}{x}$ if  \(y= (\underbrace{x+2}_{=z})^2(\underbrace{2x-3}_{=w})^3.\)

 As in example~\ref{ex:brute-force} we \emph{could} just multiply out
 the right hand side of the equation. But again this is very error
 prone, and a lot of work.

Instead notice that $y$ is the \underline{\bf{}product} of $z=(x+2)^2$ and
$w=(2x-3)^3,$ so the Product Rule seems a
likely place to begin. We have
\begin{align*}
  \d y &= z\d w + w\d z\\
       &= (x+2)^2\underbrace{\d(2x-3)^3} + (2x-3)^3\underbrace{\d(x+2)^2}\\
       &= (x+2)^2(3)(2x-3)^2\underbrace{\d(2x-3)} + (2x-3)^3(2)(x+2)\underbrace{\d(x+2)}\\
       &= (x+2)^2(3)(2x-3)^2(2)\d x + (2x-3)^3(2)(x+2)\d x\\
\intertext{so that }
\dfdx{y}{x} &= 6(x+2)^2(2x-3)^2 + 2(2x-3)^3(x+2)\\
\end{align*}

As before this form is easier to work with if we ever need\footnote{We
will.} to  set
$\dfdx{y}{x}$ equal to zero and solve for $x.$

Let's try. The prospects for solving
\begin{align*}
  6(x+2)^2(2x-3)^2 + 2(2x-3)^3(x+2) &= 0\\
\intertext{seem pretty glum at first. But, in fact, it isn't too
  bad. We just need to notice that $(x+2)$ and $(2x-3)^2$ appear as
factors in both terms on the left hand side. Factoring these out
gives}
(x+2)(2x-3)^2\left[6x(x+2)+2(2x-3)\right] &= 0
\intertext{so that}
(x+2)(2x-3)^2\left[6x^2+12x+4x-6)\right] &= 0\\
(x+2)(2x-3)^2(3x^2+8x-2) &= 0.
\end{align*}
Clearly $x=-2$ and $x=3/2$ are both solutions as are all solutions of 
\(3x^2+8x-3=0.\)
\begin{embeddedproblem}{}
  \label{embed:dum}
%  \IndexProblem{\theembeddedproblem}{dum}
  Complete this example by finding all solutions of this last equation.
\end{embeddedproblem}
\end{myexample}

\marginpar{We need a couple of simple examples of how to derive
  differential equations from a word problem. Consider the constant
  subtangent equal to one (exponential), and $x^2+y^2=1 \imp
  \dfdx{y}{x}=-\frac{x}{y}.$ Maybe one other?} 
% To start exploiting the relationship between derivatives and slopes of
% tangent lines, consider the shape of the center cable on a stable
% suspension bridge.  \\
% \centerline{\includegraphics*[height=2in,width=4in]{Figures/MackinacBridge1}}

% The shape appears to be parabolic, but how can we be sure?  To examine
% this, we will assume that the weight of the cable is negligible with
% respect to the weight of the deck (which we will assume is horizontal,
% for simplicity).  Suppose the deck is made of a material that weighs $W$
% newtons per meter.  Since the curve is symmetric about its lowest
% point, let's look at half of the cable.  Consider a point $P$ on the
% cable and all of the forces involved. 

% \centerline{\includegraphics*[height=2in,width=4in]{Figures/MackinacBridge2}}

% Here $H$ represents the (magnitude of) the horizontal tension on the
% cable and is constant throughout the cable (otherwise the cable would
% move sideways), $Wx$ represents the weight of the deck from the lowest
% point at $x=0$ to the point $P,$ (recall that we are ignoring the weight
% of the cable).  $T$ represents (the magnitude of) the tension on the
% cable at point $P$ which is tangent to the cable at that point.  For the
% cable to be stable, the vertical component of $T$ must equal the weight
% of the deck and the horizontal component must equal $H.$


% \centerline{\includegraphics*[height=2in,width=2in]{Figures/MackinacBridge2}} 

% \begin{embeddedproblem}{}
%   \begin{description}
%   \item[(a)] Use the above analysis to show that the curve which
%     represents the cable must satisfy the differential equation 
% $$
% \dfdx{y}{x} = \frac{W_x}{H}.
% $$
% \item[(b)] Show that the parabola  $y=\frac{W_xH}{2} x^2+b$  satisfies this
%   differential equation in part (a).
%   \end{description}
% \end{embeddedproblem}

% While we're on the topic of hanging cables, consider the following
% related problem.
% \begin{wrapfigure}[]{r}{2in}
% \captionsetup{labelformat=empty}
% \includegraphics*[height=1in,width=2in]{Figures/HangingChain1}
% \label{fig:HangingChain}
% \end{wrapfigure}

% In his\emph{ Discoursi} of $1638$ Galileo Galilei $(1564-1642)$
% mentioned that the shape of a chain hanging under its own weight was a
% parabola.  This reflected the common belief at the time.  Let's
% explore this belief.
 
% We will apply an analysis similar to the one above with the suspension
% bridge.  The difference is that there is no deck, and the vertical
% force will be \emph{only} the weight of the chain (cable) from the
% lowest point to $P.$


% \begin{mynotation}{}
%   Before we send you off to attack this problem, we need to introduce
%   a topic that will take on more significance later.  For now, it is
%   more notational.  If we have $y=x^3+5x-7,$ then differentiating, we
%   have
% $$
% \dfdx{y}{x}=3x^2+5.
% $$
% Differentiating again, we have
% $$
% \dfdx{}{x}\left(\dfdx{y}{x}{}\right)=\dfdx{3x^2+5}{x}=6x
% $$
% This ``second'' derivative is denoted by
% $$
% \dfdx{}{x}\left(\dfdx{}{x}(y){}\right)=\dfdxn{y}{x}{2}.
% $$

% Of course, we could keep going and compute the ``third derivative:''
% $\dfdxn{}{x}{3} (x^3+5x-7)=\dfdx{}{x} \left(\dfdxn{}{x}{2}\left(x^3+5x-7\right)\right)=\dfdx{}{x}(6x)=6. $
% \end{mynotation}

% \begin{embeddedproblem}{}
%   \begin{description}
%   \item[(a)] If we let $w$ represent the weight density of the chain in
%     newtons per meter and s represent the length of the chain from the
%     lowest point to $P$ then show that the curve represented by the
%     chain must satisfy the differential equation   
% $$
% \dfdx{y}{x}=\frac{\omega s}{H}.
% $$
% and use this to show that the curve must satisfy the differential equation
% $$
% \dfdxn{y}{x}{2}=\frac{w}{H}\dfdx{s}{x}=\frac{w}{H} \sqrt{1+\left(\dfdx{y}{x}\right)^2 }.
% $$
% \item[(b)]
% Show that the general parabola $y=ax^2+bx+c$ does not satisfy this
% differential equation (and so Galileo was mistaken!)  The curve which
% does satisfy it is called a catenary, which is not too illuminating as
% catenary is derived from the Latin word catena which means "chain." 
%   \end{description}
% \end{embeddedproblem}

% \begin{wrapfigure}[]{r}{1.5in}
% \captionsetup{labelformat=empty}
% \includegraphics*[height=2in,width=1.5in]{Figures/RefractiveAndReflectiveTelescopes}
% \label{fig:Telescopes}
% \end{wrapfigure}
% As the previous problem shows the catenary is not a parabola. Nor is
% it a cubic polynomial. In fact, there is no polynomial of any degree
% whose graph is the catenary curve.  Clearly, we'll need to extend our
% differentiation rules to a broader class of functions but before we do
% that, here is a real world  application which we can handle using the
% tools we now have.

% % \begin{wrapfigure}[]{r}{.5in}
% % \captionsetup{labelformat=empty}
% % \includegraphics*[height=1in,width=.5in]{Figures/LargeBinocular}
% % \label{fig:Binocular}
% % \end{wrapfigure}
% Refractive telescopes (the top image at the right) use two lenses to
% refract light.  Reflective telescopes (bottom image) use a
% parabolic objective (primary) mirror to reflect light toward the focus
% of the parabola where the flat (secondary) mirror reflects it to the
% eyepiece. When we say a mirror is ``parabolic'' we obviously don't
% mean that it has the shape of  a parabola, since parabolas are curved
% lines and mirrors are three dimensional shapes. What we mean is that
% every cross-section of the mirror is a parabola. The shape of a
% parabolic mirror is obtained by revolving a parabola about its axis of
% symmetry. 

% Large telescopes tend to be reflective telescopes as they
% really don't need the tube (which actually only holds the two lenses
% in the refractive telescope in place).  As an example, consider the
% Large Binocular Telescope at Mt. Graham, AZ.

% \begin{wrapfigure}[]{r}{2in}
% \captionsetup{labelformat=empty}
% \centerline{\includegraphics*[height=1in,width=2in]{Figures/SpinCasting1}}
% \label{fig:SpinCasting1}
% \end{wrapfigure}


% Each parabolic reflector mirror is $8.4$ meters in diameter.  The
% following image gives you an idea of the size of one of these
% mirrors.

% How does one make such a large parabolic mirror?  The Steward
% Observatory Mirror Laboratory ``spin casts'' these large mirrors.  That
% is, they load borosilicate glass into a revolving oven.  [See images
% below.]

% % \begin{wrapfigure}[]{l}{2in}
% % \captionsetup{labelformat=empty}
% % \centerline{\includegraphics*[height=1in,width=2in]{Figures/SpinCasting2}}
% % \label{fig:SpinCasting2}
% % \end{wrapfigure}


% \begin{embeddedproblem}{}
%   Consider a cylinder partially filled with a liquid which is rotating
%   with an angular velocity of $\omega$ radians per second.  Of course, in
%   such a setting the middle will go down and the sides will go up as
%   in the diagram below.
  
% %   \begin{wrapfigure}[]{l}{2in}
% % \captionsetup{labelformat=empty}
% \centerline{\includegraphics*[height=1.5in,width=2in]{Figures/SpinCasting3}}
% % \label{fig:SpinCasting3}
% % \end{wrapfigure}
% Our task is to find the shape of the surface of the liquid (that is
% the shape of the above curve which will be rotated around the $y$ axis.
% If we consider a point mass $m$ at point\\
% \begin{wrapfigure}[]{r}{1in}
% \captionsetup{labelformat=empty}
% \includegraphics*[height=.75in,width=1in]{Figures/SpinCasting4}
% \label{fig:Spin4}
% \end{wrapfigure}
% $(x,y)$ on the surface, then the
% force from the liquid which keeps that point elevated is perpendicular
% to the surface.  (Think about a hose with a hole in it.  The water
% sprays out in a stream perpendicular to the hose.)   

% If we separate that force into its vertical and horizontal components,
% then we get the following diagram. The (magnitude of) the vertical
% component is $mg,$ where $g$ is the acceleration due to gravity.  This is
% the force needed to counter the weight of the particle.  The
% horizontal force has a magnitude of $mx\omega^2.$  This is the centripetal
% force which is keeping the particle moving in a circle around the
% axis.  [We will see why this is the centripetal force later, but for
% now, take it on faith.] \label{SpinCasting}
% \begin{description}
% \item[(a)] 	Use the above information to show that the curve must
%   satisfy the differential equation 
% $$
% \dfdx{y}{x}=\frac{\omega^2}{g} x
% $$
% and that the curve satisfying this must be a parabola.
% \item[(b)] What will happen to the parabola as $\omega$ increases?  Does this
%   makes sense with the physical problem?
% \end{description}
% \end{embeddedproblem}
% Notice that in the problem, we furnished a formula for the centripetal
% force.  Where did this come from?  Again, we need to extend our
% differentiation rules to beyond polynomials. 
% \newpage{}
 \begin{ProblemSection}
 \begin{myproblem}{}\label{prob:1}
   Assume that $x, y,$ and $z$ all depend on $t.$
      \begin{description}
      \item[(a)] If $\alpha=xy,$ use the Product Rule to find
        $\dfdx{\alpha}{t}.$
      \item[(b)] If $\alpha=xyz,$ use the Product Rule to find
        $\dfdx{\alpha}{t}.$
    \item[(c)] If $\alpha=xyzw,$ use the Product Rule to find
      $\dfdx{\alpha}{t}.$
    \item[(d)] Find a formula for the derivative of five quantities.
    \item[(e)] Find a formula for the derivative of an arbitrary
      number of quantities. (Hint: Look for a pattern in parts (a)
      through (d).)
      \end{description}
  \end{myproblem}

    \begin{myproblem}{}
      Let $\alpha=\frac{xy}{z}.$ Use the Product Rule and the Quotient
      Rule in concert to find each of the following:
      \begin{multicols}{2}
        \begin{description}
        \item[(a)] $\dfdx{\alpha}{x}$
        \item[(b)] $\dfdx{\alpha}{y}$
        \item[(c)] $\dfdx{\alpha}{z}$
        \item[(d)] Assume that $x, y,$ and $z$ depend on $t$ and
          compute $\dfdx{\alpha}{t}.$
        \end{description}
      \end{multicols}
    \end{myproblem}

    \begin{myproblem}{}
      Suppose $x=(1+t)^3,$ $y=(1-t)^3.$ 
      \begin{description}
      \item[(a)] If $z=xy$ show that
        \[
        \dfdx{z}{t} = -6t(1-t^2)^2.
        \]
      \item[(b)] If $z=x/y$ show that
        \[
        \dfdx{z}{t} = \frac{6(1+t)^2}{(1-t)^4}.
        \]
      \end{description}
    \end{myproblem}
%   \begin{myproblem}{}
%     Use the Product Rule to compute $\d y:$
%     \begin{multicols}{2}
%       \begin{description}
%       \item[(a)] $y=x\cdot x$
%       \item[(b)] $y=x^4(x-1)^3$
%       \end{description}
%     \end{multicols}
%   \end{myproblem}

  \begin{myproblem}{}
    Compute $\dfdx{y}{x}$ for each of the following:
    \begin{multicols}{2}
      \begin{description}
      \item[(a)] $\displaystyle y=4.9x^2+15x+6$                                                   
      \item[(b)] $\displaystyle y=3.14$                                                           
      \item[(c)] $\displaystyle y=x^2-2x+4$                                                         
      \item[(d)] $\displaystyle y=\frac{x^2-2x+4}{x}$ 
      \item[(e)] $\displaystyle y=3(x^3-4x+2)$                                                    
      \item[(f)] $\displaystyle y=x^2-\sqrt{x}$                                                   
      \item[(g)] $\displaystyle y=\frac{7x^5-3x^3+x^2}{x}$                                        
      \item[(h)] $\displaystyle y=\frac{x}{x+1}$                                                  
      \item[(i)] $\displaystyle y=(2x-1)(3x^3+5x^2-6x+7)$                                         
      \item[(j)] $\displaystyle y=\left(\sqrt[3]{x}\right)^2-\left(\sqrt{x}\right)^3$             
      \item[(k)] $\displaystyle y=\left(\sqrt[3]{x}\right)^3-\left(\sqrt{x}\right)^2$             
      \item[(l)] $\displaystyle y=\left(\frac{1}{x^2}+\frac{1}{x^3}\right)\left(x^2+x^3\right)$  
      \item[(m)] $\displaystyle y=\left(x^2+x^3\right)\left(x^{-2}+x^{-3}\right)$  
      \item[(n)] $\displaystyle y=\inverse{\left(x^2+x^3\right)}\left(x^{-2}+x^{-3}\right)$  
      \item[(o)] $\displaystyle y=\frac{x^2-1}{x-2}$
      \item[(p)] $\displaystyle y=\frac{x^3-1}{x-3}$
      \end{description}
    \end{multicols}
  \end{myproblem}
  \begin{myproblem}{}
    For each of the following compute $\dfdx{y}{x},$ and find all
    solutions of the equation $\dfdx{y}{x} =0.$
    \begin{multicols}{2}
      \begin{description}
      \item[(a)] $\displaystyle y=x^3(x+2)^4$
      \item[(b)] $\displaystyle y=\frac{x^3}{(x+2)^4}$
      \item[(c)] $\displaystyle y=\sqrt{3x}\inverse{(x-1)}$
      \item[(d)] $\displaystyle y=\frac{\sqrt{3x}}{x-1}$
      \item[(e)] $\displaystyle y=(3x-1)^7$
      \item[(f)] $\displaystyle y=3(x-1)^7$
      \item[(g)] $\displaystyle y=(5x-4)^2(2x+1)^3$
      \item[(h)] $\displaystyle y=\sqrt{2x-1}+\sqrt[3]{2x-1}$
      \item[(i)] $\displaystyle y=\left(2\sqrt{x}-1\right)+\left(2\sqrt[3]{x}-1\right)$
      \item[(j)] $\displaystyle y=\sqrt{1-x^2}$
      \item[(i)] $\displaystyle y=x\sqrt{1-x^2}$
      \item[(j)] $\displaystyle y=\frac{1}{\sqrt{1-x^2}}$
      \item[(k)] $\displaystyle y=\frac{x}{\sqrt{1-x^2}}$
      \end{description}
    \end{multicols}
  \end{myproblem}

  \begin{myproblem}{}
    Find $\dfdx{y}{x}$ and $\dfdx{x}{y}$ for each of the following.
    \begin{multicols}{2}
      \begin{description}
      \item[(a)] $x^2-4xy+y^2=4$
      \item[(b)] $2x^2+xy-y^2=3x$
      \item[(c)] $2x^2+xy=y^2+3x$
      \item[(d)] $y\cos x = x^2+y^2$
      \item[(e)] $x\sec y = x^2-y^2$
      \item[(f)] $(x^2+y^2)^2 = x^2-y^2$
      \end{description}
    \end{multicols}
  \end{myproblem}

  \begin{myproblem}{}
    Compute $\d y$ for each of the following and reduce all fractions
    as much as possible. Look for a pattern.
    \begin{multicols}{2}
      \begin{description}
      \item[(a)] $\displaystyle y=\frac{x^2-1}{x-1}$
      \item[(b)] $\displaystyle y=\frac{x^3-1}{x-1}$
      \item[(c)] $\displaystyle y=\frac{x^4-1}{x-1}$
      \item[(d)] $\displaystyle y=\frac{x^5-1}{x-1}$
      \item[(e)] $\displaystyle y=\frac{x^6-1}{x-1}$
      \item[(f)] $\displaystyle y=\frac{x^n-1}{x-1},$ where $n$ is a
        positive integer.
      \item[(g)] Confirm that the formula you found in part (f) works
        for parts (a), (b), (c), (d), and (e).
      \end{description}
    \end{multicols}
  \end{myproblem}

  \begin{myproblem}{}
    For this problem $x_1, x_2, x_3, \ldots, x_n$ are all
    constants. The variable is $x,$ as usual.
    \begin{description}
    \item[(a)] Show that: $\d (x-x_1)(x-x_2) =
      \left[(x-x_1)+(x-x_2)\right]\d x$
    \item[(b)] Show that: 
      \[\d (x-x_1)(x-x_2)(x-x_3) = \left[(x-x_2)(x-x_3) +
        (x-x_1)(x-x_3) + (x-x_1)(x-x_2)\right]\d x\]
    \item[(c)] Show that: 
        \begin{multline*}
          \d (x-x_1)(x-x_2)(x-x_3)(x-x_4)\\ = [(x-x_2)(x-x_3)(x-x_4) +
          (x-x_1)(x-x_3)(x-x_4)\\ + (x-x_1)(x-x_2)(x-x_4) +
          (x-x_1)(x-x_2)(x-x_3)]\d x
        \end{multline*}
      \item[(d)] Guess what the differential of \(
        (x-x_1)(x-x_2)(x-x_3)(x-x_4)(x-x_5) \) is and show that your
        guess is correct.
      \item[(e)] Guess what the differential of \(
        (x-x_1)(x-x_2)(x-x_3)\ldots(x-x_n) \) is. Can you show that
        your guess is correct?
    \end{description}
  \end{myproblem}

  \begin{myproblem}{}
    Boyle's Law says that if a gas is held at a constant temperature
    the the product of its pressure ($P$) and it volume ($V$) is
    constant. 
    \begin{description}
    \item[(a)] Show that if the temperature is constant then 
\[\dfdx{P}{V} = -\frac{P}{V}.\]
\item[(b)] Suppose the volume of a gas is increasing at $2$ cubic
  inches per second. Find the rate of change of the pressure with
  respect to time, $\dfdx{P}{t}.$ 
    \end{description}
  \end{myproblem}

\end{ProblemSection}




% \section{The Chain Rule}
% \label{sec:chain-rule}
% Suppose for some reason we want to find the number of seconds in a
% typical $365$ day year\index{seconds per year}.  For our purposes we are not so interested in
% the actual number, which will turn out to be approximately
% $\pi\times10^7,$ as we are in the computational process involved. We
% start with the observation that the proportion 
% \[
% \left(60\
%   \frac{\text{seconds}}{\text{minute}}\right)\times\left(60\
%   \frac{\text{minutes}}{\text{hour}}\right)
% \] 
% yields \(3600\
% \frac{\text{seconds}}{\text{hour}}.\) Notice that the ``seconds'' that
% appear in the numerator of the first factor and the denominator of the
% second factor divide out\footnote{Sometimes this is called
%   ``cancelling'' but we will avoid this term as it is the source of
%   many errors.}, leaving only the proportion, seconds per hour. In the next
% step we multiply \(3600\ \frac{\text{seconds}}{\text{hour}}\) by \(24\
% \frac{\text{hours}}{\text{day}}\) to get the number of seconds per
% day. Continuing in this fashion and ``chaining'' the computations
% together we get:
% \[
% \left(60
%   \frac{\text{\small{}seconds}}{\text{\small{}minute}}\right)\times\left(60
%   \frac{\text{\small{}minutes}}{\text{\small{}hour}}\right)\times\left(24
%   \frac{\text{\small{}hours}}{\text{\small{}day}}\right)\times\left(365 \frac{\text{\small{}
%       days}}{\text{\small{}year}}\right)=31,536,000
% \frac{\text{\small{}seconds}}{\text{\small{}year}}.
% \]
% This kind of computation was called the Chain Rule for centuries
% before Calculus came along and it is easy to see why. We can get from
% one kind of proportion to another as long as we can find a  chain of
% proportions where all of intermediate numerators and denominators will
% divide out.
% \begin{wrapfigure}[]{R}{2.5in}
% %\vskip1.2cm{}
%   \includegraphics*[height=1.2in,width=2.5in]{Figures/ChainRule}
% \caption{The Chain Rule: Notice that $y$ depends on $x$ which depends
%   on $t.$}
% \label{fig:ChainRule}
% %\vskip1mm{}
% \end{wrapfigure}

% Now consider the situation depicted in figure~\ref{fig:ChainRule}
% where $y$ depends on $x$ and $x$ depends on $t.$ Suppose we know
% $\dfdx{y}{x},$ the rate of change of $y$ with respect to $x,$ and we
% know $\dfdx{x}{t},$ the rate of change of $x$ with respect to $t,$ but
% we need $\dfdx{y}{t},$ the rate of change of $y$ with respect to $t.$
% Clearly we need to compute
% \begin{equation}
% \dfdx{y}{x}\dfdx{x}{t}=\dfdx{y}{t}.
% \label{eq:ChainRule}
% \end{equation}\index{Chain Rule}
% Since this is a Chain Rule type of conversion, when Calculus was invented
% this particular computation was also called the Chain Rule. As time
% went on the older meaning dropped away so that today this particular
% operation with differentials is the only Chain Rule. 

% We formalize this as a theorem.
% \begin{mytheorem}{\bf The Chain Rule}
%   If $y$ depends on $x,$ and $x$ depends on $t$ then the rate of
%   change of $y$ with respect to $t$ is given by the formula:
% \[
% \dfdx{y}{t} = \dfdx{y}{x}\cdot\dfdx{x}{t}.
% \]
% \end{mytheorem}

% There is more in our symbolism than is at first
% apparent. Consider the following very simple example.
% \begin{example}{}
%   \label{example:cr1}
%   Suppose that $y=3x$ and that $x=5t.$ Compute $\dfdx{y}{t}.$

% This is such a simple problem that we will have no problem computing
% $\dfdx{y}{t}$ directly. Since
% \begin{align*}
%   y&=3x\\
%    &=3(5t)\\
%    &=15t\\
% \intertext{we have}
%  \d y &=15\d t \text{ or}\\
% \dfdx{y}{t}&=15
% \end{align*}
% so the rate of change of $y$ with respect to $t$ is $15.$ 

% But computing $\dfdx{y}{t}$ directly like this hides most of what is
% interesting and all of what will ultimately be useful to us.  

% So let's do it again by the following, admittedly more cumbersome
% method.  Looking again at equation~\ref{eq:ChainRule} we see that we
% need $\dfdx{y}{x},$ which for this problem is $\dfdx{y}{x}=3.$  We
% will also need $\dfdx{x}{t}$ which for this problem is $\dfdx{x}{t}=5.$
% Thus
% \[
% \dfdx{y}{t} = \dfdx{y}{x}\cdot \dfdx{x}{t} = 3\cdot5 = 15.
% \]
% Naturally we get the same answer as before. Only the method of
% computation is different. 
% \end{example}

% Example~\ref{example:cr1} is a little too simple to display all of the
% nuances of the Chain Rule. The next example is slightly more complex.

% \begin{example}{}
%   \label{example:cr2}
% Now suppose $y=3x^2$ and again $x=5t.$ Let's compute $\dfdx{y}{t}.$

% First we need $\dfdx{y}{x}.$ To find this we compute 
% \[\d y = d\left(3x^2\right).\] You may already see that 
% \[\d y = 6x\d x.\] If so, good for you! However, let's walk through
% the details of this calculation just to be sure. 

% Clearly the expression $3x^2$ is the product of $3$ and $x^2$ so by
% the Product Rule we have
% \[
% \d y =3\d (x^2)+x^2\d (3).
% \]
% We saw on page~\ref{thm:ConstantDifferential}
% that the differential of a constant is zero so $\d (3)=0.$ This leaves 
% \[\d y = 3\d (x^2). \]
% If you did problem~\ref{prob:2} you already know that $\d (x^2)=2x\d
% x.$ If you haven't done problem~\ref{prob:2},  the calculation is
% simple enough. Simply observe that $x^2$ is the product of $x$ and $x$
% and apply the Product Rule again. Either way we now have
% \[
% \d y = 3(2x\d x) = 6x\d x
% \]
% or 
% \begin{equation}
%   \label{eq:CRExample2}
%   \dfdx{y}{x} = 6x.
% \end{equation}
% We also need 
% \begin{equation}
%   \label{eq:CRExample2-1}
%   \dfdx{x}{t} =5.
% \end{equation}
% Combining equations~\ref{eq:CRExample2} and~\ref{eq:CRExample2-1}
% gives
% \begin{align}
% \nonumber  \dfdx{y}{t} &= \dfdx{y}{x}\cdot\dfdx{x}{t}\\
% \nonumber  &= (6x)(5)\\
% \intertext{so that}
% \dfdx{y}{t} &= 30x.  \label{eq:CRExample2-3}
% \end{align}

% This seems like the solution to our problem until we think about it
% for a moment. Then we realize that the symbol $\dfdx{y}{t}$ means
% ``the rate of change of $y$\index{with respect to} \underline{with
%   respect to $t.$''} 

% It is this last phrase, ``with respect to $t$,'' which is
% troubling. When we say that $y$ depends on $t$ we are thinking of $t$
% as the independent and $y$ as the dependent variable so the right side
% of equation~\ref{eq:CRExample2-3} should be in terms of $t$ not $x.$

% This is easily handled. Since $x=5t$ we have 
% \[
% \dfdx{y}{t} = 30x = 30(5t) = 150t.
% \]
% \end{example}

% Again, the next example is slightly more complex than those previous.

% \begin{example}{}
%   \label{example:CR3}
% Supose $y=4x^2$ and $x=3t^2.$ We wish to compute $\dfdx{y}{t}.$ As
% before we need both $\dfdx{y}{x}$ and $\dfdx{x}{t},$ since 
% \[
% \dfdx{y}{t} = \dfdx{y}{x}\cdot\dfdx{x}{t}.
% \]

% Thus
% \begin{align*}
%   \d y &= \d(4x^2)\\
%   &= 4\d(x^2)\\
%   &= 4(2x)\d x\\
%   &= 8x\d x \\
% \intertext{or}
% \dfdx{y}{x} &= 8x.
% \end{align*}

% Similarly, 
% \begin{align*}
%   \d x &= \d(3t^3)\\
%   &= 3\d(t^3)\\
%   % &= 3\d(t^2\cdot t)\\
%   % &= 3\left[t^2\d t+t\d(t^2)\right]\\
%   % &= 3\left[t^2\d t+t(2t)\d t\right]\\
%   % &= 3\left[t^2\d t+2t^2\right]\d t\\
%   &= 3\left(3t^2\right)\d t\\
% \intertext{so that}
% \dfdx{x}{t} &= 9t^2. \\
% \intertext{Thus}
% \dfdx{y}{t}&=\dfdx{y}{x}\cdot\dfdx{x}{t}\\
%   &= (8x)(9t^2).\\
% \intertext{As before the symbol $\dfdx{y}{t}$ indicates that we are
%   thinking of $t$ as the independent variable so we substitute
%   $x=3t^2$ to get}
% \dfdx{y}{t} &= 8(3t^2)(9t^2)\\
% \intertext{or}
% \dfdx{y}{t}&=216t^4.
% \end{align*}

% \end{example}
% \begin{ProblemSection}
%   \begin{myproblem}{}
%     Verify the computations in Example~\ref{example:CR3} by direct
%     substitution.
%   \end{myproblem}
% \end{ProblemSection}

\section{Derivatives of Transcendental Functions}
\label{sec:diff-transc-funct}

The differentiation rules we've learned so far are general in the
sense that they are applicable to a wide variety of functions. That is, we
can use them to compute the differential of each of the following:
\begin{align*}
  y&=(x-1)^3\left(\sqrt{x^2-3}\right)\\[2mm]
  y&=\frac{(x-1)^3}{\sqrt{x^2-3}}.
% \intertext{or even}
%   y&=(x-1)^3\sin(x).
\end{align*}
The formula can change, but the differentiation rules still work.

\begin{embeddedproblem}{}
  Compute $\dfdx{y}{x}$ for each of the curves above.
\end{embeddedproblem}

% But wait! Can we really use the rules we've learned so far on this
% last example: \((x-1)^3\sin(x)?\)

% Let's try.

% At first it seems easy enough. Obviously the
% Power Rule can be used. This gives:
% \[
%   \d\left((x-1)^3\sin(x)\right) = (x-1)^3]\underbrace{\d(\sin(x))}_{?} +
%                                   3(x-1)^2\sin(x).
% \]
% But in order to proceed we will need to compute \(\d(\sin(x))\) and we
% do not yet know how to do this. In fact, none of the rules we've
% developed so far will help us with this. We'll have to find another
% way if we want to find \(\d\left((x-1)^3\sin(x)\right).\)

% \begin{wrapfigure}[]{O}{2in}
% %\vskip-.7cm{}
%   \includegraphics*[height=1.5in,width=2in]{Figures-HandDrawn/DiffSin}
% \caption{The Unit Circle}
% \label{fig:UnitCircleSineDiff}
% \vskip1mm{}
% \end{wrapfigure}

As useful as the differentiation rules we've worked out so far are
they are not enough. The fact is that differentiation rules come
in two flavors: those like the ones we've seen already which are
general and work in a wide variety of situations, and the rules which
apply to specific expressions, like $\sin(x),$ which simply have to be
memorized. In this section we will work out the differentiation rules
for all of the trigonometric functions, and all of the inverses of
trigonometric functions.

This may sound like a large undertaking but it turns out that there
are many similarities in the development of each of these. These can
be exploited to make the development easier and, perhaps more
importantly, to make the memorization process a little easier too.

But make no mistake, we are developing basic tools here. Learn
them. {\sc{}Memorize them!} It will be expected hereafter that you have all
of these techniques at your fingertips at all times.

\setcounter{myexample}{0}
\subsection{The Derivative of the Sine and Cosine Functions}
\label{subsec:diff-sine-funt}
Consider the sketch below\\
\centerline{ \includegraphics*[height=1.5in,width=2in]{Figures-HandDrawn/DiffSin}}
where we have a unit circle with an angle, \(x,\) in standard
position. Notice that $\sin(x) = AB.$ If we allow the angle to vary by
an infinitesimal amount, $\d x,$ then the sides of the triangle
$\Delta BDC$ are also infinitesimally small, so that $\d(\sin(x)) =
BD,$ and $\d{x}=CD.$ Moreover, since this in a \emph{unit} circle we
also have $\d x = BC$\footnote{Because when the radius of a circle is
  one, the radian measure of the the central angle is the same as the
  arclength of the subtended angle. This is the whole point of using
  radian measure.}  
\begin{embeddedproblem}{}
  Finally, triangles $\Delta OAB$ and $\Delta DBC$ are similar, but we
  will leave that as an exercise for you.
\end{embeddedproblem}
Therefore from the properties of similar
triangles,
\[
\frac{\d(\sin(x))}{\d x} = \frac{\cos(x)}{1}
\]
or \[\frac{\d(\sin(x))}{\d x} = \cos(x)\]
 which is  our next differentiation formula.

\noindent{Objection! $BC$ is not a straight line, so $\Delta DBC$
  isn't really a triangle. This ``proportion'' is not true.}
\marginpar{We may want to have an ``Objection'' construct which we can
  use to point out logical problems. At least we need to take a moment
  to address this particular objection.}

\begin{myexample}{}
  Supose $y=3x\sin(x).$ What is $\dfdx{y}{x}?$

From the Constant Multiple Rule we see that 
\begin{align*}
\d y &= 3\d(x\sin(x)).\\
\intertext{Next the Product Rule gives}
&=3\left(x\d(\sin(x)) + \sin(x)\d x\right)\\
\intertext{so that}
\dfdx{y}{x} &=3\left(x\cos(x) + \sin(x)\right)
\end{align*}
\end{myexample}

\begin{myexample}{}
  Supose $y=(x^3-2x^2)\sin(x^2-x).$ What is $\dfdx{y}{x}?$

The Product Rule gives
\begin{align*}
  \d y &= (x^3-2x^2)\cos(x^2-x)\d(x^2-x) + \sin(x^2-x)\d(x^3-2x^2) \\
       &= (x^3-2x^2)\cos(x^2-x)(2x-1)\d x + \sin(x^2-x)(3x^2-4x)\d x \\
\intertext{so that}
  \dfdx{y}{x} &= \left[(x^3-2x^)\cos(x^2-x)(2x-1) + \sin(x^2-x)(3x^2-4x)\right]
\end{align*}
\end{myexample}

\begin{myexample}{}
  Supose $y=\sin(2x).$ What is $\dfdx{y}{x}?$

Clearly, 
\[
  \dfdx{y}{x} = \sin(2x)\d (2x) = 2\sin(2x).
\]
\end{myexample}

\begin{myexample}{}
  Supose $y=\sin(x^2).$ What is $\dfdx{y}{x}?$

As before 
\[
\dfdx{y}{x} = \sin(x^2)\d(x^2)\d x = 2x\sin(x^2).
\]
\end{myexample}


\begin{myexample}{}
  Supose $y=\sin(k-x),$ where $k$ is constant. What is $\dfdx{y}{x}?$

We see that
\begin{align*}
  \d y &= \d(\sin(k-x))\\
  &= \cos(k-x)\d(k-x)\\
  &= -\cos(k-x)\d x\\
\intertext{and so}
  \dfdx{y}{x} &=-\cos(k-x).\\
\end{align*}
\end{myexample}

This  last example actually gives us the differential of the
cosine function. We simply take $k=\frac{\pi}{2}$ so that 
\begin{align*}
  \cos(x) &= \sin\left(\pi/2-x\right)\\
\intertext{and}
  \sin(x) &= \cos\left(\pi/2-x\right).\\
\intertext{Thus}
\d(\cos(x))&= \d\left(\sin\left(\pi/2-x\right)\right)\\
          &=\cos\left(\pi/2-x\right)\d\left(\pi/2-x\right)\\
          &=-\cos\left(\pi/2-x\right)\d x\\
\intertext{and so}
\dfdx{(\cos(x))}{x}&=-\sin(x).
\end{align*}

\begin{myexample}{}
  If \(y=\cos\left(x^2+2x\right)\) find $\d y.$
  \begin{align*}
    \d y &= \d\left(\cos\left(x^2+2x\right)\right)\\
         &=-\sin\left(x^2+2x\right)\d\left(x^2+2x\right)\\
         &=-2x\sin\left(x^2+2x\right)\\
\intertext{so that}
    \dfdx{ y}{x} &=-2x\sin\left(x^2+2x\right).
  \end{align*}
\end{myexample}

\begin{embeddedproblem}{}

% \begin{wrapfigure}[]{r}{1.5in}
% \captionsetup{labelformat=empty}
% \centerline{\includegraphics*[height=1.5in,width=1.5in]{Figures/SpinCasting6}}
% \label{fig:SpinCasting6}
% \end{wrapfigure}
\begin{wrapfigure}[]{l}{1in}
\captionsetup{labelformat=empty}
\centerline{\includegraphics*[height=1in,width=1in]{Figures/Centripetal1}}
\label{fig:Centripetal1}
\end{wrapfigure}
\begin{wrapfigure}[]{r}{1in}
\captionsetup{labelformat=empty}
\centerline{\includegraphics*[height=1in,width=1in]{Figures/SpinCasting5}}
\label{fig:SpinCasting5}
\end{wrapfigure}
  Recall that in spin casting [p.~\pageref{SpinCasting}], we gave you
  the formula for centripetal force.  Specifically, for a mass $m$
  revolving in a circle of radius $r$ meters around an axis with an
  angular speed of $\omega$ radians/second, the centripetal force is a force
  directed toward the axis.  This force keeps the mass going in a
  circle instead of traveling off in a straight line.    


The formula given for the magnitude of this force was
$$
mr\omega^2
$$
We will use our knowledge of trigonometric functions to 
derive this.

First, recall that \emph{force=mass$\times$acceleration.}  Thus we want to
show that the centripetal acceleration is given by $r\omega^2.$

Let's put in some coordinates and consider the following diagram.

\begin{description}
\item[(a)] Show that the coordinates of $P$ are given by
$$
x=r\cos\theta, \ \ y=r\sin\theta.
$$
\item[(b)] The velocity vector $v$ is tangent to the circle and is
  composed of the velocity in the $x$ direction $\left(\dfdx{x}{t}\right)$ and the velocity
  in the $y$ direction $\left(\dfdx{y}{t}\right). $ Show that the
  length of v (which will be the speed) is given by
$$
\sqrt{\left(\dfdx{x}{t}\right)^2+\left(\dfdx{y}{t}\right)^2}=r\dfdx{\theta}{t}.
$$
%\centerline{\includegraphics*[height=2in,width=2in]{Figures/Centripetal1}}
\item[(c)] Let's assume that the angular velocity is given by
  $\dfdx{\theta}{t}=\omega$ for some constant $\omega$.  In this case, we only have centripetal
  acceleration represented by the following vector a composed of the
  vertical acceleration $\left(\dfdxn{y}{t}{2}\right)$ and horizontal acceleration
  $\left(\dfdxn{x}{t}{2}\right).$ 
%\centerline{\includegraphics*[height=2in,width=2in]{Figures/AngularVelocity}}
\end{description}
 
\end{embeddedproblem}
Remember all of those identities you had to memorize in trigonometry?
You can reduce the memorization burden a bit by using Calculus. It
turns out that if you differentiate an identity you get another
identity! This is because if two variable quantities are equal then
their differentials must be equal too.

\begin{myexample}{}
  Consider the double angle formula for the sine function:
  \begin{align}
    \sin2x&=2\sin x\cos x.\nonumber
            \intertext{Differentiating gives }
\cos2x\d x &= \cos^2x\d x -\sin^2x\d x\nonumber
\intertext{or}
    \cos2x &=\cos^2x-\sin^2x.\label{eq:CosDouble}
  \end{align}
which is the double angle formula for the cosine function.
\end{myexample}

\begin{embeddedproblem}{}
  What do you think will happen if we differentiate both sides of
  equation~\ref{eq:CosDouble}\ ? Try it and see.
\end{embeddedproblem}

\begin{myexample}{}
  \label{Sin2Cos2}
  We know that \(\sin^2x+\cos^2x=1\) no matter what value $x$ has, so
  we'd expect that differential of \(\sin^2x+\cos^2x\) to be
  zero. Let's check.

  By the Sum Rule
  \begin{align*}
    \d(\sin^2x+\cos^2x) &= \d(\sin^2x) + \d(\cos^2x)\\
\intertext{and by the Power Rule we have}
    &= 2\sin x\d(\sin x) + 2\cos x\d(\cos x)\\
    &= 2\sin x\cos x + 2\cos x(-\sin x)\d x\\
    &= 2(\sin x\cos x - \sin x\cos x)\d x\\
    &= 0.
  \end{align*}
\end{myexample}

\digress{What does it really mean when a differential is zero?}
This example is actually much more interesting, and important, than it
first appears to be. We will digress for a moment to see why.

We have seen that the differential
  of a constant is \emph{necessarily} zero. Is the reverse also true?
That is, if we can show that the differential of some expression
is zero can we conclude that the expression is constant? Symbolically,
does $\d y = 0$ imply that $y=c,$ where $c$ is some constant. 

To be definite, we showed in example~\ref{Sin2Cos2} that
\(\d(\sin^2x+\cos^2x)=0.\) Can\footnote{
Since we already know that \(\sin^2x+\cos^2x=1\)
the question may seem moot. However when examining new ideas or
concepts it is always a good idea to start with a question whose
answer is known. That way we can tell whether our intuition is leading
us astray.} we conclude from this that
\(\d(\sin^2x+\cos^2x)\) is a constant?
First, as a matter of logic, the statement:\\
\centerline{``If $y$ is constant, then $\d y=0.$''}
is \underline{\underline{\underline{\bf \Large not}}} the same as the
statement:\\
\centerline{``If $\d y=0,$ then $y$ is a constant.''}
To see this a more banal example is useful. Consider the statement:\\
\centerline{If it is raining, then I will open my umbrella.''}
This is clearly \underline{not} the same as:\\
\centerline{If my umbrella is open, then it is raining.''}
because I could open my umbrella for any number of reasons in addition
to rain.

So, knowing that\\
\centerline{``If $y$ is constant then $\d y=0"$} is not sufficient
to conclude that\\
\centerline{``If $\d y=0$ then $y$ is constant.''}
Nevertheless when we think about the meaning of the statement $\d y=0$
it certainly seems like it \emph{ought} to be true that $y$ is
constant. After all, this says that infinitesimal changes in $y$ are
\emph{always}  zero. Or, to put it another way, it says that $y$ is
not changing. If $y$ is not changing then $y$ is constant.

On the other hand, consider: \(y=\frac{x}{x-1}-\frac{x-2}{x^2-3x+2}.\)
It is straightforward to show that $\d y=0$ but it is not at all clear
that $y$ is constant.

\begin{embeddedproblem}{}
  Show that $\d\left(\frac{x}{x-1}-\frac{x-2}{x^2-3x+2}\right)=0.$
\end{embeddedproblem}

The point of the foregoing discussion is to convince you that the
statement\\
\centerline{``If $\d y=0$ then $y$ is constant''}
is in fact true, even though we can not yet prove that it is true\footnote{But
we will eventually.}.

Finally, to see why this is important here is a problem that we will
eventually need to be able to solve.
\begin{myexample}{}
  \label{ex:exact-differential}
  Suppose we know that\footnote{This formula is an example of what is
    known as a ``differential equation'' So called because, duh, it
    has differentials in it.}
\[ x\cos x\d y +\sin y\d x =0.\]  Can we reconstruct $y$ in terms of $x$ from
this information?

Obviously the answer is yes, else this example is pointless, but how?
Observe that if we take $z=x\sin y$ then 
\begin{align*}
\d z &= \underbrace{x\cos y\d y +\sin y\d x}_{=0} \text{ and so}\\
 \d z &=0, \text{ and therefore}\\
        z&=c \text{ where $c$ is an unknown constant. And thus}\\
  x\sin y&=c\\
  \sin y &=c/x\\
  y&=\inverse\sin\left(c/x\right).
\end{align*}
The crucial observations to this argument are:
\begin{enumerate}
\item $\d z = x\cos x\d y +\sin y\d x =0$
\item If $\d z = 0$ then $z=c.$
\end{enumerate}
\end{myexample}

The hard part of example~\ref{ex:exact-differential}, of course, was finding $z=x\sin y.$
Eventually you'll learn how to do that yourself, but not today. Also,
it would be helpful to know the value of $c.$ But again, not
today.
\enddigress{End of Digression}
\begin{ProblemSection}
  \begin{myproblem}{}
  Compute the differential, $\d y,$ for each of the following functions:
  \begin{multicols}{2}
    \begin{description}
    \item[(a)] $y=\sin(2x)$
    \item[(b)] $y=\cos(-3x)$
    \item[(c)] $y=\sin(x)\cos(x)$
    \item[(d)] $y=\sin^2(x)$
    \item[(e)] $y=\sin(x^2)$
    \item[(f)] $y=\cos^2(x)$
    \item[(g)] $y=\cos(x^2)$
    \item[(h)] $y=\cos(\sin(x))$
    \item[(i)] $y=3\sin^2(3x)-3\cos^2(3x)$
    \item[(j)] $y=\cos(5x^3-2x^2-3x+12)$
    \item[(k)] $y=\sin^2(7\pi x) + \cos^2(7\pi x)$
    \end{description}
  \end{multicols}
\end{myproblem}
\end{ProblemSection}


\subsection{The Derivatives of the Other Trigonometric Functions}
\label{subsec:diff-other-trig}

Once the differential of $\sin x$ is known, the differentials of the
other trigonometric functions are easily computed. Since our current
goal is simply to develop the tools we will need later we will not
spend any more time on this than necessary.

\subsubsection*{The Tangent and Cotangent}

Observe that by definition \(\tan(x)=\frac{\sin x}{\cos x},\) so it
seems that the Quotient Rule applies:
\begin{align*}
  \d\left(\tan x\right) &= \d\left(\frac{\sin x}{\cos x}\right)\\
                        &= \frac{\cos x \d(\sin x)-\sin x\d(\cos x)}{\cos^2x}\\
                        &= \frac{\cos^2 x\d x +\sin^2 x\d x}{\cos^2x}\\
                        &= \frac{1}{\cos^2x}\d x\\
  \intertext{and since $\sec x = 1/\cos x$ we have}
  \dfdx{\left(\tan x\right)}{x} &= \sec^2x.
\end{align*}

\begin{embeddedproblem}{}
  Use the Quotient Rule to show that 
\[
\dfdx{(\cot x)}{x} = -\csc^2x.
\]
\end{embeddedproblem}

\subsubsection*{The Secant and Cosecant}

Observe that by definition \(\sec x = \frac{1}{\cos x} = \inverse{(\cos
  x)}.\) Differentiating, we have
\begin{align*}
  \d(\sec x) &= \d\inverse{(\cos  x)}.\\
\intertext{By the Power Rule we have}
             &=(-1)(\cos x)^{-2}\d(\cos x),\\
\intertext{so that}
             &= (-1)(\cos x)^{-2}(-\sin x)\d x.\\
\intertext{Simplifying this gives}
  \d(\sec x) &= \frac{\sin x}{\cos^2x}\d x\\
  &= \frac{1}{\cos x}\cdot\frac{\sin x}{\cos x}\d x\\
\intertext{so that, finally we have}
\dfdx{(\sec x)}{x}   &= \sec x\tan x.
\end{align*}

\begin{embeddedproblem}{}
  Show by similar means that \(\dfdx{(\csc x)}{x} = -\csc x\cot x.\)
\end{embeddedproblem}

So the derivatives of all six trigonometric functions are given
below. \underline{\sc Memorize them!}
\index{Differentiation Rules!Trig Functions}
\begin{enumerate}
\item $\dfdx{ (\sin x)}{x} = \cos x $
\item $\dfdx{ (\cos x)}{x} = -\sin x $
\item $\dfdx{ (\tan x)}{x} = \sec^2 x $
\item $\dfdx{ (\cot x)}{x} = -\csc^2 x $
\item $\dfdx{ (\sec x)}{x} = \sec x\tan x $
\item $\dfdx{ (\csc x)}{x} = -\csc x\cot x $
\end{enumerate}


\begin{ProblemSection}
  Compute the differential, $\d y,$ for each of the following functions:
  \begin{multicols}{4}
    \begin{description}
    \item[(a)] $y=\tan(10x)$
    \item[(b)] $y=\cot(\pi x)$
    \item[(c)] $y=\sec(2\pi x)$
    \item[(d)] $y=\csc(-\pi x)$
    \end{description}
  \end{multicols}
\end{ProblemSection}

\subsection{The Inverse Trigonometric Functions}
\label{subsec:inverse-trig-funct}
Recall that if $x$ is in the interval $(-\pi/2,\pi/2)$ and  $\sin x =
y,$ then by definition 
$\inverse{\sin} y = x.$ The function $\inverse\sin x$ is called the
inverse function of the sine, or sometimes the $\arcsin x$ (pronounced
``arc sine of x.'') There are similar inverse functions for each of
the six trigonometric functions. In this section we will derive
differentals  of each of them.

In fact, because of the relationship between a function and its
inverse this is remarkably easy to do in complete generality. A
defining property of function inverses is this: \\[1mm]
\centerline{If $y=f(x)$ then $x=\inverse{f}(y).$}
Differentiating each of the two  equations above we have 
\begin{align}
   \d y = \d\left(f(x)\right)\d x  \label{eq:Dependentx} \\
\intertext{and}
\d x =\d\left(\inverse f(y)\right)\d y.\label{eq:Dependenty}
\end{align}
From these we see that $\dfdx{y}{x}=\d\left(f(x)\right)$ and that 
$\dfdx{x}{y}=\d\left(\inverse f(y)\right)$ from which it is clear that 
\begin{equation}
\d\left(\inverse f(y)\right)=
\frac{1}{\d\left(f(x)\right)}.\label{eq:ReciprocalInverseDeriv}
\end{equation}
In words, it seems that the differential of the inverse of a given
function is the \emph{reciprocal} of the differential of the original
function. This makes perfect sense in simple cases. For example if we
start with the line, 
\[
3x-4y+2=0
\]
then $y$ as a function of $x$ is given by: $y(x) = \frac34x+\frac12$
whereas $x$ as a function of $y$ is given by $x(y) =
\frac43y-\frac23.$
From these it is clear that $\dfdx{y}{x} = \frac34 =
\frac{1}{\frac43}=\frac{1}{\dfdx{x}{y}}$ as we expected. 

However this is a simple case. When the expressions involved are even
slightly more complex we need to be very careful and precise in our
use of notation. We illustrate this in the next example.

\begin{myexample}
  For this example we only consider positive values of $x.$ 

  Suppose $y$ is given as a function of $x$ by the formula: $y=x^2.$
  Then clearly $x$ as a function of $y$ is given by: $x=\sqrt{y}.$

  Clearly $\dfdx{y}{x} = 2x,$ and $\dfdx{x}{y}=\frac{1}{2\sqrt{y}}.$
  But wait! Aren't these supposed to be mutually reciprocal? How is
  $\frac{1}{2\sqrt{y}}$ the reciprocal of $2x?$

  In this example it is not hard to see what is happening. Since
  $y=x^2$ clearly $\sqrt{y}=x$ so that in fact
$$
\dfdx{x}{y}=\frac{1}{2\sqrt{y}} = \frac{1}{2x}
$$
which is the reciprocal of $2x$ as required. 
\begin{embeddedproblem}{}
  Verify that \(f(x) = \frac34x+\frac12,\) and \(\inverse f(x) =
  \frac43y-\frac23\) truly are mutually inverse.
\end{embeddedproblem}\end{myexample}

When the formulas involved are more complex or, as in the next
example, especially when they are more abstract it can be very
confusing and harder to resolve the confusion. We will reserve a
detailed discussion of these issues and their subtleties for a later
chapter. For now we just note we must be very careful differentiating
when inverse functions are involved. The issue has less with our
understanding than with our notation.

\begin{myexample}{{\bf A COMMON MISTAKE; DON'T DO THIS!}}\\
\noindent{}  Clearly then if $\inverse\sin x$ is the inverse of $\sin x$ we have
  \[
  \d\left(\inverse \sin x\right) = \frac{1}{\d(\sin x)}= \frac{1}{\cos
    x\d x}= \frac{\sec x}{\d x},
  \] right?

As we indicated this is a common error so obviously, this is not
correct. But it certainly seems like it ought to be doesn't it? Let's
see what's wrong with it.

% First notice that in equations~\ref{eq:Dependentx}
% and~\ref{eq:Dependenty} the roles of the $x$ and $y$ variables are
% reversed. Indeed, they \emph{must be reversed} because this is really
% what inverse functions are about. If we start with a formula, say
% \[
% 3x-4y+2=0
% \]
% there is no compelling reason to think of $x$ as the independent
% variable and $y$ as the dependent variable. To be sure \emph{one} of
% them needs to be dependent and the other independent but it doesn't
% really matter which is which. Either way we get the same line.

% Choosing $x$ as the independent variable is purely a matter of
% convention. We do this, to be quite honest, because in the English
% alphabet $x$ comes before $y.$ After all we need to choose one and
% there is no compelling reason to choose either.

% If we wanted to graph this equation we would typically, solve for $y$
% in order to get the slope-intercept form:
% \[
% y=\frac34x+\frac12.
% \]
% But suppose the English alphabet were arranged with $y$ before $x.$
% Then we would quite naturally choose $y$ as the independent variable
% and solve for $x:$
% \[
% x=\frac43y-\frac23.
% \]
% So we get\\
% \begin{description}
%   \item[$y$ as a function, $f,$ of $x:$]
%     \(y=f(x) = \frac34x+\frac12,\) and
%   \item[$x$ as the \emph{inverse} function of $y:$]
%     \(x=\inverse f(x) = \frac43y-\frac23.\)
% \end{description}
% We can now see what is wrong with the last example. 
\end{myexample}

Notice that in
equation~\ref{eq:ReciprocalInverseDeriv} we have the variable $y$ on
the left and the variable $x$ on the right side of the equation,
whereas in our example we used only the $x$ variable. This is clearly
not a correct use of our formula, but it is an incredibly easy mistake
to make. When we use general results in mathematics we must make
absolutely certain that our usage is consistent with the statement
of the result in \emph{all} of its particulars.

\subsubsection{The Derivative of the Inverse Sine and Cosine
  Functions}
\label{subsubsec:diff-inverse-sine}
In the following watch the roles of the $x$ and
$y$ variables closely. Since we are differentiating the \emph{inverse}
function we are reversing the roles of $x$ and $y.$ That is, we are
thinking of $y$ as the dependent and $x$ as the independent variable
so it seems natural to write $x = \inverse\sin(y).$ However, there is
no reason not to use $y=\sin(x)$ as long as we keep in mind that we're
thinking of $y$ as  a function of $x$ and that we are trying to
compute $\dfdx{x}{y}$ and  not its reciprocal, $\dfdx{y}{x}.$ 

So if
\begin{align*}
  y&=\sin(x) \text{ then}\\
  \d y &= \cos(x)\d x\\
\intertext{and solving for $\dfdx{x}{y}$ gives}
  \dfdx{x}{y} &= \frac{1}{\cos(x)}.
\end{align*}
This must be correct since if we take the reciprocal of both sides we
get $ \dfdx{y}{x} = \cos(x)$ which we know is correct. But this
formula is almost completely useless for our present purposes.

Since we are thinking of $x$ as the dependent and $y$ as the
independent variables ($x$ is a function of $y$) we would really like
to express $\dfdx{x}{y}$ in terms of $y,$ not $x.$ It is as if, given
$y=x^2$ we had found that $\dfdx{y}{x}=2\sqrt{y}.$ This is true (since
$\sqrt{y} =x$) but almost useless.

% we see from equation~\ref{eq:Dependenty} that
% \begin{align*}
%   \d x &= \d\left(\inverse\sin y\right)\d y\\
% \intertext{and from equation~\ref{eq:ReciprocalInverseDeriv} we have }
%   &= \frac{1}{\d(\sin x)}\d y\\ 
% \intertext{so that}  
% \d x &= \frac{1}{\cos x}\d y.\\ 
% \end{align*}
% Since $x = \inverse\sin(y)$ this becomes
% \begin{equation}
% \d x = \frac{1}{\cos\left(\inverse\sin y\right)}\d y,\label{eq:InvSinDiff}
% \end{equation}
% and we are done.

% Well, almost.

% The formula we have above is correct, but it isn't very useful as it
% is because (1) as mentioned before we have the roles of $x$ and $y$
% reversed, and (2) it isn't easy to deal with an expression like
% $\cos\left(\inverse\sin y\right).$ Neither of these is ``wrong'' in
% any general sense. They just make things a little difficult to work
% with so this is not the form you will need to memorize.

% The first issue identified above, the switching of the roles of $x$
% and $y,$ is easily handled. We just switch them back. That is, if 
% \[
% y=\inverse\sin(x)
% \]
% then 
% \begin{equation}
%   \label{eq:InvSinDiff2}
%   \d y = \frac{1}{\cos\left(\inverse\sin x\right)}\d x.
% \end{equation}

So our problem now is to express the right side of the formula
$\dfdx{x}{y} = \frac{1}{\cos(x)}$ in terms of the dependent variable
$y$ instead of the dependent variable $x.$ This is entirely an
exercise in trigonometry as we now show.

First observe that since $y=\sin(x)$ we have 
$
x=\inverse\sin y
$
so that 
$$
\dfdx{x}{y} = \frac{1}{\cos\left(\inverse\sin y\right)}.
$$
so apparently all we have to do is figure out what
$\cos\left(\inverse\sin y\right)$ is in terms of $x.$ 

This may seem a daunting task but this is an illusion caused mostly by
unwieldy notation. That is, it just looks scary. But it would be silly
to let our own notation frighten us. Besides, if we take it one piece
at a time it isn't so bad.

If $y=\sin(x)$ then $y$ is a number and $x$ is an angle and we can
visualize this as in the following sketch:

\centerline{\includegraphics*[height=2in,width=2in]{Figures/arcsin1}}

Since $x$ is the dependent variable it makes sense to read the formula
$x=\inverse\sin y$ as ``$x$ is the angle whose sine is $y$.'' Since we
want to find the $\cos(x)$ we see from our sketch and the Pythagorean
Theorem that 
$$
\cos(x)=\cos\left(\inverse\sin y\right) = \sqrt{1-y^2}.
$$
Therefore
$$
\dfdx{x}{y} = \frac{1}{\sqrt{1-y^2}}.
$$
\begin{mynotation}
  Again, this is correct and we appear to be finished. Unfortunately
  we have one more small point to make. This one will feel like we are
  just going out of our way to make things difficult but we
  aren't. This last change is not about understanding the concepts, it
  is just about how

  The inverse sine is an elementary function and equations using it
  will most commonly be written like this
$$
y = \inverse\sin x,
$$
with $x$ taking the role of the independent variable as usual.  

However, unless there is a compelling reason to use $y$ as the
independent variable the custom is to use $x.$ In our derivation of
the derivative of the inverse sine we had a compelling reason to use
$y.$ Having completed the derivation we now revert to the customary
usage.
\end{mynotation}

% Since we know what is coming we will
% switch the roles of $x$ and $y$ up front; 
We have five more trigonometric functions to invert and differentiate
so let's get going. Fortunately the derivations are all essentially
the same so we will abbreviate the discussion.

Suppose $y=\inverse\cos x$ so that $x=\cos y.$ Then
\begin{align*}
  \d x &= \frac{1}{\d(\cos y)}\d y\\
  &= \frac{1}{-\sin y}\d y\\
\intertext{and thus,}
  \dfdx{x}{y} &= \frac{-1}{\sin \left(\inverse\cos y\right)}.\\
\end{align*}
From the sketch below

\centerline{\includegraphics*[height=2in,width=2in]{Figures/arccos1}}
it should be clear that $\sin\left(\inverse\cos x\right)=
\sqrt{1-y^2},$ so
$$
  \dfdx{x}{y} = \frac{-1}{\sqrt{1-y^2}}.
$$
So again switching to the customary usage, if $y=\inverse\sin x$ we have $\dfdx{y}{x} = \frac{-1}{\sqrt{1-x^2}}.$ 

\subsubsection{The Other Inverse Trigonometric Functions}
\label{subsubsec:other-inverse-trig}

Let $y=\inverse\tan x$ so that $x=\tan y.$ Then
\begin{align*}
  \d x &= \d(\tan y)\\
       &= \sec^2 y\d y\\
\intertext{from which}
\dfdx{y}{x}  &= \frac{1}{\sec^2y}.\\
\intertext{From the identity $1+\tan^2y=\sec^2y$ we see that}
\dfdx{ y}{x}&= \frac{1}{1+\tan^2y}\\
\intertext{and thus}
  \dfdx{ y}{x} &= \frac{1}{1+x^2}.
\end{align*}

\begin{embeddedproblem}{}
  Show that if $y=\inverse\cot x,$ then  $\dfdx{ y}{x} = \frac{-1}{1+x^2}.$
\end{embeddedproblem}

Let $y=\inverse\sec x$ so that $x=\sec y.$ Then
\begin{align*}
  \d x &= \d(\sec y)\\
  \d x &= \sec y\tan y \d y\\
  \dfdx{ y}{x}&= \frac{1}{\sec\left(\inverse\sec x\right)\tan y}.\\
  &= \frac{1}{x\tan y}\d x.\\
\intertext{From the identity $1+\tan^2y=\sec^2y$ we see that $\tan y =
  \sqrt{\sec^2y-1}.$ Thus}
  &= \frac{1}{x\sqrt{\sec^2y-1}}\d x.\\
  &= \frac{1}{x\sqrt{\left[\sec\left( \inverse\sec x\right)\right]^2-1}}\d x.\\
\intertext{so that}
  \d y &= \frac{1}{x\sqrt{x^2-1}}\d{x}.
\end{align*}


\begin{embeddedproblem}{}
  Show that if $y=\inverse\csc x,$ then  $\d y = \frac{-1}{x\sqrt{x^2-1}}\d x.$
\end{embeddedproblem}

Actually, we've played a little fast and loose on these last two\marginpar{Note to self: We should include some problems that bring this issue into focus.}
formulas. Because of the domain restrictions necessary to invert the
secant and cosecant functions the correct formula for the differential
of the inverse secant function is
\[
\d (\inverse\sec x) = \frac{1}{\abs{x}\sqrt{x^2-1}} \d x,
\]
and similarly for the inverse cosecant. But we have enough on our
plate right now so we won't overly concern ourselves with this at 
the moment. All of the formulas given below are correct.
\underline{\sc Memorize them!}\index{Differentiation Rules!Inverse Trig Functions}
\begin{enumerate}
\item $\displaystyle\dfdx{ (\inverse\sin x)}{x} =  \frac{1}{\sqrt{1-x^2}} $
\item $\displaystyle\dfdx{ (\inverse\cos x)}{x} = \frac{-1}{\sqrt{1-x^2}}  $
\item $\displaystyle\dfdx{ (\inverse\tan x)}{x} = \frac{1}{1+ x^2} $
\item $\displaystyle\dfdx{ (\inverse\cot x)}{x} = \frac{-1}{1+ x^2}$
\item $\displaystyle\dfdx{ (\inverse\sec x)}{x} = \frac{1}{\abs{x}\sqrt{x^2-1}} $
\item $\displaystyle\dfdx{ (\inverse\csc x)}{x} = \frac{-1}{\abs{x}\sqrt{x^2-1}} $
\end{enumerate}



\section{Newton's Method}
\label{sec:newtons-method}

If $x$ and $y$ are variable quantities then we have interpreted the
symbols $\d x$ and $\d y$ to mean infinitely small changes in $x$ and
$y.$ This is analogous to using $\Delta x$ and $\Delta y$ to represent
``the change in $x$'' and ``the change in $y.$'' And just as the ratio
$\frac{\Delta y}{\Delta x}$ represents the slope of a straight line we
interpret the ration $\dfdx{y}{x}$ to be the slope of the graph of $y$
as a function of $x$ or, what comes to the same thing, the slope of
the line tangent to that graph.

But if that graph is not a straight line then obviously its slope
changes from point to point. When the particular point of tangency,
say $x=a,$  is
important we will use the notation 
$$
\left.\dfdx{y}{x}\right|_{x=a}
$$
to indicate this. Otherwise we will use $\dfdx{y}{x}$ as before.

Thus if $y=x^3$ then
\begin{align*}
  \left.\left.\dfdx{y}{x}\right|_{x=2}=3x^2\right|_{x=2}=3\cdot(2^2)
  &= 12.
\intertext{Or if $y=\sin(6x)$ then}
  \left.\left.\dfdx{y}{x}\right|_{x=\frac{\pi}{2}}=6\cos(6x)\right|_{x=\frac{\pi}{2}}=6\cos(3\pi)&=-6
\end{align*}

Now that we can use the derivative to find the equation of the line
tangent to a curve a natural question to ask is, ``What use is this?''


One application of the derivative that is immediate and
extraordinarily useful is the solution of equations in one
variable. Unfortunately, it is difficult to convey to modern students
just how useful this is because modern technology renders it
completely trivial. For example, most mathematical software these days
will accept  the equation $\cos x = x$ as input and give the
\underline{approximate} solution $x=0.739085$ at the click of a mouse
button. 

The problem thus appears to be simple.

But imagine yourself back in the late 17th century for a moment. The
only technology available for computation is paper and
pencil\footnote{Actually, modern pencils were invented in the late
  18th century. The point is that computations were done by
  hand.}. How would you solve this problem? How would you even
generate an approximate solution?

One possibility is to graph $y=x$ and $y=\cos x$ on the same set of
axes\\
\centerline{\includegraphics*[height=2in,width=3in]{Figures/cosxeqx}}
and look for the value of $x$ where the two graphs intersect. This
seems like a good idea until we realize that accurately graphing even
simple equations was an almost insurmountable task in those days
too. We clearly used modern technology to draw the graph above. Doing
this by hand would have been very difficult.

Here's another idea. If we rearrange the equation just a little we get 
$$
\cos x -x =0.
$$
You wouldn't think such a simple change would be helpful but it
is. For now, instead of looking for the point where two graphs
intersect we are searching for value of $x$  where the graph of the
curve $\cos x - x$ crosses the $x-$axis. This is called a root of
$\cos x -x.$ 

Now look at the graph of $y=\cos x -x.$\\
\centerline{\includegraphics*[height=2in,width=3in]{Figures/cosx-x}}
Clearly this crosses the $x-$axis somewhere between $0.5$ and $1.0.$
Moreover if we draw the line tangent to $\cos x -x$ at $x=0.5$ \\
\centerline{\includegraphics*[height=2in,width=3in]{Figures/cosx-x2}}
we see that the tangent line crosses the $x-$axis at very nearly the
same place that $\cos x- x $ does. So this $x$ value, whatever it is,
is actually a pretty good approximation to the root of $\cos x -x.$ If
we zoom in on this part of our graph this is easy to see:\\
\centerline{\includegraphics*[height=2in,width=3in]{Figures/cosx-x3}}
Let's call this  first approximation of the root, $r_1.$

But wait! If we are calling that the \underline{first} approximation
then we are clearly planning a second, and presumably better, such
approximation. Can you see how we might calculate $r_2?$

We have the $x$ coordinate of our first approximation, $r_1$ so the
$y$ coordinate is easy to compute. It is $\cos(r_1)-r_1.$ We now have
both coordinates of a point on our curve which is near to the
root. If we now find the line tangent to the graph of $\cos x -x $ at
this point we will have the following picture:\\
\centerline{\includegraphics*[height=2in,width=3in]{Figures/cosx-x4}}
Of course this last picture is useless because at this scale the graph
and its tangent line are indistinguishable. 

But this is a good thing! If the curve and its tangent are essentially
identical then the tangent line will cross the $x-$axis at very nearly
the same place as the curve. In other words, $r_2$ will be an
extremely good approximation to the solution of $\cos x -x =0.$ 

The pattern should be clear. 
\begin{enumerate}
\item We start by finding a point ``near'' the root we are seeking. In
  view of what follows it seems reasonable to call this $r_0$ our
  zeroth approximation to the root we seek.
\item We use this to generate $r_1,$ our first approximation to the
  root.
\item We then use $r_1$ to generate $r_2,$ our second such
  approximation.
\end{enumerate}
Clearly, we don't have to stop there. If we wanted a more accurate
approximation we could use $r_2$ to generate an $r_3,$ and so on. We
stop when our approximation is accurate enough for our purposes.

\begin{myexample}{}
  Let's run through this calculation quickly. Our initial
  approximation was $r_0=0.5.$ The point on our curve where we need
  the first tangent line is $(0.5), \cos(0.5)-0.5) \approx (0.5,
  0.378)$ and the slope of the tangent line at this point is:
  $-\sin(.05)-1\approx -1.48.$ 

  Thus the equation of the tangent at this point is
  \begin{align*}
    y-.378&=-1.48(x-.5)\\
\intertext{or}
    y&=-1.48(x-.5)+.378.\\
\intertext{Our first approximation to the root, $r_1$ will be the $x$
  coordinate where this line crosses the $x-$axis. That is, where
  $y=0.$ Solving this we have}
    0&=-1.48(x-.5)+.378\\
    \frac{.378}{1.48}+.5&=x \text{ or}\\
x&\approx 0.755.
  \end{align*}
\end{myexample}

\begin{embeddedproblem}{}
  Find $r_3$ for the previous example and compare it to the answer
  $0.739085$ we found by using modern technology.
\end{embeddedproblem}

Clearly this paper-and-pencil procedure is not as simple as handing
the problem off to your favorite computational software, so a natural
question to ask is, ``Why bother? Why should we learn this?''

The answer is that whatever software you end up using to generate an
approximate solution to your problem will be performing either the
computations above or something very like them, and these computations
{\bf will not always work.} If you hand this off to software without
any understanding of what the software is doing you run the risk of
getting wrong answers. More importantly, if you simply trust the
software without understanding it you run the risk of {\bf believing }
the wrong answer when you get it. Depending on what you are computing
this could mean anything from a minor annoyance (if you are
calculating $\sqrt{2}$ just for fun), to a deadly disaster (if you are
designing a control procedure for a self-driving car).

The procedure we have outlined was originally developed by Isaac
Newton in the late 17th century and is thus usually called
\underline{Newton's Method.} To understand it fully we need to get
organized.

Suppose we have some curve $y=f(x)$ whose root or roots we would like
to compute\footnote{For our purposes the symbol $f(x)$ just means
  ``some formula involving the variable $x$ where, for example, $f(1)$
  is the value of $y$ when $x=1.$ You may have seen this notation used
  with a deeper meaning.}. Newton's Method is as follows:
\begin{enumerate}
\item Find an initial guess, $r_0$ for the root. It is best if this
  guess is as close to the actual root as we can make it.
\item Find the equation of the line tangent to $y=f(x)$ and solve for
  $x$ when $y=0.$ The tangent line is given by:
$$
y-f(r_0) = \left(\left.\dfdx{y}{x}\right|_{x=r_0}\right)(x-r_0)
$$
so when $y=0$ we have
$$
x=r_0-\frac{f(r_0)}{\left.\dfdx{y}{x}\right|_{x=r_0}}.
$$
This is our second approximation so we set $r_1=x.$
\item Repeat using the most recent root approximation until the
  desired accuracy is attained. 
\end{enumerate}

Even more succinctly we have:\\
\centerline{\bf \Large \underline{Newton's Method}}
\begin{enumerate}
\item Choose $r_0.$
\item For $n=1, 2, 3, \ldots$ compute
$$
r_{n+1}=r_n-\frac{f(r_n)}{\left.\dfdx{y}{x}\right|_{x=r_n}}.
$$
\end{enumerate}

\begin{myexample}
  Newton's Method gives us an easy way to compute approximations to
  irrational numbers such as $\sqrt{2}$ or $\sqrt[5]{7}$ for
  example. All we need to do is find a function which has the given
  number as a root. For example, $\sqrt{2}$ is a root when $f(x) =
  x^2-2$ and $\sqrt[5]{7}$ is a root when $f(x) = x^5-7.$
\end{myexample}
\begin{embeddedproblem}
  Compute $\sqrt{2}$ and $\sqrt[5]{7}$ to $5$ decimals.
\end{embeddedproblem}

When Newton's Method works it usually works extremely well. That is,
it will usually find the root of a function in just a few iterations
if the initial guess is reasonably close. This made it extremely
useful in the $17$the century when such computations were done by
hand. Indeed, it computes the
square root of a number pretty quickly even if the initial guess is
very bad. For example in the following graph the dotted curve is the
graph of $f(x)=x^2-2.$ We start with an initial guess of $r_0=5$
(obviously a terrible guess) and
shoot the red tangent line down to the $x-$axis to find the next
guess, $r_1=2.7$ which is better but still terrible. Repeating we
generate the green tangent line at $(r_1,f(r_1))$ which crosses the
$x-$axis at $r_2= 1.72.$ Finally we generate the orange tangent line
at $(r_2,f(r_2))$ which crosses the $x-$axis at $r_3=1.44$ which is
correct to one decimal.
\centerline{\includegraphics*[height=2in,width=4in]{Figures/NewtonsMethod1}}
If we continue one more iteration (not shown) we get $r_4=1.1414.$
which is correct to four decimals. 
\begin{embeddedproblem}{}
  Compute $\sqrt{3}$ using the initial guess, $r_0=20.$
\end{embeddedproblem}

The problem with Newton's Method, as we mentioned, is that it doesn't
always work. In fact, it can fail in two distinct ways. One is quite
spectacular and you can see from the computations that it is
failing. The other can be quite subtle. That is, it can converge, but
not to the number we seek.


\noindent{}{\bf \underline{Spectacular Failure}}\\
The standard example of this is the following.
\begin{embeddedproblem}{}
  The only root of the function $f(x)=x^{1/3},$ is zero.  
  \begin{description}
  \item[(a)] Use Newton's Method with the initial guess, $r_0=1$ to
    see if it converges to zero. (It won't.)
  \item[(b)] Write down the iteration step (step 2) from Newton's
    Method for this function. Use this to explain why the method will
    not converge no matter what initial guess is used.
  \end{description}
\end{embeddedproblem}\\
Obviously we don't need to use Newton's Method to compute this. The
point of this example is that Newton's Method will not find the root
no matter what we do. Instead it will continue to generate larger and
larger (but alternately positive and negative) ``approximations'' to
$\frac{\pi}{2}.$

\noindent{}{\bf \underline{Subtle Failure}}\\
The best way to demostrate this is with an example.
\begin{myexample}
  Suppose we wish to compute $\frac{\pi}{2}$ by finding the first
  positive root of $\cos(x).$ If we start with an initial guess of
  $r_0=.1$ (not a great first guess, but not horrible either) we then
  get $r_1=10.07,$ which is certainly seems like it might be  a
  problem since $\frac{\pi}{2}\approx 1.7.$ If we ignore this and
  continue we get $r_2= 11.4,$ and $r_3=10.97.$

  What's going on here? On the one hand the numbers seem to be
  converging, but they are converging to the wrong answer. The
  following figure shows what the difficulty is. 
  \centerline{\includegraphics*[height=2in,width=4in]{Figures/NewtonsMethodFail1}}
  In a nutshell, our initial guess was too far away from the
  root. The slope of the tangent line at $(0.1, \cos(0.1))$ is
  $\sin(0.1)\approx -0.1$ which means that the tangent line decreases
  from left to right, but also that it is very shallow. Thus the
  tangent line crosses the $x-$axis at about $r_1=10.07,$ very far from
  the root we seek. After that the approximations will settle in on
  the root at $7\pi/2,$  but the damage has already been done. We've
  found the wrong root.

  We call this a subtle failure because, although it is glaringly
  obvious what goes wrong when we draw a picture of each
  successive approximation the fact is that most of the time
  there will be no pictures. In fact, most of the time these
  computatiuons are done in software from which the only thing that
  comes out is the final approximation to the root. Notice that,
  unlike the spectacular failure above there is nothing in the
  calculations being performed that could be detected in software to
  let the human in charge know that things have gone wrong. If this is
  not understood there is a real risk that that a ridiculous answer
  could be accepted as correct.
\end{myexample}
It is thus important that the initial guess be close enough to the
root we seek that the iterations will converge to the desired root.
This issue is particularly acute when a function has two roots which
are very close together, for example if
$f(x) =
\frac{1}{10}(10x^2-21x+11)(x-0.05)(x^3+7).$%(x-1)(x-11/10)(x^2+7).$
\begin{embeddedproblem}{}
  Find approximations to all real roots of $f(x) =
  \frac{1}{10}(10x^3-21x^2+11x).$
\end{embeddedproblem}

\begin{ProblemSection}
  \begin{myproblem}{}
    Use Newton's Method to compute the root of each of the following
    functions, using the given initial guess. 
    \begin{multicols}{3}
      \begin{description}
      \item[(a)] $f(x)=x-2,\\ \text{Guess: } a=5$
      \item[(b)] $f(x)=x^2-2,\\ \text{Guess: } a=5$
      \item[(c)] $f(x)=x^3-2,\\ \text{Guess: } a=5$
      \item[(d)] $f(x)=x-5,\\ \text{Guess: } a=2$
      \item[(e)] $f(x)=x^2-5,\\ \text{Guess: } a=2$
      \item[(f)] $f(x)=x^3-5,\\ \text{Guess: } a=2$
      \item[(g)] $f(x)=x+7,\\ \text{Guess: } a=0$
      \item[(h)] $f(x)=x^2+7,\\ \text{Guess: } a=0$
      \item[(i)] $f(x)=x^3-7,\\ \text{Guess: } a=0$
      \end{description}

    \end{multicols}
  \end{myproblem}

  \begin{myproblem}{}
    Suppose that a cubic polynomial has three real roots, that two
    of them, $r_1$ and $r_2$ are known, and we wish to approximate the
    third. Show that if we take the average of $r_1$ and $r_2$ as our
    initial guess, Newton's Method will find the third root exactly in
    one iteration.
  \end{myproblem}
\end{ProblemSection}


\section{Euler's Method}
\label{sec:eulers-method}
If we have the slope of a curve at a particular point we have
\emph{local} information. That is, if we know the slope \emph{here} we
can reasonably assume that it won't change -- much -- as long as we
don't move too far. But tells us nothing about the slope \emph{there}
if ``there'' is very far away. Newton's Method exploits this. It works
because the roots of functions is a local property, not a global
property.

On the other hand if we know the slope of a graph at \underline{every}
point then we know a great deal about the graph. We know enough, in
fact, that we can (almost) completely reconstruct the graph just by
knowing its derivative.

For example we know that if $y=x^2$ then $\dfdx{y}{x}=2x.$ Thus the
slope of our graph at any point on the graph is twice the value of the
$x-$coordinate. So
\begin{align*}
  \left.\dfdx{y}{x}\right|_{x=-4}&=2\cdot(-4)=-8,\\
  \left.\dfdx{y}{x}\right|_{x=-3}&=2\cdot(-3)=-6,\\
  \left.\dfdx{y}{x}\right|_{x=-2}&=2\cdot(-2)=-4,\\
  \left.\dfdx{y}{x}\right|_{x=-1}&=2\cdot(-1)=-2,\\
  \left.\dfdx{y}{x}\right|_{x=-1/2}&=2\cdot(-1/2)=-1,\\
\end{align*}
\begin{align*}
  \left.\dfdx{y}{x}\right|_{x=-0}&=2\cdot0=0,\\
  \left.\dfdx{y}{x}\right|_{x=1/2}&=2\cdot(1/2)=1,\\
  \left.\dfdx{y}{x}\right|_{x=1}&=2\cdot1=2,\\
  \left.\dfdx{y}{x}\right|_{x=2}&=2\cdot2=4,\\
  \left.\dfdx{y}{x}\right|_{x=3}&=2\cdot3=6,\\
  \left.\dfdx{y}{x}\right|_{x=4}&=2\cdot4=8,\\
\end{align*}

The graph of $y=x^2$ along with a few of its tangent lines indicates
how the function might be reconstructed from knowledge of
$\dfdx{y}{x}.$\\[2mm]
\centerline{\includegraphics*[height=3in,width=3in]{Figures/parab-envelope}}

The problem is that to reconstruct the whole graph we need
\underline{all} of the tangent lines. That is a lot of information.
We would normally not have that much. 

On the other hand, if we lower our expectations perhaps we can use the
local information about the slope of the tangent line to
\emph{approximate} (rather than reconstruct completely) the graph.

% It is frequently the case in science and engineering that we know the
% rate at which some quantity is changing and need to reconstruct the
% quantity itself. The following mundane example is illustrative: The
% speedometer on your car tells you how fast you're going (that is, the
% rate of change of your position) but not where you are (position).

An example is in order. Suppose we know that some curve passes through
the point $(0,1)$ and that at every point on the curve the relation
between the slope and the curve is given by the formula
$$
\dfdx{y}{x} = y.
$$
We can interpret this either as 
\begin{description}
\item[Newtonian interpretation: ] The rate of change of the $y-$coordinate is
  exactly equal to the $y-$coordinate itself.
\item[Leibnizian interpretation:] At every point on
  the curve the slope of the tangent line is given by the
  $y-$coordinate.
\end{description}
% We would like to reconstruct the curve from this
% information. Unfortunately we don't yet have all of the tools we need
% to accomplish this. So we will have to settle for an approximate
% reconstruction. 

Notice that the given information tells us that our curve passes
through the point $(0,1)$ with a slope equal the $y-$coordinate: $1.$
So the tangent of our curve is given by the graph below.\\
\centerline{\includegraphics*[height=2in,width=2in]{Figures/exp-envelope1}}
This is not a lot to work with, but let's not set our sites too
high. 

Near the point $(0,1)$ the tangent line and the curve are going to be
nearly indistiguishable. Moreover we have enough information (a slope
and a point) to write down the equation of the tangent line at
$(0,1).$ That is, we can find any point on this line. So if we stay on
the tangent line and move the $x-$coordinate just a little bit away
from zero then the corresponding $y-$coordinate \underline{on the line} will be
close to the $y-$coordinate \underline{on the curve}.

The equation of our tangent line is $y=x+1$ so when $x=0.1$ $y=1.1.$
So now we have two points on our curve: $(0,1)$ and $(0.1, 1.1).$ 

Ok, $(0.1, 1.1)$ isn't really on the curve, but it's close. Since
these two points are so close together the curve between them will be
practically a straight line and the line tangent at $(0.1, 1.1)$ will
have a slope of approximately $y=1.1.$ Adding this to our previous
graph we have\\
\centerline{\includegraphics*[height=2in,width=2in]{Figures/exp-envelope2}}
So far, so good. 

Now suppose we move to the right again, to $x=0.2$ this time. It won't
help to stay on our tangent line; that only approximates our curve
near the point of tangency. To stray too far from that is useless.

What we need now is a new tangent line at the point on our curve where
$x=0.1.$ If we had that we could just repeat the previous computation
at this point. But, of course, we can't find that line since we have
neither the slope of that line nor the $y-$coordinate of that
point. All we have is the approximation: $(0.1, 1.1).$

But wait. Can we use these approximate coordinates to get an
approximate tangent line? 

Of course we can! It's easy. We just need to pretend that the point
$(0.1, 1.1)$ is actually a point on our curve. In that case the curve
passes through this point with a slope of $\dfdx{y}{x} = y = 1.1.$  
 
If we now simply repeat this process, moving to the right again to
$x=0.2,$ we get the point $(0.2, 1.21).$ Continuing in this fashion
we compute the points:
\begin{align*}
  &(0.3,1.331)\\
  &(0.4,1.4641)\\
  &(0.5,1.61051)\\
\end{align*}
and so on.

If we plot these points and the tangent lines of each the graph
will look something like this:\\
\centerline{\includegraphics*[height=2in,width=2in]{Figures/exp-envelope3}}
An approximation to our curve will be the pieces of each tangent line
between $x=0$ and $x=.1,$ between $x=.1$ and $x=.2$ and so on.\\
\centerline{\includegraphics*[height=2in,width=2in]{Figures/exp-piecewise-linear}}


The procedure we have just outlined is known as {\bf Euler's Method};
named for the great eighteenth century mathematicial Leonard
Euler. The problem of recreating a curve given it's $y-$coordinate and
slope at a single point is called an {\bf Initial Value Problem}, or
IVP for short.

We need to recognize that this is not the actual graph of the curve
which solve our IVP:
$$
\dfdx{y}{x} =y, \ \ y(0)=1.
$$ 
The very first point we computed
was an approximation, and at each step we used the previous
approximation to compute the next, so it would be  miraculous if
we had actually found \underline{exactly} the points on the graph of
our curve.

Still, it seems reasonable to believe that the curve we've found at
least resembles the desired curve. If it does then we can draw some
conclusion about this curve based on our graph. For example this graph
clearly continues to increase as we tak greater values of $x.$

\begin{embeddedproblem}{}
  If $y=\sin(x)$ then
$$
\dfdx{y}{x}=\cos(x), \text{ and } y(0)=0.
$$

Use the method we've just developed to approximate the graph of
$y=\sin(x)$ from this Initial Value Problem. The point of this
exercise is to convince you that Euler's Method really does give a
qualitative sense of the shape of the curve, even if all of the
numbers are only approximations.
\end{embeddedproblem}

\begin{ProblemSection}
  \begin{myproblem}{}
    Construct curves satisfying the IVPs given:
  \end{myproblem}
\end{ProblemSection}
\section{Higher Order Derivatives: Derivatives of Derivatives}
\label{sec:high-order-deriv}

We have seen that if, for example, $y=x^3,$ then $\dfdx{y}{x} = 3x^2.$
If we differentiate the expression $3x^2$ we get $6x,$
but what should we call $6x$ as it relates to $y?$ Since we got $6x$
by differentiating $y=x^3$ twice the obvious choice is to call $6x$
the \emph{second derivative} of $y=x^3.$ Clearly then $3x^2$ is the
\emph{first derivative} and $6$ the \emph{third derivative} of
$y=x^3.$

All well and good, but what \underline{notation} should we use when
writing these things? 

For better or worse the notation that has been invented and become
standard is this:
\begin{align*}
  \text{ If } y &= x^3 \text{ then}\\
  \dfdx{y}{x}=\dfdx{}{x}(y) &= 3x^2 \text{ as before, and}\\
  \dfdx{}{x}\left(\dfdx{y}{x}\right)=\dfdxn{y}{x}{2} &= 6x. \text{
                                                       Notice the
                                                       placement of
                                                       the exponents.}\\
\end{align*}
Differentiating one more time we get
$$
  \dfdx{}{x}\left(\dfdxn{y}{x}{2}\right)=\dfdxn{y}{x}{3} = 6.
$$
Once more and we get $\dfdxn{y}{x}{4} = 0.$ Obviously we can continue
this process as many times as we would like, although for this
particular example since the fourth derivative and every derivative
thereafter is zero this would be rather dull. The following example is
more interesting.

\begin{myexample}
  Consider the expression $y=\sin(x).$ As we've seen $\dfdx{y}{x}=
  \cos(x).$ Continuing we have $\dfdxn{y}{x}{2} = -\sin(x),$
  $\dfdxn{y}{x}{3} = -\cos(x),$ and finally $\dfdxn{y}{x}{4} =
  \sin(x).$ Thereafter the pattern of derivatives clearly repeats.
\end{myexample}

\begin{embeddedproblem}{}
  \begin{description}
  \item[(a)] Compute $\dfdxn{}{x}{34}\left(\sin(x)\right),$   $\dfdxn{}{x}{17}\left(\sin(x)\right),$ $\dfdxn{}{x}{1999}\left(\sin(x)\right),$ and
    $\dfdxn{}{x}{103}\left(\sin(x)\right).$ 
  \item[(b)] Repeat part (a), but using the $\cos(x).$
  \end{description}
\end{embeddedproblem}

We would be remiss if we did not point out some very
troubling aspects of this notation.

Consider, when we introduced the concept of a differential in Chapter
\ref{chapt:differentials} we appealed to your understanding of
\emph{finite} differences. That is, if $x_1<x_2$ then the finite
difference 
$$
\Delta{}x=x_2-x_1
$$
is a measure of the actual separation between $x_2$ and $x_1.$
Extending this to differential, if we imagine that $x_2$ and $x_1$ are
\underline{infinitely} close together -- but still distinct -- then we
represented the distance between them as
$$
\d x = x_2-x_1.
$$

It is all well and good to \emph{write}  expressions like $\d x$ or
$\d y$ but what does $\d x$ represent, really?

If $x_2$ and $x_1$ are infinitely close together doesn't that mean that
they are \underline{not} separated at all? If they are not separated
then $x_2=x_1.$ In that case isn't $\d x =x_2-x_1$ just zero? And if
$\d x =0$ then isn't the expression $\dfdx{y}{x}$ dividing by zero,
which we know is meaningless?

Now consider the meaning of the \underline{second} derivative. If the
first derivative is obtained by computing the ratio of quantities that
are infinitely close together, what can it possibly mean to
differentiate it again? If we've already taken numbers that are
infinitesimally close together to compute the first derivative, what
can we possibly be using to compute the second, or the third
derivative?

What is the infinitesimal difference ($\d^2 x$) of an infinitesimal
difference ($\d x)?$ 

This is all very puzzling and it is tempting to toss it all away,
except that we have a great deal of confidence that
$\dfdx{\left(x^3\right)}{x}=3x^2,$ and that
$\dfdx{\left(3x^2\right)}{x}=6x.$ We've seen both of these used in the
last section to compute the roots of polynomials correctly. There must
be something to this, but it is extremely hard to tell at this
point what it is.

If we are honest with ourselves we must acknowledge these
questions. There is something here that we do not understand. We
would like to understand it, but we do not yet. We will proceed for
now, but very cautiously.

\digress{An Extension of the Tangent Line Idea}
When $\frac{-\pi}{2}\le x\le\frac{\pi}{2}$ a well known approximation
of $\cos(x)$ is
\[
\cos(x)\approx 1-\frac{x^2}{2}+\frac{x^4}{24}.
\]
An obvious question is, ``How do we obtain this approximation?'' That
is, when you compare the graphs it is clear that this polynomial stays
close to $\cos(x)$ for small values of $x,$ but how would anyone come
up with it in the first place?

This  is actually a straightforward consequence of the 
construction of tangent lines using the derivative.

We begin by constructing the tangent line of this graph at the point
\((0,1).\) This is not hard. If
\(y=\cos(x)\) then \(\left.\dfdx{y}{x}\right|_{x=0}=0. \) Therefore an equation of the
tangent line at \((0,1)\) is
\[
y=\underbrace{\left[\left.\dfdx{y}{x}\right|_{x=0}\right]}_{=0}(x-0) +1
\]
or 
\[y=1.\]
Most likely you knew that already.
 From following figure\\[2mm]
  \centerline{\includegraphics*[height=1.2in,width=2in]{Figures/CosApprox}}
we see that at the point of tangency the graphs of \(y=\cos(x)\) and
\(y=1\) share two interesting features: 
\begin{enumerate}
\item Both pass through the point \((0,1)\) and
\item  Both have a slope of zero at that point.
\end{enumerate}

That is, when $x=0$ the slope of $\cos(x)$ is the same as the slope of
our tangent line. Indeed, this is fairly obvious. That's what a
tangent line is. But suppose we could find a function whose first
\emph{and second} derivatives match $\cos(x)$ at $x=0?$ What would
that look like? 

That function is $y=1-\frac{x^2}{2}$ and it looks like this:
\InsertGraphic{}

You can see where this is going, right? The curve $y=
1-\frac{x^2}{2}+\frac{x^4}{24}$ is the polynomial with the property
that its first four derivatives at $x=0$ match the first four
derivatives of $\cos(x),$ also at  $x=0.$
\enddigress{}

% \subsection{Logarithmic and Exponential Functions}
% % \marginpar{Note to self: As per skype discussion with Bob, I need to
% %   re-imagine the beginning of this section along the lines of: There
% %   are a bazillion log and exp functions. What are their essential
% %   features? What makes something a logarithm? Why is the ``natural
% %   logarithm'' called \underline{natural?} What's so natural about it?
% %   It turns out that answering this last question answers them all.}

% There are two more kinds of functions whose differentials we will
% need. These are functions of the form $y=a^x$ and $y=\log_a(x)$ where
% $a$ is a a positive constant. These are called \emph{exponential}
% functions and \emph{logarithmic} functions respectively and you should
% be familiar with the unique properties of these functions from your
% earlier studies so we will not discuss them in depth here. On the
% other hand the properties of these functions are significant and
% useful so a brief reminder seems to be in order.

% \subsubsection*{Exponential Functions}
% Exponentials have the form $y = a^x,$ where $a$ is some positive
% number. An example would be \(
% y=10^x.
% \)
% A table of representative values for this is:\\
% \begin{center}
%   \begin{tabular}{c|c}
%     $x$&$y=10^x$\\\hline
%     $1$&$10$\\[2mm]
%     $2$&$100$\\[2mm]
%     $3$&$1000$\\[2mm]
%     $0$&$1$\\[2mm]
%     $-1$&$\frac{1}{10}$\\[2mm]
%     $-2$&$\frac{1}{100}$\\[2mm]
%     $-3$&$\frac{1}{1000}$\\
%   \end{tabular}
% \end{center}
% Another example is $y=2^x,$ and the corresponding table is:\\
% \begin{center}
%   \begin{tabular}{c|c}
%     $x$&$y=2^x$\\\hline
%     $1$&$2$\\[2mm]
%     $2$&$4$\\[2mm]
%     $3$&$8$\\[2mm]
%     $0$&$1$\\[2mm]
%     $-1$&$\frac{1}{2}$\\[2mm]
%     $-2$&$\frac{1}{4}$\\[2mm]
%     $-3$&$\frac{1}{8}$\\
%   \end{tabular}
% \end{center}

% \begin{embeddedproblem}{}
%   Compute the corresponding table for each of the following  exponential functions:
%   \begin{multicols}{4}
%     \begin{description}
%     \item[(a)] $y=7^x$
%     \item[(b)] $y=200^x$
%     \item[(c)] $y=(1.4)^x$
%     \item[(d)] $y=(0.5)^x$
%     \end{description}
%   \end{multicols}
% \end{embeddedproblem}

% Obviously there are many exponential functions. Their is one for each
% positive constant, $a.$ This is rather more than we'll want to think
% about, at least at first, so we will mostly consider a few
% representative examples.

% From the tables listed above one property of exponentials pops out
% immediately: the point $(0,1)$ is on the graph of all
% exponentials. This is because if $a$ is any positive real number then
% $a^0=1.$ This seems a fairly mundane observation but it will soon have
% important consequences so we take particular notice of it here.

% Otherwise the most consequential question about exponential functions
% is: How do we solve an equation like $b=a^x$ for $x?$ In some cases
% this is simplicity itself. For example the solution of \(100=10^x\) is
% obviously $x=2.$ Similarly the solution of \(16=2^x\) is $x=4.$ The
% solution of $1/9=3^x$ is a little harder but after a few seconds of
% thought it is clear that it is $x=-2.$

% But what is the solution of: \(5=10^x?\) The more you think about this
% one, the more opaque it seems to become. For what number $x$ could it
% possibly be true that \(5=10^x?\) It is clear from the graph of
% $y=10^x$ that there is such a number. It \emph{exists.} We can even
% say a few things about it that must be true. For example, it is the
% $x$ coordinate of the point where the graph of \(y=10^x\) intersects
% the graph of \(y=5.\) We just can't say what it is. Yet.

% We'll have to come at this from a slightly different direction if
% we're to have any hope of finding this number. Let's begin by looking
% at the equations we \emph{could} solve. We observed that the solution
% of \(100=10^x\) was $x=2.$ This is obviously true, but why? 

% If we write $100$ as $10^2$ we get \[10^2=10^x\]
% so clearly $x=2$ is the solution. Similarly, to solve \(1/9=3^x\)
% we write
% \begin{align*}
%   1/9&=3^x\\
%   1/3^2&=3^x\\
%   3^{-2}&=3^x  
% \end{align*}
% so clearly $x=-2.$ But to employ this technique to solve \(5=10^x\)
% we'd have to be able to express $5$ as a power of $10,$ so the whole
% thing looks hopeless.

% However \emph{if we could express $5$ as a power of $10$} we'd be
% done. That exponent would be our solution. This seems to be the whole
% problem: to express $5$ as a power of $10!$ Ok, so let's give that
% number a name. The name that has come down to us historically is
% ``logarithm'' which is rather a lot to say and write, so let's
% abbreviate it. Let just use the first three letters ``log''. Of course
% we'll want the name to indicate that it is the solution
% of the equation \(5=10^x\) so let's use the symbols $5$ and $10$ in
% the name as well. Let's use $\log_{10}(5)$ to represent the solution
% of \(5=10^x.\)

% Clearly then, $2=\log_{10}(100).$ But notice that $100=10^2.$ Using
% this gives $2=\log_{10}(10^2).$ Similarly,
% \begin{align*}
%   4&= \log_2(2^4)=\log_2(16)\\
% \intertext{because $4$ is the number we have
%       to raise $2$ to to get $16,$ and}
%   -2&= \log_3(3^{-2})=\log_3(1/9) \\
% \intertext{because $-2$ is the number we have
%      to raise $3$ to to get $1/9,$ and, significantly,}
%   0&= \log_a(1)\\
% \intertext{ because $0$ is the number we have
%      to raise any positive real number, $a$ to to get $1.$}
% \end{align*}

% So, it appears that the solution of $5=10^x$ is $x=\log_{10}(5),$
% because $\log_{10}(5)$ is the number we have to raise $10$ to to get
% $5.$ This is because ``$\log_{10}(5)$'' is the name we've given to that
% number.

% \subsubsection*{Logarithmic Functions}
% Logarithms are hard to think about. If you
% still have trouble thinking about these things, if the sight of the
% word ``logarithm'' sent a cold shiver down your spine please don't
% waste time berating yourself over it. And don't give up.  Keep
% working. Keep trying. If you need a more extensive review than we are
% giving here go get it. Use whatever resources are available to
% you. But remember this: If this \emph{seems} hard it is because it
% \emph{is} hard, not because you aren't capable of understanding
% it. Keep working. Keep trying. With practice comes familiarity. With
% familiarity comes comfort. And with comfort comes facility. With
% facility comes understanding. \underline{Keep trying.}


% Suppose $a$ is some positive number\footnote{It doesn't matter at the
%   moment which positive number it is; think of it as $2$ or $10$ if
%   that makes it easier for you to think about.}. Then ``the logarithm
% function with base $a$,'' is denoted with the symbols $\log_a(x).$
% This is too much to say (or write) each time we need it so when
% speaking we usually just say ``log base $a$.'' When referring to a
% logarithm function in writing, naturally we just use the notation
% $\log_a(x),$ since it was invented for that purpose.

% If $y=\log_a(x)$ then the following are true:\label{LogProp}
% \begin{enumerate}
% \item $\log_a(1)=0$
% \item $\log_a(xy)=\log_a(x)+\log_a(y)$
% \item $\log_a(x^y)=y\log_a(x)$
% \end{enumerate}
% These properties are definitive in the sense that if a function
% satisfies all three of them then that function is a log base $a$ for
% some positive number $a.$

% Our goal in this section is to determine the differential of a
% logarithm function. That is, if $y=\log_a(x)$ we'd like a
% differentiation formula for $\d y = \d (\log_a(x)).$ The following two
% observations will be helpful.

% \begin{description}
% \item [First Observation]
%   Our first observation is not new. In
%   section~\ref{subsec:diff-sine-funt} we stated and used (although we
%   didn't prove) the following: ``If the differential of a function is
%   zero then the function is constant.''  We emphasize that although
%   this seems intuitively obvious, there is room for doubt as we
%   indicated in section~\ref{subsec:diff-sine-funt}. It would be
%   comforting to prove this rigorously and remove any doubt. But for
%   now we simply rely on our intuition and assume that this is true.
% \item[Second Observation] 
% We have seen that $\d (x^n)=nx^{n-1}.$ Suppose we reverse this. That
% is, suppose we know that $\d y = x^n,$ and ask: What does $y$
% have\footnote{Actually, a more fundamental question is, ``Does there
%   \emph{have} to be a $y$ such that $\d y = x^n?$'' But it seems
%   intuitively clear that there must be, and indeed, we are about to
%   show you what it is so we will leave this for another time.} to be?
% Clearly all we have to do is run the
% Product Rule backwards. A few example
% suffices to show the pattern.

% \begin{enumerate}
% \item If $\d y = x\d x$ then $y=\frac{x^2}{2}.$
% \item If $\d y = x^2\d x$ then $y=\frac{x^3}{3}.$
% \item If $\d y = x^3\d x$ then $y=\frac{x^4}{4}.$
% \end{enumerate}
% In general, if $\d y=x^n\d x$ then $y=\frac{x^{n+1}}{n+1}.$ This would
% seem to be the end of the matter except that we know the
% Power Rule works for both positive and
% negative values of $n.$

% What if $n=-1?$

% In that case we have $\d y = \inverse{x}\d x = \frac1x\d x$ so that
% $y= \frac{x^0}{-1+1}$ which is utterly devoid of meaning since
% division by zero is not a meaningful operation.

% This is weird! If $n$ is \underline{any rational number whatsoever},
% except $-1,$ we can find $y$ if we know that $\d y = x^n\d x.$ But if
% $n=-1$ we get this apparently meaningless result when we try to find
% $y.$ Does this mean that there is no function whose differential is
% $\inverse{x},$ or does it simply mean that we don't know (yet) how to
% find it? At this point we have no way to determine which of these is
% correct, however in our experience nature is far more clever than we
% are so we'd be willing to bet that there is such a function. We just
% don't know (yet) how to find it. Let's proceed on the assumption that
% $y=\log_a(x),\ a>0,$ and that $\d y = \frac1x\d x$ and see where that
% takes us.
% \end{description}

% % We will need a name for this function. For now let's call it $W\!F(x)$
% % (for ``weird function''), and derive as many of it's properties as we
% % can. Maybe that will give us a hint as to its nature.

% So far we know only that if $y= \log_a(x)$ then $\d y = \frac1x\d x.$ We
% know this because that is how we defined $\log_a(x).$ 

% This is a rather strange way to define a logarightm though. In the
% past we've always started with a base, say $a=10,$ and examined the
% exponential, $y=10^x.$ The logarithm with base 10 is then the
% function which ``undoes'' the exponential: $\log_{10}(10^x)=x.$ 

% Since we don't know the base for our proposed logarithm it is hard to know
% how to proceed from here. However the conditions given on
% page~\pageref{LogProp} are definitive in the sense that any function
% which satisfies them must be a logarithm for \emph{some} base $a.$ So our
% task is to show that $\log_a(x)$ satisfies all three of those
% properties.

% We start simply. We ask: If  $k\ne0$ and 
% $y=\log_a(kx),$ what is $\d y?$ This is not hard. We have
% \begin{align*}
%   \d y &= \d(\log_a(kx))\\
%        &= \frac{1}{kx}\d(kx)\\
%        &= \frac{1}{kx}k\d x \text{ or}\\
%   \d y &= \frac{1}{x}\d x
% \end{align*}
% Now this is weird. It seems that the differential of $\log_a(kx)$ is
% exactly the same as the differential of $\log_a(x).$ This seems
% counterintuitive at best. Certainly $\log_a(x)$ and $\log_a(ax)$ are
% related, but they are decidedly different functions\footnote{Just as
%   $f(x)=x^2$ and $f(2x)=(2x)^2=4x^2$ are related but
%   different.}. 
% However there doesn't seem to be anything wrong with 
% our computations so let's accept the idea that two \emph{different}
% functions might have the same differential. What would be the
% consequence of that?

% Well, for one thing, the differential of their \emph{difference} would
% be zero, wouldn't it. Symbolically, if $\d(\log_a(kx)) = \d(\log_a(x))$ then
% $\d(\log_a(kx)) - \d(\log_a(x))=0.$ But wait a minute! From the
% Sum Rule we know that
% \[
%   \d\left[\log_a(kx)-\log_a(x)\right]=  \d(\log_a(kx))-\d(\log_a(x))
% \]
% so it follows that 
% \[
%   \d\left[\log_a(kx)-\log_a(x)\right]= 0
% \]
% and from our first observation above it follows that
% % \marginpar{Note to self: The notation used here and in the next few pages is very inconsistent. Sometimes I use x and y, sometimes x and a. Should probably reserve a for the base of the log and stick with x and y. In any case it needs to be fixed.}
% $  \log_a(kx)-\log_a(x) $ is a constant. That is,
% \[
%  \log_a(kx)-\log_a(x) = c
% \]
% for some constant $c.$ Or, to rearrange things just a bit:
% \begin{equation}
%  \log_a(ax)= \log_a(x) + c.\label{eq:LogProp1}
%  \end{equation}
% Since this is an \emph{identity} it is true no matter what value the
% variable $x$ has. Letting $x=1$ gives: $ \log_a(k)- \log_a(1) = c.$
% Substituting this into equation~\ref{eq:LogProp1} gives
% \begin{equation}
%   \label{eq:LogProp2Almost}
%    \log_a(kx)= \log_a(x) + \log_a(k)- \log_a(1).
% \end{equation}

% Now look back at the properties of logarithms listed on
% page~\pageref{LogProp}. The second property listed is
% \[
% \log_a(xy)=\log_a(x)+\log_a(y).
% \]
% If we knew that $\log_a(1) =0$ then formula~\ref{eq:LogProp2Almost}
% would look exactly like the second property of a logarithm. In that
% case we would have shown that $\!\log_a(x)$ satisfes two of the three
% definitive properties of a logarithm function, suggesting that it
% actually \emph{is} a logarithm function.

% Unfortunately nothing we have done so far allows us to \emph{conclude}
% that $\log_a(1)=0$ and, in fact, it is not always true. We'll need
% another idea, and another function.

% But wait! We've invested quite a bit of time and energy into our weird
% little function already. Let's not give up on it too soon. If we can't
% \emph{show} that $\log_a(1)=0,$ perhaps we can force the issue. That
% is, maybe we can modify this function ever-so-slightly, keeping the
% crucial additive property:\(\log_a(xy)=\log_a(x)+\log_a(y)\)
% but also forcing it's value at $1$ to be $0.$

% This sounds harder than it is. All we really have to do is define a
% new function, we'll call it $\log_e(x)$ 
% as follows:
% \begin{equation}
%   \label{eq:NatLog}
% \log_e(x) = \log_a(x)-\log_a(1),
% \end{equation}
% so that 
% \[
% \log_e(1) = \log_a(1)-\log_a(1) =0,
% \]
% as required. 

% The function $\log_e(x)$ satisfies all three of the properties on
% page!\pageref{LogProp} as we now show.

% \begin{description}
% \item[Property 1:] \(\log_e(1) = \log_a(1)-\log_a(1) =0.\)
% \item[Property 2:] Since adding or subtracting a
%   constant from a function doesn't change its differential we have
%   $\d(\log_e(x) = \d(\log_a(x) + \log_a(1)) = \frac1x.$
% \item[Property 3:] Notice that the differential of
%   $\log_e(x^n)$ is
% \[
% \d\left[\log_e(x^n)\right] = \frac{\d(x^n)}{x^n}  = 
% \frac{nx^{n-1}\d x}{x^n} = \frac{n}{x}\d x,
% \]
% and this is also the differential of the product $n\cdot \log_e(x).$
% So it follows that the difference $\log_e(x^n)-n\cdot \log_e(x)$ is a
% constant:
% \[
% \log_e(x^n)-n\cdot \log_e(x) = c.
% \]
% But this last formula is an identity. That is, it is true no matter
% what value of $x$ is used. Suppose $x=1.$ Then we have
% \[
% \log_e(1)-n\cdot \log_e(1) = c.
% \]
% But since $\log_e(1)=0$ this says that $c=0$ and therefore 
% \[
% \log_e(x^n)=n\cdot \log_e(x)
% \]
% \end{description}



% % \begin{embeddedproblem}{}
% %   Show that if $a$ is a constant then 
% % \[
% % \d(\log_e(ax) = \d(\log_e(x)).
% % \]
% % \end{embeddedproblem}


% % \begin{embeddedproblem}{}
% %   Show that $\log_e(ax) = \log_e(x) + \log_e(a).$
% % \end{embeddedproblem}

% % Our function $\log_e(x)$ satisfies two of the three definitive
% % properties of a logarithm. We make the following conjecture:
% % \begin{conjecture}
% %   Our new weird function, $\log_e(x)$ satisfies all of the following 
% %   \begin{enumerate}
% %   \item $\log_e(1)=0$
% %   \item $\log_e(xy)=\log_e(x)+\log_e(y)$
% %   \item $\log_e(x^y)=y\cdot \log_e(x)$
% %   \end{enumerate}
% %   and so it is actually a logarithm function for some, as yet unknown,
% %   base $a.$ 
% % \end{conjecture}
% % \begin{proof}[of our Conjecture]
% %   In order to show that our conjecture is true we need to show that
% %   all three of the properties of logarithms hold for $\log_e(x).$ The
% %   way we constructed $\log_e(x)$ guarantees the first property and
% %   the second property is the content of our last \emph{Problem in
% %     Context,} so apparently all that is left is to show the third
% %   property: $\log_e(x^y) = y\log_e(x).$ 

% %   The proof of the third property is very similar to the proof of the
% %   second. 
% % which is the third property of a logarithm.

% Since our function  $\log_e(x)$ satisfies all three of the definitive
% properties of a logarithm we can conclude that it is in fact a
% logarithm function. 

% This is a most odd way to define a logarithm. As we've observed
% previously, in the past we defined an exponential, say $10^x,$ and
% then defined the associated logarithm. Here we have defined a function
% and shown that it has all of the properties required of a logarithm,
% but we have \emph{not} identified the associated exponential. It is
% true that we've used the notation $\log_e(x)$ so it \emph{looks} like
% maybe the base of this logarithm is some number called $e,$ in which
% case the exponential will be $e^x.$ 

% But what is $e?$ If you look back at equation~\ref{eq:NatLog} you will see that
% we just slipped this notation in without explanation. So at this point
% the notation $\log_e$ is just a symbol that \emph{looks like} the
% symbol $\log_{10}.$ On the other hand if there is such a number, $e,$
% then we don't need to think of our new logarithm in such abstract
% terms. We can just think of it as the function that undoes $e^x$ in
% the same way that $\log_{10}$ undoes $10^x.$ So it is worth asking if
% we can find the number $e$ if indeed it exists.

% In principle, this is not hard to do. It is elementary that
% $\log_2(2)=1,$ that $\log_{10}(10)=1,$ and that in general the
% logarithm base $a$ of $a$ is always equal to one:
% \[
% \log_a(a)=1.
% \]
% Thus the number $e$ will, in principle, be the number such that
% $\log_e(e)=1.$ We say ``in principle'' because, although there clearly
% is such a number, it is nearly impossible to figure out much about it
% just yet. Strangely, this will not trouble us much at first so we will
% leave this for another time.

% The function $\log_e(x)$  is called the \emph{natural logarithm.} It
% is called ``natural'' because it arises quite naturally from the
% question, ``What is the function whose differential is $\frac1x\d x?$
% To distinguish it from all other logarithms we give it it's own
% special notation. Mathematicians simply called it $\log(x)$ because
% for us it is the only one that really matters. Engineers and
% scientists usually use \(\ln(x).\) For the sake of simplicity and
% consistency we will use $\ln(x)$ henceforth in this text.

% The defining properties of the natural logarithm are the three
% defining properties of \underbar{any} logarithm:
%   \begin{enumerate}
%   \item $\ln(1)=0$
%   \item $\ln(xy)=\ln(x)+\ln(y)$
%   \item $\ln(x^y)=y\ln(x),$ \\[2mm] 
% \noindent{}along with the differential property we started with:
%   \item $\d\left(\ln(x)\right)=\frac1x\d x.$
%   \end{enumerate}

% \subsubsection*{The Natural Exponential}
% \label{subsubsec:natural-exponential}
% The last function whose differential we will need to be able to
% compute is the function 
% \[
% y=e^x
% \]
% which is called the \emph{natural} exponential for the same reason
% reason that $\ln(x)$ is called the \emph{natural}
% logarithm. Fortunately all of the groundwork has been laid so this is
% very easy. Observe that since $\ln(x)$ and $e^x$ are mutually inverse
% we have 
% \begin{align*}
%   \ln(e^x) &= x\\
% \intertext{so that differentiating both sides gives}
%   \frac{1}{e^x}\d(e^x) &= \d x.\\
% \intertext{Solving for $\d(e^x)$ we see that }  
%   \d(e^x) &= e^x\d x.
% \end{align*}

% \begin{embeddedproblem}{}
%   Use the fact that $a=e^{\ln(x)}$ to show that \[\d(a^x) =
%   a^x\ln(a)\d x.\]
% \end{embeddedproblem}

% \begin{embeddedproblem}{}
%   Use the fact that $\log_a(x)=\frac{ln(x)}{\ln(a)}$ to show that \[\d(\log_a(x)) =
%   \frac{1}{x\ln(a)}\d x.\]
% \end{embeddedproblem}

% \begin{embeddedproblem}{}
%   Show that $\log_a(x)=\frac{\ln(x)}{\ln(a)}.$ 
% \end{embeddedproblem}


This completes the introductory portion of a first course
in Calculus. At this point we have derived all of the differentiation
rules you will need to differentiate a very large class of functions. 
\begin{ProblemSection}
\end{ProblemSection}


% Thus $\d x$ is a small (an infinitesimally small) displacement of $x.$
% Physicists and engineers call $\d x$ a ``virtual'' displacement,
% but they mean the same thing we do. In most cases it will be safe to
% think of the infinitesimal displacement, $\d x,$ and the finite
% displacement, $\Delta x,$ as the same thing. 

% Next we again consider a point moving along the $x$ axis, but this
% time we will denote its position as $r$ because we want to think of it
% as the radius of the inner circle in figure~\ref{fig:CircDiff}.

% We ask, ``What is the differential of the area of the inner circle?''
% That is, if we increment the radius $r$ by $\d r$ what is the
% corresponding increment, $\d A,$ in the area $A=\pi r^2?$ From
% figure~\ref{fig:CircDiff} it should be clear that $\d A$ is the area
% of the small ring between the inner and outer circles. The thickness
% of the ring is  $\d r.$

% Clearly when the radius increases from $r$ to $r+\d r$ the area of our
% circle increases also. The amount of increase is the area of the outer
% ring in the figure. This is $\d A,$ the differential chang in the
% area. Clearly $\d A$ is related to $\d r.$

% We will return to this example shortly in order to compute $\d A$
% exactly. For now our goal is simply to understand what the symbols
% $\d r$ and $\d A$ represent in figure~\ref{fig:CircDiff}. We repeat
% for emphasis: $\d r$ is an
% infinitesimal change in the the radius of the circle and $\d A$ is the
% corresponding infinitesimal change in its area. In our figure it is
% the area of the ring with radius $\d r.$

% Consider the situation in figure~\ref{fig:RectDiff}
% where one side of a rectangle has a fixed length of $5$ and the
% other is allowed to vary. Imagine that the varying length,
% $x,$ is being traced out by the motion of a point moving on the $x$
% axis. Then the area of the rectangle is given by the formula:
% \[A=5x\]
% and $\d A$ is clearly the small rectangular region indicated in the
% figure. Since $\d A$ is also a rectangle it is clear that 
% \[\d A =5\d x\]
% as expected.

% It is tempting to stop here but the fact is that rectangles are
% easy. More complex shapes are not. So we will do this computation
% again, this time by a method which may seem unnecessarily wieldy but
% which illuminates the process a little more clearly.

% Observe that when our point is at $x$ we have a rectangle whose area
% is 
% \[
% A(x) = 5x.
% \]
% When our point has moved on by an infinitesimal amount $\d x,$ we have
% a rectangle whose area
% is 
% \[
% A(x+\d x) = 5(x+\d x).
% \]
% From figure~\ref{fig:RectDiff} it is clear that 
% % we can find the
% % differential of $A$ by subtracting the smaller rectangle form the larger:
% \begin{align*}
% \d A &= A(x+\d x) - A(x)\\
%      &= 5(x+\d x) - 5x\\
% \d A &= 5\d x.
% \end{align*}

% We interpret this last formula as follows: If $x$ is incremented by
% $\d x$ then $A$ will be incremented by $5$ times $\d x.$ Dividing both
% sides by $\d x$ we get:
% \[
% \dfdx{A}{x} = 5
% \]
% which we interpret to mean that the area of our rectangle is
% increasing five times faster than its length. 

%\end{chapter}

%%% Local Variables: 
%%% mode: latex
%%% outline-minor-mode: t
%%% TeX-master: "Calculus"
%%% End: 
